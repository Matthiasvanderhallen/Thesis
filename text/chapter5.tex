\chapter{Proving Full Abstraction}
\label{chap:InformalProof}
Some formal techniques of proving full abstraction exist and the idea of one of these proof techniques, based on trace semantics, is sketched in \myref{sec}{sec:prooftechniques}.
%Using the ideas of this proof technique, \myref{sec}{sec:informalproof} aims to give a short and informal proof for the compilation scheme presented earlier.

\section{Formal Proof Techniques}
\label{sec:prooftechniques}
Recalling from \myref{sec}{sec:fullabstraction}, \emph{full abstraction} is a compiler property.
It states that \emph{contextual equivalence} for source-level objects is preserved by and reflected from their target-level translations.
\[
    O_{1} \simeq O_{2} \iff \compiled{O_{1}} \simeq \compiled{O_{2}}
\]

Proving full abstraction requires a proof for soundness  and completeness.
\begin{description}
\item[Soundness]
Soundness expresses that the compilation of two source-level objects does not `introduce' contextual equivalence in the target language.
Instead, for the target-level objects to be contextually equivalent, the source-level objects have to contextually equivalent already.
\[
    \compiled{O_{1}} \simeq \compiled{O_{2}} \implies O_{1} \simeq O_{2}
\]
Soundness corresponds closely to the informal notion of compiler \emph{correctness}.
Indeed, formulating the logically equivalent contrapositive of the soundness property gives:
\[
    O_{1} \not\simeq O_{2} \implies \compiled{O_{1}} \not\simeq \compiled{O_{2}}
\]

This expresses that two contextually \emph{un}equivalent source-level objects result in contextually \emph{un}equivalent translations.
If a compilation scheme would \emph{not} be sound, there would exist two contextually \emph{un}equivalent source-level objects, whose translations would be contextually equivalent.

Clearly, this compiler does not function `correctly', as there is a context in which the source-level objects behave differently, but the translations do not.
One of these translations does not accurately behave like the source-object it is derived from.
\item[Completeness]
Completeness says that all contextually equivalent source-level objects are translated to contextually equivalent target-level objects.
It expresses that the contextual equivalence of source-level objects, which provides certain security guarantees, are preserved when compiling.
\[
    O_{1} \simeq O_{2} \implies \compiled{O_{1}} \simeq \compiled{O_{2}}
\]
\end{description}

As most compilers are expected to be `correct' or sound, the most important part of the full abstraction proof is the proof of completeness.
Proving completeness can be done using \emph{trace semantics} and looking at the contrapositive of completeness.
\[
    \compiled{O_{1}} \not\simeq \compiled{O_{2}} \implies O_{1} \not\simeq O_{2}
\]
The following section will detail how trace semantics can be used to prove completeness of a compilation scheme.

\subsection{Trace Semantics}
\label{sec:tracesemantics}
Trace semantics~\cite{Rathke, Patrignani:TraceSemantics} describe the behavior of a component as a series of method calls and returns.

Full abstraction proofs based on trace semantics, such as~\cite{Patrignani,Agten:2012:SCM:2354412.2355247}, first show that the operational semantics of the target-language are equivalent to the proposed trace semantics.
As a consequence, the interaction of context objects $O_{c}$ with target-level objects $O_{1}$ and $O_{2}$ in the definition of contextual equivalence can be represented by traces $T_{1}$ \& $T_{2}$.

If two target-level objects $\compiled{O_1}$ and $\compiled{O_2}$ are not contextually equivalent, their traces $T_1$ and $T_2$ must be different.
The proof of fully abstract compilation then must show that if two source-level objects $O_1$,$O_2$ have different target-level traces $T_1$ and $T_2$, a source-level context object $O_C$ can be created from those two traces.
This $O_C$ will be able to serve as a source-level object that differentiates between two objects $O_1$ and $O_2$.
The existence of such an object effectively means the source-level objects $O_1$ and $O_2$ are not contextually equivalent, which concludes the completeness proof.

%\section{Informal Proof}
%\label{sec:informalproof}
%
%This section gives a short and informal reasoning that suggests that the compilation scheme presented in \myref{sec}{sec:formalizedcompiler2} is fully abstract.
%The argument is highly influenced by the formal proof technique described in \myref{sec}{sec:tracesemantics}.
%\\[1em]
%The first important step in a formal proof would be showing that the operational semantics of the target-language have an equivalent trace semantics.
%The secure compilation ensures this by limiting the information passed between insecure context and secure code.
%
%This is achieved by clearing flags and registers~\cite{Agten:2012:SCM:2354412.2355247} when control flow passes from secure code to insecure context.
%Secondly, the leaking of information about memory locations is prevented.
%This is partly achieved by the use of masking~\cite{Patrignani}, and for the other part by making fields available using the generic structure value entry point and a (masked) reference to a frame.
%
%Information leakage about the order of definitions, whether it be value definitions of structure definitions, is prevented by imposing an alphabetical ordering on the definitions.
%An alphabetic reordering at compilation makes sure that the order of fields within a frame, or frames within the f-list is insensitive to a reordering of the source-level code.
%The addition of functors requires additional care to ensure that there is no way of telling whether a structure was created by a static binding or using functor application.
%\\[1em]
%The next step would be showing that any pair of low-level traces $T_1$ and $T_2$ that differentiate two compiled objects $\compiled{O_1}$ and $\compiled{O_2}$ makes it possible to construct a high level context $O_c$ that could differentiate between $O_1$ and $O_2$.
%To ensure this, tampering with the control flow must be prevented.
%
%In the compilation scheme presented here, the control flow is secured by the memory access model, which limits the execution rights of the insecure context to the entry points specified.
%Only by jumping to these entry points can the execution switch between the insecure context and the secure code.
%
%However, jumping to one of the entry points corresponds to making a call to something in the API of the source-level object.
%Because every call from insecure context to secure code listed in the trace corresponds to a call to the API of the source-level object, it can be shown that a pair of traces can be merged to a source-level context object. 
%This source-level context object will be able to differentiate between the source-level objects, just as the pair of traces could for the target-level translations.
%\\[1em]
%The existence of such a source-level context object shows that the source-level objects are not contextually equivalent.
%This proves the completeness property, using its contrapositive form $\compiled{O_{1}} \not\simeq \compiled{O_{2}} \implies O_{1} \not\simeq O_{2}$.
%%Find other references as well!