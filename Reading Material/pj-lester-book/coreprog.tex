\chapter{Example Core-language programs}
\label{sect:core-progs}

In this Appendix we give a few Core-language programs which are useful
for testing some of the implementations developed in the book.
They assume that the functions defined in the prelude
(Section~\ref{sect:prelude}) are defined.

\section{Basic programs}

The programs in this section require only integer constants and
function application.

\subsection{Ultra-basic tests}

This program should return the value \mbox{\tt 3} rather quickly!
\begin{verbatim}
        main = I 3
\end{verbatim}
The next program requires a couple more steps before returning \mbox{\tt 3}.
\begin{verbatim}
        id = S K K ;
        main = id 3
\end{verbatim}
This one makes quite a few applications of \mbox{\tt id} (how many?).
\begin{verbatim}
        id = S K K ;
        main = twice twice twice id 3
\end{verbatim}

\subsection{Testing updating}

This program should show up the difference between a system which does
updating and one which does not.
If updating occurs, the evaluation of \mbox{\tt (I\ I\ I)} should take place only once;
without updating it will take place twice.
\begin{verbatim}
        main = twice (I I I) 3
\end{verbatim}

\subsection{A more interesting example}

This example uses a functional representation of lists
(see Section~\ref{sect:template:data-str-hof}) to build an infinite
list of \mbox{\tt 4}'s, and then takes its second element.  The functions for
head and tail (\mbox{\tt hd} and \mbox{\tt tl}) return \mbox{\tt abort}\indexTT{abort} if their argument is an empty
list.  The \mbox{\tt abort}\indexTT{abort} supercombinator just generates an infinite loop.
\begin{verbatim}
        cons a b cc cn = cc a b ;
        nil      cc cn = cn ;
        hd list = list K abort ;
        tl list = list K1 abort ;
        abort = abort ;

        infinite x = cons x (infinite x) ;

        main = hd (tl (infinite 4))
\end{verbatim}

\section{\mbox{\tt let}\indexTT{let} and \mbox{\tt letrec}\indexTT{letrec}}

If updating is implemented, then this program will execute in fewer steps
than if not, because the evaluation of \mbox{\tt id1}\indexTT{id1} is shared.
\begin{verbatim}
        main = let id1 = I I I
               in id1 id1 3
\end{verbatim}
We should test nested \mbox{\tt let}\indexTT{let} expressions too:
\begin{verbatim}
        oct g x = let h = twice g
                  in let k = twice h
                  in k (k x) ;
        main = oct I 4
\end{verbatim}
The next program tests \mbox{\tt letrec}\indexTT{letrec}s, using `functional lists' based on
the earlier definitions of \mbox{\tt cons}\indexTT{cons}, \mbox{\tt nil}\indexTT{nil}, etc.
\begin{verbatim}
        infinite x = letrec xs = cons x xs
                     in xs ;
        main = hd (tl (tl (infinite 4)))
\end{verbatim}

\section{Arithmetic}

\subsection{No conditionals}

We begin with simple tests which do not require the conditional.
\begin{verbatim}
        main = 4*5+(2-5)
\end{verbatim}
This next program needs function calls to work properly.  Try replacing
\mbox{\tt twice\ twice} with \mbox{\tt twice\ twice\ twice} or \mbox{\tt twice\ twice\ twice\ twice}.  Predict
what the result should be.
\begin{verbatim}
        inc x = x+1;
        main = twice twice inc 4
\end{verbatim}
Using functional lists again, we can write a length function:
\begin{verbatim}
        length xs = xs length1 0 ;
        length1 x xs = 1 + (length xs) ;

        main = length (cons 3 (cons 3 (cons 3 nil)))
\end{verbatim}

\subsection{With conditionals}

Once we have conditionals we can at last write `interesting' programs.
For example, factorial:
\begin{verbatim}
        fac n = if (n==0) 1 (n * fac (n-1)) ;
        main = fac 5
\end{verbatim}
The next program computes the greatest common divisor of two integers,
using Euclid's algorithm:
\begin{verbatim}
        gcd a b = if (a==b)
                        a
                  if (a<b) (gcd b a) (gcd b (a-b)) ;
        main = gcd 6 10
\end{verbatim}

The \mbox{\tt nfib}\indexTT{nfib} function is interesting because its result (an integer) gives
a count of how many function calls were made during its execution.  So
the result divided by the execution time gives a performance measure in
function calls per second.  As a result, \mbox{\tt nfib}\indexTT{nfib} is quite widely used as a
benchmark.  The `nfib-number' for a particular implementation needs to
be taken with an enormous
dose of salt, however, because it is critically dependent
on various rather specialised optimisations.
\begin{verbatim}
        nfib n = if (n==0) 1 (1 + nfib (n-1) + nfib (n-2)) ;
        main = nfib 4
\end{verbatim}

\section{Data structures}

This program returns a list of descending integers.  The evaluator
should be expecting a list as the result of the program.
\mbox{\tt cons}\indexTT{cons} and \mbox{\tt nil}\indexTT{nil} are now expected to be implemented in the
prelude as \mbox{\tt Pack{\char'173}2,2{\char'175}}\indexTT{Pack} and \mbox{\tt Pack{\char'173}1,0{\char'175}}\indexTT{Pack} respectively.
\begin{verbatim}
        downfrom n = if (n == 0)
                          nil
                          (cons n (downfrom (n-1))) ;
        main = downfrom 4
\end{verbatim}

The next program implements the Sieve of Eratosthenes to generate the
infinite list of primes, and takes the first few elements of the result list.
If you arrange that output is printed incrementally, as it is generated, you
can remove the call to \mbox{\tt take}\indexTT{take} and just print the infinite list.
\begin{verbatim}
        main = take 3 (sieve (from 2)) ;

        from n = cons n (from (n+1)) ;

        sieve xs = case xs of
                        <1> -> nil ;
                        <2> p ps -> cons p (sieve (filter (nonMultiple p) ps)) ;

        filter predicate xs
                = case xs of
                        <1> -> nil ;
                        <2> p ps -> let rest = filter predicate ps
                                    in
                                    if (predicate p) (cons p rest) rest ;

        nonMultiple p n = ((n/p)*p) ~= n ;

        take n xs = if (n==0)
                        nil
                        (case xs of
                                <1> -> nil ;
                                <2> p ps -> cons p (take (n-1) ps))

\end{verbatim}
