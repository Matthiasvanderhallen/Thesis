\chapter{Formal Specification \label{chapter:formalspecification}}
In this chapter the source language, \MiniML, and the target language, LLVM IR, will be formally specified. 

First the syntax of \MiniML\ within which a program can be defined will be introduced.
Next, the typing rules that a correct program must adhere to is shown.
Lastly, the operational semantics determine what a correct program must do once it runs.
\section{MiniML}
\subsection{Syntax}
First we introduce the syntax of a \MiniML\ program, as seen in \myref{fig}{fig:Syntax}.
Like ML, the \MiniML\ syntax is composed of three parts: the \emph{Core} language, a \emph{Module} language\cite{Milner:1997:DSM:549659}, and the concept of a \emph{Program}. 
The three languages, \emph{Core}, \emph{Module} and \emph{Program} each have their corresponding valid \emph{expression}s.
The division between \emph{Core} and \emph{Module} mostly correlates to the concepts of `programming in the small' and `programming in the large'\cite{Milner:1997:DSM:549659,DeRemer:1975:PLV:390016.808431} respectively.
This separation of programming in the large and programming in the small aims to help programmers to introduce the right degree of modularity in their software.

\subsubsection{Core language}
The \emph{Core} mainly consists of \emph{value expressions} \cmath{e}, these express the manipulation of values and execution of functions to implement small algorithms or control logic.
Every value expression \cmath{e} needs to have a corresponding type \cmath{\tau}, otherwise the expression is not \emph{sound}.
The \emph{Core} language syntax is shown in \myref{fig}{fig:CoreSyntax}.

\begin{figure}[htb]
%\begin{subfigure}{0.4\textwidth}
%\centering
\begin{align*}
\begin{aligned}
%\text{Program } \\
%P\; ::= \; & \overline{\mathit{Mod}};\;e & \\
%\\
\text{Val Exp } \\
e \; ::= \; &\mathit{num \; n \;}     &\expl{(natural number)} \\
&|\;\mathit{id}                       &\expl{(value identifier)}\\
&|\;\mathit{StrId.id}                 &\expl{(struct value)}\\
&|\;e_{1}e_{2}                        & \expl{(function application)}\\
&|\;\lambda(x:\tau)\;.\;e             &\expl{(lambda function)}\\
&|\;\text{\lsttext{let}}\; x : \tau \; =\; e_{1}\; \text{\lsttext{in}} \; e_{2}
                                      &\expl{(let binding)}\\
%&\longspace \longspace \;\;\;\   \\
&|\;(e_{1}, e_{2})                    &\expl{(pair)}\\
&|\;e.\#1 \;| \;e.\#2                 &\expl{(pair projection)}\\
&|\;[]                                &\expl{(empty list)}\\
&|\;e::e                              &\expl{(list concatenation)}
% &|\;\mathit{letrec} \; p \; = \; e_{1} \; in \; e_{2} \\
%&|\; \mathit{if(e_{1}) \; then \; e_{2} \; else \; e_{3}}\\
% &|\;\mathit{p.left}\; | \; \mathit{p.right} \\
% &|\;\mathit{fix\;e} \\
\\
\text{Identifiers } \\
x \; ::= \; &id                      & \expl{(value identifier)}\\ 
&|\;StrId                            & \expl{structure identifier}\\
% &|\;F_{i}\\
% &|\;M_{i}.id\\
\\
\text{Types }\\
\tau \; ::= \; &\text{\lsttext{int}} &\expl{(int type)}\\
&| \; \mathit{StrId}.t                &\expl{(struct type)}\\
% &| \; \mathit{bool} \\
&| \; \tau_{1} \rightarrow \tau_{2}  & \expl{(function type)}\\
&| \; \alpha                         & \expl{(type variable)}\\
&| \; \tau_{1} \times \tau_{2}       & \expl{(pair type)}\\
&| \; [\tau]                         & \expl{(array type)}\\
% &| \; \alpha\\
%\\
%\text{Access Path } \mathit{path} \; ::= \; &\mathit{M_{i}}\\
% &|\;F_{i}(\overline{\mathit{M_{i}}})\\
%&|\;\mathit{this}\\
%\\
%\\
\end{aligned}
\end{align*}
\caption{Core language syntax\label{fig:CoreSyntax}}
\end{figure}


\subsubsection{Module language}
The \emph{Module} language uses \emph{module expressions} \cmath{me} to how the small parts of \emph{Core} expressions can be `glued together' or \emph{composed} into larger, working programs.
Its syntax is shown in \myref{fig}{fig:ModuleSyntax}.

The \emph{Module} language brings encapsulation and namespaces to \MiniML.
The \emph{Module} language does this by introducing the concept of a \emph{structure}, as was informally explained in \myref{sec}{sec:MiniML}.
A structure is defined using a \lsttext{struct} expression, and bound to an identifier \cmath{StrId} using the \lsttext{structure} expression.
It consists of a body of definitions which is denoted in the syntax as $\overline{\mathit{\delta}}$ using the bar notation for lists\footnote{
The bar notation uses $\emptyset$ as the empty set and the comma (,) as the prepend operator.
For example: $\overline{\mathit{\delta}}$ could be the empty set $\emptyset$ or it could be $id = e:\tau, \overline{\delta}$.}.
In its body, a structure \cmath{StrId} can bind values \cmath{id} to a value expression \cmath{e}, or can define a new type \cmath{t}.

Just as types in the \emph{Core} language limit the number of valid or \emph{sound} value expressions, signatures restrict the number of well-typed module expressions.
Signatures are named and bound to their identifier using the \lsttext{signature} expression. 
They are defined using the \lsttext{sig} expression, and their body consists of abstract type definitions, type synonyms and value declarations.

A structure can be ascribed a signature using \emph{transparant} ($:$) or \emph{opaque} ($:>$) ascription.
\begin{description}
\item[Transparant ascription] Transparant ascription \cmath{StrId : SigId} lets the implementation of type definitions in the underlying structure \cmath{StrId} propagate through, while hiding from external view any values that were not declared in the signature \cmath{SigId}.
\item[Opaque ascription] Opaque ascription \cmath{StrId :> SigId} restricts the external view of the structure to the values and types declared inside the signature \cmath{SigId}.
If a type \cmath{t} is declared abstract in \cmath{SigId}, its implementation is not known to any code not local to the structure.
\end{description}

In order to obtain a more simple language to study, \MiniML\ signatures of restrict their declarations to a subset \cmath{T} of the \emph{Core} types \cmath{\tau}. 
The subset \cmath{T} contains only opaque types defined by structures, the int and function types.
As a result, any value that is accessible from outside the structure itself can only get parameters and return values of type \lsttext{int}, of an opaque type or of a function combination of those types.

This results in arrays \cmath{[\tau]} and pairs \cmath{\tau_{1} \times \tau_{2}} not being primitive types for module expressions.
Only within a structure can a value be treated as an array or pair, if the value is of an opaque type defined within the same structure with an implementation containing the array or pair type.

\begin{figure}[!htb]
\newcommand{\ova}{\overline{\alpha}}
\begin{align*}
\begin{aligned}
%
\text{Mod Exp }\\
\mathit{me} ::= \; & \text{\lsttext{structure}} \;  \mathit{StrId : SigId} = \text{\lsttext{struct}}\; \overline{\mathit{d}}\; \text{\lsttext{end}}
                                             & \expl{(Trans. struct. binding)}\\
&|\; \text{\lsttext{structure}} \;  \mathit{StrId :> SigId} = 
\text{\lsttext{struct}}\; \overline{\mathit{d}}\; \text{\lsttext{end}}
                                             & \expl{(Opaque struct. binding)}\\
&|\; \text{\lsttext{signature}} \; SigId = 
\text{\lsttext{sig}}\; \Sigma \; \text{\lsttext{end}} 
                                             & \expl{(sig. binding)}\\
%
\text{Definition }\\
d \; ::= \; &\mathit{id}=e:\tau                    & \expl{(value def.)} \\
& | \text{\lsttext{type}}\ \ova\ t= \tau           & \expl{(type def.)} \\
% TODO: type variables
%
\text{Signature body } \\
\Sigma \; ::= \; &\overline{\delta}                    & \expl{(Signature body)}\\
%
\text{Declaration }\\
\delta \; ::= \; & \text{\lsttext{type}}\ \ova\ t  & \expl{(abstract type declaration.)} \\
& | \; \text{\lsttext{type}}\ \ova\ t = T          & \expl{(type synonym)} \\
& | \; \text{\lsttext{val}}\ \mathit{id}:T         & \expl{(value declaration)} \\
%
\text{Shared type } \\
T \; ::= \; &\text{\lsttext{int}}                  & \expl{(type int)} \\
&| \; p.t                                          & \expl{(struct type)}\\
&| \; T_{1} \rightarrow T_{2}                      & \expl{(function type)}\\
&| \; \alpha                                       & \expl{(type variable)}
\end{aligned}
\end{align*}
\caption{Module language syntax \label{fig:ModuleSyntax}}
\label{fig:Syntax}
\end{figure}


\subsubsection{Program}
A \MiniML\ program consists of a set of \emph{Module expressions}.
%, denoted as \cmath{\overline{\mathit{me}}}. 
It is then concluded by a single naked value expression \cmath{e}, functioning as the main entry point of the program.
This description of a program allows us to first specify a set of signatures as well as a set of structures conforming to those signatures.
The syntax of a Program is shown in \myref{fig}{fig:ProgramSyntax}

\begin{figure}[!htb]
\captionsetup{skip=0pt}
\begin{align*}
\begin{aligned}
\text{Program } \\
%\mathit{P} ::= \; &\overline{\mathit{me}}, e & \expl{(Program definition)}
\mathit{P} ::= \; & \mathit{me}, P & \expl{(Module expression prefix)}\\
&|\; e & \expl{(Program entry point)}\\
\end{aligned}
\end{align*}
\caption[Syntax: Program Language]{Program language syntax. \label{fig:ProgramSyntax}}
\label{fig:Syntax}
\end{figure}


%A signature $\Sigma$ is a module type and is represented by a list of \emph{declarations}. A declaration $\Delta$ specifies the type of an value identifier or the signature of a module identifier.
%\\[2ex] 
%A module can be seen as a special case of functors. A module specifies a signature and module body and is uniquely identified with an identifier $M_{i}$. The module body is represented as a list of definitions $\overline{d}$. A module $M_{i}$ with body $\overline{d}$ conforms to a signature $S_{i}$ if every identifier in $S_{i}$ has a definition in the module body, and its typing does not violate the one specified in the signature.
%\\[2ex]
%Functors $F_{i}$, presented as a generalization of modules, specify \todo{Is it always possible to specify the interface of a module? In our simple system, yes, There can be no problem with opaque types since we don't support them.} their own signature, $S_{i}$, as well as a set of signatures upon which it depends, $\overline{S_{n}}$. It then specifies a functor body in which those modules can be used. The functor can be given a set of modules that conform to the dependent signatures. We say the functor is applied to a set of modules. The result of this application behaves as a module that conforms to the interface $S_{i}$ that the functor specified for itself.

%\subsubsection{Syntax example}
%We now give an example of a syntactically correct \mbox{MiniML} program.

%\begin{figure}[!htbp]
%\begin{verbatim}
%test
%\end{verbatim}
%\caption{Syntax example}
%\label{code:SyntaxExample}
%\end{figure}

\subsection{Type system}
Having defined the syntax for \MiniML and its parts, this section formalizes the static semantics, also called its \emph{type system}.
The type system restricts the world of possible programs to those that consist of \emph{sound} or \emph{well-typed} module and value expressions. 

\subsubsection{Type-schemes and contexts}
First, the concept of a type-scheme is introduced, as shown in \myref{fig}{fig:Type-schemesAndContexts}.
A type-scheme, sometimes called polytype, introduces polymorphism by making use of the type variable $\alpha$ in the definition of $\tau$, and quantifying it with the universal quantifier $\forall$.
This allows any concrete types $\tau$ to 'match' to the type variable.
For example, the identity function \inlinecode{id} is typed $id:\forall \alpha. \alpha \rightarrow \alpha$, thereby introducing parametrized polymorphism which enables one to use the same \inlinecode{id} function everywhere regardless of the arguments type.

This concept of a type-scheme will later be used to provide %\todo{explain further}
let-polymorphism. 
Note that the definition of a Type-Scheme assures that the resulting type-scheme is in \emph{prenex normal form}, i.e. a string of quantifiers concluded by a quantifier-free ending.
\\[2ex]
Our type system will also need to keep track of the type assumptions and the module, functor and signature definitions. This represents the notion of a \emph{context}. It is in this context that typing will happen. While type checking, the context is what the type checker uses to keep track of the facts it already knows.
\\[2ex]
To access mappings from these contexts, we will introduce projections. For example $\Gamma[M_{i}].\overline{d}$ will look up the mapping $(M_{i} \mapsto \lbrace S,\overline{d}\rbrace)$ in $\Gamma$ and project this to the $\overline{d}$ specified in the mapping. A lookup will \emph{fail} if the identifier has no mapping in the context.

\begin{figure}[!htb]
\begin{align*}
\begin{aligned}
\text{Context }\Gamma ::=\; &\emptyset \\
&| \; (x:\sigma),\Gamma \\
&| \; (M_{i} \mapsto \lbrace \Sigma,\overline{d}\rbrace), \Gamma \\
% &| \; (F_{i} \mapsto \lbrace \Sigma, \overline{\Sigma_{n}}, \overline{d} \rbrace), \Gamma \\
&| \; (S_{i} \mapsto \Sigma), \Gamma
\end{aligned}
\begin{aligned}
\longspace
\end{aligned}
\begin{aligned}
\text{Type-Scheme } \sigma \; ::= \; &\tau \\
&| \; \forall \alpha . \sigma \\
\\
\\
\end{aligned}
\end{align*}
\caption{Type-schemes and contexts in the MiniML type system.}
\label{fig:Type-schemesAndContexts}
\end{figure}

We are now in a position to define a few helpful relations between type-schemes, types and contexts: type-scheme specialization and type-scheme generalization.

\subsubsection{Type-scheme specialization}
The specialization relation $\sigma_{1} \geq \sigma_{2}$ expresses that $\sigma_{2}$ is more specialised than $\sigma_{1}$. This means that the following rule holds:
% $\sigma_{2}$ can be expressed as $\forall \beta_{i}...\beta_{m}.\sigma_{2}'$ and $\sigma_{1}$ as $\forall \alpha_{1}...\forall \alpha_{n}.\sigma_{1}'$, $\sigma_{2}$ is more specialised than $\sigma_{1}$ iff $\sigma_{2}'=[\alpha_{i} \mapsto \sigma_{i}]\sigma_{1}'$ and $\beta_{i} \in free(\sigma_{1})$. In other words, 
%

\[
\tag{specialization}
\frac{\tau_{2}=[\alpha_{i} \mapsto \tau_{i}]\tau_{1} \longspace \beta_{i} \not\in\mathit{free(\alpha_{1}...\forall \alpha_{n}.\tau_{1})}}
{\forall\alpha_{1}...\forall\alpha_{n}.\tau_{1}\geq \forall \beta_{i}...\forall \beta_{m}\tau_{2}'}
\]

In other words, the quantifier-free ending of the more specialized type-scheme can be obtained by consistently replacing all quantified type variables $\alpha_{i}$ in the more general type-scheme by a type $\tau_{i}$, which can possibly contain type variables itself, resulting in the quantifier-free ending of the more specialized type scheme. Furthermore, only variables that were not free in the more general type-scheme can be bound in the specialized type-scheme.

The first condition gives one the possibility to specify the type of a type variable. This second condition forbids one to \emph{rescope} a type variable in the process.

\subsubsection{Type-scheme generalization}
Type-scheme generalisation is the opposite process of type-scheme specialization. However, whereas specialization can be expressed independent of the context, whether or not one is allowed to generalize, is dependent on the context. Generalisation allows one to quantify an unquantified variable, as long as it does not appear unquantified in any type expression in the current context.

\[
\tag{generalization}
\frac{\Gamma \vdash e:\Sigma \longspace \alpha \not\in \mathit{free(\Gamma)}}{\Gamma \vdash e : \forall \alpha . \sigma}
\]


\subsubsection{Typing judgements}
To type check our program, the type checker will perform typing judgements. These typing judgements, which can bee seen in \myref{fig}{fig:TypingJudgements}, are relations between the context and parts of the syntax. They convey the meaning that an expression or other part of the syntax is well-typed in the context $\Gamma$. The typing of a module body and its definitions generates a new typing context $\Gamma'$ for the module. In this resulting context, the declarations must be well-typed.

The $\Gamma \vdash \Diamond$ judgement is a statement of well-formedness of a context $\Gamma$. A context is well-formed if the keyset of the lookup table it represents conforms to the standard notion of a set, meaning every key is used only once.

\begin{figure}[!htb]
\begin{align*}
\text{ExpressionTyping } ::=\;&\Gamma \vdash e: \sigma \\
\text{ModuleTyping } ::= \; &\Gamma \vdash \mathit{Mod} \\
\text{DefinitionTyping } ::= \; &\Gamma \vdash d \rightarrow \Gamma' \\
\text{DeclarationTyping } ::= \;&\Gamma \vdash \Delta \\
\text{Well-formedness } ::=\;&\Gamma \vdash \Diamond
\end{align*}
\caption{Typing judgements in the MiniML type system.}
\label{fig:TypingJudgements}
\end{figure}

\subsubsection{Rules}
%\begin{figure}
\begin{align*}
&\Gamma \vdash true : bool \tag{T-True} \\
&\Gamma \vdash false : bool \tag{T-False} \\
&\Gamma \vdash num \; n : nat \tag{T-Num} \\ \\
\tag{T-Mono}
&\frac{\sigma \geq \tau \longspace id:\sigma \in \Gamma}{\Gamma \vdash id:\tau}\\ \\
\tag{T-App}
&\frac{\Gamma \vdash e_{1}:\tau_{2} \rightarrow \tau_{1} \longspace \Gamma \vdash e_{2}:\tau_{2}}
{\Gamma \vdash e_{1}e_{2}:\tau_{1}} \\ \\
\tag{BuildContext1}
& id:\sigma \rightarrow \emptyset, (id:\sigma) \\ \\
\tag{BuildContext2}
&\frac{p_{1}:\sigma_{1} \rightarrow \Gamma_{1} \longspace p_{2}:\sigma_{2}\rightarrow \Gamma_{2}}
{(p_{1},p_{2}):\sigma_{1}\times \sigma_{2} \rightarrow \Gamma_{1}\cup \Gamma_{2}} \\ \\
\tag{T-Fun}
&\frac{p:\tau_{2} \rightarrow \Gamma_{2} \longspace \Gamma_{2} \cup \Gamma_{1} \vdash e:\tau_{1}}
{\Gamma_{1} \vdash \lambda(p:\tau).e:\tau_{2} \rightarrow \tau_{1}} \\ \\
\tag{T-IfThenElse}
&\frac{\Gamma \vdash e_{1}:bool \longspace \Gamma \vdash e_{2}:\tau \longspace \Gamma \vdash e_{3} : \tau}
{\Gamma \vdash if \; e_{1} \; then \; e_{2} \; else \; e_{3} : \tau} \\ \\
%\tag{T-Pair}
%&\frac{\Gamma \vdash e_{1}:\tau_{1} \longspace \Gamma \vdash e_{2}:%\tau_{2}}
%{\Gamma \vdash (e_{1},e_{2}) : \tau_{1}\times\tau_{2}} \\ \\
%\tag{T-PairLeft}
%&\frac{\Gamma \vdash p:\tau_{1}\times\tau_{2}}
%{\Gamma \vdash \mathit{p.left} : \tau_{1}} \\
%\\
% \tag{T-PairRight}
%&\frac{\Gamma \vdash p:\tau_{1}\times\tau_{2}}
%{\Gamma \vdash \mathit{p.right} : \tau_{2}} \\
%\\
\tag{T-Let}
&\frac{\Gamma \vdash e_{2}:\tau_{2} \;\;\; \sigma=gen(\Gamma,\tau)\;\;\;p:\sigma\rightarrow \Gamma_{2} \;\;\; \Gamma \cup \Gamma_{2} \vdash e_{1}:\tau}
{\Gamma \vdash let\;p\;=\;e_{2}\;in\;e_{1}:\tau} 
%\\ \\
%\tag{T-Letrec}
%&\frac{\Gamma \vdash let\;p\;=\;\mathit{fix}\;(\lambda p.e_{2})\;in\%;e_{1}:\tau}
%{\Gamma \vdash letrec\;p\;=\;e_{2}\;in\;e_{1}:\tau} \\ \\
%\tag{T-Fix}
%&\frac{\Gamma \vdash e : \tau \rightarrow \tau}
%{\Gamma \vdash \mathit{fix\;e} : \tau} \\
%\displaybreak
\\
\tag{T-ModVarThis}
&\frac{\sigma \geq \tau \longspace this.id:\sigma \in \Gamma}
{\Gamma \vdash this.id : \tau} \\ 
\\
\tag{T-ModVarOther}
&\frac{
\Gamma \vdash M_{i}
\longspace id:\tau \in \Gamma[M_{i}].\Sigma}
{\Gamma \vdash \mathit{M_{i}.id} : \tau} \\
\\
%\tag{T-FunctorVar}
%&\frac{\overline{\Gamma \vdash M_{1..n}}
%\longspace
%\overline{\Sigma_{3} \succeq \Gamma[M_{1..n}].\Sigma}
%\longspace
%\Gamma \vdash F_{i}(\overline{M_{1..n}})
%\longspace
%id:\tau \in \Gamma[F_{i}].\Sigma_{1}}
%{\Gamma \vdash \mathit{F_{i}(\overline{M_{1..n}}).id}:\tau} \\
%\\
\displaybreak
\tag{T-Module}
&\frac{
\emptyset \vdash \Gamma[M_{i}].\overline{d}\rightarrow \Gamma' \longspace \Gamma' \vdash \Gamma[M_{i}].\Sigma_{i}}
{\Gamma \vdash M_{i}} \\
\\
%\tag{T-Functor}
%&\frac{
%\emptyset \vdash [\overline{M_{1..n} \mapsto M_{arg}}]\overline{d} \rightarrow \Gamma' \longspace \Gamma' \vdash \Gamma[F_{i}].\Sigma}
%{\Gamma \vdash F_{i}(\overline{M_{args}})}\\
\\
\tag{T-ModInterfaceField}
&\frac{(x:\tau) \in \Gamma \longspace \Gamma \vdash \Delta}
{\Gamma \vdash (x:\tau),\Delta}\\
\\
\tag{T-ModInterfaceModule}
&\frac{(M=\lbrace \Sigma_{2}, \overline{d} \rbrace^{M}) \in \Gamma
\longspace \Sigma_{1} \succeq \Sigma_{2} 
\longspace \Gamma \vdash \Delta}
{\Gamma \vdash (M:\Sigma_{1}),\Delta}\\
\\
\tag{T-ModBodyV}
&\frac{ (x:\tau),\Gamma \vdash \overline{d} \rightarrow \Gamma' \longspace \Gamma \vdash e:\tau}
{\Gamma \vdash (x=e:\tau),\overline{d} \rightarrow (x:\tau),\Gamma'} \\
\\
\tag{T-ModBodyM}
&\frac{\Gamma \vdash \lbrace\Sigma, \overline{d} \rbrace^{M_{i}} }
{\Gamma \vdash (\mathit{M_{i}=\overline{d}}:\Sigma),\overline{d} \rightarrow (M_{i}=\lbrace \Sigma,\overline{d} \rbrace),\Gamma'} \\
\\
\tag{T-EmptySet}
&\frac{\Gamma \vdash \Diamond}
{\Gamma \vdash \emptyset}
\end{align*}
%\end{figure}
\todo{Must specify $\succeq$ to mean the specialization of an interface}

\subsection{Operational semantics}
\begin{align*}
\text{Value }v ::=\;&\mathit{num\;n} \; | \; \mathit{true} \; | \; \mathit{false} \\
% &| (v,v) \\
&| \lambda p.e\\
\\
\text{Module Table } T\; ::= \;&\emptyset \\
&| \; (M_{i} \mapsto \lbrace \Sigma,\overline{d}\rbrace), T \\
%&| \; (F_{i} \mapsto \lbrace \Sigma, \overline{\Sigma_{n}}, \overline{d} \rbrace), T \\
&| \; (S_{i} \mapsto \Sigma), T\\
\\
\text{Evaluation } ::= T \vdash &e \rightarrow T \vdash e' \\
\end{align*}

The operational semantics defines a module table T, containing mappings from the signature-%, module and functor identifiers to their definition and and the evaluation relation. 
and module identifiers to their definition and and the evaluation relation. 

The module table T allows looking up the definition behind a certain identifier and accessing a certain part of it using projection. $T[M_{i}].\Sigma$ will give access to the $\Sigma$ in the definition of $M_{i}$. 

The evaluation relation allows the evaluation of an expression $e$ to a (simpler) expression $e'$, while potentially making a lookup in T.

\subsubsection{Rules}
\begin{align*}
\tag{E-IfTrue}
&T \vdash if \; true  \; then \; e_{1} \; else \; e_{2} \rightarrow T \vdash e_{1}\\
\tag{E-IfFalse}
&T \vdash if \; false \; then \; e_{1} \; else \; e_{2} \rightarrow T \vdash e_{2}\\ \\
\tag{E-IfThenElse}
&\frac{T \vdash e_{1} \rightarrow T \vdash e_{1}'}
{T \vdash if \; e_{1} \; then \; e_{2} \; else \; e_{3} \rightarrow T \vdash if \; e_{1}' \; then \; e_{2} \; else \; e_{3}}\\ \\
%\tag{E-PairLeft}
%&\frac{T \vdash e_{1} \rightarrow T \vdash e_{1}'}
%{T \vdash (e_{1},e_{2}) \rightarrow T \vdash (e_{1}',e_{2})} \\ \\
%\tag{E-PairRight}
%&\frac{T \vdash e_{2} \rightarrow T \vdash e_{2}'}
%{T \vdash (e_{1},e_{2}) \rightarrow T \vdash (e_{1},e_{2}')} \\ \\
\tag{E-Let}
&\frac{T \vdash e_{1}\rightarrow T \vdash e_{1}'}
{T \vdash let \; p \; = \; e_{1} \; in \; e_{2} \rightarrow T \vdash let \; p \; = \; e_{1}' \; in \; e_{2}}
\\ \\
\tag{E-LetV}
&T \vdash let \; id \; = \; v \; in \; e \rightarrow T \vdash [id \mapsto v]e \\ \\
%\tag{E-LetRec}
%& T \vdash letrec\;p=\;e_{1} \; in \; e_{2} \rightarrow T \vdash let \; p \; = %fix(\lambda p.e_{1}') \; in \; e_{2} \\ \\ 
%\tag{E-Fix}
%&\frac{T \vdash e\rightarrow T \vdash e'}
%{T \vdash fix(e) \rightarrow T \vdash fix(e')}\\ \\
%\tag{E-FixRec}
%&T \vdash fix(\lambda(p.e)) \rightarrow T \vdash [p \mapsto (fix %(\lambda(p.e))]e \\
\\
\tag{E-PatternMatch}
&T \vdash let \; (p_{1},p_{2}) \; = \; (e_{1},e_{2}) \; in \; e_{3} \rightarrow
let \; p_{1} \; = \; e_{1} \; in \;
(let \; p_{2}  \; = \; e_{2} \; in \; e_{3}) \\ \\
\tag{E-App1}
&\frac{T \vdash e_{1} \rightarrow T \vdash e_{1}'}
{T \vdash e_{1} e_{2} \rightarrow T \vdash e_{1}' e_{2}}\\ \\
\tag{E-App2}
&\frac{T \vdash e_{2} \rightarrow T \vdash e_{2}'}
{T \vdash v\;e_{2} \rightarrow T \vdash v\;e_{2}'}\\ \\
\tag{E-Lambda}
&T \vdash (\lambda x . e) \; v \rightarrow T \vdash [x \mapsto v]e \\ \\
\tag{E-MatchLambda}
&T \vdash (\lambda (p_{1},p_{2}) . e_{3}) \; (e_{1},e_{2}) \rightarrow T \vdash (\lambda p_{1}.(\lambda p_{2}.e_{3})\;e_{2})\; e_{1} \\
%\displaybreak
\\
\tag{E-ModVar}
&\frac{\longspace (x=e':\tau) \in T[M_{i}].\overline{d} \longspace e=[this.y\mapsto M.y]e'\;\forall (this.y \in e')}
{T \vdash M.x \rightarrow T \vdash e}\\
\\
%\tag{E-FunVar}
%&\frac{\longspace (x=e':\tau) \in T[F_{i}].\overline{d} \longspace %e=[M_{1..n} \mapsto M_{args}][this.y\mapsto F_{i}%(\overline{M_{args}}).y]e'\;\forall (this.y \in e')}
%{T \vdash F_{1}(\overline{M_{args}}).x \rightarrow T \vdash e}
\end{align*}
%\todo{How to express the substitution of all references to argument placeholder module names to the modules names given at execution}
%\todo{provide a desugar function}
%\end{flushleft}

\subsubsection{Structural Typing\label{sec:StructuralTyping}}
%TODO: VERY, Hugely important!

\section{LLVM Intermediate Representation}
The LLVM Intermediate Representation is a language very reminiscent of assembly. 
\subsubsection{Syntax}
In \myref{fig}{fig:LLVMSyntax}, the reduced syntax of LLVM is given\todo{Formalize this citation into the correct style}\todo{Expand the syntax}.\footnote{Taken from "Formalizing the LLVM Intermediate
Representation for Verified Program
Transformations"}
\begin{figure}[!htb]
\begin{align*}
\begin{aligned}
\text{Modules }\mathit{mod} ::= &\overline{\mathit{prod}} \\
\text{Products }prod ::= & \mathit{global}\ \mathit{typ}\ \mathit{const}\ \mathit{align}\ | \mathit{define\ typ\ id(\overline{arg})\{\overline{b}\}}\\
&| \mathit{declare\ typ\ id(\overline{arg})} \\
\text{Types } \mathit{typ} ::= &\mathit{ isz\ |\ void\ |\ typ*\ |\ \left[sz \times typ\right]\ |\ \lbrace\ \overline{typ_{j}}^{j}\ \rbrace\ |\ typ\ \overline{typ_{j}}^{j}\ \rbrace\ | id}
\end{aligned}
\end{align*}
\label{fig:LLVMSyntax}
\caption{The reduced LLVM Syntax, taken from Jianzhou Zhao et al.}
\end{figure}

%\text{Types } \mathit{typ\ ::= }&\mathit{ isz | void | typ* | } 

%\end{align*}

\section{Formalized Compiler}
\newcommand{\compile}[1]{\left[\left[#1\right]\right]}
\newcommand{\makes}{& \rightarrow}
\begin{align*}
\begin{aligned}
\compile{\bar{S};\bar{M};e} \makes \compile{\bar{M}}^{\bar{S}};\compile{e}\\ 
\compile{\bar{M}}^{\bar{S}}\makes \compile{M_{i}:S_{i} = \bar{d}};\compile{\bar{M}}^{\bar{S}}\\
\compile{M_{i}:S_{i} = \bar{d}} \makes \compile{\bar{d}}^{S_{i}} \mathit{\ with\ } S_{i} \in \bar{S} \\
\compile{d:\bar{d}}^{S_{i}} \makes \compile{d}^{S_{i}};\compile{\bar{d}}^{S_{i}}
\end{aligned}
\end{align*}