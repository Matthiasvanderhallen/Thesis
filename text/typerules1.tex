\begin{figure}[htbp]
\begin{subfigure}{\textwidth}
\begin{align*}
%&\Gamma \vdash true : bool \tag{T-True} \\
%&\Gamma \vdash false : bool \tag{T-False} \\
&\Gamma \vdash num \; n : nat \tag{T-Num} \\ \\
\tag{T-Mono}
&\frac{\sigma \geq \tau \longspace id:\sigma \in \Gamma}{\Gamma \vdash id:\tau}\\ \\
\tag{T-App}
&\frac{\Gamma \vdash e_{1}:\tau_{2} \rightarrow \tau_{1} \longspace \Gamma \vdash e_{2}:\tau_{2}}
{\Gamma \vdash e_{1}e_{2}:\tau_{1}} \\ \\
\tag{BuildContext1}
& id:\sigma \rightarrow \emptyset, (id:\sigma) \\ \\
\tag{BuildContext2}
&\frac{p_{1}:\sigma_{1} \rightarrow \Gamma_{1} \longspace p_{2}:\sigma_{2}\rightarrow \Gamma_{2}}
{(p_{1},p_{2}):\sigma_{1}\times \sigma_{2} \rightarrow \Gamma_{1}\cup \Gamma_{2}} \\ \\
\tag{T-Fun}
&\frac{p:\tau_{2} \rightarrow \Gamma_{2} \longspace \Gamma_{2} \cup \Gamma_{1} \vdash e:\tau_{1}}
{\Gamma_{1} \vdash \lambda(p:\tau).e:\tau_{2} \rightarrow \tau_{1}} \\ \\
%
%\tag{T-IfThenElse}
%&\frac{\Gamma \vdash e_{1}:bool \longspace \Gamma \vdash e_{2}:\tau \longspace \Gamma \vdash e_{3} : \tau}
%{\Gamma \vdash if \; e_{1} \; then \; e_{2} \; else \; e_{3} : \tau} \\ \\
%
%\tag{T-Pair}
%&\frac{\Gamma \vdash e_{1}:\tau_{1} \longspace \Gamma \vdash e_{2}:%\tau_{2}}
%{\Gamma \vdash (e_{1},e_{2}) : \tau_{1}\times\tau_{2}} \\ \\
%\tag{T-PairLeft}
%&\frac{\Gamma \vdash p:\tau_{1}\times\tau_{2}}
%{\Gamma \vdash \mathit{p.left} : \tau_{1}} \\
%\\
% \tag{T-PairRight}
%&\frac{\Gamma \vdash p:\tau_{1}\times\tau_{2}}
%{\Gamma \vdash \mathit{p.right} : \tau_{2}} \\
%\\
\tag{T-Let}
&\frac{\Gamma \vdash e_{2}:\tau_{2} \;\;\; \sigma=gen(\Gamma,\tau)\;\;\;p:\sigma\rightarrow \Gamma_{2} \;\;\; \Gamma \cup \Gamma_{2} \vdash e_{1}:\tau}
{\Gamma \vdash let\;p\;=\;e_{2}\;in\;e_{1}:\tau}
\end{align*}
\caption{Typing rules for the Core language. \label{fig:TypeRulesCore}}
\end{subfigure}
\\ \\
\begin{subfigure}{\textwidth}
%\end{figure}
%\begin{figure}
\begin{align*}
\tag{T-Structure}
&\frac
{\Gamma[SigId].\Sigma \succeq \mathit{PS}(\overline{d})}
{\Gamma \vdash StrId : SigId = \overline{d} \rightarrow (\mathit{StrId} \mapsto \lbrace \mathit{ES}(\overline{d}), \overline{d}\rbrace),\Gamma}
%\\
%\\
%\tag{T-Structure-opaque}
%&\frac
%{\Gamma[SigId].\Sigma \succeq PS(\overline{d})}
%{\Gamma \vdash StrId :> SigId = \overline{d} \rightarrow (\mathit{StrId} \mapsto \lbrace%\Gamma[SigId].\Sigma, \overline{d} \rbrace ),\Gamma}
\\
\\
\tag{T-ModVar}
&\frac
{StrId.id:\tau \in \Gamma[StrId].\Sigma}
{\Gamma \vdash \mathit{StrId.id} : \tau}
\\
\\
\tag{T-ModType}
&
\frac
{\Gamma \vdash e:\tau \longspace t = \tau' \in \gamma[\mathit{StrId}].\Sigma \longspace \tau' \geq \tau}
{\Gamma \vdash e : \mathit{StrId.t}}
%This rule says that if an expression evaluates to a StrId.t type in an expression, the type must be transparent, or it must come from StrId.id with that type.
\\
\\
\tag{T-Signature}
&\Gamma \vdash SigId = \Sigma \rightarrow (\mathit{SigId} \mapsto \Sigma),\Gamma
%\\
%\\
%&\frac{
%\emptyset \vdash \overline{d}\rightarrow \Gamma' }
%{PS(\overline{d})} \\
\end{align*}
\caption{Typing rules for the Module language. \label{fig:TypeRulesModule}}
\end{subfigure}
\\ \\
\begin{subfigure}{\textwidth}
%\end{figure}
%\begin{figure}

\begin{align*}
&\frac
{\Gamma \vdash me \rightarrow \Gamma' \longspace \Gamma' \vdash \Diamond \longspace \Gamma' \vdash P}
{\Gamma \vdash me, P}
\end{align*}
\caption{Typing rules for the Program language. \label{fig:TypeRulesProgram}}
\end{subfigure}
\caption{Typing rules for \MiniML.}
\end{figure}


%\begin{figure}
%\begin{align*}
%%\\ \\
%%\tag{T-Letrec}
%%&\frac{\Gamma \vdash let\;p\;=\;\mathit{fix}\;(\lambda p.e_{2})\;in\%;e_{1}:\tau}
%%{\Gamma \vdash letrec\;p\;=\;e_{2}\;in\;e_{1}:\tau} \\ \\
%%\tag{T-Fix}
%%&\frac{\Gamma \vdash e : \tau \rightarrow \tau}
%%{\Gamma \vdash \mathit{fix\;e} : \tau} \\
%%\displaybreak
%\\
%\tag{T-ModVarThis}
%&\frac{\sigma \geq \tau \longspace this.id:\sigma \in \Gamma}
%{\Gamma \vdash this.id : \tau} \\ 
%\\
%\tag{T-ModVarOther}
%&\frac{
%\Gamma \vdash M_{i}
%\longspace id:\tau \in \Gamma[M_{i}].\Sigma}
%{\Gamma \vdash \mathit{M_{i}.id} : \tau} \\
%\\
%%\tag{T-FunctorVar}
%%&\frac{\overline{\Gamma \vdash M_{1..n}}
%%\longspace
%%\overline{\Sigma_{3} \succeq \Gamma[M_{1..n}].\Sigma}
%%\longspace
%%\Gamma \vdash F_{i}(\overline{M_{1..n}})
%%\longspace
%%id:\tau \in \Gamma[F_{i}].\Sigma_{1}}
%%{\Gamma \vdash \mathit{F_{i}(\overline{M_{1..n}}).id}:\tau} \\
%%\\
%\tag{T-Module}
%&\frac{
%\emptyset \vdash \Gamma[M_{i}].\overline{d}\rightarrow \Gamma' \longspace \Gamma' \vdash \Gamma[M_{i}].\Sigma_{i}}
%{\Gamma \vdash M_{i}} \\
%\\
%%\tag{T-Functor}
%%&\frac{
%%\emptyset \vdash [\overline{M_{1..n} \mapsto M_{arg}}]\overline{d} \rightarrow \Gamma' \longspace \Gamma' \vdash \Gamma[F_{i}].\Sigma}
%%{\Gamma \vdash F_{i}(\overline{M_{args}})}\\
%%\\
%\tag{T-ModInterfaceField}
%&\frac{(x:\tau) \in \Gamma \longspace \Gamma \vdash \Delta}
%{\Gamma \vdash (x:\tau),\Delta}\\
%\\
%\tag{T-ModInterfaceModule}
%&\frac{(M=\lbrace \Sigma_{2}, \overline{d} \rbrace^{M}) \in \Gamma
%\longspace \Sigma_{1} \succeq \Sigma_{2} 
%\longspace \Gamma \vdash \Delta}
%{\Gamma \vdash (M:\Sigma_{1}),\Delta}\\
%\\
%\tag{T-ModBodyV}
%&\frac{ (x:\tau),\Gamma \vdash \overline{d} \rightarrow \Gamma' \longspace \Gamma \vdash e:\tau}
%{\Gamma \vdash (x=e:\tau),\overline{d} \rightarrow (x:\tau),\Gamma'} \\
%\\
%\tag{T-ModBodyM}
%&\frac{\Gamma \vdash \lbrace\Sigma, \overline{d} \rbrace^{M_{i}} }
%{\Gamma \vdash (\mathit{M_{i}=\overline{d}}:\Sigma),\overline{d} \rightarrow (M_{i}=\lbrace \Sigma,\overline{d} \rbrace),\Gamma'} \\
%\\
%\tag{T-EmptySet}
%&\frac{\Gamma \vdash \Diamond}
%{\Gamma \vdash \emptyset}
%\end{align*}
%\caption{Typing rules for the Module language. \label{fig:TypeRulesModule}}
%\end{figure}
