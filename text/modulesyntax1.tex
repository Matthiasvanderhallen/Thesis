\begin{figure}[!htb]
\begin{align*}
\begin{aligned}
%
\text{Mod Exp }\\
\mathit{me} ::= \; & \text{\lsttext{structure}} \;  \mathit{StrId : SigId} = \text{\lsttext{struct}}\; \overline{\mathit{d}}\; \text{\lsttext{end}}
                                             & \expl{(trans. struct. binding)}\\
&|\; \text{\lsttext{structure}} \;  \mathit{StrId :> SigId} = 
\text{\lsttext{struct}}\; \overline{\mathit{d}}\; \text{\lsttext{end}}
                                             & \expl{(opaque struct. binding)}\\
&|\; \text{\lsttext{signature}} \; SigId = 
\text{\lsttext{sig}}\; \Sigma \; \text{\lsttext{end}} 
                                             & \expl{(sig. binding)}\\
%
\text{Definition }\\
%d \; ::= \; &\mathit{id}=e:\tau                    & \expl{(value def.)} \\
d \; ::= \; &\text{\lsttext{val}}\ \mathit{id}=e:\tau   & \expl{(value def.)} \\
& |\; \text{\lsttext{fun}}\  \mathit{id}\ \overline{x}\ =e:\tau   & \expl{(value def.)} \\
& |\; \text{\lsttext{type}}\ \ova\ t= \tau           & \expl{(type def.)} \\
% TO DO: type variables
%
\text{Signature body } \\
\Sigma \; ::= \; &\overline{\delta}                    & \expl{(signature body)}\\
%
\text{Declaration }\\
\delta \; ::= \; & \text{\lsttext{type}}\ \ova\ t  & \expl{(abstract type declaration.)} \\
& | \; \text{\lsttext{type}}\ \ova\ t = T          & \expl{(type synonym)} \\
& | \; \text{\lsttext{val}}\ \mathit{id}:T         & \expl{(value declaration)} \\
%
\text{Shared type } \\
T \; ::= \; &\text{\lsttext{int}}                  & \expl{(type int)} \\
&| \; \mathit{StrId}.t \ \overline{\tau}           & \expl{(struct type)}\\
&| \; \overline{T_{1}} \rightarrow T_{2}           & \expl{(function type)}\\
&| \; \alpha                                       & \expl{(type variable)}
%\text{Fun type} \\
%T_{F} \; ::= \; & T \rightarrow T_{F}              & \expl{(function type)}
%T_{}
\end{aligned}
\end{align*}
\caption[Syntax: Module Language]{Module language syntax. \label{fig:ModuleSyntax}}
\label{fig:Syntax}
\end{figure}
