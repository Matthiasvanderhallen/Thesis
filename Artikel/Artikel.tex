\documentclass[10pt]{article}
\title{Secure Compilation and The ML Language}
\author{Matthias van der Hallen}
\date{}
\usepackage{listings}
\usepackage{lmodern}
\usepackage{amsmath}
\usepackage{amsfonts}
\usepackage{amssymb}
\usepackage{todonotes}
\usepackage{hyperref}
\usepackage{listings}
\usepackage{float}
\usepackage{lmodern}
\usepackage{graphicx}
\usepackage{caption}
\usepackage{subcaption}
\usepackage{xcolor,colortbl}
%\usepackage{longtable}
\usepackage{ltablex}
\usepackage{appendix}
\usepackage{amsthm}
\newcommand{\com}[1]{#1^\downarrow}
\newcounter{line}


\DeclareMathOperator\compat{\ensuremath{\raisebox{1mm}{$\frown$}}}
\DeclareMathOperator\nsimeq{\ensuremath{\simeq\!\!\!\!\!/\ }}
\DeclareMathOperator\rbisim{\ensuremath{\mc{R}}} 
\DeclareMathOperator\impl{\ensuremath{\Rightarrow}} 
\DeclareMathOperator\teq{\ensuremath{\simeq_{T}}}
\DeclareMathOperator\nteq{\ensuremath{\nsimeq\!\!_{T}}}
\DeclareMathOperator\oeq{\ensuremath{\simeq_{O}}}
\DeclareMathOperator\wbis{\ensuremath{\approx}}

%%%%%%%%%%%%%%%%%%%%% fonts
\newcommand{\mt}[1]{\ensuremath{\texttt{#1}}}
\newcommand{\mtt}[1]{\ensuremath{\mathtt{#1}}}
\newcommand{\mf}[1]{\ensuremath{\mathbf{#1}}}
\newcommand{\mi}[1]{\ensuremath{\mathit{#1}}}
\newcommand{\mc}[1]{\ensuremath{\mathcal{#1}}}
\newcommand{\ms}[1]{\ensuremath{\mathsf{#1}}}
\newcommand{\mb}[1]{\ensuremath{\mathbb{#1}}}

%%%%%%%%%%%%%%%%%%%%% shorts
\newcommand{\acron}[0]{\ensuremath{\mbox{FPMAC}}\xspace}
\newcommand{\acrons}[0]{\ensuremath{\mbox{FPMACs}}\xspace}

\newcommand{\OB}[1]{\ensuremath{\overline{#1}}}
\newcommand{\subt}[0]{\ensuremath{<:}}
\newcommand{\xto}[1]{\ensuremath{\xrightarrow{~#1~}}}
\newcommand{\Xto}[1]{\ensuremath{\xRightarrow{~#1~}}}
\newcommand{\xtol}[1]{\ensuremath{\xrightarrow{~#1~}\low}}
\newcommand{\Xtol}[1]{\ensuremath{\xRightarrow{~#1~}\Low}}
\newcommand{\nXtol}[1]{\ensuremath{\xRightarrow{~#1~}\nLow}}
\newcommand{\tol}[0]{\ensuremath{\to}} %{\to\low}}
\newcommand{\low}[0]{\ensuremath{\!\!\!\!\to} }
\newcommand{\Low}[0]{\ensuremath{\!\!\!\!\Rightarrow} }
\newcommand{\nLow}[0]{\ensuremath{\!\!\!\!\!\!\!\!/\!\!\Rightarrow} }
\newcommand{\da}[0]{\ensuremath{^{\downarrow}}}
\newcommand{\ua}[1]{\ensuremath{\uparrow\!\!(#1)}}
\newcommand{\Ua}[0]{\ensuremath{\Uparrow}}
\newcommand{\myset}[2]{\ensuremath{\{#1 ~|~ #2\}}}
\newcommand{\sta}[1]{\ensuremath{\widehat{#1}}}

\newcommand{\toli}[0]{\ensuremath{\to^{\!\!\!\!\!i}\ }}
\newcommand{\tole}[0]{\ensuremath{\to^{\!\!\!\!\!e}\ }}


\newcommand{\jjr}[0]{Java~Jr.\xspace}

\newcommand{\divr}[0]{\ensuremath{\!\!\Uparrow\xspace}}

\newcommand{\totr}[1]{\ensuremath{\lfloor #1 \rfloor}}
\newcommand{\toex}[1]{\ensuremath{\lceil #1 \rceil}}

\newcommand{\fun}[2]{\ensuremath{\mtt{#1}(#2)}}
\newcommand{\dom}[1]{\ensuremath{\fun{dom}{#1}}}
\newcommand{\mex}[1]{\ensuremath{\fun{m_{ext}}{#1}}}
\newcommand{\mse}[1]{\ensuremath{\fun{m_{sec}}{#1}}}
\newcommand{\stu}[0]{\ensuremath{^{\bot}\xspace}}
\newcommand{\ldiv}[0]{\ensuremath{^{\Uparrow L}\xspace}}

\newcommand{\assline}[1]{\arabic{line}~~ \mtt{#1}\stepcounter{line}}

\newcommand{\sps}[0]{\ms{SP_{sec}}\xspace}
\newcommand{\spe}[0]{\ms{SP_{ext}}\xspace}

\newcommand{\TL}[1]{\ensuremath{\ms{Tr}(#1)}}
\newcommand{\trl}[0]{\ensuremath{\ms{Tr^L}}\xspace}
\newcommand{\trs}[0]{\ensuremath{\ms{Tr^S}}\xspace}

\newcommand{\blk}[0]{\ensuremath{(\ms{unknown},m,s)}\xspace}

\newcommand{\comm}[1]{}
%%%%%%%%%%%%%%%%%%%%%%%% typing & rules 
\newcommand{\typerule}[3]{\ensuremath{\begin{array}{c}\textsf{\scriptsize ({#1})} \\#2 \\\hline\raisebox{-3pt}{\ensuremath{#3}}\end{array}}}

\newcommand{\raisedrule}[2][0em]{\leaders\hbox{\rule[#1]{1pt}{#2}}\hfill}

\newcommand{\toprule}[1]{\begin{center}\vspace{1mm}\noindent\raisedrule{0.3mm}\ \raisebox{-0.5ex}{\emph{#1}} \raisedrule{0.3mm}\!\raisedrule{0.3mm}\!\raisedrule{0.3mm}}

\newcommand{\botrule}{\vspace*{2mm}\HRule\end{center}}

\newcommand{\HRule}[0]{\rule{\linewidth}{0.3mm}\vspace*{-1mm}}%\rule{\linewidth}{0.3mm}}

%%%%%%%%%%%%%%%%%%%%%%%% misc
\newcommand{\MP}[1]{\todo[color=blue!30]{TODO:#1}}
\newcommand{\myparagraph}[1]{\smallskip \noindent\noindent\textit{#1.}~~}


%%%%%%%%%%%%%%%%%%%%%%%% environments
\newtheorem{assumption}{Assumption}
\newtheorem{notation}{Notation}


\newenvironment{proofsketch}{\trivlist\item[]\emph{Proof Sketch}.\xspace}{\unskip\nobreak\hskip 1em plus 1fil\nobreak$\Box$\parfillskip=0pt\endtrivlist}


%%%%%%%%%%%%%%%%%%%%%%%% pictures
\pgfdeclarelayer{background}
\pgfdeclarelayer{back2}
\pgfdeclarelayer{foreground}
\pgfsetlayers{background,back2,main,foreground}

\newcommand{\tikzpic}[1]{
\begin{tikzpicture}[shorten >=1pt,auto,node distance=6mm,rounded corners]
\tikzstyle{state} =[fill=white,minimum size=4pt]
#1
\end{tikzpicture}
}
\newcommand{\myfig}[3]{\begin{figure} [!h]
#1
\caption{\label{fig:#2}#3}
\end{figure}}

\newcommand{\myfigb}[3]{\begin{figure} [!b]
#1
\caption{\label{fig:#2}#3}
\end{figure}}

%%%%%%%%%%%%%    Easier way to use references   
\newcommand{\customref}[2]{\hyperref[#2]{#1~\ref*{#2}}}

\newcommand{\myref}[2]{%
  \ifthenelse{\equal{#1}{lst}}{\customref{Listing}{#2}}{
  \ifthenelse{\equal{#1}{fig}}{\customref{Fig.}{#1:#2}}{%
  \ifthenelse{\equal{#1}{def}}{\customref{Definition}{#1:#2}}{%
  \ifthenelse{\equal{#1}{lem}}{\customref{Lemma}{#1:#2}}{%
  \ifthenelse{\equal{#1}{thm}}{\customref{Theorem}{#1:#2}}{%
  \ifthenelse{\equal{#1}{not}}{\customref{Notation}{#1:#2}}{%
  \ifthenelse{\equal{#1}{prop}}{\customref{Proposition}{#1:#2}}{%
  \ifthenelse{\equal{#1}{ax}}{\customref{Axiom}{#1:#2}}{%
  \ifthenelse{\equal{#1}{ex}}{\customref{Example}{#1:#2}}{%
  \ifthenelse{\equal{#1}{propr}}{\customref{Property}{#1:#2}}{%
  \ifthenelse{\equal{#1}{ass}}{\customref{Assumption}{#1:#2}}{%
  \ifthenelse{\equal{#1}{tab}}{\customref{Table}{#1:#2}}{%
  \ifthenelse{\equal{#1}{sec}}{\hyperref[#2]{Section~\ref*{#2}}}{%
  \ifthenelse{\equal{#1}{eq}}{\hyperref[#1:#2]{Equation~(\ref*{#1:#2})}}{%
    \ifthenelse{\equal{#1}{for}}{\hyperref[#1:#2]{Formula~(\ref*{#1:#2})}}{%
  \autoref{#1:#2}%
  }}}}}}}}}}}}}}}}
  
  
%use only when citing Example 1 and 2.  The Example 1 is cited with \myref, in order to obtain the right 2, use \myrefdrop
\newcommand{\myrefand}[2]{%
  \ifthenelse{\equal{#1}{fig}}{\customref{}{#1:#2}}{%
  \ifthenelse{\equal{#1}{def}}{\customref{}{#1:#2}}{%
  \ifthenelse{\equal{#1}{lem}}{\customref{}{#1:#2}}{%
  \ifthenelse{\equal{#1}{not}}{\customref{}{#1:#2}}{%
  \ifthenelse{\equal{#1}{thm}}{\customref{}{#1:#2}}{%
  \ifthenelse{\equal{#1}{prop}}{\customref{}{#1:#2}}{%
  \ifthenelse{\equal{#1}{ax}}{\customref{}{#1:#2}}{%
  \ifthenelse{\equal{#1}{ex}}{\customref{}{#1:#2}}{%
  \ifthenelse{\equal{#1}{propr}}{\customref{}{#1:#2}}{%
  \ifthenelse{\equal{#1}{ass}}{\customref{}{#1:#2}}{%
  \ifthenelse{\equal{#1}{tab}}{\customref{}{#1:#2}}{%
    \ifthenelse{\equal{#1}{lis}}{\customref{}{#1:#2}}{%
  \ifthenelse{\equal{#1}{sec}}{\hyperref[#1:#2]{\ref*{#1:#2}}}{%
  \ifthenelse{\equal{#1}{eq}}{\hyperref[#1:#2]{(\ref*{#1:#2})}}{%
    \ifthenelse{\equal{#1}{for}}{\hyperref[#1:#2]{(\ref*{#1:#2})}}{%
  \autoref{#1:#2}%
  }}}}}}}}}}}}}}}}
  
\newcommand{\etal}[0]{\textit{et al.}\xspace} 



%%%%%%%%%%%%%%%%%%%%%%listing options
\lstdefinelanguage{Java} %thanks to whoever did this
{morekeywords={abstract, all, and, as, assert, but, check, disj, else, exactly, extends, fact, for, fun, iden, if, iff, implies, in, Int, int, let, lone, module, no, none, not, one, open, or, part, pred, run, seq, set, sig, some, sum, then, univ, package, class, public, private, null, return, new, interface, extern, object, implements, System, static},
sensitive=true,morecomment=[l][\small\itshape]{--},morecomment=[l][\small\itshape]{//},morecomment=[s][\small\itshape]{/*}{*/},basicstyle=\small,
%basicstyle=\ttfamily,
numbers=left,numberstyle=\scriptsize,tabsize=2,numbersep=3pt,breaklines=true,lineskip=-2pt,stepnumber=1,captionpos=b,breaklines=true,breakatwhitespace=false,showspaces=false,showtabs=false,
%frame=single,
columns=fullflexible,escapeinside={(*@}{@*)},
literate={->}{{$\to$}}1 {^}{{$\mspace{-3mu}\widehat{\quad}\mspace{-3mu}$}}1
{<}{$<$ }2 {>}{$>$ }2 {>=}{$\geq$ }2 {=<}{$\leq$ }2
{<:}{{$<\mspace{-3mu}:$}}2 {:>}{{$:\mspace{-3mu}>$}}2
{=>}{{$\Rightarrow$ }}2 {+}{$+$ }2 {++}{{$+\mspace{-8mu}+$ }}2
{<=>}{{$\Leftrightarrow$ }}2 {+}{$+$ }2 {++}{{$+\mspace{-8mu}+$ }}2
{\~}{{$\mspace{-3mu}\widetilde{\quad}\mspace{-3mu}$}}1
{!=}{$\neq$ }2 {*}{${}^{\ast}$}1 %{.}{$\cdot$}1
{\#}{$\#$}1
}
\lstset{language=Java,numbersep=5pt,frame=single}


\lstset{language={Java}, basicstyle=\footnotesize\ttfamily, columns=flexible}
\begin{document}
\maketitle
\abstract{
The development of software is a daunting task, even more so when high security becomes an issue. 
Meanwhile, in todays technology driven world, computer security is not so much a luxury as it is a necessity. 
Nearly all development happens in a high-level programming language, mainly because it allows for easier reasoning about the program being written.
This simplification of the thought process is a direct consequence of the abstract computing model that high-level languages usually offer to the programmer.

The computing model offered by real architectures differs on many accounts from the one high-level languages offer.
It is during the compilation process that many bugs are introduced to seemingly correct software, because the abstract properties of the high-level language are lost.
These bugs can even pop up in software whose workings were formally verified using tools such as verifast~\cite{Verifast:paper}.
The goal of secure compilation is strengthening the compilation process such that any security guarantees derived from the abstract computing model are preserved when compiling.

This work aims to bring secure compilation to a language with an ML style module system, which uses signature matching and functors to provide modularization, encapsulation and information hiding.
}

\section{Introduction}
The amount of software in todays world is growing at rapid rates.
As software programs take on a larger and larger role in our lives, and security sensitive applications run side by side with software of the garden-variety, it is important that these applications behave as expected.

Most computer software is written using a high-level language, for example Java or ML.
This high-level language offers a computing model that abstracts away many subtleties of real architectures.
These subtleties of the computing model however are reintroduced when the software is compiled from those high-level language to a low-level language.
Some of the subtleties that are abstracted away are:
\begin{itemize}
\item The existence of registers and memory.
\item The fact that the software code and the values or objects it create must all be stored in this same memory space.
\item How the flow of control, i.e. the next command to be executed, is managed.
\item Objects created by software must provide an implementation, they no longer are \emph{algebraic data types} described only by their functionality.
\end{itemize}

Many of the abstractions allow the programmer to make certain assumptions about the security of their software.
The confidentiality of certain values and their integrity, for example.
Another abstraction is the atomicity of a function.
A software programmer never assumes that a function could be executed only partially, taking no real note of the possible harm that could be done if an attacker would be able to bypass the part of a function where permissions are checked.

The reintroduction of these subtleties in the low level can cause these security assumptions to become void.
Many attacks exist that abuse the way compilation reintroduces these subtleties, breaking the security guarantees assumed by the programmer or any formal verification software.
For example, Y. Erlinsson et al~\cite{OVSPaper} list a number of ways that low-level attacks might manipulate control flow or values whose integrity was guaranteed in on source level.

Because the severe problems that these bugs can introduce, breaking into security sensitive parts of an application exploiting its bindings with less secure parts of the application, a strengthening of the compilation process is needed.
The goal of \emph{secure compilation} is to provide compilation that preserves all high-level security guarantees in the low-level.

\section{Problem Setting}
Informally, the goal of secure compilation was formulated as compilation that preserves all high-level security guarantees in the low-level.
Contextual equivalence allows us to formalize what exactly are the security guarantees offered by the high-level language.

Contextual equivalence~\cite{Agten:2012:SCM:2354412.2355247} does this by introducing a homomorphic equivalence relation $\simeq$ on programs or their components.
Two objects or components $O1$ and $O2$ are contextually equivalent if no third object $O_C$, called the \emph{context}, is able to distinguish between the two components when it is run together with one of the objects as a programming.
The two objects are thus perfectly substitutable, as their outward behaviour is the same: they function as one and the same blackbox.

\[
 \forall O_C : O_C[O_1] \rightarrow^* c \iff O_C[O_2] \rightarrow^* c
\]

Contextual equivalence captures security guarantees such as the public/private access modifier or the atomicity of function executions. For example, if two components $O_1$ and $O_2$, containing functions \myref{lst}{lst:SimpleReturn} and \myref{lst}{lst:UnreachableCode} respectively, are contextually equivalent then the atomicity of function execution of \lstinline{f} is guaranteed. Being able to break function atomicity would result in the normally unreachable \lstinline{return} in \myref{lst}{lst:UnreachableCode} becoming reachable, which means substituting $O_1$ by $O_2$ would be observable from some context object $O_C$.

\begin{minipage}{0.40\textwidth}
\begin{lstlisting}[label={lst:SimpleReturn}, caption={Simple return}]
public void f(){
    return 0;
    
}
\end{lstlisting}
\end{minipage}
\begin{minipage}{0.40\textwidth}
\begin{lstlisting}[label={lst:UnreachableCode}, caption={Unreachable code}]
public void f(){
    return 0; 
    return 1; //Unreachable
}
\end{lstlisting}
\end{minipage}

Now that contextual equivalences ability to express the security guarantees of a language is established, secure compilation can be formalized by the notion of \emph{full abstraction}~\cite{Abadi}, or the preservation of contextual equivalence during the compilation process.
If $O_1$ is a high-level component and $\com{O_1}$ its low-level result from compilation, full abstraction becomes:
\[
 \forall O_1, O_2 : O_1 \simeq O_2 \iff \com{O_1} \simeq \com{O_2}
\]

This definition formulates that for any two programs that are contextually equivalent for any high-level context, their compiled representation should be contextually equivalent for any low-level context as well. 
This means compilation must preserve contextual equivalence, i.e. all security guarantees provided in the high-level language.

It also says that contextual equivalence should be reflected.
This property is called soundness, and is closely linked to what is expected of a `correct' compiler.
Indeed, if the results of compilation are contextually equivalent but the high-level objects are not, there exists a high-level context $O_C$ that can distinguish between $O_1$ and $O_2$, but without a low-level translation.

\section{Contributions}
This works introduces a secure compilation scheme for a language with an ML style module language.
The ML language is a functional language that provides modularization of programs through its \emph{module} language~\cite{Milner:1997:DSM:549659}.

One of the primary concepts that the module language offers is that of a \emph{structure}. A structure is simply a set of type and value definitions.
Structures allow a programmer to divide a large program in sets of smaller units containing closely related type and value definitions.
These units dependencies and connections are well-defined and explicit.

Another concept of the module language is a signature.
A signature declares the interface of a structure.


\section{Related Work}

\section{Conclusion}

\bibliographystyle{ieeetr}
\bibliography{bibliography}
\end{document}
