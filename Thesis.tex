\documentclass[10pt,a4paper]{report}
\usepackage[utf8]{inputenc}
\usepackage{amsmath}
\usepackage{amsfonts}
\usepackage{amssymb}
\usepackage{todonotes}
\usepackage{listings}
\begin{document}

%\begin{flushleft}

\chapter{A Compilation Example}

The secure compilation scheme is shown from an example program written in the source language, MiniML, which is a subset of the ML language. Its syntax and semantics are reminiscent of those used in Standard ML.

\section{A Cipher In ML}
Our example program will consist of the definition of a signature that represents symmetric cyphers, a concept used in cryptography. We chose this example since the modules related to cryptography are usually under more scrutiny with regards to the privacy of their internal values. 

\begin{figure}[!htbp]
\begin{lstlisting}[frame=single]
signature SYMMETRICCIPHER = sig 
    type cred
    val newcredentials : cred
    val encrypt: int -> cred -> int
    val decrypt: int -> cred -> int
end

structure Caesar :> SYMMETRICCIPHER = struct
    type cred = int
    fun newcredentials = rand
    fun encrypt(a,cred) = (a + cred)%26
    fun decrypt(a, cred) = (a - cred)%26
    val seed = 3
    fun rand = time.now * seed
end
\end{lstlisting}
\caption{ML example}
\label{code:Example}
\end{figure}

In Figure \ref{code:Example}, we define a signature \texttt{SYMMETRICCIPHER} that describes a module that implements a symmetric cipher. In order to implement a symmetric cipher, one must have a credential, i.e. the key, and two functions, \texttt{encrypt} and \texttt{decrypt}, which take data and credentials. The \texttt{encrypt} function takes the raw data and encodes it in a way only those with knowledge of the correct credentials can later use the \texttt{decrypt} function to transform the encoded data back into the raw data.

The second part of Figure \ref{code:Example} is the definition of a structure called \texttt{Caesar}. This structure provides the \texttt{newcredential}, \texttt{encrypt} and \texttt{decrypt} functions. For internal use it also possesses the necessary characteristics of a pseudorandom number generator, namely a seed value and a rand function that provides a pseudorandom number. It is necessary to hide the seed value from users since this would allow people to predict the output of the pseudorandom number generator.

The \texttt{Caesar} structure is forced to conform to the signature \texttt{SYMMETRICCIPHER} by means of \emph{opaque ascription (\texttt{:>})}. This not only forces the module to implement all the necessary elements of the signature, but it also restricts the means of interaction with the module to those elements that are explicitly mentioned in the interface. It is this notion of \emph{opaque ascription} that dictates what it means for this module to be secure. Concretely, to be secure, this module hides its \texttt{rand} function and its seed value from any outside code, only 

This example can later be broadened so that the \texttt{Caesar} structure becomes a functor that is parameterized to use an external module as its PRNG.

\section{Expected LLVM Intemediate Representation}
This section introduces the expected LLVM intermediate representation code for the example in Figure \ref{code:Example}. This means the high level abstractions made in MiniML, for example \emph{signatures} and \emph{structures} must be mapped to lower level constructs that are available in the LLVM intermediate representation. In order to focus on the security of the compilation, the more aggressive optimisation capabilities of LLVM will not be used.


\chapter{Formal Specification}
In this chapter the source language, MiniML, will be formally specified. First the syntax within which a program can be defined will be introduced. Next, the typing rules that a correct program must adhere to is shown. Lastly, the operational semantics govern what a correct program must do once it runs.

\section{Syntax}
\newcommand{\longspace}{\;\;\;\;\;\;}
\newcommand{\inlinecode}{\texttt}
First we introduce the syntax of a MiniML program, as seen in figure \ref{fig:Syntax}. A program consists of a set of module expressions, denoted as $\overline{\mathit{Mod}}$ using the bar notation for lists \footnote{The bar notation uses $\emptyset$ as the empty set and , as the prepend operator. For example:$\overline{\mathit{Mod}}$ can be the empty set, $\emptyset$, or $\lbrace \Delta, \overline{\mathit{d}}\rbrace^{name}, \overline{Mod}$}. It is then concluded by a single naked value expression, functioning as the main entry point of the program.
This description of a program allows us to first specify a set of signatures as well as a set of module and functors conforming to those signatures.
\\[2ex]
A signature $\Sigma$ is a module type and is represented by a list of \emph{declarations}. A declaration $\Delta$ specifies the type of an value identifier or the signature of a module identifier.
\\[2ex] 
A module can be seen as a special case of functors. A module specifies a signature and module body and is uniquely identified with an identifier $M_{i}$. The module body is represented as a list of definitions $\overline{d}$. A module $M_{i}$ with body $\overline{d}$ conforms to a signature $S_{i}$ if every identifier in $S_{i}$ has a definition in the module body, and its typing does not violate the one specified in the signature.
\\[2ex]
Functors $F_{i}$, presented as a generalization of modules, specify \todo{Is it always possible to specify the interface of a module? In our simple system, yes, There can be no problem with opaque types since we don't support them.} their own signature, $S_{i}$, as well as a set of signatures upon which it depends, $\overline{S_{n}}$. It then specifies a functor body in which those modules can be used. The functor can be given a set of modules that conform to the dependent signatures. We say the functor is applied to a set of modules. The result of this application behaves as a module that conforms to the interface $S_{i}$ that the functor specified for itself.

\begin{figure}[!htbp]
\begin{align*}
\begin{aligned}
\text{Program} ::= \; & \overline{\mathit{Mod}};\;e\\
\\
\text{Value Expression }e \; ::= \; &\mathit{num \; n \;} | \; \mathit{false} \; | \; \mathit{true} \\
&|\;\mathit{id}  \\
&|\;\mathit{path.id} \\
&|\;e_{1}e_{2} \\
&|\;(e_{1},e_{2}) \\
&|\;\lambda(p:\tau)\;.\;e \\
&|\;\mathit{let }\; p \; = \; e_{1} \; in \; e_{2} \\
&|\;\mathit{letrec} \; p \; = \; e_{1} \; in \; e_{2} \\
&|\; \mathit{if(e_{1}) \; then \; e_{2} \; else \; e_{3}}\\
&|\;\mathit{p.left}\; | \; \mathit{p.right} \\
&|\;\mathit{fix\;e} \\
\\
\text{Identifiers } ::= \; & \; id \\ 
&|\;M_{i}\\
&|\;F_{i}\\
&|\;S_{i}\\
\\
\text{Access Path } \mathit{path} \; ::= \; &\mathit{M_{i}}\\
&|\;F_{i}(\overline{\mathit{M_{i}}})\\
&|\;\mathit{this}\\
\\
\text{Mod Expr } \mathit{me} ::= \; & \mathit{sig} \; S_{i} = \Sigma\\
&|\; \mathit{mod} \;  M_{i} : S_{i} = \overline{\mathit{d}} \\
&|\; \mathit{funct} \; F_{i} (\overline{M_{n}:S_{n}}):S_{i} = \overline{d}\\
\\
\end{aligned}
\begin{aligned}
\text{Mod Types } \mathit{Mod} ::=\;&\lbrace \Sigma, \overline{\mathit{d}} \rbrace^{M_{i}} \\
&|\; \lbrace \Sigma, \overline{\Sigma}, \overline{d} \rbrace^{F_{i}} \\
\\
\text{Signature } \Sigma \; ::=\; & \overline{\Delta}\\
\\
\text{Declaration } \Delta \; ::=\; & (\mathit{id}:\tau)\\
& | \; (\mathit{M_{i}}:S_{i})\\
\\
\text{Definition } d \; ::= \; &(\mathit{id}=e:\tau)\\
& | \; (\mathit{M_{i}} = \overline{d} : S_{i}) \\
\\
\text{Type }\tau \; ::= \; &nat \\
&| \; \mathit{bool} \\
&| \; \tau_{1} \rightarrow \tau_{2} \\
&| \; \tau_{1} \times \tau_{2} \\
&| \; \alpha\\
\\
\text{Pattern }p \; ::= \; & \mathit{id} \\
&| \; (p,p)\\
\\
\\
\end{aligned}
\end{align*}
\caption{The syntax of MiniML}
\label{fig:Syntax}
\end{figure}

%\subsubsection{Syntax example}
%We now give an example of a syntactically correct MiniML program.

%\begin{figure}[!htbp]
%\begin{verbatim}
%test
%\end{verbatim}
%\caption{Syntax example}
%\label{code:SyntaxExample}
%\end{figure}

\subsection{Type system}
Having defined the syntax for MiniML, we are now able to formalize its type system.

\subsubsection{Type-schemes and contexts}
First, we introduce the concept of a type-scheme, as seen in figure \ref{fig:Type-schemesAndContexts}. A type-scheme, sometimes called polytype, introduces polymorphism by making use of the type variable $\alpha$ in the definition of $\tau$, and quantifying it with the universal quantifier $\forall$. This allows any concrete types $\tau$ to 'match' to the type variable. For example, the identity function \inlinecode{id} is typed $id:\forall \alpha. \alpha \rightarrow \alpha$, thereby introducing parametrized polymorphism which enables one to use the same \inlinecode{id} function everywhere regardless of the arguments type.

This concept of a type-scheme will later be used to provide \todo{explain further}let-polymorphism.  Note that the definition of a Type-Scheme assures that the resulting type-scheme is in \emph{prenex normal form}, i.e. a string of quantifiers concluded by a quantifier-free ending.
\\[2ex]
Our type system will also need to keep track of the type assumptions and the module, functor and signature definitions. This represents the notion of a \emph{context}. It is in this context that typing will happen. While type checking, the context is what the type checker uses to keep track of the facts it already knows.
\\[2ex]
To access mappings from these contexts, we will introduce projections. For example $\Gamma[M_{i}].\overline{d}$ will look up the mapping $(M_{i} \mapsto \lbrace S,\overline{d}\rbrace)$ in $\Gamma$ and project this to the $\overline{d}$ specified in the mapping. A lookup will \emph{fail} if the identifier has no mapping in the context.

\begin{figure}[!htbp]
\begin{align*}
\begin{aligned}
\text{Context }\Gamma ::=\; &\emptyset \\
&| \; (x:\sigma),\Gamma \\
&| \; (M_{i} \mapsto \lbrace \Sigma,\overline{d}\rbrace), \Gamma \\
&| \; (F_{i} \mapsto \lbrace \Sigma, \overline{\Sigma_{n}}, \overline{d} \rbrace), \Gamma \\
&| \; (S_{i} \mapsto \Sigma), \Gamma
\end{aligned}
\begin{aligned}
\longspace
\end{aligned}
\begin{aligned}
\text{Type-Scheme } \sigma \; ::= \; &\tau \\
&| \; \forall \alpha . \sigma \\
\\
\\
\end{aligned}
\end{align*}
\caption{Type-schemes and contexts in the MiniML type system.}
\label{fig:Type-schemesAndContexts}
\end{figure}

We are now in a position to define a few helpful relations between type-schemes, types and contexts: type-scheme specialization and type-scheme generalization.

\subsubsection{Type-scheme specialization}
The specialization relation $\sigma_{1} \geq \sigma_{2}$ expresses that $\sigma_{2}$ is more specialised than $\sigma_{1}$. This means that the following rule holds:
% $\sigma_{2}$ can be expressed as $\forall \beta_{i}...\beta_{m}.\sigma_{2}'$ and $\sigma_{1}$ as $\forall \alpha_{1}...\forall \alpha_{n}.\sigma_{1}'$, $\sigma_{2}$ is more specialised than $\sigma_{1}$ iff $\sigma_{2}'=[\alpha_{i} \mapsto \sigma_{i}]\sigma_{1}'$ and $\beta_{i} \in free(\sigma_{1})$. In other words, 
%

\[
\tag{specialization}
\frac{\tau_{2}=[\alpha_{i} \mapsto \tau_{i}]\tau_{1} \longspace \beta_{i} \not\in\mathit{free(\alpha_{1}...\forall \alpha_{n}.\tau_{1})}}
{\forall\alpha_{1}...\forall\alpha_{n}.\tau_{1}\geq \forall \beta_{i}...\forall \beta_{m}\tau_{2}'}
\]

In other words, the quantifier-free ending of the more specialized type-scheme can be obtained by consistently replacing all quantified type variables $\alpha_{i}$ in the more general type-scheme by a type $\tau_{i}$, which can possibly contain type variables itself, resulting in the quantifier-free ending of the more specialized type scheme. Furthermore, only variables that were not free in the more general type-scheme can be bound in the specialized type-scheme.

The first condition gives one the possibility to specify the type of a type variable. This second condition forbids one to \emph{rescope} a type variable in the process.

\subsubsection{Type-scheme generalization}
Type-scheme generalisation is the opposite process of type-scheme specialization. However, whereas specialization can be expressed independent of the context, whether or not one is allowed to generalize, is dependent on the context. Generalisation allows one to quantify an unquantified variable, as long as it does not appear unquantified in any type expression in the current context.

\[
\tag{generalization}
\frac{\Gamma \vdash e:\Sigma \longspace \alpha \not\in \mathit{free(\Gamma)}}{\Gamma \vdash e : \forall \alpha . \sigma}
\]


\subsubsection{Typing judgements}
To type check our program, the type checker will perform typing judgements. These typing judgements, which can bee seen in figure \ref{fig:TypingJudgements}, are relations between the context and parts of the syntax. They convey the meaning that an expression or other part of the syntax is well-typed in the context $\Gamma$. The typing of a module body and its definitions generates a new typing context $\Gamma'$ for the module. In this resulting context, the declarations must be well-typed.

The $\Gamma \vdash \Diamond$ judgement is a statement of well-formedness of a context $\Gamma$. A context is well-formed if the keyset of the lookup table it represents conforms to the standard notion of a set, meaning every key is used only once.

\begin{figure}[!htbp]
\begin{align*}
\text{ExpressionTyping } ::=\;&\Gamma \vdash e: \sigma \\
\text{ModuleTyping } ::= \; &\Gamma \vdash \mathit{Mod} \\
\text{DefinitionTyping } ::= \; &\Gamma \vdash d \rightarrow \Gamma' \\
\text{DeclarationTyping } ::= \;&\Gamma \vdash \Delta \\
\text{Well-formedness } ::=\;&\Gamma \vdash \Diamond
\end{align*}
\caption{Typing judgements in the MiniML type system.}
\label{fig:TypingJudgements}
\end{figure}

\subsubsection{Rules}
%\begin{figure}
\begin{align*}
&\Gamma \vdash true : bool \tag{T-True} \\
&\Gamma \vdash false : bool \tag{T-False} \\
&\Gamma \vdash num \; n : nat \tag{T-Num} \\ \\
\tag{T-Mono}
&\frac{\sigma \geq \tau \longspace id:\sigma \in \Gamma}{\Gamma \vdash id:\tau}\\ \\
\tag{T-App}
&\frac{\Gamma \vdash e_{1}:\tau_{2} \rightarrow \tau_{1} \longspace \Gamma \vdash e_{2}:\tau_{2}}
{\Gamma \vdash e_{1}e_{2}:\tau_{1}} \\ \\
\tag{BuildContext1}
& id:\sigma \rightarrow \emptyset, (id:\sigma) \\ \\
\tag{BuildContext2}
&\frac{p_{1}:\sigma_{1} \rightarrow \Gamma_{1} \longspace p_{2}:\sigma_{2}\rightarrow \Gamma_{2}}
{(p_{1},p_{2}):\sigma_{1}\times \sigma_{2} \rightarrow \Gamma_{1}\cup \Gamma_{2}} \\ \\
\tag{T-Fun}
&\frac{p:\tau_{2} \rightarrow \Gamma_{2} \longspace \Gamma_{2} \cup \Gamma_{1} \vdash e:\tau_{1}}
{\Gamma_{1} \vdash \lambda(p:\tau).e:\tau_{2} \rightarrow \tau_{1}} \\ \\
\tag{T-IfThenElse}
&\frac{\Gamma \vdash e_{1}:bool \longspace \Gamma \vdash e_{2}:\tau \longspace \Gamma \vdash e_{3} : \tau}
{\Gamma \vdash if \; e_{1} \; then \; e_{2} \; else \; e_{3} : \tau} \\ \\
\tag{T-Pair}
&\frac{\Gamma \vdash e_{1}:\tau_{1} \longspace \Gamma \vdash e_{2}:\tau_{2}}
{\Gamma \vdash (e_{1},e_{2}) : \tau_{1}\times\tau_{2}} \\ \\
\tag{T-PairLeft}
&\frac{\Gamma \vdash p:\tau_{1}\times\tau_{2}}
{\Gamma \vdash \mathit{p.left} : \tau_{1}} \\
\\
\tag{T-PairRight}
&\frac{\Gamma \vdash p:\tau_{1}\times\tau_{2}}
{\Gamma \vdash \mathit{p.right} : \tau_{2}} \\
\\
\tag{T-Let}
&\frac{\Gamma \vdash e_{2}:\tau_{2} \;\;\; \sigma=gen(\Gamma,\tau)\;\;\;p:\sigma\rightarrow \Gamma_{2} \;\;\; \Gamma \cup \Gamma_{2} \vdash e_{1}:\tau}
{\Gamma \vdash let\;p\;=\;e_{2}\;in\;e_{1}:\tau} \\ \\
\tag{T-Letrec}
&\frac{\Gamma \vdash let\;p\;=\;\mathit{fix}\;(\lambda p.e_{2})\;in\;e_{1}:\tau}
{\Gamma \vdash letrec\;p\;=\;e_{2}\;in\;e_{1}:\tau} \\ \\
\tag{T-Fix}
&\frac{\Gamma \vdash e : \tau \rightarrow \tau}
{\Gamma \vdash \mathit{fix\;e} : \tau} \\
\displaybreak
\\
\tag{T-ModVarThis}
&\frac{\sigma \geq \tau \longspace this.id:\sigma \in \Gamma}
{\Gamma \vdash this.id : \tau} \\ 
\\
\tag{T-ModVarOther}
&\frac{
\Gamma \vdash M_{i}
\longspace id:\tau \in \Gamma[M_{i}].\Sigma}
{\Gamma \vdash \mathit{M_{i}.id} : \tau} \\
\\
\tag{T-FunctorVar}
&\frac{\overline{\Gamma \vdash M_{1..n}}
\longspace
\overline{\Sigma_{3} \succeq \Gamma[M_{1..n}].\Sigma}
\longspace
\Gamma \vdash F_{i}(\overline{M_{1..n}})
\longspace
id:\tau \in \Gamma[F_{i}].\Sigma_{1}}
{\Gamma \vdash \mathit{F_{i}(\overline{M_{1..n}}).id}:\tau} \\
\\
\tag{T-Module}
&\frac{
\emptyset \vdash \Gamma[M_{i}].\overline{d}\rightarrow \Gamma' \longspace \Gamma' \vdash \Gamma[M_{i}].\Sigma_{i}}
{\Gamma \vdash M_{i}} \\
\\
\tag{T-Functor}
&\frac{
\emptyset \vdash [\overline{M_{1..n} \mapsto M_{arg}}]\overline{d} \rightarrow \Gamma' \longspace \Gamma' \vdash \Gamma[F_{i}].\Sigma}
{\Gamma \vdash F_{i}(\overline{M_{args}})}\\
\\
\tag{T-ModInterfaceField}
&\frac{(x:\tau) \in \Gamma \longspace \Gamma \vdash \Delta}
{\Gamma \vdash (x:\tau),\Delta}\\
\\
\tag{T-ModInterfaceModule}
&\frac{(M=\lbrace \Sigma_{2}, \overline{d} \rbrace^{M}) \in \Gamma
\longspace \Sigma_{1} \succeq \Sigma_{2} 
\longspace \Gamma \vdash \Delta}
{\Gamma \vdash (M:\Sigma_{1}),\Delta}\\
\\
\tag{T-ModBodyV}
&\frac{ (x:\tau),\Gamma \vdash \overline{d} \rightarrow \Gamma' \longspace \Gamma \vdash e:\tau}
{\Gamma \vdash (x=e:\tau),\overline{d} \rightarrow (x:\tau),\Gamma'} \\
\\
\tag{T-ModBodyM}
&\frac{\Gamma \vdash \lbrace\Sigma, \overline{d} \rbrace^{M_{i}} }
{\Gamma \vdash (\mathit{M_{i}=\overline{d}}:\Sigma),\overline{d} \rightarrow (M_{i}=\lbrace \Sigma,\overline{d} \rbrace),\Gamma'} \\
\\
\tag{T-EmptySet}
&\frac{\Gamma \vdash \Diamond}
{\Gamma \vdash \emptyset}
\end{align*}
%\end{figure}
\todo{Must specify $\succeq$ to mean the specialization of an interface}

\subsection{Operational semantics}
\begin{align*}
\text{Value }v ::=\;&\mathit{num\;n} \; | \; \mathit{true} \; | \; \mathit{false} \\
&| (v,v) \\
&| \lambda p.e\\
\\
\text{Module Table } T\; ::= \;&\emptyset \\
&| \; (M_{i} \mapsto \lbrace \Sigma,\overline{d}\rbrace), T \\
&| \; (F_{i} \mapsto \lbrace \Sigma, \overline{\Sigma_{n}}, \overline{d} \rbrace), T \\
&| \; (S_{i} \mapsto \Sigma), T\\
\\
\text{Evaluation } ::= T \vdash &e \rightarrow T \vdash e' \\
\end{align*}

The operational semantics defines a module table T, containing mappings from the signature-, module and functor identifiers to their definition and and the evaluation relation. 

The module table T allows looking up the definition behind a certain identifier and accessing a certain part of it using projection. $T[M_{i}].\Sigma$ will give access to the $\Sigma$ in the definition of $M_{i}$. 

The evaluation relation allows the evaluation of an expression $e$ to a (simpler) expression $e'$, while potentially making a lookup in T.

\subsubsection{Rules}
\begin{align*}
\tag{E-IfTrue}
&T \vdash if \; true  \; then \; e_{1} \; else \; e_{2} \rightarrow T \vdash e_{1}\\
\tag{E-IfFalse}
&T \vdash if \; false \; then \; e_{1} \; else \; e_{2} \rightarrow T \vdash e_{2}\\ \\
\tag{E-IfThenElse}
&\frac{T \vdash e_{1} \rightarrow T \vdash e_{1}'}
{T \vdash if \; e_{1} \; then \; e_{2} \; else \; e_{3} \rightarrow T \vdash if \; e_{1}' \; then \; e_{2} \; else \; e_{3}}\\ \\
\tag{E-PairLeft}
&\frac{T \vdash e_{1} \rightarrow T \vdash e_{1}'}
{T \vdash (e_{1},e_{2}) \rightarrow T \vdash (e_{1}',e_{2})} \\ \\
\tag{E-PairRight}
&\frac{T \vdash e_{2} \rightarrow T \vdash e_{2}'}
{T \vdash (e_{1},e_{2}) \rightarrow T \vdash (e_{1},e_{2}')} \\ \\
\tag{E-Let}
&\frac{T \vdash e_{1}\rightarrow T \vdash e_{1}'}
{T \vdash let \; p \; = \; e_{1} \; in \; e_{2} \rightarrow T \vdash let \; p \; = \; e_{1}' \; in \; e_{2}}
\\ \\
\tag{E-LetV}
&T \vdash let \; id \; = \; v \; in \; e \rightarrow T \vdash [id \mapsto v]e \\ \\
\tag{E-LetRec}
& T \vdash letrec\;p=\;e_{1} \; in \; e_{2} \rightarrow T \vdash let \; p \; = fix(\lambda p.e_{1}') \; in \; e_{2} \\ \\ 
\tag{E-Fix}
&\frac{T \vdash e\rightarrow T \vdash e'}
{T \vdash fix(e) \rightarrow T \vdash fix(e')}\\ \\
\tag{E-FixRec}
&T \vdash fix(\lambda(p.e)) \rightarrow T \vdash [p \mapsto (fix (\lambda(p.e))]e \\
\\
\tag{E-PatternMatch}
&T \vdash let \; (p_{1},p_{2}) \; = \; (e_{1},e_{2}) \; in \; e_{3} \rightarrow
let \; p_{1} \; = \; e_{1} \; in \;
(let \; p_{2}  \; = \; e_{2} \; in \; e_{3}) \\ \\
\tag{E-App1}
&\frac{T \vdash e_{1} \rightarrow T \vdash e_{1}'}
{T \vdash e_{1} e_{2} \rightarrow T \vdash e_{1}' e_{2}}\\ \\
\tag{E-App2}
&\frac{T \vdash e_{2} \rightarrow T \vdash e_{2}'}
{T \vdash v\;e_{2} \rightarrow T \vdash v\;e_{2}'}\\ \\
\tag{E-Lambda}
&T \vdash (\lambda x . e) \; v \rightarrow T \vdash [x \mapsto v]e \\ \\
\tag{E-MatchLambda}
&T \vdash (\lambda (p_{1},p_{2}) . e_{3}) \; (e_{1},e_{2}) \rightarrow T \vdash (\lambda p_{1}.(\lambda p_{2}.e_{3})\;e_{2})\; e_{1} \\
\displaybreak
\\
\tag{E-ModVar}
&\frac{\longspace (x=e':\tau) \in T[M_{i}].\overline{d} \longspace e=[this.y\mapsto M.y]e'\;\forall (this.y \in e')}
{T \vdash M.x \rightarrow T \vdash e}\\
\\
\tag{E-FunVar}
&\frac{\longspace (x=e':\tau) \in T[F_{i}].\overline{d} \longspace e=[M_{1..n} \mapsto M_{args}][this.y\mapsto F_{i}(\overline{M_{args}}).y]e'\;\forall (this.y \in e')}
{T \vdash F_{1}(\overline{M_{args}}).x \rightarrow T \vdash e}
\end{align*}
%\todo{How to express the substitution of all references to argument placeholder module names to the modules names given at execution}
%\todo{provide a desugar function}
%\end{flushleft}
\end{document}