\documentclass[11pt]{article}
\title{Secure Compilation and The ML Language}
\author{Matthias van der Hallen}
\date{}
\usepackage{listings}
\usepackage{lmodern}
\usepackage{amsmath}
\usepackage{amsfonts}
\usepackage{amssymb}
\usepackage{todonotes}
\usepackage{hyperref}
\usepackage{listings}
\usepackage{float}
\usepackage{lmodern}
\usepackage{graphicx}
\usepackage{caption}
\usepackage{subcaption}
\usepackage{xcolor,colortbl}
%\usepackage{longtable}
\usepackage{ltablex}
\usepackage{appendix}
\usepackage{amsthm}

\newcommand{\mypageref}[1]{Page~\pageref{#1}}
\newcommand{\earlier}[2]{{\protect\myref{#1}{#2}} on {\protect\mypageref{#2}}}
\newcommand{\com}[1]{#1^\downarrow}
\newcommand{\cmath}[1]{\ensuremath{\mathit{#1}}}
\newcommand{\gray}{\cellcolor{lightgray}}
\newcounter{line}


\DeclareMathOperator\compat{\ensuremath{\raisebox{1mm}{$\frown$}}}
\DeclareMathOperator\nsimeq{\ensuremath{\simeq\!\!\!\!\!/\ }}
\DeclareMathOperator\rbisim{\ensuremath{\mc{R}}} 
\DeclareMathOperator\impl{\ensuremath{\Rightarrow}} 
\DeclareMathOperator\teq{\ensuremath{\simeq_{T}}}
\DeclareMathOperator\nteq{\ensuremath{\nsimeq\!\!_{T}}}
\DeclareMathOperator\oeq{\ensuremath{\simeq_{O}}}
\DeclareMathOperator\wbis{\ensuremath{\approx}}

%%%%%%%%%%%%%%%%%%%%% fonts
\newcommand{\mt}[1]{\ensuremath{\texttt{#1}}}
\newcommand{\mtt}[1]{\ensuremath{\mathtt{#1}}}
\newcommand{\mf}[1]{\ensuremath{\mathbf{#1}}}
\newcommand{\mi}[1]{\ensuremath{\mathit{#1}}}
\newcommand{\mc}[1]{\ensuremath{\mathcal{#1}}}
\newcommand{\ms}[1]{\ensuremath{\mathsf{#1}}}
\newcommand{\mb}[1]{\ensuremath{\mathbb{#1}}}

%%%%%%%%%%%%%%%%%%%%% shorts
\newcommand{\acron}[0]{\ensuremath{\mbox{FPMAC}}\xspace}
\newcommand{\acrons}[0]{\ensuremath{\mbox{FPMACs}}\xspace}

\newcommand{\OB}[1]{\ensuremath{\overline{#1}}}
\newcommand{\subt}[0]{\ensuremath{<:}}
\newcommand{\xto}[1]{\ensuremath{\xrightarrow{~#1~}}}
\newcommand{\Xto}[1]{\ensuremath{\xRightarrow{~#1~}}}
\newcommand{\xtol}[1]{\ensuremath{\xrightarrow{~#1~}\low}}
\newcommand{\Xtol}[1]{\ensuremath{\xRightarrow{~#1~}\Low}}
\newcommand{\nXtol}[1]{\ensuremath{\xRightarrow{~#1~}\nLow}}
\newcommand{\tol}[0]{\ensuremath{\to}} %{\to\low}}
\newcommand{\low}[0]{\ensuremath{\!\!\!\!\to} }
\newcommand{\Low}[0]{\ensuremath{\!\!\!\!\Rightarrow} }
\newcommand{\nLow}[0]{\ensuremath{\!\!\!\!\!\!\!\!/\!\!\Rightarrow} }
\newcommand{\da}[0]{\ensuremath{^{\downarrow}}}
\newcommand{\ua}[1]{\ensuremath{\uparrow\!\!(#1)}}
\newcommand{\Ua}[0]{\ensuremath{\Uparrow}}
\newcommand{\myset}[2]{\ensuremath{\{#1 ~|~ #2\}}}
\newcommand{\sta}[1]{\ensuremath{\widehat{#1}}}

\newcommand{\toli}[0]{\ensuremath{\to^{\!\!\!\!\!i}\ }}
\newcommand{\tole}[0]{\ensuremath{\to^{\!\!\!\!\!e}\ }}


\newcommand{\jjr}[0]{Java~Jr.\xspace}

\newcommand{\divr}[0]{\ensuremath{\!\!\Uparrow\xspace}}

\newcommand{\totr}[1]{\ensuremath{\lfloor #1 \rfloor}}
\newcommand{\toex}[1]{\ensuremath{\lceil #1 \rceil}}

\newcommand{\fun}[2]{\ensuremath{\mtt{#1}(#2)}}
\newcommand{\dom}[1]{\ensuremath{\fun{dom}{#1}}}
\newcommand{\mex}[1]{\ensuremath{\fun{m_{ext}}{#1}}}
\newcommand{\mse}[1]{\ensuremath{\fun{m_{sec}}{#1}}}
\newcommand{\stu}[0]{\ensuremath{^{\bot}\xspace}}
\newcommand{\ldiv}[0]{\ensuremath{^{\Uparrow L}\xspace}}

\newcommand{\assline}[1]{\arabic{line}~~ \mtt{#1}\stepcounter{line}}

\newcommand{\sps}[0]{\ms{SP_{sec}}\xspace}
\newcommand{\spe}[0]{\ms{SP_{ext}}\xspace}

\newcommand{\TL}[1]{\ensuremath{\ms{Tr}(#1)}}
\newcommand{\trl}[0]{\ensuremath{\ms{Tr^L}}\xspace}
\newcommand{\trs}[0]{\ensuremath{\ms{Tr^S}}\xspace}

\newcommand{\blk}[0]{\ensuremath{(\ms{unknown},m,s)}\xspace}

\newcommand{\comm}[1]{}
%%%%%%%%%%%%%%%%%%%%%%%% typing & rules 
\newcommand{\typerule}[3]{\ensuremath{\begin{array}{c}\textsf{\scriptsize ({#1})} \\#2 \\\hline\raisebox{-3pt}{\ensuremath{#3}}\end{array}}}

\newcommand{\raisedrule}[2][0em]{\leaders\hbox{\rule[#1]{1pt}{#2}}\hfill}

\newcommand{\toprule}[1]{\begin{center}\vspace{1mm}\noindent\raisedrule{0.3mm}\ \raisebox{-0.5ex}{\emph{#1}} \raisedrule{0.3mm}\!\raisedrule{0.3mm}\!\raisedrule{0.3mm}}

\newcommand{\botrule}{\vspace*{2mm}\HRule\end{center}}

\newcommand{\HRule}[0]{\rule{\linewidth}{0.3mm}\vspace*{-1mm}}%\rule{\linewidth}{0.3mm}}

%%%%%%%%%%%%%%%%%%%%%%%% misc
\newcommand{\MP}[1]{\todo[color=blue!30]{TODO:#1}}
\newcommand{\myparagraph}[1]{\smallskip \noindent\noindent\textit{#1.}~~}


%%%%%%%%%%%%%%%%%%%%%%%% environments
\newtheorem{assumption}{Assumption}
\newtheorem{notation}{Notation}


\newenvironment{proofsketch}{\trivlist\item[]\emph{Proof Sketch}.\xspace}{\unskip\nobreak\hskip 1em plus 1fil\nobreak$\Box$\parfillskip=0pt\endtrivlist}


%%%%%%%%%%%%%%%%%%%%%%%% pictures
\pgfdeclarelayer{background}
\pgfdeclarelayer{back2}
\pgfdeclarelayer{foreground}
\pgfsetlayers{background,back2,main,foreground}

\newcommand{\tikzpic}[1]{
\begin{tikzpicture}[shorten >=1pt,auto,node distance=6mm,rounded corners]
\tikzstyle{state} =[fill=white,minimum size=4pt]
#1
\end{tikzpicture}
}
\newcommand{\myfig}[3]{\begin{figure} [!h]
#1
\caption{\label{fig:#2}#3}
\end{figure}}

\newcommand{\myfigb}[3]{\begin{figure} [!b]
#1
\caption{\label{fig:#2}#3}
\end{figure}}

%%%%%%%%%%%%%    Easier way to use references   
\newcommand{\customref}[2]{\hyperref[#2]{#1~\ref*{#2}}}

\newcommand{\myref}[2]{%
  \ifthenelse{\equal{#1}{lst}}{\customref{Listing}{#2}}{
  \ifthenelse{\equal{#1}{fig}}{\customref{Fig.}{#1:#2}}{%
  \ifthenelse{\equal{#1}{def}}{\customref{Definition}{#1:#2}}{%
  \ifthenelse{\equal{#1}{lem}}{\customref{Lemma}{#1:#2}}{%
  \ifthenelse{\equal{#1}{thm}}{\customref{Theorem}{#1:#2}}{%
  \ifthenelse{\equal{#1}{not}}{\customref{Notation}{#1:#2}}{%
  \ifthenelse{\equal{#1}{prop}}{\customref{Proposition}{#1:#2}}{%
  \ifthenelse{\equal{#1}{ax}}{\customref{Axiom}{#1:#2}}{%
  \ifthenelse{\equal{#1}{ex}}{\customref{Example}{#1:#2}}{%
  \ifthenelse{\equal{#1}{propr}}{\customref{Property}{#1:#2}}{%
  \ifthenelse{\equal{#1}{ass}}{\customref{Assumption}{#1:#2}}{%
  \ifthenelse{\equal{#1}{tab}}{\customref{Table}{#1:#2}}{%
  \ifthenelse{\equal{#1}{sec}}{\hyperref[#2]{Section~\ref*{#2}}}{%
  \ifthenelse{\equal{#1}{eq}}{\hyperref[#1:#2]{Equation~(\ref*{#1:#2})}}{%
    \ifthenelse{\equal{#1}{for}}{\hyperref[#1:#2]{Formula~(\ref*{#1:#2})}}{%
  \autoref{#1:#2}%
  }}}}}}}}}}}}}}}}
  
  
%use only when citing Example 1 and 2.  The Example 1 is cited with \myref, in order to obtain the right 2, use \myrefdrop
\newcommand{\myrefand}[2]{%
  \ifthenelse{\equal{#1}{fig}}{\customref{}{#1:#2}}{%
  \ifthenelse{\equal{#1}{def}}{\customref{}{#1:#2}}{%
  \ifthenelse{\equal{#1}{lem}}{\customref{}{#1:#2}}{%
  \ifthenelse{\equal{#1}{not}}{\customref{}{#1:#2}}{%
  \ifthenelse{\equal{#1}{thm}}{\customref{}{#1:#2}}{%
  \ifthenelse{\equal{#1}{prop}}{\customref{}{#1:#2}}{%
  \ifthenelse{\equal{#1}{ax}}{\customref{}{#1:#2}}{%
  \ifthenelse{\equal{#1}{ex}}{\customref{}{#1:#2}}{%
  \ifthenelse{\equal{#1}{propr}}{\customref{}{#1:#2}}{%
  \ifthenelse{\equal{#1}{ass}}{\customref{}{#1:#2}}{%
  \ifthenelse{\equal{#1}{tab}}{\customref{}{#1:#2}}{%
    \ifthenelse{\equal{#1}{lis}}{\customref{}{#1:#2}}{%
  \ifthenelse{\equal{#1}{sec}}{\hyperref[#1:#2]{\ref*{#1:#2}}}{%
  \ifthenelse{\equal{#1}{eq}}{\hyperref[#1:#2]{(\ref*{#1:#2})}}{%
    \ifthenelse{\equal{#1}{for}}{\hyperref[#1:#2]{(\ref*{#1:#2})}}{%
  \autoref{#1:#2}%
  }}}}}}}}}}}}}}}}
  
\newcommand{\etal}[0]{\textit{et al.}\xspace} 



%%%%%%%%%%%%%%%%%%%%%%listing options
\lstdefinelanguage{Java} %thanks to whoever did this
{morekeywords={abstract, all, and, as, assert, but, check, disj, else, exactly, extends, fact, for, fun, iden, if, iff, implies, in, Int, int, let, lone, module, no, none, not, one, open, or, part, pred, run, seq, set, sig, some, sum, then, univ, package, class, public, private, null, return, new, interface, extern, object, implements, System, static},
sensitive=true,morecomment=[l][\small\itshape]{--},morecomment=[l][\small\itshape]{//},morecomment=[s][\small\itshape]{/*}{*/},basicstyle=\small,
%basicstyle=\ttfamily,
numbers=left,numberstyle=\scriptsize,tabsize=2,numbersep=3pt,breaklines=true,lineskip=-2pt,stepnumber=1,captionpos=b,breaklines=true,breakatwhitespace=false,showspaces=false,showtabs=false,
%frame=single,
columns=fullflexible,escapeinside={(*@}{@*)},
literate={->}{{$\to$}}1 {^}{{$\mspace{-3mu}\widehat{\quad}\mspace{-3mu}$}}1
{<}{$<$ }2 {>}{$>$ }2 {>=}{$\geq$ }2 {=<}{$\leq$ }2
{<:}{{$<\mspace{-3mu}:$}}2 {:>}{{$:\mspace{-3mu}>$}}2
{=>}{{$\Rightarrow$ }}2 {+}{$+$ }2 {++}{{$+\mspace{-8mu}+$ }}2
{<=>}{{$\Leftrightarrow$ }}2 {+}{$+$ }2 {++}{{$+\mspace{-8mu}+$ }}2
{\~}{{$\mspace{-3mu}\widetilde{\quad}\mspace{-3mu}$}}1
{!=}{$\neq$ }2 {*}{${}^{\ast}$}1 %{.}{$\cdot$}1
{\#}{$\#$}1
}
\lstset{language=Java,numbersep=5pt,frame=single}


\lstset{language={Java}, basicstyle=\footnotesize\ttfamily, columns=flexible, breaklines=true}
\begin{document}
\maketitle
\abstract{
The development of software is a daunting task, even more so when high security becomes an issue. 
Meanwhile, in todays technology driven world, computer security is not so much a luxury as it is a necessity. 
Nearly all development happens in a high-level programming language, mainly because it allows for easier reasoning about the program being written.
This simplification of the thought process is a direct consequence of the abstract computing model that high-level languages usually offer to the programmer.

The computing model offered by real architectures differs on many accounts from the one high-level languages offer.
It is during the compilation process that many bugs are introduced to seemingly correct software, because the abstract properties of the high-level language are lost.
These bugs can even pop up in software whose workings were formally verified using tools such as verifast~\cite{Verifast:paper}.
The goal of secure compilation is strengthening the compilation process such that any security guarantees derived from the abstract computing model are preserved when compiling.

This work aims to bring secure compilation to a language with an ML style module system, which uses signature matching and functors to provide modularization, encapsulation and information hiding.
}

\section{Introduction}
The amount of software in todays world is growing at rapid rates.
As software programs take on a larger and larger role in our lives, and security sensitive applications run side by side with software of the garden-variety, it is important that these applications behave as expected.

Most computer software is written using a high-level language, for example Java or ML.
This high-level language offers a computing model that abstracts away many subtleties of real architectures.
These subtleties of the computing model however are reintroduced when the software is compiled from those high-level language to a low-level language.
Some of the subtleties that are abstracted away are:
\begin{itemize}
\item The existence of registers and memory.
\item The fact that the software code and the values or objects it create must all be stored in this same memory space.
\item How the flow of control, i.e. the next command to be executed, is managed.
\item Objects created by software must provide an implementation, they no longer are \emph{algebraic data types} described only by their functionality.
\end{itemize}

Many of the abstractions allow the programmer to make certain assumptions about the security of their software.
The confidentiality of certain values and their integrity, for example.
Another abstraction is the atomicity of a function.
A software programmer never assumes that a function could be executed only partially, taking no real note of the possible harm that could be done if an attacker would be able to bypass the part of a function where permissions are checked.

The reintroduction of these subtleties in the low level can cause these security assumptions to become void.
Many attacks exist that abuse the way compilation reintroduces these subtleties, breaking the security guarantees assumed by the programmer or any formal verification software.
For example, Y. Erlinsson et al~\cite{OVSPaper} list a number of ways that low-level attacks might manipulate control flow or values whose integrity was guaranteed in on source level.

Because the severe problems that these bugs can introduce, breaking into security sensitive parts of an application exploiting its bindings with less secure parts of the application, a strengthening of the compilation process is needed.
The goal of \emph{secure compilation} is to provide compilation that preserves all high-level security guarantees in the low-level.

\section{Problem Setting}
Informally, the goal of secure compilation was formulated as compilation that preserves all high-level security guarantees in the low-level.
Contextual equivalence allows us to formalize what exactly are the security guarantees offered by the high-level language.

Contextual equivalence~\cite{Agten:2012:SCM:2354412.2355247} does this by introducing a homomorphic equivalence relation $\simeq$ on programs or their components.
Two objects or components $O1$ and $O2$ are contextually equivalent if no third object $O_C$, called the \emph{context}, is able to distinguish between the two components when it is run together with one of the objects as a programming.
The two objects are thus perfectly substitutable, as their outward behaviour is the same: they function as one and the same blackbox.

\[
 \forall O_C : O_C[O_1] \rightarrow^* c \iff O_C[O_2] \rightarrow^* c
\]

Contextual equivalence captures security guarantees such as the public/private access modifier or the atomicity of function executions. For example, if two components $O_1$ and $O_2$, containing functions \myref{lst}{lst:SimpleReturn} and \myref{lst}{lst:UnreachableCode} respectively, are contextually equivalent then the atomicity of function execution of \lstinline{f} is guaranteed. Being able to break function atomicity would result in the normally unreachable \lstinline{return} in \myref{lst}{lst:UnreachableCode} becoming reachable, which means substituting $O_1$ by $O_2$ would be observable from some context object $O_C$.

\begin{minipage}{0.40\textwidth}
\begin{lstlisting}[label={lst:SimpleReturn}, caption={Simple return}]
public void f(){
    return 0;
    
}
\end{lstlisting}
\end{minipage}
\begin{minipage}{0.40\textwidth}
\begin{lstlisting}[label={lst:UnreachableCode}, caption={Unreachable code}]
public void f(){
    return 0; 
    return 1; //Unreachable
}
\end{lstlisting}
\end{minipage}

Now that contextual equivalences ability to express the security guarantees of a language is established, secure compilation can be formalized by the notion of \emph{full abstraction}~\cite{Abadi}, or the preservation of contextual equivalence during the compilation process.
If $O_1$ is a high-level component and $\com{O_1}$ its low-level result from compilation, full abstraction becomes:
\[
 \forall O_1, O_2 : O_1 \simeq O_2 \iff \com{O_1} \simeq \com{O_2}
\]

This definition formulates that for any two programs that are contextually equivalent for any high-level context, their compiled representation should be contextually equivalent for any low-level context as well. 
This means compilation must preserve contextual equivalence, i.e. all security guarantees provided in the high-level language.

It also says that contextual equivalence should be reflected.
This property is called soundness, and is closely linked to what is expected of a `correct' compiler.
Indeed, if the results of compilation are contextually equivalent but the high-level objects are not, there exists a high-level context $O_C$ that can distinguish between $O_1$ and $O_2$, but without a low-level translation.

\section{Contributions}
\lstset{frame=single, language=ML, numbers=left, columns=flexible, breaklines=true, escapeinside={(*@}{@*)}}
This works introduces a secure compilation scheme for a language with an ML style module language.
The ML language is a functional language that provides modularization of programs through its \emph{module} language~\cite{Milner:1997:DSM:549659}.

\subsection{structures}
One of the primary concepts that the module language offers is that of a \emph{structure}. A structure is simply a set of type and value definitions.
Structures allow a programmer to divide a large program in sets of smaller units containing closely related type and value definitions.
These units dependencies and connections are well-defined and explicit.
An example of a structure is shown in \myref{lst}{lst:DictionaryStructureExample}.

\begin{lstlisting}[frame=single, language=ML,caption={[Dictionary Definition Example]An example structure showing the definition of a dictionary in ML.}, label=lst:DictionaryStructureExample,numbers=left]
structure Dictionary =
    struct
        type dictionary = (string * string) list
        val emptyDictionary = []
        fun insert d, x, y = (x,y)::d
    end
\end{lstlisting}

\subsection{Signatures}
Another concept of the module language is a signature.
A signature is a set of type declarations and value declarations. It can provide type declarations without giving a specific implementation.
An example of such a signature is shown in \myref{lst}{lst:DictionarySignatureExample}.

\begin{lstlisting}[frame=single, language=ML, caption={[Dictionary Declaration Example]An example signature showing the declaration of a dictionary in ML.}, label=lst:DictionarySignatureExample, numbers=left]
signature DICTIONARYSIGNATURE =
    sig
        type dictionary
        val emptyDictionary : dictionary
        val insert: dictionary -> string -> string -> dictionary
    end
\end{lstlisting}

A signature can describe the interface of a structure.
Any \lstinline{struct} expression has a so called \emph{principal signature}.
When the \lstinline{struct} expression is bound to a name using the \lstinline{structure} keyword, it can be ascribed with a signature, either opaquely or transparent.
This causes the interface of the structure to be type checked, and then altered, based on the type of ascription.
This technique of structural typing is called \emph{signature matching}~\cite{Pierce:Adv}.

The interface of \emph{view} of a structure \cmath{Str} ascribed with a signature \cmath{Sig} can be computed as follows:
\begin{itemize}
\item Values defined in the \lstinline{struct} expression but not declared in the ascribed signature are never available for code outside the \lstinline{struct} expression.
They do \emph{not} become part of the interface of structure \cmath{Str}.
\item
Only types declared in the ascribed signature \cmath{Sig} are part of the interface of \cmath{Str}.
%The declaration of types inside the \lstinline{struct} expression are visible if they correspond to type definition in the ascribed signature. 
Whether or not their \emph{definition} is propagated depends on whether ascription was opaque or transparent.
\end{itemize}

\subsection{Functors}
The last concept of the module language is a functor.
Functors behave as a function mapping an argument structure \cmath{StrIn} to output structures \cmath{StrOut} and are used to create \emph{parametric} dependencies of one structure on another.
An example of such a functor is shown in \myref{lst}{lst:DictionaryFunctorExample}.

\subsubsection{Functor Definition}
\begin{lstlisting}[label={lst:DictionaryFunctorExample}, caption={The Dictionary as a functor. The structures and signatures that this example depends on are shown in {\protect\myref{lst}{lst:AdditionalSignatures}} in Appendix {\protect\ref{app:AdditionalCode}}.}]
functor DictionaryFn (KeyStruct:EQUAL) :> DICTIONARY where type key = KeyStruct.t =
    struct
        type key = KeyStruct.t
        type 'a dictionary = (key * 'a) list
        val emptyDictionary = []
        fun insert d x y = (x,y)::d
        fun lookup |[] x = error
                   |(key,value):ds x = if(KeyStruct.equal key x)
                                     then value
                                     else (lookup ds x)
    end
    
structure StringDict = DictionaryFn(StringEqual); (*@ \label{ln:Application} @*)
\end{lstlisting}

Functor definitions specify an identifier for an argument structure (\lstinline{KeyStruct}) and define an output structure that can use values defined by the argument structure. 

To limit the set of structures allowed as an argument, a functor also specifies a signature(\lstinline{EQUAL)} with which the argument structure should match.
This signature ensures the functor that the output structure it defines can trust the values it uses from the argument structure are really defined by the argument structure.

As functors themselves define an output structure as well, this output structure can be ascribed with a signature, in example \myref{lst}{lst:DictionaryFunctorExample} this is would be the signature \lstinline{DICTIONARY}.

\subsubsection{Functor Application}
On top of being defined, functors are also applied.
The application of a functor is shown in \myref{lst}{lst:DictionaryFunctorExample} on \myref{ln}{ln:Application}.
A functor application binds the structure that results as output of a functor to a name.
Of course the functor application must be supplied a concrete structure as an argument, whose interface matches the expected argument signature.

\subsubsection{Secure Compilation}

The contribution of this work is a secure compilation scheme for a language that provides an ML style module language.
In order to be able to preserve the security guarantees of ML when compiling to a low-level language, it is assumed that the low-level architecture offers certain access control semantics for the computer memory.
The same access control semantics as specified by Agten et al.~\cite{Agten:2012:SCM:2354412.2355247} are chosen (\myref{fig}{fig:PCBAC}).

\begin{figure}[htb]
    \centering
	\begin{tabular}{|c|c|c|c|c|}
		\hline
		From \textbackslash To & \multicolumn{3}{c|}{Protected} & Unprotected \\
		& Entry Point & Code & Data & \\ \hline
		Protected & r x & r x & r w & r w x \\ \hline
		Unprotected & x & & & r w x \\ \hline
	\end{tabular}
    \caption[PCBAC Semantics]{Program counter based access control semantics as specified in Agten et al.\cite{Agten:2012:SCM:2354412.2355247}. \label{fig:PCBAC}}
\end{figure}

They already specify a secure compilation scheme from a simple high-level language to this low-level architecture.
Patrignani et al.~\cite{Patrignani} expand on this work, adding several concepts from \emph{object oriented programming}.

When a collection of structures, signatures and functors is compiled with the secure compilation scheme, it is assumed that execution will happen by loading the securely compiled code to the protected code memory and linking it with an insecure context that resides in unprotected memory. Memory allocations in the protected code will allocate memory inside the protected data section.

In order to securely compile a language with an ML style module system, it is important to preserve the basic security precautions that were described by Agten et al.~\cite{Agten:2012:SCM:2354412.2355247}. These are summarized here shortly:

\begin{itemize}
\item Every function has a single entry point.
The access control semantics of the low-level architecture make sure that control flow can only switch from the insecure code to the secure code by passing through such an entry point.
This protects the `atomicity' of function execution.
\item A shadow stack should be used for all stack operations done in the secure code. When control flow switches from the insecure context to the secure module or back, the stack pointer and the shadow stack pointer should be switched.
\item When control flow passes from the secure code to the insecure code, all flags and all registers not used to pass the return value should be cleared.\footnote{With the exception of callee saved registers.}
\item A reordering of function definitions does not break contextual equivalence. Therefore, the order in which functions are defined in the low-level language should not provide a way to distinguish between the compilation results of contextually equivalent modules.
\end{itemize}

The compilation scheme has to be modified to allow for an ML style module language.
\begin{description}
\item[Masking] The first modification consists of adding the technique of masking, as introduced by Patrignani~\cite{Patrignani}.
This means that values created in the secure code do not leave the secure memory, as this might provide details over their implementation, and neither do direct memory pointers.
ML allows for opaque types, meaning two structures that provide the same type but with a different implementation are contextually equivalent. 
As a result these implementation must be hidden from the insecure context.
Instead of passing the values directly or passing a pointer to their memory location, every value created in secure code but passed to the insecure context is represented within insecure context as its index in a \emph{masking} list.

\item[Typing]
As a consequence of ML allowing for opaque types, simply having the correct structural implementation does not mean a function can operate on a certain value.
For this to be allowed, the value must also be of the correct opaque type, which means the masking must keep track of type information.

\item[Frames]
Because functors are not hidden on source-level from the insecure code, the insecure code can create new structures by applying functors.
Security-wise, the resulting structures are still expected to be secured in the same way as if they were written in secure code itself.
For example, applying the dictionary from \myref{lst}{lst:DictionaryFunctorExample} should not result in anyone being able to tell whether the functor implements the dictionary type using a list of pairs, as in the case of the example, or a pair of lists.

As a result, it is not adequate to statically compile away functor applications.
This means that the compiled secure code must provide a way to \emph{dynamically} create new structures that correspond to functor applications.

This is achieved by compiling functors values using generic code that takes an additional parameter: the argument structure that the functor was applied to.
For this, structures need to have a runtime representation containing a list with pointers to all their value definitions.

These runtime representations are called \emph{frames}, and are collected in a list called the \emph{f-list}.
The insecure code can refer to frames using their index in the \emph{f-list}

\item[Trimming map]
Code that represents the functor can now address values of the argument structure using the list of pointers inside the argument frame, and an offset.
Because the argument structure can define more values than the expected signature declares, the offset of a value needed by the functor within the argument structure might differ from the offset of this same value in the argument signature.
To solve this, a mapping between these offsets must be established as well, called the \emph{trimming map}. This is unique for every functor application.

\item[Functor Applications]
Functor applications introduce new structures. 
That these structures are the result of functor applications is not a distinguishable feature in the ML source language.
When a program defines a structure statically, this program is still contextually equivalent to a program that defines this structure as the result of a functor application.
Note that this introduces two new problems:
\begin{itemize}
\item 
These structures can be used as an argument to a functor again. This means that a frame must be created for these structures as well.
This has to happen dynamically, checking the argument structure to make sure that it matches with the expected signature.

\item When calling a value of a structure resulting from functor application, this call is not allowed to look different from calls that acces a value defined in a static structure defintion.
This means that all value calls must include a frame as an argument.

This frame can only be the frame that represents the structure containing the value itself.
Frames that represent the output structure of functor application must contain a reference to the frame that represents the argument of the functor application, so that the generic code representing the functor still gets a reference to the argument frame.
\\[1em]
As a result, all frames now look like this, with an empty trimming map and pointer if the frame represents a statically defined structure.

\begin{tabular}{|c c c|}
\hline
Ptr to argument frame & \gray trimming map &  list of values\\
\hline
\end{tabular}
\end{itemize}

\item[Ordering of Structure Bindings]
The order in which structures are defined provides no way to distinguish two ML programs.
As a result, the order of frames in the f-list has to be alphabetical for any frames that correspond to structures defined within the secure code.

Frames that represent structures resulting from \emph{functor applications} within \emph{insecure context} are appended in the order of their bindings.

\item[A single entry point]
Whether a structure was defined using functor application or statically is not a distinguishing feature in ML.
This has an effect on how structure values should be called.

If the functions that represent these values were direct entry points to the module, two structures that both result from functor applications would share the same entry points.
The inverse is true as well: sharing entry points is a proof of being defined using functor application.

This can be used in the low-level to distinguish between two programs, where one defines a structure using functor application and the other writes out the structure definition statically.

As a result, only a single entry point into the module is created. 
All values now share the same entry point.
Specifying which value is called can be done by providing an index in the \emph{f-list}, uniquely defining a structure, and the offset for the value within the list of values stored in the frame.
\end{description}

Implementing these checks and security precautions results in a secure compilation scheme for a language implementing an ML-style modules system.

\section{Related Work}
Lots of work trying to preserve the security of a source languages when compiling exists.
The idea of using full abstraction to formalize secure compilation is introduced by Abadi~\cite{Abadi}.

Different techniques to achieve this were developed, for example using \emph{Adress Space Layout Randomisation} or ASLR.% introduced by Abadi~\cite{AbadiASLR} as well.
The idea of ASLR catched on, and ASLR saw implementations in common operating systems such as Windows Vista, OS X Mountain Lion and some Linux distributions. 
The idea also raised scientific study, for example by Abadi and Plotkin~\cite{AbadiASLR} or Jagadeesan, et al.\cite{Jagadeesan} and criticism~\cite{Shacham:2004:EAR:1030083.1030124,Strackx:2009:BMS:1519144.1519145}.

Other techniques work by introducing security guarantees to memory access.
For example, Agten et al.~\cite{Agten:2012:SCM:2354412.2355247} already provide a secure compilation scheme for an object based language, when access to memory is restricted based on the value of the program counter. This technique is called \emph{Program Counter Based Access Control}, or \emph{PCBAC}~\cite{PCBAC}.
Later work by Patrignani et al.~\cite{Patrignani} introduced additional object oriented to the fully abstract compilation scheme.

The restricted access of memory can be implemented in hardware\cite{Sancus,SGX} or using software\cite{Fides,Salus}.
This choice effects the size of the trusted computing base or \emph{TCB}.
Even with fully abstract compilation, security issues in the TCB could lead to low-level attaques.
A recent innovation in restricting access on a hardway level is Intel\textregistered Software Guard Extensions, or \emph{SGX}~\cite{SGX}.

\section{Conclusion}
This work aims to bring a secure compilation scheme to languages that implement an ML style module system.
This work shows that such a secure compilation scheme is possible on a low-level architecture providing certain access control semantics as specified by Agten et al.~\cite{Agten:2012:SCM:2354412.2355247}. 
This was already shown to allow several advanced OOP concepts by Patrignani et al.~\cite{Patrignani}.

\clearpage
\begin{appendices}
\section{Additional Code}
\label{app:AdditionalCode}
\begin{lstlisting}[label={lst:AdditionalSignatures}, caption={The auxilary signatures and structures for the functor example of \earlier{lst}{lst:DictionaryFunctorExample}.}]
signature DICTIONARY =
    sig
        type key
        type 'a dictionary
        val emptyDictionary : 'a dictionary
        val insert : 'a dictionary -> key -> 'a -> 'a dictionary
        val lookup : 'a dictionary -> key -> 'a
    end

signature EQUAL =
    sig
        type t
        val equal : t -> t -> bool
    end

structure StringEqual: EQUAL = 
    struct
        type t = string
        fun equal t1 t2  = case String.compare(t1,t2)
                                of EQUAL => true
                                | _ => false
    end
\end{lstlisting}
\end{appendices}
\bibliographystyle{ieeetr}
\bibliography{bibliography}
\end{document}
