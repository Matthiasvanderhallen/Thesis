% $Date: 91/09/10 14:48:43 $
% $Revision: 1.9 $
% (c) 1991 Simon Peyton Jones & David Lester.
\begin{verbatim}
> module Template where
> import Language
> import Utils
\end{verbatim}
\indexDTT{module}%
\indexDTT{import}%
\indexDTT{import}%
\chapter{Template instantiation}
\label{sect:template}

This chapter introduces the simplest possible implementation of
a functional language: a graph reducer based
on \stressD{template instantiation}.

The complete source code for an initial version (Mark 1) is given, followed
by a series of improvements and variations on the basic design.
We begin with a review of graph reduction and template instantiation.

\section{A review of template instantiation}

We begin with a brief overview of template instantiation.
This material is covered in more detail in
Chapters 11 and 12 of \cite{PJBook}.

We recall the following key facts:
\begin{itemize}
\item
A functional program is `executed' by {\em evaluating an expression}.
\item
The expression is represented by a {\em graph}\indexD{graph}.
\item
Evaluation takes place by carrying out a sequence of
\stressD{reductions}.
\item
A reduction replaces (or {\em updates\/}\index{updates})
a \stressD{reducible expression}
 in the graph
by its reduced form. The term `reducible expression' is often
abbreviated to `redex'\index{redex}.
\item
Evaluation is complete when there are no more redexes; we say that the
expression is in \stressD{normal form}.
\item
At any time there may be more than one redex in the expression being
evaluated, so there is a choice about which one to reduce next.
Fortunately, whatever reduction sequence we choose, we will always get the
same answer (that is, normal form).  There is one caveat: some reduction
sequences may fail to terminate.
\index{termination}
\item
However, if
any choice of redexes makes evaluation terminate, then the policy of
always selecting the outermost redex will also do so.
This choice of reduction order is called
\stressD{normal order reduction}, and it is the one we will always use.
\end{itemize}

\begin{together}
Thus the process of evaluation can be described as follows:
\begin{verbatim}
        until there are no more redexes
                select the outermost redex
                reduce it
                update the (root of the) redex with the result
        end
\end{verbatim}
\end{together}

\subsection{An example}

As an example, consider the following Core-language program:
\begin{verbatim}
        square x = x * x ;
        main = square (square 3)
\end{verbatim}
The program consists of a set of definitions, called {\em supercombinators\/};
\mbox{\tt square}\indexTT{square} and \mbox{\tt main}\indexTT{main} are both supercombinators\indexD{supercombinator}.
By convention, the expression to be evaluated is the supercombinator \mbox{\tt main}\indexTT{main}.
Hence, to begin with the expression to be evaluated is represented by the
following rather trivial tree
(remember that a tree is just a special sort of graph):
\begin{verbatim}
        main
\end{verbatim}
Now, since \mbox{\tt main}\indexTT{main} has no arguments, it itself is a redex, so we replace it
by its body:
\begin{verbatim}
        main            reduces to      @
                                       / \
                                 square   @
                                         / \
                                   square   3
\end{verbatim}
Applications are represented by {\tt @} signs in these pictures and all
subsequent ones.

Now the outermost redex is the outer application of \mbox{\tt square}\indexTT{square}.
To reduce a function application we replace the redex with
an instance of the body of the function, substituting a pointer to the
argument for each occurrence of the formal parameter, thus:
\begin{verbatim}
                @!      reduces to      @!
               / \                     / \
         square   @                   @   \
                 / \                 / \___@
           square   3               *     / \
                                    square   3
\end{verbatim}
The root of the redex, which is overwritten with the result, is marked with
a \mbox{\tt !}.
Notice that the inner \mbox{\tt square\ 3} redex has become shared, so that
the tree has become a graph.

In the definition of \mbox{\tt square}\indexTT{square} the expression \mbox{\tt x*x} (in which the \mbox{\tt *} is
written infix) is just short for \mbox{\tt ((*\ x)\ x)}, the application of \mbox{\tt *} to
two arguments. We use {\em currying\/} to write functions of several
arguments in terms of one-argument applications: \mbox{\tt *} is a function
which, when applied to an argument \mbox{\tt p}, gives a function which, when
applied to another argument \mbox{\tt q}, returns the product of \mbox{\tt p} and \mbox{\tt q}.

Now the only redex is the inner application of \mbox{\tt square}\indexTT{square} to \mbox{\tt 3}.  The
application of \mbox{\tt *} is not reducible because \mbox{\tt *} requires its arguments to
be evaluated.
The inner application is reduced like this:
\begin{verbatim}
                @       reduces to      @
               / \                     / \
              @   \                   @   \
             / \___@!                / \___@
            *     / \               *     / \
            square   3                   @   \
                                        / \___3
                                       *
\end{verbatim}
There is still only one redex, the inner multiplication.  We replace
the redex with the result of the multiplication, \mbox{\tt 9}:
\begin{verbatim}
                @       reduces to      @
               / \                     / \
              @   \                   @   \
             / \___@!                / \___9
            *     / \               *
                 @   \
                / \___3
               *
\end{verbatim}

Notice that by physically updating the root of the redex with the
result, both arguments of the outer multiplication `see' the result
of the inner multiplication.
The final reduction is simple:
\begin{verbatim}
                @       reduces to      81
               / \
              @   \
             / \___9
            *
\end{verbatim}

\subsection{The three steps}

As we saw earlier, graph reduction consists of repeating the following
three steps until a normal form is reached:
\begin{enumerate}
\item
Find the next redex.
\item
Reduce it.
\item
Update the (root of the) redex with the result.
\end{enumerate}
As can be seen from the example in the previous section,
there are two sorts of redex, which are
reduced in different ways:
\begin{description}
\item[Supercombinators.]
If the outermost function application is a supercombinator application,
then it is certainly also a redex, and it can be reduced as described
below (Section~\ref{sect:templ:supercomb-review}).
\item[Built-in primitives.]
If the outermost function application is the application of a built-in
primitive\index{primitives}%
, then the application may or may not be a redex, depending on
whether the arguments are evaluated.
If not, then the arguments must be evaluated.
This is done using exactly the same process: repeatedly find the outermost
redex of the argument and reduce it.
Once this is done, we can return to the reduction of the outer application.
\end{description}

\subsection{Unwinding the spine to find the next redex}

The first step of the reduction cycle is to find the site of the next
reduction to be performed;
that is, the outermost reducible function application.
It is easy to find the outermost function application (though it may
not be reducible) as follows:
\begin{enumerate}
\item
Follow the
left branch of the application nodes, starting at the root, until you
get to a supercombinator or built-in primitive.
This left-branching chain of application nodes is called the {\em spine\/}
of the expression, and this process is called {\em unwinding\/}\index{unwind}
the spine.
Typically a {\em stack\/}\index{stack} is used to remember
the addresses of the nodes encountered on the way down.
\item
Now, check how many arguments the supercombinator or primitive
takes and go back up
that number of application nodes; you have now found the root of the
outermost function application.
\end{enumerate}
For example, in the expression \mbox{\tt (f\ E1\ E2\ E3)}, where \mbox{\tt f} takes two arguments,
say, the outermost function application is \mbox{\tt (f\ E1\ E2)}.  The expression and
stack would look like this:
\begin{verbatim}
         Stack
      -----------
        |  ---|-------> @
        -------        / \
        |  ---|-----> @!  E3
        -------      / \
        |  ---|---> @   E2
        -------    / \
        |  ---|-> f   E1
        -------
\end{verbatim}
The (root of the) outermost function application is marked with a \mbox{\tt !}.

If the result of an evaluation could be a partial
application\index{partial applications}, as would be the case if \mbox{\tt f} took
four arguments instead of two, then step 2 above needs to be preceded
by a check there are enough application nodes in the spine.  If not,
the expression has reached {\em weak head normal form\/} (WHNF).
\indexD{weak head normal form}
The
sub-expressions \mbox{\tt E1}, \mbox{\tt E2} and \mbox{\tt E3} might still contain redexes, but
most evaluators will stop when they reach WHNF rather than trying to
reduce the sub-expressions also.  If the program has been type-checked,
and the result is guaranteed to be a number, say, or a list, then this
underflow check\index{stack underflow check}
can be omitted.

Notice that we have only found the root of the outermost {\em function
application}.  It may or may not be a {\em redex\/} as well.  If the function
is a supercombinator, then it will certainly be a redex, but if it is
a primitive, such as \mbox{\tt +}, then it depends on whether its arguments are
evaluated.  If they are, we have found the outermost redex.  If not, we have
more work to do.

If a primitive\index{primitives}
requires the value of a currently unevaluated argument,
we must
evaluate the argument before the primitive reduction can proceed.
To do this, we must put the current stack on one
side, and begin with a new stack to reduce the argument, in the same
way as before.  This was the situation in the example of the previous section
when we reached the stage
\begin{verbatim}
                @
               / \
              @   \
             / \___@
            *     / \
            square   3
\end{verbatim}

We need to evaluate the argument \mbox{\tt (square\ 3)} on a new
stack.  During this evaluation, we might again encounter a primitive
with an unevaluated argument, so we would need to start a new evaluation
again.  We need to keep track of all the `old' stacks, so that we come
back to them in the right order.  This is conveniently done by keeping
a stack of stacks, called the {\em dump}.  When we need to evaluate an
argument, we push the current stack onto the dump\index{dump}%
; when we have finished
an evaluation we pop the old stack off the dump.

Of course, in a real implementation we would not copy whole stacks!
Since the `new' stack will be finished with before the `old' one is again
required, the `new' one could be built
physically on top of the `old' one.
The dump stack would then just keep track of where the boundary between
`new' and `old' was.  Conceptually, though, the dump is a stack of stacks,
and we will model it in this way.

\subsection{Supercombinator redexes}
\label{sect:templ:supercomb-review}

A supercombinator redex is reduced by substituting the arguments into
its body.  More precisely:
\begin{important}
{\em Supercombinator reduction}.  A supercombinator redex is reduced
by replacing the
redex with an instance of the
supercombinator body, substituting pointers to the actual arguments for
corresponding occurrences of the formal parameters.
Notice that the arguments are
not copied; rather, by the device of using pointers
to them, they are shared.\index{instantiation}
\end{important}

A supercombinator body may contain \mbox{\tt let}\indexTT{let} and \mbox{\tt letrec}\indexTT{letrec} expressions.
For example:
\begin{verbatim}
        f x = let y = x*x
              in y+y
\end{verbatim}
\mbox{\tt let}\indexTT{let} and \mbox{\tt letrec}\indexTT{letrec} expressions are treated as {\em textual descriptions of
a graph}.  Here, for example, is a possible use of the definition of \mbox{\tt f}:
\begin{verbatim}
        @       reduces to      @
       / \                     / \
      f   3                   @   \
                             / \___@y
                            +     / \
                                 @   \
                                / \___3
                               *
\end{verbatim}
The \mbox{\tt let}\indexTT{let} expression defines a sub-expression \mbox{\tt x*x}, which is named \mbox{\tt y}.
The body of the \mbox{\tt let}\indexTT{let} expression, \mbox{\tt y+y}, uses pointers to the sub-expression
in place of \mbox{\tt y}.  Thus ordinary expressions describe trees; \mbox{\tt let}\indexTT{let} expressions
describe acyclic graphs; and \mbox{\tt letrec}\indexTT{letrec} expressions describe cyclic
graphs.

\subsection{Updates}
\label{sect:templ:update-review}

After performing a reduction, we must update the root of the redex with
the result, so that if the redex is shared (as it was in the example
\mbox{\tt (square\ (square\ 3))}) the reduction is only done once.
This updating is the essence of {\em lazy
evaluation}.
A redex may not be evaluated at all but, if it is evaluated, the
update ensures that the cost of doing so is incurred at most once.
\index{updates}

Omitting the updates does not cause any errors; it will just mean that
some expressions may be evaluated more than once, which is inefficient.

There is one case that requires a little care when performing updates.
Consider the program
\begin{verbatim}
        id x = x
        f p = (id p) * p
        main = f (sqrt 4)
\end{verbatim}
After the \mbox{\tt f} reduction has taken place, the graph looks like this:
\begin{verbatim}
                @
               / \
              @   \
             / \   \
            *   @   \
               / \___@
             id     / \
                sqrt   4
\end{verbatim}
We assume \mbox{\tt sqrt}\indexTT{sqrt} is a built-in primitive for taking square roots.
Now, suppose that the next redex selected is the first argument of the \mbox{\tt *},
namely the application of \mbox{\tt id}.
(It might equally well be the second argument of \mbox{\tt *}, since neither argument
is in normal form, but we will suppose it is the first.)
What should we overwrite the root of the redex with after performing the
\mbox{\tt id} reduction?  {\em We should certainly not overwrite it with a
copy of the \mbox{\tt (sqrt\ 4)} application node,
because then \mbox{\tt (sqrt\ 4)} would be evaluated twice!}

The easiest way out of this dilemma is to add a new sort of graph node,
an {\em indirection node}\index{indirections},
which will be depicted as a \mbox{\tt {\char'43}} sign.  An
indirection node can be used to update the root of a redex to point to
the result of the reduction:
\begin{verbatim}
                @       reduces to      @
               / \                     / \
              @   \                   @   \
             / \   \                 / \   \
            *   @   \               *   #   \
               / \___@                   \___@
             id     / \                     / \
                sqrt   4                sqrt   4
\end{verbatim}
Section 12.4 of \cite{PJBook} contains further discussion of the
issues involved in updates.

\subsection{Constant applicative forms\advanced}
\label{sect:caf}

Some supercombinators have no arguments; they are called {\em constant
applicative forms}, or CAFs.
\index{CAF}
For example, \mbox{\tt fac20}\indexTT{fac20} is a CAF:
\begin{verbatim}
        fac20 = factorial 20
\end{verbatim}
The interesting thing about CAFs is that {\em the supercombinator itself
is a redex}.  We do not want to instantiate a new copy of \mbox{\tt factorial\ 20}
whenever \mbox{\tt fac20}\indexTT{fac20} is called, because that would mean repeating the
computation of \mbox{\tt factorial\ 20}.  Rather, the supercombinator \mbox{\tt fac20}\indexTT{fac20} is
the root of the \mbox{\tt fac20}\indexTT{fac20}-reduction, and should be overwritten with the
result of instantiating its body.

The practical consequence is that supercombinators should be represented
by graph nodes, in order that they can be updated in the usual way.
We will see this happening in practice in each of our implementations.

This concludes our review of graph reduction.

\section{State transition systems}

We now turn our attention to implementing graph reduction.
We will describe each of our implementations using a
{\em state transition system}.
In this section we introduce state transition systems.

A state transition system
\index{state transition system} is a notation for describing the
behaviour of a sequential machine.
At any time, the machine is in some {\em state}, beginning with a specified
{\em initial state}.
If the machine's state {\em matches\/} one of the
{\em state transition rules}, the rule {\em fires\/} and specifies a
new state for the machine.
When no state transition rule matches, execution halts.
If more than one rule matches, then one is chosen arbitrarily to fire; the
machine is then {\em non-deterministic}.  All our machines will be
deterministic.

Here is a simple example of a state transition system used to
specify a (rather inefficient) multiplication machine.  The state is a
quadruple $(n,m,d,t)$.  The numbers to be multiplied are $n$ and $m$, and
the running total is $t$, and
the machine is initialised to the state $(n,m,0,0)$.

The operation of the machine is specified by two transition rules.
The $d$ component is
repeatedly decremented towards zero while simultaneously incrementing $t$,
as specified by the first rule:
\begin{flushleft}
\qquad $\begin{array}{|lllll|}
\hline
                & n & m & d & t         \\
\Longrightarrow & n & m & d-1 & t+1     \\
                & \multicolumn{4}{l|}{\mbox{where}~d>0}  \\
\hline
\end{array}$
\end{flushleft}
We always write transition rules with each component of the new state
directly underneath the same component of the old state, so that it is
easy to see which components have changed.

When $d$ reaches zero it is initialised again to $n$, and $m$ is decremented,
until $m$ reaches zero.  This is specified by the second rule:
\begin{flushleft}
\qquad $\begin{array}{|lllll|}
\hline
                & n & m & 0 & t         \\
\Longrightarrow & n & m-1 & n & t       \\
                & \multicolumn{4}{l|}{\mbox{where}~m>0} \\
\hline
\end{array}$
\end{flushleft}
The machine terminates when no rule applies.  At this point it will be
in a state $(n,0,0,t)$, where $t$ is the product of $n$ and $m$ from the
initial state.

\begin{exercise}
Run the machine by hand
starting with initial state $(2,3,0,0)$, specifying
which rule fires at each step.  Verify that the final state is $(2,0,0,6)$.
\end{exercise}

\begin{exercise}
An invariant\index{invariant} of a sequence of states is a predicate
which is true of all of the states.  Find an invariant which expresses
the relationship between the initial value of $n$ and $m$ (call them
$N$ and $M$), and the current values of $m$, $d$ and $t$.  Hence {\em
prove\/} the conjecture that the machine performs multiplication.  To do
the proof you need to show that
\begin{enumerate}
\item
The invariant is true for the initial state.
\item
If the invariant is true for a particular state, then it is
true for its successor state.
\item
Given the invariant and the termination condition ($m=d=0$),
then $t = N*M$.
\item
The machine terminates.
\end{enumerate}
\end{exercise}

State transition systems are convenient for our purposes, because:
\begin{itemize}
\item
They are sufficiently {\em abstract\/} that we do not get tangled up in very
low-level details.
\item
They are sufficiently {\em concrete\/} that we can be sure we are
not `cheating' by hiding a lot of complexity in the rules.
\item
We can transliterate a state transition system directly into Miranda
to give an executable implementation of the system.
\end{itemize}

To illustrate the last point, we will transliterate the multiplication
machine into Miranda.  We begin by giving a type synonym to define the
type of a state in this machine:
\begin{verbatim}
> type MultState = (Int, Int, Int, Int)     -- (n, m, d, t)
\end{verbatim}
\indexDTT{type}%
Next, the function \mbox{\tt evalMult}\indexTT{evalMult} takes a state and
returns the list consisting of that state followed by all
the states which follow it:
\begin{verbatim}
> evalMult :: MultState -> [MultState]
> evalMult state = if multFinal state 
>                    then [state]
>                    else state : evalMult (stepMult state)
\end{verbatim}
\indexDTT{evalMult}%
\indexDTT{evalMult}%
The function \mbox{\tt stepMult}\indexTT{stepMult} takes a non-final state and returns the next state.
There is one equation for \mbox{\tt stepMult}\indexTT{stepMult} for each transition rule:
\begin{verbatim}
> stepMult (n, m, d, t) | d > 0  = (n, m,   d-1, t+1)
> stepMult (n, m, d, t) | d == 0 = (n, m-1, n,   t)
\end{verbatim}
\indexDTT{stepMult}%
\indexDTT{stepMult}%
The function \mbox{\tt multFinal}\indexTT{multFinal} takes a state and
tests whether the state is a final state:
\begin{verbatim}
> multFinal :: MultState -> Bool
\end{verbatim}
\indexDTT{multFinal}%
\begin{exercise}
Define the function \mbox{\tt multFinal}\indexTT{multFinal}, and run the resulting machine on
the initial state $(2,3,0,0)$, checking that the last state of the result
list is $(2,0,0,6)$.  You may find the standard function \mbox{\tt layn}\indexTT{layn} is useful
to help lay out the results more legibly.
\end{exercise}

\section{Mark 1: A minimal template instantiation graph reducer}
\index{template instantiation machine!Mark 1}

We are now ready to begin the definition of a rather simple graph reducer.
Even though it is simple, it contains many of the parts
that more sophisticated graph reducers have, so it takes a few pages
to explain.

\subsection{Transition rules for graph reduction}
\label{sect:templ:transition-rules}

The state of the template instantiation graph reduction machine
is a quadruple
\[
\mbox{\em (stack, dump, heap, globals)}
\]
or {\em(s,d,h,f)\/} for short.
\begin{itemize}
\item
The {\em stack\/} is a stack of {\em addresses}, each of which
identifies a {\em node\/} in the heap.
These nodes form the spine of the expression being
evaluated.
The notation $a_1:s$ denotes a stack whose top
element is $a_1$ and the rest of which is $s$.
\item
The {\em dump\/}\index{dump} records the state of the spine stack
\index{spine stack}
prior to
the evaluation of an argument of a strict primitive.
The dump will not be used at all in the Mark 1 machine, but it will
be useful for subsequent versions.
\item
The {\em heap\/}\index{heap} is a collection of tagged {\em nodes}\index{node}.
The notation $h[a:node]$ means that
in the heap $h$
the address $a$ refers to the node $node$.
\item
For each supercombinator (and later for each primitive),
{\em globals\/}\index{global}
gives the address of heap node representing the supercombinator
(or primitive).
\end{itemize}

A heap node can take one of three forms (for our most primitive machine):
\begin{itemize}
\item
$\mbox{\tt NAp}\indexTT{NAp}~a_1~a_2$ represents
the application of the node whose address is $a_1$ to that
whose address is $a_2$.
\item
$\mbox{\tt NSupercomb}\indexTT{NSupercomb}~args~body$ represents a supercombinator
with arguments $args$ and body $body$.
\item
$\mbox{\tt NNum}\indexTT{NNum}~n$ represents the number $n$.
\end{itemize}

There are only two state transition rules for this primitive
template instantiation machine.  The first one describes how to
unwind\index{unwind}
a single application node onto the spine stack:
\tirule{
        \tistate{a:s}{d}{h[a:\mbox{\tt NAp}\indexTT{NAp}~ a_1~ a_2]}{f}
}{
        \tistate{a_1:a:s}{d}{h}{f}
}
\label{rule:unwind}
(The heap component of the second line of this rule still includes the
mapping of address $a$ to $\mbox{\tt NAp}\indexTT{NAp}~a_1~a_2$, but we do not write it out
again, to save clutter.)
Repeated application of this rule will unwind the entire spine of
the expression onto the stack, until the node on top of the stack
is no longer a \mbox{\tt NAp}\indexTT{NAp} node.

The second rule describes how to perform a
supercombinator reduction.\index{supercombinator!reduction}
\tirulew{
        \tistate{a_0:a_1:\ldots:a_n:s}{d}
                {h[a_0:\mbox{\tt NSupercomb}\indexTT{NSupercomb}~ [x_1,\ldots,x_n]~body]}
                {f}
}{
        \tistate{a_r:s}{d}{h'}{f}
}{
        (h',a_r) = instantiate~ body~ h~
                        f[x_1 \mapsto a_1,~ \ldots,~ x_n \mapsto a_n]
}
\label{rule:sc1}
Most of the interest in this rule is hidden inside the
function $instantiate$.  Its arguments are:
\begin{itemize}
\item
the expression to instantiate,
\item
a heap,
\item
the global mapping of names to heap addresses, $f$, augmented
by the mapping of argument names to their addresses obtained from the
stack.
\end{itemize}
It returns a new heap and the address of the (root of the) newly constructed
instance.
Such a powerful operation is
really at variance with the spirit of state transition systems,
where each step is meant to be a simple atomic action, but that is the
nature of the template instantiation machine.
The implementations of later chapters will all have truly atomic actions!

Notice that the root of the redex is not itself affected by this rule;
it is merely replaced on the stack by the root of the result.
In other words, these rules describe a tree-reduction machine,
which does {\em not\/} update\index{updates}
the root of the redex, rather
than a graph-reduction
machine.
We will improve on this later in Section \ref{sect:templ:update}.

\subsection{Structure of the implementation}

Now that we have a specification of our machine, we are ready
to embark on its implementation.

Since we are writing the implementation in a functional language,
we must write a function \mbox{\tt run}\indexTT{run}, say,
to do the job.  What should its type be?
It should take a filename, run the program therein, and print
out the results, which might be either the final result or some kind
of execution trace.
So the type of \mbox{\tt run}\indexTT{run} is given by the following type signature:
\begin{verbatim}
> runProg :: [Char] -> [Char]      -- name changed to not conflict
\end{verbatim}
\indexDTT{runProg}%
\par
Now we can think about how \mbox{\tt run}\indexTT{run} might be built up.  Running a program
consists of four stages:
\begin{enumerate}
\item
Parse the program from the expression found in a specified file.
The \mbox{\tt parse}\indexTT{parse} function takes a filename and returns the parsed program.
\begin{verbatim}
>    parse :: [Char] -> CoreProgram
\end{verbatim}
\item
Translate the program
into a form suitable for execution.  The \mbox{\tt compile}\indexTT{compile} function,
which performs this task, takes a program and
produces the initial state of the template instantiation machine:
\begin{verbatim}
> compile :: CoreProgram -> TiState
\end{verbatim}
\indexDTT{compile}%
\mbox{\tt tiState}\indexTT{tiState} is the type of the state of the template instantiation machine.
(The prefix `\mbox{\tt ti}' is short for template instantiation.)
\item
Execute the program, by performing repeated state transitions until
a final state is reached.  The result is a list of all the states passed
through; from this we can subsequently
either extract the final state, or get a trace
of all the states.
For the present we will restrict ourselves to programs which return
a number as their result, so we call this execution function \mbox{\tt eval}\indexTT{eval}.
\begin{verbatim}
> eval :: TiState -> [TiState]
\end{verbatim}
\indexDTT{eval}%
\item
Format the results for printing.  This is done by the function \mbox{\tt showResults}\indexTT{showResults},
which selects which information to print, and formats it into a
list of characters.
\begin{verbatim}
> showResults :: [TiState] -> [Char]
\end{verbatim}
\indexDTT{showResults}%
\end{enumerate}
The function \mbox{\tt run}\indexTT{run} is just the composition of these four functions:
\begin{verbatim}
> runProg = showResults . eval . compile . parse  -- "run": name conflict
\end{verbatim}
\indexDTT{runProg}%
We will devote a subsection to each of these phases.


\subsection{The parser}

The source language, including the \mbox{\tt parse}\indexTT{parse} function,
is defined in a separate module \mbox{\tt language}\indexTT{language}, defined
in Chapter~\ref{sect:language}.
We make it available using the
\mbox{\tt {\char'45}include} directive to import the module:
\begin{verbatim}
> -- import Language
\end{verbatim}
\subsection{The compiler}

In this section we define the \mbox{\tt compile}\indexTT{compile} function.  We will need
the data types and functions defined in the \mbox{\tt utils}\indexTT{utils} module, so we
use \mbox{\tt {\char'45}include} to make it available.
\begin{verbatim}
> -- import Utils
\end{verbatim}
Now we need to consider the representation of the data types the
compiler manipulates.

\subsubsection{Data types}

The compiler produces the initial state of the machine, which has
type \mbox{\tt tiState}\indexTT{tiState}, so the next thing to do is to define how machine states
are represented, using a type synonym\index{type synonym}:
\begin{verbatim}
> type TiState = (TiStack, TiDump, TiHeap, TiGlobals, TiStats)
\end{verbatim}
\indexDTT{type}%
The state of the machine is a quintuple whose first four components correspond
exactly to those given in Section~\ref{sect:templ:transition-rules}, and whose
fifth component is used to accumulate statistics.

Next, we need to consider the representation of each of these components.

\begin{itemize}
\item
The {\em spine stack\/}\index{spine stack}
is just a stack of {\em heap addresses\/}:
\begin{verbatim}
> type TiStack = [Addr]
\end{verbatim}
\indexDTT{type}%
We choose to represent the stack as a list.  The elements of the stack
are members of the abstract data type \mbox{\tt addr}\indexTT{addr} defined in the \mbox{\tt utils}\indexTT{utils}
module (Appendix~\ref{sect:heap}).
They represent heap addresses, and by making them abstract we ensure that
we can only use the operations provided on them by the \mbox{\tt utils}\indexTT{utils} module.
Thus it is impossible for us to add one to an address, say, by mistake.

\item
The {\em dump\/} is
not required until Section~\ref{sect:templ:primitives}, but we make it
part of the state already because adding it later would require many tiresome
alterations to the state transition rules.  For now we give it a trivial
type definition, consisting of just a single constructor with no arguments.
\begin{verbatim}
> data TiDump = DummyTiDump
> initialTiDump = DummyTiDump
\end{verbatim}
\indexDTT{data}%
\indexDTT{initialTiDump}%
\item
The {\em heap\/} is represented by the \mbox{\tt heap}\indexTT{heap} abstract data type defined in
the \mbox{\tt utils}\indexTT{utils} module.  We have to say what the heap contains, namely objects
of type \mbox{\tt node}\indexTT{node} (yet to be defined):
\begin{verbatim}
> type TiHeap = Heap Node
\end{verbatim}
\indexDTT{type}%
Heap \mbox{\tt node}\indexTT{node}s are represented by the following algebraic data type
declaration, which corresponds to the list of possibilities given in
Section~\ref{sect:templ:transition-rules}:
\begin{verbatim}
> data Node = NAp Addr Addr                     -- Application
>             | NSupercomb Name [Name] CoreExpr -- Supercombinator
>             | NNum Int                        -- A number
\end{verbatim}
\indexDTT{data}%
\indexDTT{NSupercomb}%
\indexDTT{NNum}%
The only difference is that we have added an extra field of type
\mbox{\tt name}\indexTT{name} to the \mbox{\tt NSupercomb}\indexTT{NSupercomb} constructor, which is used to hold the name
of the supercombinator.  This is used only for documentation and
debugging purposes.

\item
The {\em globals\/} component associates each
supercombinator name with
the address of a heap node containing its definition:
\begin{verbatim}
> type TiGlobals = ASSOC Name Addr
\end{verbatim}
\indexDTT{type}%
The \mbox{\tt assoc}\indexTT{assoc} type is defined in the \mbox{\tt utils}\indexTT{utils} module, along with its
operations (Appendix~\ref{sect:assoc}).
It is actually defined there as a type synonym (not an abstract
data type) because it is so convenient to be able to manipulate associations
using the built-in syntax for lists.  There is a tension here between
abstraction and ease of programming.

\item
The \mbox{\tt tiStats}\indexTT{tiStats} component of the state is not mentioned in the
transition rules, but we will use it to collect run-time performance
statistics\index{statistics} on what the machine does.
So that we can easily change what statistics are collected,
we will make it an abstract type.  To begin with, we will record only
the number of steps taken:
\begin{verbatim}
> tiStatInitial  :: TiStats
> tiStatIncSteps :: TiStats -> TiStats
> tiStatGetSteps :: TiStats -> Int
\end{verbatim}
\indexDTT{tiStatInitial}%
\indexDTT{tiStatIncSteps}%
\indexDTT{tiStatGetSteps}%
The implementation is rather simple:
\begin{verbatim}
> type TiStats = Int
> tiStatInitial    = 0
> tiStatIncSteps s = s+1
> tiStatGetSteps s = s
\end{verbatim}
\indexDTT{type}%
\indexDTT{tiStatInitial}%
\indexDTT{tiStatIncSteps}%
\indexDTT{tiStatGetSteps}%
A useful function \mbox{\tt applyToStats}\indexTT{applyToStats} applies a given function to the
statistics\index{statistics}
component of the state:
\begin{verbatim}
> applyToStats :: (TiStats -> TiStats) -> TiState -> TiState
> applyToStats stats_fun (stack, dump, heap, sc_defs, stats)
>  = (stack, dump, heap, sc_defs, stats_fun stats)
\end{verbatim}
\indexDTT{applyToStats}%
\indexDTT{applyToStats}%
\end{itemize}
This completes our definition of the data types involved.

\subsubsection{The compiler itself}
\label{sect:ti:compiler}
\label{sect:mapAccuml-example}

The business of the compiler is to take a program, and from
it create the initial state of the machine:
\begin{verbatim}
> compile program
>  = (initial_stack, initialTiDump, initial_heap, globals, tiStatInitial)
>    where
>    sc_defs = program ++ preludeDefs ++ extraPreludeDefs
>
>    (initial_heap, globals) = buildInitialHeap sc_defs
>
>    initial_stack = [address_of_main]
>    address_of_main = aLookup globals "main" (error "main is not defined")
\end{verbatim}
\indexDTT{compile}%
\par
Let us consider each of the definitions in the \mbox{\tt where}\indexTT{where} clause in turn.
The first, \mbox{\tt sc{\char'137}defs}, is just a list
of all the supercombinator definitions involved in the program.
Recall that \mbox{\tt preludeDefs}\indexTT{preludeDefs} was defined in Section~\ref{sect:prelude}
to be the list of standard supercombinator definitions which
are always included in every program.  \mbox{\tt extraPreludeDefs}\indexTT{extraPreludeDefs} is a list of
any further standard functions we may want to add; for the present it is
empty:
\begin{verbatim}
> extraPreludeDefs = []
\end{verbatim}
\indexDTT{extraPreludeDefs}%
\par
The second definition uses an auxiliary function, \mbox{\tt buildInitialHeap}\indexTT{buildInitialHeap}, to
construct an initial heap containing an \mbox{\tt NSupercomb}\indexTT{NSupercomb} node for
each supercombinator, together with an association list \mbox{\tt globals}\indexTT{globals} which
maps each supercombinator name onto the address of its node.

Lastly, \mbox{\tt initial{\char'137}stack} is defined to contain just one item, the address
of the node for the supercombinator \mbox{\tt main}\indexTT{main}, obtained from \mbox{\tt globals}\indexTT{globals}.

Now we need to consider the definition of \mbox{\tt buildInitialHeap}\indexTT{buildInitialHeap},
which is a little
tricky.  We need to do something for each element of the list \mbox{\tt sc{\char'137}defs},
but what makes it awkward is that the `something' involves heap allocation.
Since each heap allocation produces a new heap, we need to find a way of
passing the heap along from one element of \mbox{\tt sc{\char'137}defs} to the next.
This process starts with the empty heap, \mbox{\tt hInitial}\indexTT{hInitial}
(Appendix~\ref{sect:heap}).

We encapsulate this idea in a higher-order function\index{higher-order function}
\mbox{\tt mapAccuml}\indexTT{mapAccuml}, which
we will use quite a lot in this book.  \mbox{\tt mapAccuml}\indexTT{mapAccuml} takes three
arguments: $f$, the `processing function'; $acc$, the
`accumulator'; and a list $[x_1, \ldots, x_n]$.  It takes each
element of the input list, and applies $f$ to it and the current
accumulator\index{accumulator}.  $f$ returns a pair of results, an
element of the result list and a new value for the accumulator.
\mbox{\tt mapAccuml}\indexTT{mapAccuml} passes the accumulator along from one call of $f$ to the
next, and eventually returns a pair of results: $acc'$, the final
value of the accumulator; and the result list $[y_1, \ldots, y_n]$.
Figure~\ref{fig:mapAccuml} illustrates this plumbing.  The definition
of \mbox{\tt mapAccuml}\indexTT{mapAccuml} is given in Appendix~\ref{sect:util-funs}.
\begin{figure} %\centering
\input{map_acc.tex}
\caption{A picture of $\mbox{\tt mapAccuml}\indexTT{mapAccuml}~f~acc~[x_1,\ldots,x_n]$}
\label{fig:mapAccuml}
\end{figure}

In our case, the `accumulator' is the heap, with initial value
\mbox{\tt hInitial}\indexTT{hInitial}.  The list $[x_1,\ldots,x_n]$ is the supercombinator
definitions, \mbox{\tt sc{\char'137}defs}, while the result list $[y_1, \ldots, y_n]$ is
the association of supercombinator names and addresses, \mbox{\tt sc{\char'137}addrs}.
Here, then, is the definition of \mbox{\tt buildInitialHeap}\indexTT{buildInitialHeap}.
\begin{verbatim}
> buildInitialHeap :: [CoreScDefn] -> (TiHeap, TiGlobals)
> buildInitialHeap sc_defs = mapAccuml allocateSc hInitial sc_defs
\end{verbatim}
\indexDTT{buildInitialHeap}%
\indexDTT{buildInitialHeap}%
\par
The `processing function', which we will call \mbox{\tt allocateSC}\indexTT{allocateSC},
allocates a single supercombinator, returning a new heap and a member
of the \mbox{\tt sc{\char'137}addrs} association list.
\begin{verbatim}
> allocateSc :: TiHeap -> CoreScDefn -> (TiHeap, (Name, Addr))
> allocateSc heap (name, args, body)
>  = (heap', (name, addr))
>    where
>    (heap', addr) = hAlloc heap (NSupercomb name args body)
\end{verbatim}
\indexDTT{allocateSc}%
\indexDTT{allocateSc}%
That completes the definition of the compiler. Next, we turn our attention
to the evaluator.

\subsection{The evaluator\index{evaluator}}

The evaluator \mbox{\tt eval}\indexTT{eval} takes an initial machine state,
and runs the machine one step at a time, returning the list of
all states it has been through.

\mbox{\tt eval}\indexTT{eval} always returns the current state as the first element of its
result.  If the current state is a final state, no further states are
returned; otherwise, \mbox{\tt eval}\indexTT{eval} is applied recursively to the next state.
The latter is obtained by
taking a single step (using \mbox{\tt step}\indexTT{step}), and then calling \mbox{\tt doAdmin}\indexTT{doAdmin} to
do any administrative work required between steps.
\begin{verbatim}
> eval state = state : rest_states
>              where
>              rest_states | tiFinal state = []
>                          | otherwise = eval next_state
>              next_state  = doAdmin (step state)
\end{verbatim}
\indexDTT{eval}%
\begin{verbatim}
> doAdmin :: TiState -> TiState
> doAdmin state = applyToStats tiStatIncSteps state
\end{verbatim}
\indexDTT{doAdmin}%
\indexDTT{doAdmin}%
\subsubsection{Testing for a final state}

The function \mbox{\tt tiFinal}\indexTT{tiFinal} detects the final state\index{final state}.
We are only finished if the stack contains a single object, and it is
either a number or a data object.
\begin{verbatim}
> tiFinal :: TiState -> Bool
>
> tiFinal ([sole_addr], dump, heap, globals, stats)
>  = isDataNode (hLookup heap sole_addr)
>
> tiFinal ([], dump, heap, globals, stats) = error "Empty stack!"
> tiFinal state = False              -- Stack contains more than one item
\end{verbatim}
\indexDTT{tiFinal}%
\indexDTT{tiFinal}%
\indexDTT{tiFinal}%
\indexDTT{tiFinal}%
Notice that the stack element is an address, which we need to look up in
the heap before we can check whether it is a number or not.
We should also produce a sensible error message if the stack should be empty
(which should never happen).

Finally, we can define \mbox{\tt isDataNode}\indexTT{isDataNode}:
\begin{verbatim}
> isDataNode :: Node -> Bool
> isDataNode (NNum n) = True
> isDataNode node     = False
\end{verbatim}
\indexDTT{isDataNode}%
\indexDTT{isDataNode}%
\indexDTT{isDataNode}%
\subsubsection{Taking a step}

The function \mbox{\tt step}\indexTT{step} maps one state into its successor:
\begin{verbatim}
> step :: TiState -> TiState
\end{verbatim}
\indexDTT{step}%
It has to do case analysis on the node on top
of the spine stack\index{spine stack}, so it extracts
this node from the heap,
and uses \mbox{\tt dispatch}\indexTT{dispatch} to call an appropriate function to
do the hard work for each form of node.
\begin{verbatim}
> step state
>  = dispatch (hLookup heap (hd stack))
>    where
>    (stack, dump, heap, globals, stats) = state
>
>    dispatch (NNum n)                  = numStep state n
>    dispatch (NAp a1 a2)               = apStep  state a1 a2
>    dispatch (NSupercomb sc args body) = scStep  state sc args body
\end{verbatim}
\indexDTT{step}%
\par
We can deal with the cases for numbers and applications with very little
trouble.
It is an error for there to be a number
on top of the stack, since
a number should never be applied as a function.
(If it was the only object on the stack, execution will have been halted
by \mbox{\tt tiFinal}\indexTT{tiFinal}.)
\begin{verbatim}
> numStep :: TiState -> Int -> TiState
> numStep state n = error "Number applied as a function!"
\end{verbatim}
\indexDTT{numStep}%
\indexDTT{numStep}%
Dealing with an application node is described by the unwind\index{unwind}
rule
(\ruleref{rule:unwind}),
which can be translated directly into Miranda:
\begin{verbatim}
> apStep :: TiState -> Addr -> Addr -> TiState
> apStep (stack, dump, heap, globals, stats) a1 a2
>  = (a1 : stack, dump, heap, globals, stats)
\end{verbatim}
\indexDTT{apStep}%
\indexDTT{apStep}%

\subsubsection{Applying a supercombinator}
\label{instantiate}

To apply a supercombinator, we must
instantiate its body,
binding the argument names to the argument addresses found
in the stack (\Ruleref{rule:sc1}).
Then we discard the arguments from the stack, including
the root of the redex, and push the (root of the) result of the
reduction onto the stack instead.
(Remember, in this first version of the machine we are not performing updates.)
\label{page:sc-step}
\begin{verbatim}
> scStep   :: TiState -> Name -> [Name] -> CoreExpr -> TiState
> scStep (stack, dump, heap, globals, stats) sc_name arg_names body
>  = (new_stack, dump, new_heap, globals, stats)
>    where
>    new_stack = result_addr : (drop (length arg_names+1) stack)
>
>    (new_heap, result_addr) = instantiate body heap env
>    env = arg_bindings ++ globals
>    arg_bindings = zip2 arg_names (getargs heap stack)
\end{verbatim}
\indexDTT{scStep}%
\indexDTT{scStep}%
\par
In order to apply supercombinators and primitives, we need an
auxiliary function.
The function \mbox{\tt getArgs}\indexTT{getArgs} takes a stack (which must consist
of a supercombinator on top of a stack of application nodes),
and returns a list formed from the argument of each of the application nodes
on the stack.
\begin{verbatim}
> -- now getargs since getArgs conflicts with Gofer standard.prelude
> getargs :: TiHeap -> TiStack -> [Addr]
> getargs heap (sc:stack)
>  = map get_arg stack
>    where get_arg addr = arg  where (NAp fun arg) = hLookup heap addr
\end{verbatim}
\indexDTT{getargs}%
\indexDTT{getargs}%
\par
The \mbox{\tt instantiate}\indexTT{instantiate} function takes an expression, a heap and
an environment associating
names with addresses.\index{instantiation}\index{template instantiation}
It creates an instance of the expression in the heap, and
returns the new heap and address of the root of the instance.
The environment is used by \mbox{\tt instantiate}\indexTT{instantiate}
to specify the addresses to be substituted for supercombinators and
local variables.
\begin{verbatim}
> instantiate :: CoreExpr              -- Body of supercombinator
>                -> TiHeap             -- Heap before instantiation
>                -> ASSOC Name Addr    -- Association of names to addresses
>                -> (TiHeap, Addr)     -- Heap after instantiation, and
>                                      -- address of root of instance
\end{verbatim}
\indexDTT{instantiate}%
\par
\label{page:instantiate}
The case for numbers is quite straightforward.
\begin{verbatim}
> instantiate (ENum n) heap env = hAlloc heap (NNum n)
\end{verbatim}
\indexDTT{instantiate}%
\par
The case for applications is also simple; just instantiate the
two branches, and build the application node.
Notice how we `thread' the heap though the recursive calls to \mbox{\tt instantiate}\indexTT{instantiate}.
That is, the first call to instantiate is given a heap and produces a new
heap; the latter is given to the second call to instantiate, which produces
yet another heap; the latter is the heap in which the new application node
is allocated, producing a final heap which is returned to the caller.
\begin{verbatim}
> instantiate (EAp e1 e2) heap env
>  = hAlloc heap2 (NAp a1 a2) where (heap1, a1) = instantiate e1 heap  env
>                                   (heap2, a2) = instantiate e2 heap1 env
\end{verbatim}
\indexDTT{instantiate}%
\par
For variables, we simply look up the name in the environment
we are given, producing a suitable error message if we do not find
a binding for it.
\begin{verbatim}
> instantiate (EVar v) heap env
>  = (heap, aLookup env v (error ("Undefined name " ++ show v)))
\end{verbatim}
\indexDTT{instantiate}%
\mbox{\tt aLookup}\indexTT{aLookup}, which is defined in Appendix~\ref{sect:assoc}, looks up
a variable in an association list, but returns its third argument if the
lookup fails.

We postpone the question of instantiating
constructors and \mbox{\tt let(rec)}\indexTT{let(rec)} expressions
by calling auxiliary functions \mbox{\tt instantiateConstr}\indexTT{instantiateConstr} and \mbox{\tt instantiateLet}\indexTT{instantiateLet},
which each give errors for the present; later we will
replace them with operational definitions.  Lastly, the template machine is
unable to handle \mbox{\tt case}\indexTT{case} expressions at all, as we will see.
\begin{verbatim}
> instantiate (EConstr tag arity) heap env
>               = instantiateConstr tag arity heap env
> instantiate (ELet isrec defs body) heap env
>               = instantiateLet isrec defs body heap env
> instantiate (ECase e alts) heap env = error "Can't instantiate case exprs"
\end{verbatim}
\indexDTT{instantiate}%
\indexDTT{instantiate}%
\indexDTT{instantiate}%
\begin{verbatim}
> instantiateConstr tag arity heap env
>            = error "Can't instantiate constructors yet"
> instantiateLet isrec defs body heap env
>            = error "Can't instantiate let(rec)s yet"
\end{verbatim}
\indexDTT{instantiateConstr}%
\indexDTT{instantiateLet}%
\subsection{Formatting the results}

The output from \mbox{\tt eval}\indexTT{eval} is a list of states, which are rather
voluminous if printed in their entirety.
Furthermore, since the heaps and stacks are abstract objects, Miranda
will not print them at all.
So the \mbox{\tt showResults}\indexTT{showResults} function formats the output for us, using the
\mbox{\tt iseq}\indexTT{iseq} data type introduced in Section~\ref{sect:pretty}.
\begin{verbatim}
> showResults states
>  = iDisplay (iConcat [ iLayn (map showState states),
>                      showStats (last states)
>           ])
\end{verbatim}
\indexDTT{showResults}%
We display the state just by showing the contents of the stack.  It is
too tiresome to print the heap in its entirety after each step, so
we will content ourselves with printing the contents of nodes referred to
directly from the stack.  The other components of the state do not change,
so we will not print them either.
\begin{verbatim}
> showState :: TiState -> Iseq
> showState (stack, dump, heap, globals, stats)
>  = iConcat [ showStack heap stack, iNewline ]
\end{verbatim}
\indexDTT{showState}%
\indexDTT{showState}%
\par
We display the stack, topmost element first, by displaying the address on
the stack, and the contents of the node to which it points.  Most of these
nodes are application nodes, and for each of these we
also display the contents of its argument node.
\begin{verbatim}
> showStack :: TiHeap -> TiStack -> Iseq
> showStack heap stack
>  = iConcat [
>        iStr "Stk [",
>        iIndent (iInterleave iNewline (map show_stack_item stack)),
>        iStr " ]"
>    ]
>    where
>    show_stack_item addr
>     = iConcat [ showFWAddr addr, iStr ": ",
>                 showStkNode heap (hLookup heap addr)
>       ]
\end{verbatim}
\indexDTT{showStack}%
\indexDTT{showStack}%
\begin{verbatim}
> showStkNode :: TiHeap -> Node -> Iseq
> showStkNode heap (NAp fun_addr arg_addr)
>  = iConcat [   iStr "NAp ", showFWAddr fun_addr,
>                iStr " ", showFWAddr arg_addr, iStr " (",
>                showNode (hLookup heap arg_addr), iStr ")"
>    ]
> showStkNode heap node = showNode node
\end{verbatim}
\indexDTT{showStkNode}%
\indexDTT{showStkNode}%
\indexDTT{showStkNode}%
\par
\mbox{\tt showNode}\indexTT{showNode} displays the value of a \mbox{\tt node}\indexTT{node}.
It prints only the name stored
inside \mbox{\tt NSupercomb}\indexTT{NSupercomb} nodes, rather than printing the
complete value; indeed this is the only reason the name is stored inside
these nodes.
\begin{verbatim}
> showNode :: Node -> Iseq
> showNode (NAp a1 a2) = iConcat [ iStr "NAp ", showAddr a1,
>                                  iStr " ",    showAddr a2
>                        ]
> showNode (NSupercomb name args body) = iStr ("NSupercomb " ++ name)
> showNode (NNum n) = (iStr "NNum ") `iAppend` (iNum n)
\end{verbatim}
\indexDTT{showNode}%
\indexDTT{showNode}%
\indexDTT{showNode}%
\indexDTT{showNode}%
\begin{verbatim}
> showAddr :: Addr -> Iseq
> showAddr addr = iStr (show addr)
>
> showFWAddr :: Addr -> Iseq    -- Show address in field of width 4
> showFWAddr addr = iStr (space (4 -  length str) ++ str)
>                   where
>                   str = show addr
\end{verbatim}
\indexDTT{showAddr}%
\indexDTT{showAddr}%
\indexDTT{showFWAddr}%
\indexDTT{showFWAddr}%
\par
\mbox{\tt showStats}\indexTT{showStats} is responsible for printing out the accumulated statistics:
\begin{verbatim}
> showStats :: TiState -> Iseq
> showStats (stack, dump, heap, globals, stats)
>  = iConcat [ iNewline, iNewline, iStr "Total number of steps = ",
>              iNum (tiStatGetSteps stats)
>    ]
\end{verbatim}
\indexDTT{showStats}%
\indexDTT{showStats}%
%\subsection{Exercises}

\begin{exercise}
\label{ex:templ:test}
Test the implementation given so far.  Here is a suitable test program:
\begin{verbatim}
        main = S K K 3
\end{verbatim}
The result should be the number 3.
Invent a couple more test programs and check that they work.
Remember, we have not yet defined any arithmetic operations!
\end{exercise}

\begin{exercise}
Modify \mbox{\tt showState}\indexTT{showState} so that it prints out the entire contents of the heap.
(Hint: use \mbox{\tt hAddresses}\indexTT{hAddresses} to discover the addresses of all the nodes in the
heap.)
In this way you can see how the heap evolves from one step to the next.
\end{exercise}

\begin{exercise}
\mbox{\tt scStep}\indexTT{scStep} will fail
if the supercombinator or
primitive is applied to too few arguments.  Add a suitable
check and error message to \mbox{\tt scStep}\indexTT{scStep} to detect this case.
\end{exercise}

\begin{exercise}
\label{ex:templ:stats}
Modify your interpreter to collect more execution statistics.
For example, you could accumulate:
\begin{itemize}
\item
The number of reductions, perhaps split into supercombinator
reductions and primitive reductions.
\item
The number of each kind of heap operation, especially allocations.
The most convenient way to do this is to modify the \mbox{\tt heap}\indexTT{heap} abstract
data type to accumulate this information itself, though this only
works for heap operations which return a new heap as part of the
result.
\item
The maximum stack depth.
\end{itemize}
\end{exercise}
\begin{exercise}
In the definition of \mbox{\tt scStep}\indexTT{scStep}, the environment \mbox{\tt env}\indexTT{env} which is passed to
\mbox{\tt instantiate}\indexTT{instantiate} is defined as
\begin{verbatim}
        env = arg_bindings ++ globals
\end{verbatim}
What difference would it make if the arguments to \mbox{\tt ++} were reversed?
\end{exercise}
\begin{exercise}
\label{ex:templ:eval}
(Slightly tricky.)
You might think that the following definition for \mbox{\tt eval}\indexTT{eval} would be more
obvious than the one given:
\begin{verbatim}
        eval state = [state],                   tiFinal state
                   = state : eval next_state,   otherwise
\end{verbatim}
(where \mbox{\tt next{\char'137}state} is defined as before).
Why is this an inferior definition?  (Hint: think about what
would happen if all the states were being formatted by \mbox{\tt showResults}\indexTT{showResults}, and
some error occurred when evaluating \mbox{\tt tiFinal\ state}, such as an attempt
to access a non-existent heap node.
Would the state which caused the error be printed?  If not, why not?)
\end{exercise}

\section{Mark 2: \mbox{\tt let(rec)}\indexTT{let(rec)} expressions}
\index{template instantiation machine!Mark 2}
\index{let expressions@\mbox{\tt let}\indexTT{let} expressions!in template-instantiation machine}
\index{letrec expressions@\mbox{\tt letrec}\indexTT{letrec} expressions!in template-instantiation machine}

Our first enhancement is to make the machine capable of
dealing with \mbox{\tt let}\indexTT{let} and \mbox{\tt letrec}\indexTT{letrec} expressions.
As discussed in Section~\ref{sect:templ:supercomb-review}, the bodies
of supercombinators may contain \mbox{\tt let(rec)}\indexTT{let(rec)} expressions, which are
regarded as textual descriptions of a graph.

It follows that the only change we have to make to our implementation is to
enhance \mbox{\tt instantiate}\indexTT{instantiate}, so that it has an equation for the \mbox{\tt ELet}\indexTT{ELet} constructor.

\begin{exercise}
\label{ex:inst-let}
Add an equation to \mbox{\tt instantiate}\indexTT{instantiate} for {\em non-recursive\/}
\mbox{\tt let}\indexTT{let} expressions.
What you will need to do to instantiate \mbox{\mbox{\tt (ELet\ nonRecursive\ defs\ body)}}
is:
\begin{enumerate}
\item
instantiate the
right-hand side of each of the definitions in \mbox{\tt defs}\indexTT{defs};
\item
augment the
environment to bind the names in \mbox{\tt defs}\indexTT{defs} to the addresses of the
newly constructed instances;
\item
call \mbox{\tt instantiate}\indexTT{instantiate} passing the augmented
environment and the expression \mbox{\tt body}\indexTT{body}.
\end{enumerate}
\end{exercise}
This still only takes care of \mbox{\tt let}\indexTT{let} expressions.
The result of instantiating a \mbox{\tt letrec}\indexTT{letrec} expression is a {\em cyclic\/} graph,
whereas \mbox{\tt let}\indexTT{let} expressions give rise to acyclic graphs.

\begin{exercise}
\label{ex:inst-letrec}
Copy your equation for the non-recursive \mbox{\tt ELet}\indexTT{ELet} of \mbox{\tt instantiate}\indexTT{instantiate},
and modify
it to work for the recursive case (or modify your definition
to deal with both).

(Hint: do everything exactly as in the \mbox{\tt let}\indexTT{let} case, except that in
Step 1 pass the {\em augmented\/} environment (constructed in
Step 2) to \mbox{\tt instantiate}\indexTT{instantiate}, instead of the {\em existing\/} environment.)
\end{exercise}

The hint in this exercise
seems curious, because it requires the name-to-address bindings produced in
Step 2 to be used as an input to Step 1.
If you try this in Miranda it all works perfectly because,
as in any non-strict functional language, the inputs to a function do
not have to be evaluated before the function is called.
In a real implementation we would have to do this trick `by hand',
by working out the addresses at which each of the (root) nodes in
the \mbox{\tt letrec}\indexTT{letrec} will be allocated, augmenting the environment to
reflect this information, and then instantiating the right-hand sides.

Here is a test program, to see if your implementation works:
\begin{verbatim}
    pair x y f = f x y ;
    fst p = p K ;
    snd p = p K1 ;
    f x y = letrec
                a = pair x b ;
                b = pair y a
            in
            fst (snd (snd (snd a))) ;
    main = f 3 4
\end{verbatim}
The result should be 4.
Can you figure out how this program works?  (All will be revealed in
Section~\ref{sect:template:data-str-hof}.)

\begin{exercise}
Consider the program
\begin{verbatim}
    main = letrec f = f x in f
\end{verbatim}
What happens if you run this program?  Could this problem ever arise
in a strongly typed language such as Miranda?
\end{exercise}

\section{Mark 3: Adding updating}
\label{sect:templ:update}
\index{template instantiation machine!Mark 3}
\index{updates!in template-instantiation machine}

So far our reduction machine does not perform any
updates,
so shared sub-expressions may be evaluated many times.
As explained in Section~\ref{sect:templ:update-review}
the easiest way to fix the problem is to update the root of the
redex with an indirection\index{indirections} node pointing to the result.

We can express this by modifying the state transition rule (\ref{rule:sc1})
for supercombinator redexes:
\tirulew{
        \tistate{a_0:a_1:\ldots:a_n:s}{d}
                {h[a_0:\mbox{\tt NSupercomb}\indexTT{NSupercomb}~ [x_1,\ldots,x_n]~body]}
                {f}
}{
        \tistate{a_r:s}{d}{h'[a_n:\mbox{\tt NInd}\indexTT{NInd}~a_r]}{f}
}{
        (h',a_r) = instantiate~ body~ h~
                        f[x_1 \mapsto a_1,~ \ldots,~ x_n \mapsto a_n]
}
\label{rule:sc2}
The difference is that the heap $h'$ returned by the
$instantiate$ function is further modified by overwriting the node $a_n$
(the root of the redex) with an indirection to $a_r$ (the root of the result,
returned by $instantiate$).  Notice that if the supercombinator is
a CAF\index{CAF}
(see Section~\ref{sect:caf}), then $n=0$ and the node to be modified is
the supercombinator node itself.

One further modification is required.  Since we may now encounter
indirections\index{indirections}
during unwinding\index{unwind}
the spine, we need to add a new rule to cope with this case:
\tirule{
        \tistate{a:s}{d}{h[a:\mbox{\tt NInd}\indexTT{NInd}~ a_1]}{f}
}{
        \tistate{a_1:s}{d}{h}{f}
}
\label{rule:unwind-ind}
The address of the indirection node\index{indirections},
$a$, is removed from the stack, just
as if it had never been there.

There are several things we need to do to implement these new rules:
\begin{itemize}
\item
Add a new node constructor, \mbox{\tt NInd}\indexTT{NInd}, to the \mbox{\tt node}\indexTT{node} data type.
This gives the following revised definition:
\begin{verbatim}
> data Node = NAp Addr Addr                      -- Application
>            | NSupercomb Name [Name] CoreExpr   -- Supercombinator
>            | NNum Int                          -- Number
>            | NInd Addr                         -- Indirection
\end{verbatim}
\indexDTT{data}%
\indexDTT{NSupercomb}%
\indexDTT{NNum}%
\indexDTT{NInd}%
We need to add a new equation to \mbox{\tt showNode}\indexTT{showNode} to take account of this
extra constructor.
\item
Modify \mbox{\tt scStep}\indexTT{scStep} to use \mbox{\tt hUpdate}\indexTT{hUpdate} to
update the root of the redex with an indirection
to the result (\ruleref{rule:sc2}).
\item
Add an equation to the definition of \mbox{\tt dispatch}\indexTT{dispatch} to cope with indirections
(\ruleref{rule:unwind-ind}).
\end{itemize}

\begin{exercise}
Make the modifications to perform updates with indirection nodes.
Try out the effect of your changes by running the following program
on both the Mark 1 and Mark 3 versions of your reduction machine:
\begin{verbatim}
        id x = x ;
        main = twice twice id 3
\end{verbatim}
(Recall that \mbox{\tt twice}\indexTT{twice} is defined in \mbox{\tt preludeDefs}\indexTT{preludeDefs}
-- Section~\ref{sect:prelude}.)
Try to figure out what would happen by reducing it by hand first.
What happens if you define \mbox{\tt main}\indexTT{main} to be \mbox{\tt twice\ twice\ twice\ id\ 3}?
\end{exercise}

\subsection{Reducing the number of indirections\index{indirections!reducing occurrences of}}
\label{sect:templ:inst-and-upd}

Often we will be updating the root of the redex with an indirection
to a node newly created by \mbox{\tt instantiate}\indexTT{instantiate} (or, as we shall see, by a
primitive).
Under these circumstances, rather than use an indirection,
it would be safe to build the root node of the result directly on top
of the root of the redex.  Because the root of the result is newly created,
no sharing can be lost by doing this, and it saves building (and subsequently
traversing) an extra indirection node.

We can do this by defining a new instantiation function, 
\mbox{\tt instantiateAndUpdate}\indexTT{instantiateAndUpdate},
which is just like
\mbox{\tt instantiate}\indexTT{instantiate} except that it takes an extra argument,
the address of the node to be updated with the result, and it does
not return the address of the resulting graph.
\begin{verbatim}
> instantiateAndUpdate 
>     :: CoreExpr             -- Body of supercombinator
>        -> Addr              -- Address of node to update
>        -> TiHeap            -- Heap before instantiation
>        -> ASSOC Name Addr   -- Associate parameters to addresses
>        -> TiHeap            -- Heap after instantiation
\end{verbatim}
\indexDTT{instantiateAndUpdate}%
\par
Here, for example, is the definition of \mbox{\tt instantiateAndUpdate}\indexTT{instantiateAndUpdate} in the
case
when the expression is an application:
\begin{verbatim}
> instantiateAndUpdate (EAp e1 e2) upd_addr heap env
>  = hUpdate heap2 upd_addr (NAp a1 a2)
>    where
>    (heap1, a1) = instantiate e1 heap  env
>    (heap2, a2) = instantiate e2 heap1 env
\end{verbatim}
\indexDTT{instantiateAndUpdate}%
Notice that the recursive instantiations are still performed by the old
\mbox{\tt instantiate}\indexTT{instantiate}; only the root node needs to be updated.
\begin{exercise}
Complete the definition of \mbox{\tt instantiateAndUpdate}\indexTT{instantiateAndUpdate}. The following
points need a little care:
\begin{itemize}
\item
When the expression to be instantiated is a simple variable,
you will still need to use an indirection.
Why?
\item
Think carefully about the recursive instantiations in the equations for
\mbox{\tt let(rec)}\indexTT{let(rec)} expressions.
\end{itemize}

Modify \mbox{\tt scStep}\indexTT{scStep}
to call \mbox{\tt instantiateAndUpdate}\indexTT{instantiateAndUpdate} instead of \mbox{\tt instantiate}\indexTT{instantiate}, passing
the root of the redex as the address of the node to be updated.
Remove the update code from \mbox{\tt scStep}\indexTT{scStep} itself.

Measure the effect of this modification on the number of reductions and
the number of heap nodes allocated.
\end{exercise}

\section{Mark 4: Adding arithmetic}
\label{sect:templ:primitives}
\index{template instantiation machine!Mark 4}
\index{arithmetic!in template-instantation machine}

In this section we will add arithmetic primitives.
This will involve using the dump for the first time.

\subsection{Transition rules for arithmetic}
\label{sect:templ:arithmetic}

First of all we develop the state transition rules for
arithmetic.
We begin with negation, because it is a simple unary operation.
The rules for other arithmetic operations are similar.
Here is a plausible-looking rule when the argument is evaluated:
\tirule{
        \tistate{a:a_1:[]}{d}{h\left[
                \begin{array}{l}
                a:\mbox{\tt NPrim\ Neg} \\
                a_1:\mbox{\tt NAp}\indexTT{NAp}~ a~ b \\
                b:\mbox{\tt NNum}\indexTT{NNum}~ n
                \end{array}
        \right]}{f}
}{
        \tistate{a_1:[]}{d}{h[a_1:\mbox{\tt NNum}\indexTT{NNum}~ (-n)]}{f}
}
\label{rule:negate}
Notice that the rule specifies that the stack should contain {\em only\/}
the argument to the negation operator,
because anything else would be a type error.

Suppose that the argument is not evaluated: what should happen then?
We need to evaluate the argument on a fresh stack (so that the evaluations
do not get mixed up with each other) and,
when this is complete, restore the old stack and try again.
We need a way to keep track of the old stack, so we introduce the
\stressD{dump} for the first time.
The dump is just a stack of stacks.

The \mbox{\tt Neg}\indexTT{Neg} rule to start an evaluation is like this:
\tirule{
        \tistate{a:a_1:[]}{d}{h\left[
                \begin{array}{l}
                a:\mbox{\tt NPrim\ Neg} \\
                a_1:\mbox{\tt NAp}\indexTT{NAp}~ a~ b
                \end{array}
        \right]}{f}
}{
        \tistate{b:[]}{(a:a_1:[]):d}{h}{f}
}
\label{rule:negate-eval}
This rule is used only if the previous one (which is a special case of
this one, with the node at address $b$ being an \mbox{\tt NNum}\indexTT{NNum} node) does not apply.

Once the evaluation is complete, we need a rule to restore the old
stack:
\tirule{
        \tistate{a:[]}{s:d}{h[a:\mbox{\tt NNum}\indexTT{NNum}~ n]}{f}
}{
        \tistate{s}{d}{h}{f}
}
\label{rule:num-return}
Once the old stack has been restored, the negation primitive\index{primitives}
will be found on top of the stack again, but this time the argument
will be in normal form.

But we need to take care!  The argument will indeed have been
reduced to normal form, but the root node of the argument will
have been updated, {\em so it may now
be an indirection node}.
Hence, the first rule for \mbox{\tt Neg}\indexTT{Neg} will not see the \mbox{\tt NNum}\indexTT{NNum} node directly.
(For example, consider the expression \mbox{\tt (negate\ (id\ 3))}.)

The easiest way around this is to add an extra transition rule
just before the rule which unwinds an application node (\ruleref{rule:unwind}).
In the special case where the argument of the application is
an indirection\index{indirections},
the rule updates the application with a new one whose argument points past
the indirection:
\tirule{
        \tistate{a:s}{d}{h\left[\begin{array}{l}
                                a:\mbox{\tt NAp}\indexTT{NAp}~ a_1~ a_2 \\
                                a_2: \mbox{\tt NInd}\indexTT{NInd}~ a_3
                        \end{array}\right]}{f}
}{
        \tistate{a:s}{d}{h[a:\mbox{\tt NAp}\indexTT{NAp}~ a_1~ a_3]}{f}
}
\label{rule:unwind-indarg}
In order to
bring this rule into play, we need to modify \ruleref{rule:negate-eval}
so that it unwinds anew from the root of the redex after the evaluation is
completed:
\tirule{
        \tistate{a:a_1:[]}{d}{h\left[
                \begin{array}{l}
                a:\mbox{\tt NPrim\ Neg} \\
                a_1:\mbox{\tt NAp}\indexTT{NAp}~ a~ b
                \end{array}
        \right]}{f}
}{
        \tistate{b:[]}{(a_1:[]):d}{h}{f}
}
\label{rule:negate-eval1}
This is rather tiresome; the implementations developed in
subsequent chapters will do a better job.

\begin{exercise}
Write the state transition rules for addition.
(The other dyadic arithmetic operations are practically identical.)
\end{exercise}

\subsection{Implementing arithmetic}
\label{sect:templ:primitives-impl}

To implement arithmetic we need to make a number of changes.
First, we need to redefine the type \mbox{\tt tiDump}\indexTT{tiDump} to be a stack of stacks, whose
initial value is empty.\index{dump!in template-instantiation machine}
\begin{verbatim}
> type TiDump = [TiStack]
> initialTiDump = []
\end{verbatim}
\indexDTT{type}%
\indexDTT{initialTiDump}%
Next, we need to add a new kind of heap node:
\mbox{\tt NPrim\ n\ p} represents a primitive\index{primitives} whose name is \mbox{\tt n} and
whose value is \mbox{\tt p}, where \mbox{\tt p} is of type \mbox{\tt primitive}\indexTT{primitive}.
As in the case of \mbox{\tt NSupercomb}\indexTT{NSupercomb} nodes, the name is present in the
\mbox{\tt NPrim}\indexTT{NPrim} node solely for debugging and documentation reasons.
\begin{verbatim}
> data Node = NAp Addr Addr                       -- Application
>             | NSupercomb Name [Name] CoreExpr   -- Supercombinator
>             | NNum Int                          -- Number
>             | NInd Addr                         -- Indirection
>             | NPrim Name Primitive              -- Primitive
\end{verbatim}
\indexDTT{data}%
\indexDTT{NSupercomb}%
\indexDTT{NNum}%
\indexDTT{NInd}%
\indexDTT{NPrim}%
As usual, \mbox{\tt showNode}\indexTT{showNode} needs to be augmented as well, to display \mbox{\tt NPrim}\indexTT{NPrim} nodes.
The transition rules given in the previous section suggest that
the data type \mbox{\tt primitive}\indexTT{primitive} should be defined like this:
\begin{verbatim}
> data Primitive = Neg | Add | Sub | Mul | Div
\end{verbatim}
\indexDTT{data}%
\indexDTT{Add}%
\indexDTT{Sub}%
\indexDTT{Mul}%
\indexDTT{Div}%
with one constructor for each desired primitive.

Now, just as we needed to allocate an \mbox{\tt NSupercomb}\indexTT{NSupercomb} node in the
initial
heap for each supercombinator, so we need to allocate an \mbox{\tt NPrim}\indexTT{NPrim} node in
the initial heap for each primitive.  Then we can add extra bindings
to the
$globals$ component of the machine state,
which
map the name of each primitive to the address of its node, just as we
did for supercombinators.  We can do this easily by modifying the definition
of \mbox{\tt buildInitialHeap}\indexTT{buildInitialHeap}, like this:
\begin{verbatim}
> buildInitialHeap :: [CoreScDefn] -> (TiHeap, TiGlobals)
> buildInitialHeap sc_defs
>  = (heap2, sc_addrs ++ prim_addrs)
>    where
>    (heap1, sc_addrs)   = mapAccuml allocateSc hInitial sc_defs
>    (heap2, prim_addrs) = mapAccuml allocatePrim heap1 primitives
\end{verbatim}
\indexDTT{buildInitialHeap}%
\indexDTT{buildInitialHeap}%
\par
We define an association
list giving the mapping from variable names to primitives, thus:
\begin{verbatim}
> primitives :: ASSOC Name Primitive
> primitives = [ ("negate", Neg),
>                ("+", Add),   ("-", Sub),
>                ("*", Mul),   ("/", Div)
>              ]
\end{verbatim}
\indexDTT{primitives}%
\indexDTT{primitives}%
To add further primitives, just add more constructors to the
\mbox{\tt primitive}\indexTT{primitive} type, and more elements to the \mbox{\tt primitives}\indexTT{primitives} association list.

We can then define \mbox{\tt allocatePrim}\indexTT{allocatePrim}, very much as we defined \mbox{\tt allocateSc}\indexTT{allocateSc}:
\begin{verbatim}
> allocatePrim :: TiHeap -> (Name, Primitive) -> (TiHeap, (Name, Addr))
> allocatePrim heap (name, prim)
>  = (heap', (name, addr))
>    where
>    (heap', addr) = hAlloc heap (NPrim name prim)
\end{verbatim}
\indexDTT{allocatePrim}%
\indexDTT{allocatePrim}%
\par
Next, we need to augment the \mbox{\tt dispatch}\indexTT{dispatch} function in \mbox{\tt step}\indexTT{step} to
call \mbox{\tt primStep}\indexTT{primStep} when it finds a \mbox{\tt NPrim}\indexTT{NPrim} node.
\mbox{\tt primStep}\indexTT{primStep} performs case analysis on the primitive being used,
and then calls one of a family of
auxiliary functions, \mbox{\tt primNeg}\indexTT{primNeg}, \mbox{\tt primAdd}\indexTT{primAdd} and so on, which actually
perform the operation.  For the present we content ourselves with
negation.
\begin{verbatim}
> primStep state Neg   = primNeg state
\end{verbatim}
\indexDTT{primStep}%
\mbox{\tt primNeg}\indexTT{primNeg} needs to do the following:
\begin{itemize}
\item
Use \mbox{\tt getArgs}\indexTT{getArgs} to extract
the address of the argument from the stack, and \mbox{\tt hLookup}\indexTT{hLookup} to get the node
pointed to by this address.
\item
Use the auxiliary function \mbox{\tt isDataNode}\indexTT{isDataNode} to
check if the argument node is evaluated.

\item
If it is not evaluated, use \ruleref{rule:negate-eval1} to set up the new
state ready to evaluate the argument.
This involves pushing the current stack on the dump, and making a new stack
whose only element is the argument to \mbox{\tt negate}\indexTT{negate}.

\item
If it is evaluated, use \mbox{\tt hUpdate}\indexTT{hUpdate} to overwrite
the root of the redex with an \mbox{\tt NNum}\indexTT{NNum} node
containing the result, and return, having modified the stack appropriately.
\end{itemize}

Next, we need to implement the new rules for unwinding and
for numbers.
The definition of \mbox{\tt numStep}\indexTT{numStep} must be changed
to implement \ruleref{rule:num-return}.
If the stack contains
just one item, the address of an \mbox{\tt NNum\ node}, and the dump is non-empty,
\mbox{\tt numStep}\indexTT{numStep} should pop the top element of the dump and make it into the new stack.
If these conditions do not apply, it should signal an error.
Similarly, the definition of \mbox{\tt apStep}\indexTT{apStep} must be changed to implement
\ruleref{rule:unwind-indarg}.
It can do this by checking for an indirection\index{indirections}
in the argument, and using
\mbox{\tt hUpdate}\indexTT{hUpdate} to update the heap if so.

Lastly, we need to make a change to \mbox{\tt tiFinal}\indexTT{tiFinal}.  At present it halts execution
when the stack contains a single \mbox{\tt NNum}\indexTT{NNum}; {\em but it must now only do this
if the dump is empty}, otherwise the new \ruleref{rule:num-return} will never
get a chance to execute!

\begin{exercise}
Implement all these changes to add negation,
and test some programs involving negation.
For example,
\begin{verbatim}
        main = negate 3
\end{verbatim}
or
\begin{verbatim}
        main = twice negate 3
\end{verbatim}
You should also test the following program, to show that the handling of
indirections is working:
\begin{verbatim}
        main = negate (I 3)
\end{verbatim}
\end{exercise}

The obvious extension now is to implement addition,
subtraction and the other arithmetic primitives.
If we rush ahead blindly
we will find that all these dyadic arithmetic primitives have a rather
stereotyped form; indeed they are identical except for the fact that at
one point we use \mbox{\tt *} or \mbox{\tt /} rather than \mbox{\tt +}.

To avoid this duplication, we can instead define a single generic function
\mbox{\tt primArith}\indexTT{primArith} and pass to it the required operation
as an argument, thus:
\begin{verbatim}
> primStep state Add = primArith state (+)
> primStep state Sub = primArith state (-)
> primStep state Mul = primArith state (*)
> primStep state Div = primArith state (div)
\end{verbatim}
\indexDTT{primStep}%
\indexDTT{primStep}%
\indexDTT{primStep}%
\indexDTT{primStep}%
\begin{verbatim}
> primArith :: TiState -> (Int -> Int -> Int) -> TiState
\end{verbatim}
\indexDTT{primArith}%
This is a simple example of the way in which higher-order functions
\index{higher-order function} can
enable us to make programs more modular.

\begin{exercise}
Implement \mbox{\tt primArith}\indexTT{primArith}, and test your implementation.
\end{exercise}


\section{Mark 5: Structured data}
\label{sect:templ:tidata}
\index{template instantiation machine!Mark 5}
\index{data structures!in template-instantiation machine}

In this section we will add structured
data types to our reduction machine.
It would be nice to give an implementation for the \mbox{\tt case}\indexTT{case} expressions
of our core language, but it turns out that it is rather hard to do
so within the framework of a template instantiation machine.
(Our later implementations will not have this problem.)
Instead we will use a collection of built-in
functions, such as \mbox{\tt if}, \mbox{\tt casePair}\indexTT{casePair} and
\mbox{\tt caseList}\indexTT{caseList}, which allow us to manipulate certain structured types.  The
template machine will remain unable to handle general structured objects.

\begin{exercise}
Why is it hard to introduce \mbox{\tt case}\indexTT{case} expressions into the template instantiation
machine?  (Hint: think about what \mbox{\tt instantiate}\indexTT{instantiate} would do with a
\mbox{\tt case}\indexTT{case} expression.)
\end{exercise}

\subsection{Building structured data}

Structured data is built with the family of constructors $\mbox{\tt Pack{\char'173}}t,a\mbox{\tt {\char'175}}\indexTT{Pack}$
where $t$ gives the tag\index{tag!of constructor} of the constructor, and $a$
gives its arity\index{arity!of constructor}
(Section ~\ref{sect:lang:constructors}),
so we need a representation
for these constructor functions in the graph.
They are really a new form of primitive, so we can do this by adding
a new constructor \mbox{\tt PrimConstr}\indexTT{PrimConstr} to the \mbox{\tt primitive}\indexTT{primitive} type.
Now in the equation for \mbox{\tt instantiateConstr}\indexTT{instantiateConstr}, we can instantiate an
expression $\mbox{\tt EConstr}\indexTT{EConstr}~ t~ a$ to the heap node
$\mbox{\tt NPrim\ "Pack"}~ (\mbox{\tt PrimConstr}\indexTT{PrimConstr}~ t~ a)$.

Next the question arises of how this primitive is implemented.  We need
to add a case to \mbox{\tt primStep}\indexTT{primStep} to match the \mbox{\tt PrimConstr}\indexTT{PrimConstr} constructor,
which calls a new auxiliary function \mbox{\tt primConstr}\indexTT{primConstr}.
This should check that it is given enough arguments, and if so build
a structured data object in the heap.

To do this we need to add a new constructor, \mbox{\tt NData}\indexTT{NData}, to the
\mbox{\tt node}\indexTT{node} type to represent structured data objects.
The \mbox{\tt NData}\indexTT{NData} constructor contains the
{\em tag\/}\index{tag!of constructor} of the object, and its
{\em components}\index{components!of constructor}.
\begin{verbatim}
> data Node = NAp Addr Addr                     -- Application
>           | NSupercomb Name [Name] CoreExpr   -- Supercombinator
>           | NNum Int                          -- Number
>           | NInd Addr                         -- Indirection
>           | NPrim Name Primitive              -- Primitive
>           | NData Int [Addr]                  -- Tag, list of components
\end{verbatim}
\indexDTT{data}%
\indexDTT{NSupercomb}%
\indexDTT{NNum}%
\indexDTT{NInd}%
\indexDTT{NPrim}%
\indexDTT{NData}%
\par
We can now give the rule for $\mbox{\tt NPrim}\indexTT{NPrim}~ (\mbox{\tt PrimConstr}\indexTT{PrimConstr}~ t~ n)$:
\tirule{
        \tistate{a:a_1:\ldots:a_n:[]}{d}{h\left[
                \begin{array}{l}
                a:\mbox{\tt NPrim\ (PrimConstr\ }t~ n\mbox{\tt )} \\
                a_1:\mbox{\tt NAp}\indexTT{NAp}~ a~ b_1 \\
                \ldots \\
                a_n:\mbox{\tt NAp}\indexTT{NAp}~ a_{n-1}~ b_n
                \end{array}
        \right]}{f}
}{
        \tistate{a_n:[]}{d}{h[a_n:\mbox{\tt NData}\indexTT{NData}~ t~ [b_1,\ldots,b_n]]}{f}
}
\label{rule:prim-constr}
So much for building structured objects.  The next question is how to take them
apart, which is expressed by \mbox{\tt case}\indexTT{case} expressions in the Core language.
As already mentioned, it is hard to implement \mbox{\tt case}\indexTT{case} expressions directly,
so we content ourselves with a few special cases, beginning with booleans.

\subsection{Conditionals\index{conditional!in template-instantation machine}}
\label{sect:templ:conditional}

The boolean type\index{booleans}
might be declared in Miranda like this:
\begin{verbatim}
        boolean ::= False | True
\end{verbatim}
There are two constructors, \mbox{\tt True}\indexTT{True} and \mbox{\tt False}\indexTT{False}.  Each
has arity zero, and we arbitrarily assign a tag of one to \mbox{\tt False}\indexTT{False} and two
to \mbox{\tt True}\indexTT{True}.  So we can give the following Core-language definitions:
\begin{verbatim}
        False = Pack{1,0}
        True = Pack{2,0}
\end{verbatim}

Since we cannot have general \mbox{\tt case}\indexTT{case} expressions, it will suffice to add
a conditional primitive,
with the reduction rules:
\begin{verbatim}
        if Pack{2,0} t e = t
        if Pack{1,0} t e = e
\end{verbatim}
Operationally, \mbox{\tt if} evaluates its first argument, which it expects to be a
data object, examines its tag, and selects either its second or third argument
depending on whether the tag is 2 (\mbox{\tt True}\indexTT{True}) or 1 (\mbox{\tt False}\indexTT{False}) respectively.

\begin{exercise}
Write the state transition rules for the conditional\index{conditional} primitive.
You need three rules: two to perform the reduction if the boolean
condition is already evaluated; and one to start evaluation if the
condition is not evaluated, by pushing the old
stack on the dump, and pushing the
address of the condition on the new empty stack (cf. \ruleref{rule:negate-eval1}).
You should find that you need to use an indirection in the update for
the first two rules.

One further rule is missing.  What is it?  (Hint: when evaluation
of the condition is complete, how does the conditional get re-tried?)
\end{exercise}

Once you have \mbox{\tt if}, you can give Core-language definitions for
the other boolean operators in terms of
it and \mbox{\tt False}\indexTT{False} and \mbox{\tt True}\indexTT{True}.  For example:
\begin{verbatim}
        and x y = if x y False
\end{verbatim}

\begin{exercise}
\label{ex:templ:prelude-bool}
Give Core-language definitions for \mbox{\tt or}, \mbox{\tt xor}\indexTT{xor} and \mbox{\tt not}\indexTT{not}.
Add all of these Core-language definitions to \mbox{\tt extraPreludeDefs}\indexTT{extraPreludeDefs}.
\end{exercise}

Finally, we need some way of comparing numeric values, which requires
new primitives \mbox{\tt >}, \mbox{\tt >=} and so on.

\subsection{Implementing structured data}

Here is a list of the changes required to the implementation to add
structured data objects, conditionals and comparison operations.
\begin{itemize}
\item
Add the \mbox{\tt NData}\indexTT{NData} constructor to the \mbox{\tt node}\indexTT{node} data type.  Extend \mbox{\tt showNode}\indexTT{showNode} to
display \mbox{\tt NData}\indexTT{NData} nodes.
\item
Add \mbox{\tt PrimConstr}\indexTT{PrimConstr}, \mbox{\tt If}, \mbox{\tt Greater}\indexTT{Greater}, \mbox{\tt GreaterEq}\indexTT{GreaterEq}, \mbox{\tt Less}\indexTT{Less}, \mbox{\tt LessEq}\indexTT{LessEq}, \mbox{\tt Eq}, \mbox{\tt NotEq}\indexTT{NotEq}
to the \mbox{\tt primitive}\indexTT{primitive} type.
For all except the first, add suitable pairs to the \mbox{\tt primitives}\indexTT{primitives} association
list, so that the names of these primitives can be mapped to their
values by \mbox{\tt instantiateVar}\indexTT{instantiateVar}.
\item
Add a definition for \mbox{\tt instantiateConstr}\indexTT{instantiateConstr} (and \mbox{\tt instantiateAndUpdateConstr}\indexTT{instantiateAndUpdateConstr}
if necessary).
\item
The \mbox{\tt isDataNode}\indexTT{isDataNode} function should identify \mbox{\tt NData}\indexTT{NData} nodes as well as \mbox{\tt NNum}\indexTT{NNum} nodes.
\item
The \mbox{\tt dispatch}\indexTT{dispatch} code in \mbox{\tt step}\indexTT{step} needs an extra case for \mbox{\tt NData}\indexTT{NData} nodes, calling
a new auxiliary function \mbox{\tt dataStep}\indexTT{dataStep}.
\item
Define \mbox{\tt dataStep}\indexTT{dataStep}; it is very similar to \mbox{\tt numStep}\indexTT{numStep}.
\item
Extend \mbox{\tt primStep}\indexTT{primStep} to cope with the new primitives \mbox{\tt PrimConstr}\indexTT{PrimConstr}, \mbox{\tt If}, \mbox{\tt Greater}\indexTT{Greater}
and so on.
For \mbox{\tt PrimConstr}\indexTT{PrimConstr} and \mbox{\tt If} it should call new auxiliary functions \mbox{\tt primConstr}\indexTT{primConstr}
and \mbox{\tt primIf}\indexTT{primIf}.
The comparison primitives can almost, but not quite, use \mbox{\tt primArith}\indexTT{primArith}.
What we need is a slight generalisation of \mbox{\tt primArith}\indexTT{primArith}:
\begin{verbatim}
> primDyadic :: TiState -> (Node -> Node -> Node) -> TiState
\end{verbatim}
\indexDTT{primDyadic}%
which takes a node-combining function instead of a number-combining one.
It is simple to define \mbox{\tt primArith}\indexTT{primArith}, and a similar function \mbox{\tt primComp}\indexTT{primComp} for
comparison primitives, in terms of \mbox{\tt primDyadic}\indexTT{primDyadic}; and to
define \mbox{\tt primDyadic}\indexTT{primDyadic} by generalising the definition of \mbox{\tt primArith}\indexTT{primArith}.
\end{itemize}

\begin{exercise}
Make all these changes.
Now, at last, we can write sensible recursive functions, because we
have a conditional to terminate the recursion.
Try, for example, the factorial function
\begin{verbatim}
        fac n = if (n == 0) 1 (n * fac (n-1)) ;
        main = fac 3
\end{verbatim}
\end{exercise}

\subsection{Pairs}

The constructors for booleans both have arity zero.  Next, we will
add the data type of pairs\index{pairs}, which might be declared in Miranda like this:
\begin{verbatim}
        pair * ** ::= MkPair * **
\end{verbatim}

We can build pairs using the \mbox{\tt Pack{\char'173}1,2{\char'175}}\indexTT{Pack} constructor:
\begin{verbatim}
        MkPair = Pack{1,2}
\end{verbatim}
How about taking them apart, still without using \mbox{\tt case}\indexTT{case} expressions?
For example, consider the following Core-language program involving
a \mbox{\tt case}\indexTT{case} expression:
\begin{verbatim}
        f p = case p of
                <1> a b -> b*a*a end
\end{verbatim}
Lacking \mbox{\tt case}\indexTT{case} expressions, we can translate it instead as follows:
\begin{verbatim}
        f p = casePair p f'
        f' a b = b*a*a
\end{verbatim}
Here, \mbox{\tt f'} is an auxiliary function, and \mbox{\tt casePair}\indexTT{casePair} is a built-in primitive
defined like this:
\begin{verbatim}
        casePair (Pack{1,2} a b) f = f a b
\end{verbatim}

Operationally, \mbox{\tt casePair}\indexTT{casePair} evaluates its first argument, which it expects to
yield a pair; it then applies its second argument to the two components of
the pair.
You can implement this by adding yet another constructor \mbox{\tt PrimCasePair}\indexTT{PrimCasePair} to
the \mbox{\tt primitive}\indexTT{primitive} type, and writing some more code to handle it.

We can, for example, define \mbox{\tt fst}\indexTT{fst} and \mbox{\tt snd}\indexTT{snd} which extract the
first and second components of a pair, with the following Core-language
definitions:
\begin{verbatim}
        fst p = casePair p K
        snd p = casePair p K1
\end{verbatim}

\begin{exercise}
Write the state transition rules for \mbox{\tt casePair}\indexTT{casePair}.  As usual, you will need
two rules: one to perform the reduction if the first argument is evaluated,
and one to start its evaluation if not.

Make the necessary changes to implement pairs, as described above.

Test your implementation with the following program (and others of your own):
\begin{verbatim}
        main = fst (snd (fst (MkPair (MkPair 1 (MkPair 2 3)) 4)))
\end{verbatim}
\end{exercise}

\subsection{Lists}

Now that you have done pairs and booleans, lists should be easy.
The list\index{lists} data type might be defined in Miranda like this:
\begin{verbatim}
        list * ::= Nil | Cons * (list *)
\end{verbatim}
We assign the tag 1 to \mbox{\tt Nil}\indexTT{Nil} and 2 to \mbox{\tt Cons}\indexTT{Cons}.

The only question is what exactly
the \mbox{\tt caseList}\indexTT{caseList} primitive, which takes a list apart, should do.
We recall that the \mbox{\tt casePair}\indexTT{casePair} has one `continuation', a function which
takes the components
of the pair as its arguments.  \mbox{\tt if} has two `continuations', and selects one
or other depending on the value of its first argument.
So \mbox{\tt caseList}\indexTT{caseList} is just a combination of both these ideas:
\begin{verbatim}
        caseList Pack{1,0}        cn cc = cn
        caseList (Pack{2,2} x xs) cn cc = cc x xs
\end{verbatim}
It takes three arguments, and evaluates the first one.  If it is an empty
list (i.e.\ has a tag
of 1 and no components) then \mbox{\tt caseList}\indexTT{caseList} simply selects its second argument \mbox{\tt cn}.
Otherwise it must be a list cell (i.e.\ a tag of 2 and two components), and
\mbox{\tt caseList}\indexTT{caseList} applies its third argument \mbox{\tt cc} to these components.

For example, suppose we wanted to implement the \mbox{\tt length}\indexTT{length} function, which in
Miranda would be written
\begin{verbatim}
        length [] = 0
        length (x:xs) = 1 + length xs
\end{verbatim}
With the aid of \mbox{\tt caseList}\indexTT{caseList} we could write length like this:
\begin{verbatim}
        length xs = caseList xs 0 length'
        length' x xs = 1 + length xs
\end{verbatim}

\begin{exercise}
Write Core-language definitions for
\mbox{\tt Cons}\indexTT{Cons}, \mbox{\tt Nil}\indexTT{Nil}, \mbox{\tt head}\indexTT{head} and
\mbox{\tt tail}\indexTT{tail}. To define \mbox{\tt head}\indexTT{head} and \mbox{\tt tail}\indexTT{tail},
you will need to introduce a new
primitive \mbox{\tt abort}\indexTT{abort}, which is returned if you take the \mbox{\tt head}\indexTT{head} or \mbox{\tt tail}\indexTT{tail} of
an empty list.
\mbox{\tt abort}\indexTT{abort} can conveniently be implemented by
calling Miranda's \mbox{\tt error}\indexTT{error} primitive to stop the program.
\end{exercise}

\begin{exercise}
Write the state transition rules for \mbox{\tt caseList}\indexTT{caseList},
implement it and \mbox{\tt abort}\indexTT{abort},
and add definitions to \mbox{\tt preludeDefs}\indexTT{preludeDefs} for \mbox{\tt Cons}\indexTT{Cons}, \mbox{\tt Nil}\indexTT{Nil}, \mbox{\tt head}\indexTT{head} and
\mbox{\tt tail}\indexTT{tail}.

Write some programs to test your implementation.
\end{exercise}

You should now be able to write a suitable \mbox{\tt case}\indexTT{case} primitive for any
structured data type you care to think of.
\begin{exercise}
What is the main disadvantage of taking apart structured data types
with \mbox{\tt case}\indexTT{case} primitives,
rather than implementing full \mbox{\tt case}\indexTT{case} expressions?
\end{exercise}

\subsection{Printing\index{printing} lists}
\label{sect:templ:print-list}

So far we have been implicitly assuming that the result of the
whole program is a number.
What would we have to do to allow a {\em list\/} of numbers to be the result?
If this was the case, after evaluating \mbox{\tt main}\indexTT{main} for a while,
we would eventually expect to find the address of a list object
on top of the stack.
If it is an empty list, the program terminates.
If it is a \mbox{\tt Cons}\indexTT{Cons} cell, and
its head is not evaluated we need to begin a recursive evaluation
of its head; if the head is evaluated we need to print the head, and then
repeat the whole exercise on the tail.

As soon as we start to write state transition rules to describe this, we
have to decide how to express the idea of `printing a number' in our state
transition world.  The neatest solution is to add a new component
to the state, called the {\em output}, and model `printing a number'
by `appending a number to the output'.  We will also add two
new primitives, \mbox{\tt Print}\indexTT{Print} and \mbox{\tt Stop}\indexTT{Stop}.

The \mbox{\tt Stop}\indexTT{Stop} primitive is easy: it makes the stack empty.
(\mbox{\tt tiFinal}\indexTT{tiFinal} will be altered to stop the machine when it sees an empty stack,
rather than giving an error, which is what it does now.)
\mbox{\tt Stop}\indexTT{Stop} expects the dump to be empty.
\tiruleO{
        \tistateO{o}{a:[]}{[]}{h[a:\mbox{\tt NPrim\ Stop}]}{f}
}{
        \tistateO{o}{[]}{[]}{h}{f}
}
\label{rule:stop}

The \mbox{\tt Print}\indexTT{Print} primitive evaluates its first argument to an integer,
and attaches its value
to the output list; then it returns its second argument as its result.
It also expects the dump to be empty.
The first rule applies if the first argument is already evaluated:
\tiruleO{
        \tistateO{o}{a:a_1:a_2:[]}{[]}
                {h\left[\begin{array}{l}
                        a:\mbox{\tt NPrim\ Print} \\
                        a_1:\mbox{\tt NAp}\indexTT{NAp}~a~b_1 \\
                        a_2:\mbox{\tt NAp}\indexTT{NAp}~a_1~b_2 \\
                        b_1:\mbox{\tt NNum}\indexTT{NNum}~n
                \end{array}\right]}{f}
}{
        \tistateO{o \plusplus [n]}{b_2:[]}{[]}{h}{f}
}
\label{rule:print}

\mbox{\tt Print}\indexTT{Print} is a rather weird supercombinator,
because it has a side-effect
on the output $o$.  \mbox{\tt Print}\indexTT{Print} must obviously be used with care!
The second rule applies if \mbox{\tt Print}\indexTT{Print}'s first argument is not evaluated: it
starts an evaluation in the usual way.
\tiruleO{
        \tistateO{o}{a:a_1:a_2:[]}{[]}
                {h\left[\begin{array}{l}
                        a:\mbox{\tt NPrim\ Print} \\
                        a_1:\mbox{\tt NAp}\indexTT{NAp}~a~b_1 \\
                        a_2:\mbox{\tt NAp}\indexTT{NAp}~a_1~b_2 \\
                \end{array}\right]}{f}
}{
        \tistateO{o}{b_1:[]}{(a_2:[]):[]}{h}{f}
}
\label{rule:print-eval}

Now, we define the following extra functions in \mbox{\tt extraPreludeDefs}\indexTT{extraPreludeDefs}:
\begin{verbatim}
        printList xs = caseList xs stop printCons
        printCons h t = print h (printList t)
\end{verbatim}
where \mbox{\tt "print"} is bound by \mbox{\tt primitives}\indexTT{primitives} to the \mbox{\tt Print}\indexTT{Print} primitive,
and \mbox{\tt "stop"} is bound to \mbox{\tt Stop}\indexTT{Stop}.

Finally, we modify the \mbox{\tt compile}\indexTT{compile} function so that the stack initially contains
the address of the expression \mbox{\tt (printList\ main)}.
It should not take long to convince yourself that this does the right
thing.

\begin{exercise}
Implement these changes, and test your implementation by writing a program
which returns a list of numbers.
\end{exercise}

\section{Alternative implementations\advanced}

These exercises explore some alternative implementations for things we have
done.

\subsection{An alternative representation for primitives}
\label{sect:ti:alternative-data}

Since {\em the only thing we ever
do to a primitive is execute it}, we
can play the following trick: instead of making \mbox{\tt primitive}\indexTT{primitive} be
an enumeration type on which we perform case analysis, we could
make it a {\em function\/} which takes a \mbox{\tt tiState}\indexTT{tiState} to a \mbox{\tt tiState}\indexTT{tiState}, like this:
\begin{verbatim}
>  Type Primitive = TiState -> TiState
\end{verbatim}
The constructors \mbox{\tt Add}\indexTT{Add}, \mbox{\tt Sub}\indexTT{Sub} and so on have vanished altogether.
Now the `case analysis' done by \mbox{\tt primStep}\indexTT{primStep} is rather easy:
just apply the function!
Here are the revised definitions for \mbox{\tt primStep}\indexTT{primStep} and \mbox{\tt primitives}\indexTT{primitives}.
\begin{verbatim}
> primStep state prim = prim state
> primitives = [ ("negate", primNeg),
>               ("+", primArith (+)), ("-", primArith (-)),
>               ("*", primArith (*)), ("/", primArith (/))
>              ]
\end{verbatim}
\indexDTT{primStep}%
\indexDTT{primitives}%
This has a direct counterpart in real implementations: instead of
storing a small integer tag in a \mbox{\tt NPrim}\indexTT{NPrim} node to distinguish among
primitives, we store a code pointer, and jump to it to execute the
primitive.

\begin{exercise}
Implement and test this change.
\end{exercise}

\subsection{An alternative representation of the dump\index{dump!alternative representation}}

At present we have implemented the dump as a stack of stacks, but in
a real implementation we would doubtless build the new stack directly
on top of the old one.  The dump would then contain offsets from the
base of the
spine stack, telling where one sub-stack ends and the next begins.

We can model this directly in our machine with the following type
declaration:
\begin{verbatim}
> type TiDump = Stack Num
\end{verbatim}
\indexDTT{type}%
\begin{exercise}
Implement and test this change.  You will need to modify the equations
that deal with beginning and ending an evaluation of an argument,
and the definition of \mbox{\tt tiFinal}\indexTT{tiFinal}.
\end{exercise}

\subsection{An alternative representation for data values}
\label{sect:template:data-str-hof}
\index{data structures!as higher-order functions}

There is another way to implement booleans\index{booleans!as higher-order functions}
which is quite instructive.
The reduction rules given in Section~\ref{sect:templ:conditional}
for \mbox{\tt if} simply select one or other of
the second or third arguments.
Now, suppose that instead of representing a boolean value as a structured
data object, we represented it as a {\em function,
which selects one or other of its arguments}.  That is, \mbox{\tt True}\indexTT{True} and
\mbox{\tt False}\indexTT{False} are redefined like this:
\begin{verbatim}
        True  t f = t
        False t f = f
\end{verbatim}
Now boolean operators can be defined like this:
\begin{verbatim}
        if = I
        and b1 b2 t f = b1 (b2 t f) f
        or b1 b2 t f = b1 t (b2 t f)
        not b t f = b f t
\end{verbatim}
These definitions can all be included in \mbox{\tt extraPreludeDefs}\indexTT{extraPreludeDefs}.
Now the only primitives required are the arithmetic comparison operators!
There is no need to define a primitive \mbox{\tt if}, or
to add \mbox{\tt NData}\indexTT{NData} to the \mbox{\tt node}\indexTT{node} type.

We can apply exactly the same trick for pairs\index{pairs}.  A pair is represented
as a {\em function\/} which takes a single argument and applies it to the
two components of the pair:
\begin{verbatim}
        pair a b f = f a b
        casePair = I
        fst p = p K
        snd p = p K1
\end{verbatim}
The same trick works for lists\index{lists}, but now we need two `extra' arguments,
one to use if the list is a \mbox{\tt Cons}\indexTT{Cons} cell, and the other to use if it is empty:
\begin{verbatim}
        cons a b cn cc = cc a b
        nil      cn cc = cn
        caseList = I
\end{verbatim}

\begin{exercise}
Implement booleans, pairs and lists in this way, and measure
their performance.
What advantages and disadvantages
can you see relative to the previous implementation?
\end{exercise}

\section{Garbage collection\index{garbage collection}\advanced}

As execution proceeds, more and more nodes will be allocated in the
heap, so the Miranda data structure representing the heap will become
larger and larger.  Eventually, Miranda will run out of space.  This
comes as no surprise, because it corresponds directly to real
implementations.  As nodes are allocated, the heap becomes larger and
larger, and eventually fills up.  We need to perform {\em garbage
collection\/} to free up some space.

More specifically, we need to define a function \mbox{\tt gc}, with type


whose result state behaves exactly like its input state, except that
it has a (hopefully) smaller heap.  This smaller heap contains all the
nodes which are accessible from the other components of the machine
state, directly or indirectly.  \mbox{\tt gc} makes the heap smaller by
calling \mbox{\tt hFree}\indexTT{hFree} on the addresses of nodes which are no longer required
(see Appendix~\ref{sect:heap} for a description of \mbox{\tt hFree}\indexTT{hFree}).

The \mbox{\tt doAdmin}\indexTT{doAdmin} function can check the
heap size (using \mbox{\tt hSize}\indexTT{hSize}) after each step, and call the garbage collector
if it is larger than some given size.

\subsection{Mark-scan collection}\index{mark-scan collection}

To begin with, we will develop a {\em mark-scan collector}.
This works in three phases:
\begin{enumerate}
\item
The first phase identifies all the {\em roots\/}; that is,
all the heap addresses contained in the machine state.  Where can such addresses
be lurking?  We can easily find out by looking at the types involved in
the machine state for occurrences of \mbox{\tt addr}\indexTT{addr}.  The answer is that addresses
can occur in the stack, the dump and the globals.  So we need the following
functions:
\begin{verbatim}
> findStackRoots  :: TiStack -> [Addr]
> findDumpRoots   :: TiDump -> [Addr]
> findGlobalRoots :: TiGlobals -> [Addr]
\end{verbatim}
\indexDTT{findStackRoots}%
\indexDTT{findDumpRoots}%
\indexDTT{findGlobalRoots}%
\item
In the {\em mark phase}, each node whose address is in the machine
state is {\em marked}.  When a node is marked, all its descendants
are also marked, and so on recursively.
The \mbox{\tt markFrom}\indexTT{markFrom} function takes a heap and an address, and returns a new
heap in which all the nodes accessible from the address have been marked.
\begin{verbatim}
> markFrom :: TiHeap -> Addr -> TiHeap
\end{verbatim}
\indexDTT{markFrom}%
\item
In the {\em scan phase}, all the nodes in the heap (whether marked or not)
are examined.  Unmarked nodes are freed, and marked nodes are unmarked.
\begin{verbatim}
> scanHeap :: TiHeap -> TiHeap
\end{verbatim}
\indexDTT{scanHeap}%
\end{enumerate}
\begin{exercise}
Write a definition for \mbox{\tt gc} in terms of \mbox{\tt findRoots}\indexTT{findRoots}, \mbox{\tt markFrom}\indexTT{markFrom} and \mbox{\tt scanHeap}\indexTT{scanHeap},
and call it appropriately from \mbox{\tt doAdmin}\indexTT{doAdmin}.
\end{exercise}
\begin{exercise}
Write a definition for \mbox{\tt findRoots}\indexTT{findRoots}.
\end{exercise}

Before we can implement \mbox{\tt markFrom}\indexTT{markFrom} and \mbox{\tt scanHeap}\indexTT{scanHeap} we need to
have a way to {\em mark\/} a node.
In a real implementation this
is done by using a bit in the node to indicate
whether or not the node is marked.
We will model this by adding a new constructor, the \mbox{\tt node}\indexTT{node} type, as follows:
\begin{verbatim}
> data Node = NAp Addr Addr                     -- Application
>           | NSupercomb Name [Name] CoreExpr   -- Supercombinator
>           | NNum Int                          -- Number
>           | NInd Addr                         -- Indirection
>           | NPrim Name Primitive              -- Primitive
>           | NData Int [Addr]                  -- Tag, list of components
>           | NMarked Node                      -- Marked node
\end{verbatim}
\indexDTT{data}%
\indexDTT{NSupercomb}%
\indexDTT{NNum}%
\indexDTT{NInd}%
\indexDTT{NPrim}%
\indexDTT{NData}%
\indexDTT{NMarked}%
\par
The new kind of node is an \mbox{\tt NMarked}\indexTT{NMarked} node, and it contains inside it the \mbox{\tt node}\indexTT{node}
which was there before the marking happened.
The \mbox{\tt node}\indexTT{node} inside an \mbox{\tt NMarked}\indexTT{NMarked} node
is never another \mbox{\tt NMarked}\indexTT{NMarked} node.

Now we are ready to define \mbox{\tt markFrom}\indexTT{markFrom}.
Given an address $a$ and a heap $h$, it does the following:
\begin{enumerate}
\item
It looks up $a$ in $h$, giving a node $n$.  If it is already marked, \mbox{\tt markFrom}\indexTT{markFrom}
returns immediately.  This is what prevents the marking process from going
on forever when it encounters a cyclic structure in the heap.
\item
It marks the node by using \mbox{\tt hUpdate}\indexTT{hUpdate} to replace it with $\mbox{\tt NMarked}\indexTT{NMarked}~n$.
\item
It extracts any addresses from inside $n$ (there may be zero or more
such addresses), and calls \mbox{\tt markFrom}\indexTT{markFrom} on each of them.
\end{enumerate}
All that remains is \mbox{\tt scanHeap}\indexTT{scanHeap}.   It uses \mbox{\tt hAddresses}\indexTT{hAddresses} to extract the list of
all the addresses used in the heap, and examines each in turn.  If the node
to which it refers is unmarked (that is, not an \mbox{\tt NMarked}\indexTT{NMarked} node), it calls
\mbox{\tt hFree}\indexTT{hFree} to free the node.  Otherwise, it unmarks the node by
using \mbox{\tt hUpdate}\indexTT{hUpdate} to replace it with the node found inside the
\mbox{\tt NMarked}\indexTT{NMarked} constructor.
\begin{exercise}
Write definitions for \mbox{\tt markFrom}\indexTT{markFrom} and \mbox{\tt scanHeap}\indexTT{scanHeap}.
\end{exercise}

That completes the mark-scan garbage collector.

Mark-scan is not the only way to perform garbage collection, and
we now suggest some directions for further exploration.
A brief survey of garbage-collection techniques can be found in
Chapter 17 of \cite{PJBook}; a more comprehensive review is \cite{Cohen}.

\subsection{Eliminating indirections}

\begin{figure} %\centering
\input{gc-ind.tex}
\caption{Eliminating indirections during garbage collection}
\label{fig:gc-ind}
\end{figure}
We begin with an optimisation to the collector we have just developed.
During evaluation we may introduce indirection nodes, and it would be nice
to eliminate them, by readjusting pointers as suggested in
Figure~\ref{fig:gc-ind}.
To do this, we need to change the functionality of \mbox{\tt markFrom}\indexTT{markFrom} a bit.
It should now take an address and a heap, mark all the nodes accessible
from the address, and return a new heap {\em together with a new address
which should be used instead of the old one}.
\begin{verbatim}
> markFrom :: TiHeap -> Addr -> (TiHeap, Addr)
\end{verbatim}
\indexDTT{markFrom}%
In the picture, calling
\mbox{\tt markFrom}\indexTT{markFrom} with the address of node (a) should mark node (c) (but not node
(b)), and return the address of node (c).

How do we make use of the address returned by \mbox{\tt markFrom}\indexTT{markFrom}?
It must be inserted in place
of the address with which \mbox{\tt markFrom}\indexTT{markFrom} was called.  The easiest way to do
this is to merge the first two phases, so that as each root is identified
in the machine state,
\mbox{\tt markFrom}\indexTT{markFrom} is called, and the returned address is used to replace the
original root in the machine state.
So we replace \mbox{\tt findStackRoots}\indexTT{findStackRoots} and its companions with:
\begin{verbatim}
> markFromStack   :: TiHeap -> TiStack   -> (TiHeap,TiStack)
> markFromDump    :: TiHeap -> TiDump    -> (TiHeap,TiDump)
> markFromGlobals :: TiHeap -> TiGlobals -> (TiHeap,TiGlobals)
\end{verbatim}
\indexDTT{markFromStack}%
\indexDTT{markFromDump}%
\indexDTT{markFromGlobals}%
\begin{exercise}
Implement the revised version of \mbox{\tt markFrom}\indexTT{markFrom}, making it `skip over'
indirections without marking them, and update the addresses inside
each node as it calls itself recursively.
Then implement the other marking functions in terms of \mbox{\tt markFrom}\indexTT{markFrom},
and glue them together
with a new version of \mbox{\tt gc}.
Measure the improvement, by comparing the heap sizes obtained with this
new collector to the ones you obtained before.  (You can easily revert to
the one before by removing the special handling of \mbox{\tt NInd}\indexTT{NInd} from
\mbox{\tt markFrom}\indexTT{markFrom}.)
\end{exercise}

\subsection{Pointer reversal}\index{pointer reversal}

If all the $N$ nodes in the heap happened to be linked together into a single
long list, then \mbox{\tt markFrom}\indexTT{markFrom} would call itself recursively $N$ times.
In a real implementation this would build up a stack which is as deep as the
heap is large.  It is very tiresome to have to allocate a stack as large
as the heap to account for a very unlikely situation!

There is a neat trick called {\em pointer reversal},
which can eliminate the stack by linking together the
very nodes which are being marked \cite{PointerReversal}.
The only extra requirement placed by the algorithm is that
marked nodes need a few extra bits of state information.  We can express
this by expanding the \mbox{\tt NMarked}\indexTT{NMarked} constructor somewhat:
\begin{verbatim}
> data Node = NAp Addr Addr                     -- Application
>           | NSupercomb Name [Name] CoreExpr   -- Supercombinator
>           | NNum Int                          -- Number
>           | NInd Addr                         -- Indirection
>           | NPrim Name Primitive              -- Primitive
>           | NData Int [Addr]                  -- Tag, list of components
>           | NMarked MarkState Node            -- Marked node
\end{verbatim}
\indexDTT{data}%
\indexDTT{NSupercomb}%
\indexDTT{NNum}%
\indexDTT{NInd}%
\indexDTT{NPrim}%
\indexDTT{NData}%
\indexDTT{NMarked}%
\begin{verbatim}
> data markState = Done         -- Marking on this node finished
>                | Visits Int   -- Node visited n times so far
\end{verbatim}
\indexDTT{data}%
\indexDTT{Visits}%
The meaning of the constructors for \mbox{\tt markState}\indexTT{markState} will be explained shortly.

\begin{figure} %\centering
\input{gc}
\caption{Marking a graph using pointer reversal}
\label{fig:gc}
\end{figure}
We can describe the pointer-reversal
algorithm with the aid of another (quite separate) state
transition system.
The state of the
marking machine has three components $(f,b,h)$, the \stressD{forward pointer},
the \stressD{backward pointer} and the heap.  Each call to \mbox{\tt markFrom}\indexTT{markFrom}
initiates a new run of the machine.  When \mbox{\tt markFrom}\indexTT{markFrom} is called with
address $a$ and heap $h_{init}$ the machine is started from the state
\[
(a,\mbox{\tt hNull}\indexTT{hNull},h_{init})
\]
(\mbox{\tt hNull}\indexTT{hNull} is a distinguished value of type \mbox{\tt addr}\indexTT{addr} which does not address any
object in the heap, and which can be distinguished from ordinary addresses.)
The machine terminates when it is in a state
\[
(f,~\mbox{\tt hNull}\indexTT{hNull},~ h[f:\mbox{\tt NMarked\ Done}~n])
\]
that is, when $f$ points to a marked node, and $b=\mbox{\tt hNull}\indexTT{hNull}$.
(It is possible that the initial state is also a final state, if the
node pointed to by $f$ is already marked.)

We begin the transition rules by describing how the machine handles
unmarked nodes.  \mbox{\tt NData}\indexTT{NData} nodes will be ignored for the present.
First, we deal with the case of applications.  When we encounter an
unmarked application, we `go down' into its first sub-graph,
recording the old back-pointer in the first field of the \mbox{\tt NAp}\indexTT{NAp} node.
The new forward-pointer addresses the first sub-graph, and the
new back-pointer addresses the application node itself.
The state information, \mbox{\tt Visits\ 1}, records the fact that the back-pointer
is kept in the first field of the \mbox{\tt NAp}\indexTT{NAp} node.
\gcrule{
   \gcstate{f}{b}{h[f:\mbox{\tt NAp}\indexTT{NAp}~a_1~a_2]}
}{
   \gcstate{a_1}{f}{h[f:\mbox{\tt NMarked}\indexTT{NMarked}~ (\mbox{\tt Visits\ 1})~(\mbox{\tt NAp}\indexTT{NAp}~b~a_2)]}
}
This is illustrated in Figure~\ref{fig:gc}(a) and (b).  In this figure the
marks are abbreviated to `\mbox{\tt V1}' for `\mbox{\tt Visits\ 1}', `\mbox{\tt V2}' for
`\mbox{\tt Visits\ 2}' and `\mbox{\tt D}' for `\mbox{\tt Done}\indexTT{Done}'.  Notice the way that
a chain of reversed pointers builds up in the application nodes which
have been visited.

The next rule says that unmarked \mbox{\tt NPrim}\indexTT{NPrim} nodes should be marked as completed,
using the \mbox{\tt NMarked\ Done} constructor (Figure~\ref{fig:gc}(c)):
\gcrule{
   \gcstate{f}{b}{h[f:\mbox{\tt NPrim}\indexTT{NPrim}~p]}
}{
   \gcstate{f}{b}{h[f:\mbox{\tt NMarked\ Done}~(\mbox{\tt NPrim}\indexTT{NPrim}~p)]}
}
\mbox{\tt NSupercomb}\indexTT{NSupercomb} and \mbox{\tt NNum}\indexTT{NNum} nodes are treated similarly, since they
do not contain any further addresses.

So much for unmarked nodes.  When the machine finds that $f$ points to
a marked node, it inspects the node which $b$ points to.  If it is \mbox{\tt hNull}\indexTT{hNull},
the machine terminates.  Otherwise, it must be a
marked \mbox{\tt NAp}\indexTT{NAp} node.  Let us deal first with the case where the
state information is \mbox{\tt (Visits\ 1)}, saying that the node has been
visited once. We have therefore completed marking the first sub-graph
of the \mbox{\tt NAp}\indexTT{NAp} node, and should now mark the second, which we do by making
$f$ point to it, leaving $b$ unchanged, moving the back-pointer saved
in the node ($b'$) from the first field to the second, and changing the state
information (Figure~\ref{fig:gc}(d)):
\gcrule{
   \gcstate{f}{b}
        {h\left[\begin{array}[c]{l}
                f:\mbox{\tt NMarked\ Done}~n \\
                b:\mbox{\tt NMarked}\indexTT{NMarked}~(\mbox{\tt Visits\ 1})~(\mbox{\tt NAp}\indexTT{NAp}~b'~a_2)
                \end{array}\right]}
}{
  \gcstate{a_2}{b}{h[b:\mbox{\tt NMarked}\indexTT{NMarked}~ (\mbox{\tt Visits\ 2})~(\mbox{\tt NAp}\indexTT{NAp}~f~b')]}
}

Some time later, the machine will complete marking the second sub-graph,
in which case it can restore the node to its original form, and
back up the chain of back-pointers one stage (Figure~\ref{fig:gc}(e)):
\gcrule{
   \gcstate{f}{b}
        {h\left[\begin{array}[c]{l}
                f:\mbox{\tt NMarked\ Done}~n \\
                b:\mbox{\tt NMarked}\indexTT{NMarked}~ (\mbox{\tt Visits\ 2})~ (\mbox{\tt NAp}\indexTT{NAp}~a_1~b')
                \end{array}\right]}
}{
  \gcstate{b}{b'}{h[b:\mbox{\tt NMarked\ Done}~ (\mbox{\tt NAp}\indexTT{NAp}~a_1~f)]}
}

Lastly, we deal with indirections\index{indirections}.  They are skipped over by
changing $f$ but not $b$.  The heap is left unchanged, so the indirection itself
is not marked.  When garbage collection is completed, all indirections will
therefore be reclaimed.  As you can see, `shorting out' indirections during
garbage collection is very easy with this marking algorithm.
\gcrule{
   \gcstate{f}{b}{h[f:\mbox{\tt NInd}\indexTT{NInd}~a]}
}{
   \gcstate{a}{b}{h}
}

That completes the state transitions for the pointer-reversal algorithm.
\begin{exercise}
Add rules for the \mbox{\tt NData}\indexTT{NData} node.
\end{exercise}
\begin{exercise}
Implement the algorithm.  The main changes required are to the \mbox{\tt node}\indexTT{node} type
and to the \mbox{\tt markFrom}\indexTT{markFrom} function.  \mbox{\tt scan}\indexTT{scan} needs to change in a trivial way,
because the format of \mbox{\tt NMarked}\indexTT{NMarked} constructors has changed.
\end{exercise}

\subsection{Two-space garbage collection}\index{two-space garbage collection}

Another very popular way of performing garbage collection is
to copy all the live data from one heap to another, the so-called
two-space collector invented by \cite{Fenichel} (see also
\cite{Baker,Cheney}). The collector works in two stages:
\begin{enumerate}
\item
All the nodes pointed to by the machine state (stack, dump, etc.)
are \stressD{evacuated} from the old heap (called \stressD{from-space}) into
the initially empty new heap (called \stressD{to-space}).
A node is evacuated by allocating
a copy of it in to-space, and overwriting the from-space copy with a
\stressD{forwarding pointer} containing the to-space address of the new
node.
Like \mbox{\tt markFrom}\indexTT{markFrom}, the
evacuation routine returns the to-space address of the new node, which
is used to replace the old address in the machine state.
\item
Then all the nodes in to-space are scanned linearly, starting at the first,
and each is \stressD{scavenged}.
A node $n$ is scavenged by evacuating any nodes
to which it points, replacing their addresses in $n$ with their new
to-space addresses.
Scanning stops when the scanning pointer catches up with the allocation
pointer.
\end{enumerate}

To implement this, we have to add yet another variant of the \mbox{\tt node}\indexTT{node} type,
this time with an \mbox{\tt NForward}\indexTT{NForward} constructor, which contains a single adddress
(the to-space address).  (\mbox{\tt NMarked}\indexTT{NMarked} is not needed for this collector.)
Instead of \mbox{\tt markFromStack}\indexTT{markFromStack} we need \mbox{\tt evacuateStack}\indexTT{evacuateStack} with type:
\begin{verbatim}
> evacuateStack :: TiHeap -> TiHeap -> TiStack -> (TiHeap, TiStack)
\end{verbatim}
\indexDTT{evacuateStack}%
\par
The call $(\mbox{\tt evacuateStack}\indexTT{evacuateStack}~fromheap~toheap~stk)$ evacuates all the
nodes in $fromheap$ referred to from $stk$ into $toheap$, returning
the new $toheap$ and the new $stk$.  Similar functions are required for
the dump and globals.

Lastly, we need a function
\begin{verbatim}
> scavengeHeap :: TiHeap -> TiHeap -> TiHeap
\end{verbatim}
\indexDTT{scavengeHeap}%
where the call $(\mbox{\tt scavengeHeap}\indexTT{scavengeHeap}~fromheap~toheap)$ scavenges nodes in
$toheap$, evacuating when necessary nodes from $fromheap$ into $toheap$.

\begin{exercise}
Implement this garbage collector.
\end{exercise}

% end of chapter 2
