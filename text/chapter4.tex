\chapter{Advanced Concepts: Formalization}
\label{chap:FormalizationOfAdvancedConcepts}
First, \myref{sec}{sec:formalspec2} of this chapter will extend the formalization of \MiniML\ made in \myref{sec}{sec:MiniMLFormalSpecification} with the additions described in \myref{chap}{chap:AdvancedConcepts}.
Later, \myref{sec}{sec:formalizedcompiler2} extends upon the secure compiler as it was formalized in \myref{sec}{sec:formalizedcompiler}.

\section{Formal Specification of \MiniML}
\label{sec:formalspec2}
The \MiniML\ language as specified in \myref{chap}{chap:formalspecification} consisted of a very limited subset of the ML language.
In \myref{chap}{chap:AdvancedConcepts}, this issue was addressed by adding two more concepts to the subset:
\begin{description}
\item[Higher-order functions]
The concept of \emph{higher-order functions} allows that functions receive other functions as their input, or return functions as their output.
By allowing them to be assigned as a value to a variable, functions essentially become \emph{first-class values}, going by the name \emph{closure}.

To implement closures, the ability to create anonymous functions using the $\lambda$ notation is added.
This addition makes it possible to define a function as: \cmath{\lambda(\overline{x}:\overline{\sigma}).e}.
This expression is considered as the construction of a closure.

These expressions are also the \emph{only} way of constructing a closure, as explained in \myref{sec}{sec:OnlyLambdaClosures} on page \pageref{sec:OnlyLambdaClosures}. 
Essentially, when a named function is used as a closure, this is considered to be syntactic sugar.
Its desugared form is an expression in which a closure is explicitly constructed using the $\lambda$-notation to wrap around the named function.

\item[Functors]
The idea of functors is that they act as functions from structures to structures.
They provide a way to parametrizing structures.

A parametrized structure is a functor \cmath{F} that, instead of depending on another structure \cmath{Str_{2}} by name, receives this structure as an argument.
To ensure that the structure given as an argument defines the necessary bindings, the argument structure is restricted by a signature \cmath{Sig}.
The functor definition defines all bindings of the parametrized structure.
In short, the definition of a functor corresponds to \lsttext{functor $\ \cmath{id}$($\cmath{ArgStr:Sig_{1}}$):>$\cmath{Sig_{2}}$ = struct $\ \overline{d}$ end}.

The application of a functor \cmath{F} to a \cmath{Str_{1}}, written as \cmath{F(Str_{1})} then results in a new structure \cmath{Str_{2}} that can be bound to an identifier.
This resulting structure, or \emph{output} structure, can be used like any regular structure. Any time the functor bindings uses \cmath{ArgStr.x}, \cmath{Str_{2}} uses \cmath{Str_{1}.x}.
\end{description}

These additions have important consequences for the way \MiniML\ is compiled.
To be able to describe these consequences, the formalization of \MiniML\ as given in \myref{chap}{chap:formalspecification} is revisited.

\subsection{Syntax}
The additional concepts introduce some new syntactical constructs to the \MiniML\ syntax.
The additions to the Core syntax are seen in \myref{fig}{fig:UpdatedCoreSyntax}, additions to the Module syntax in \myref{fig}{fig:UpdatedModuleSyntax}.
The Program syntax remains unchanged by the additions from \myref{chap}{chap:AdvancedConcepts}.

\subsubsection{Closures}
Closures introduce the \cmath{\lambda}-expression to the Core syntax, \myref{fig}{fig:UpdatedCoreSyntax}.
It specifies a number of argument, their identifiers and their types, as well as the expression that represents the function body.

\begin{figure}[htb]
\begin{align*}
\begin{aligned}
\text{Val Exp } \\
e \; ::= &\ \ldots \\
&|\ \lambda\ \overline{x}.e:(\overline{\tau_{1}}\rightarrow\tau_{2}) & \expl{(Anonymous function)}
\end{aligned}
\end{align*}
\caption[Updated Core Syntax]{The Core syntax, updated from {\protect\myref{fig}{fig:CoreSyntax}}.\label{fig:UpdatedCoreSyntax}}
\end{figure}
\vspace{-1em} %Todo: LaTeX magic, see if this is necessary!
\subsubsection{Functors}
Functors introduce the functor definition and functor application as module expressions to the Module syntax \myref{fig}{fig:UpdatedModuleSyntax}.

A functor definition specifies:
\begin{itemize}
\item the functor identifier,
\item the identifier \cmath{ArgStr} for the argument structure,
\item the signature \cmath{Sig_{1}} to which the argument structure must comply,
\item the signature \cmath{Sig_{2}} to which the functors body must comply,
\item and the body \cmath{\overline{d}}.
\end{itemize}
A functor application specifies the identifier to which the resulting structure is bound, the functor identifier, and the identifier for the structure that is passed as an argument.
\begin{figure}[htb]
\caption[Updated Module Syntax]{The Module Syntax, updated from {\protect\myref{fig}{fig:ModuleSyntax}}.\label{fig:UpdatedModuleSyntax}}
\end{figure}

\subsection{Type system}
The type system also needs some minor modifications to allow the new concepts.
As earlier, due to its simplicity the Program type system sees no modifications.

First, the concept of a \emph{context} $\Gamma$ is extended in \myref{fig}{fig:UpdatedMiniMLContexts}, so that it can contain conclusions related to functors: The signature it conforms to, its body of definitions, and the name of the signature that its argument must conform to.

\begin{figure}[!htb]
\begin{align*}
\begin{aligned}
\text{Context }\\
\Gamma ::=\; & \ldots \\ 
&| \; (\mathit{FunId} \mapsto \lbrace \Sigma, \overline{d}, SigId\rbrace), \Gamma
                                           & \expl{(functor definition)}
\end{aligned}
\end{align*}
\caption[Updated Contexts]{Contexts in the MiniML type system, extending \earlier{fig}{fig:MiniMLContexts}.}
\label{fig:UpdatedMiniMLContexts}
\end{figure}

Next, the typing rules are updated, as shown in \myref{fig}{fig:UpdatedTypeRulesCore} and \myref{fig}{fig:UpdatedTypeRulesModule}.

\subsubsection{Closures}
The addition of closures, as a Core concept, modifies the function application rule T-App inside the Core typing system in \myref{fig}{fig:UpdatedTypeRulesCore}, so that it allows arguments or return values to be of the function type.
It also adds the $\lambda$-expression typing rule T-Lambda. 
This rule describes that a $\lambda$-expression is well typed if its defining expression can be typed to the result type, assuming each argument is of the stated argument type.

\begin{figure}[htb]
\begin{align*}
\tag{T-App}
&\frac
{\Gamma \vdash e_{1}:\overline{\tau_{2}} \rightarrow \tau_{1}
    \longspace
    \Gamma \vdash \overline{e_{2}}:\overline{\tau_{2}}}
{\Gamma \vdash e_{1} \overline{e_{2}}:\tau_{1}}
\\[1em]
\tag{T-Lambda}
&\frac
{\overline{(x:\tau_{2})} \cup \Gamma \vdash e : \tau_{1}}
{\Gamma \vdash \lambda\ \overline{x}.e : \overline{\tau_{2}} \rightarrow \tau_{1}}
\end{align*}
\caption[Updated Typing Rules: Core Language]{Updated typing rules for the Core language, extending \earlier{fig}{fig:TypeRulesCore}. \label{fig:UpdatedTypeRulesCore}}
\end{figure}

\subsubsection{Functors}
Functors are strictly a Module language concept, which means modifications are limited to Module language, updated in \myref{fig}{fig:UpdatedTypeRulesModule}.
It adds two rules:
\begin{description}
\item[T-FunctorDef]
A rule to type a functor definition.
This rule says that the \emph{principal signature} of the body (\cmath{PS(\overline{d})}) must match with the signature specified in the ascription of the functor.
Furthermore, each definition in the body must be correctly typed, assuming the well-typedness of all other definitions and the values defined in the argument structure. 
\item[T-FunctorApp]
A rule that types functor applications.
This rule says that a functor application is well typed if the effective signature of the structure that was passed as an argument matches to the expected signature.
Furthermore, it specifies that the structure is saved in the context.
The output structures identifier maps to the result signature of the functor and the body of the functor after substituting all references to the argument structure.
\end{description}


\begin{figure}[htb]
\begin{align*}
\tag{T-FunctorDef}
&\frac
{\Gamma[SigId].\Sigma \succeq \mathit{PS}(\overline{d})
\longspace
\forall d_{1} \in \overline{d} : PS(\overline{d}\setminus \{ d_{1}\}) \cup \Gamma \vdash d_{1}}
{\Gamma \vdash StrId : SigId = \overline{d} \rightarrow (\mathit{StrId} \mapsto \lbrace \mathit{ES}(\overline{d}), \overline{d}\rbrace),\Gamma}
\\
\\
\tag{T-FunctorApp}
&\frac
{\Gamma[SigId_{2}].\Sigma \succeq \mathit{PS}(\overline{d})
\longspace
\forall d_{1} \in \overline{d} : PS(\overline{d}\setminus \{ d_{1}\}) \cup \Gamma \vdash d_{1}}
{\Gamma \vdash \cmath{FunId(ArgStr:SigId_{1}):SigId_{2}} = \overline{d} \rightarrow (\cmath{FunId \mapsto \lbrace ES(\overline{d}), \overline{d}, SigId_{1}\rbrace}),\Gamma}
\end{align*}
\caption[Updated Typing Rules: Module Language]{Updated typing rules for the Module language, extending \earlier{fig}{fig:TypeRulesModule}. \label{fig:UpdatedTypeRulesModule}}
\end{figure}
\subsection{Operational Semantics}
The updated formalization of \MiniML\ is concluded with a description of the updates to the operational semantics.

Fist, the concepts of values v is extended in \myref{fig}{fig:UpdatedValues} to reflect the addition of closures as a \emph{first class value}.
A closure is a pair, consisting of an expression and an environment in which this expression is evaluated.
The environment consists of a set of bindings of identifiers to values, as shown in \myref{fig}{fig:Environments}.

\begin{figure}[htb]
\begin{align*}
\text{Value }v ::=\;& \ldots \\
&| (e, \rho)\ 
\end{align*}
\caption[Updated Value Concept]{Updated concept of `values', extending \earlier{fig}{fig:MiniMLOperationalSemanticEntitiesAndRelations}.\label{fig:UpdatedValues}}
\end{figure}

\begin{figure}[htb]
\begin{align*}
\text{Environment }\rho ::=\;& \emptyset \\
&| (x \mapsto v), \rho\ 
\end{align*}
\caption[Environment entity]{Environments.\label{fig:Environments}}
\end{figure}

Next, the concept of the module table T is extended, as shown in \myref{fig}{fig:UpdatedModuleTable} so that it can keep track of the definitions of functors.
The module table maps functor identifiers to their body of definitions.

\begin{figure}[htb]
\begin{align*}
\text{Module Table } T\; ::= \;& \ldots \\
&| \; \cmath{(FunId) \mapsto \overline{d}}, T
\end{align*}
\caption[Updated Module Table Concept]{Updated concept of a `module table', extending \earlier{fig}{fig:MiniMLOperationalSemanticEntitiesAndRelations}.\label{fig:UpdatedModuleTable}}
\end{figure}

\subsubsection{Closures}
The addition of closures adds two evaluation rules, one concerning their definition and one for their application, shown in \myref{fig}{fig:ClosureEvaluationRules}.
\begin{description}
\item[E-ClosureDef]
This rule describes how a $\lambda$-expression is evaluated.
The $\lambda$-expression creates a closure value, consisting of the expression $e$ and the environment.
The environment contains bindings for all variables used in the defining expression~$e$, but not defined within the expression. These are called the non-local of free variables of e, defined as \cmath{FreeVar(e)}.
\item[E-ClosureApplication]
This rule describes how the evaluation of a closure happens.
When the first expression of regular function application is evaluated to a closure value, it the function application is evaluated to the defining expression with all the arguments identifiers substituted with their values, and all mappings inside the environment applied.
\end{description}

\begin{figure}[htb]
\begin{align*}
\tag{E-ClosureDef}
&\frac
{\rho = \cmath{FreeVar(e)}}
{T \vdash \lambda \overline{x}.e \rightarrow T \vdash (e,\rho)}
\\
\\
\tag{E-ClosureApplication}
&\frac
{T \vdash e_{1} \rightarrow T \vdash (e_{2}, \rho)}
{T \vdash e_{1} \overline{v} \rightarrow [\overline{x} \mapsto \rho \cup \overline{v}] e_{2}}
\end{align*}
\caption[Closure Evaluation Rules]{The evaluation rules concerning closures, extending on \earlier{fig}{fig:MiniMLOperationalSemantics}. \label{fig:ClosureEvaluationRules}}
\end{figure}


\subsubsection{Functors}
Adding functors to the language means there must be a way to evaluate functor definitions and functor applications.
This means the additions of functors introduces two evaluation rules, as described in \myref{fig}{fig:FunctorEvaluationRules}
\begin{description}
\item[E-FunctorDef]
When a functor definition is encountered, a mapping from the functor identifier to its definition body should be saved in the module table T.
\item[E-FunctorApp]
When a structure is bound by a functor application, a mapping for the structure identifier is added in the module table T.

\end{description}

\begin{figure}[htb]
\begin{align*}
\tag{E-FunctorDef}
&\cmath{T \vdash FunId(ArgStr:SigId_{1}) = \overline{d}, P \rightarrow (FunId \mapsto \overline{d}), T \vdash P}
\\
\\
\tag{E-FunctorApp}
&\cmath{T \vdash StrId_{1} = FunId(StrId_{2}), P \rightarrow (StrId_{1} \mapsto [ArgStr \mapsto StrId_{2}]\overline{d}), T \vdash P}
\end{align*}
\caption[Functor Evaluation Rules]{The evaluation rules concerning functors, extending on \earlier{fig}{fig:MiniMLOperationalSemantics} and \myref{fig}{fig:ClosureEvaluationRules}. \label{fig:FunctorEvaluationRules}}
\end{figure}


\section{Formalized Compiler}
\label{sec:formalizedcompiler2}
