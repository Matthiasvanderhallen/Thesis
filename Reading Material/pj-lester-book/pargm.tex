% $Date: 91/11/18 07:13:42 $
% $Revision: 1.7 $
% (c) 1991 Simon Peyton Jones & David Lester.
\begin{verbatim}
> module ParGM where
> import Utils
> import Language
> --import GM
\end{verbatim}
\indexDTT{module}%
\indexDTT{import}%
\indexDTT{import}%
\chapter{A Parallel G-machine\index{parallel G-Machine}}
\label{sect:par-g-machine}

\section{Introduction}

In this chapter we develop an abstract machine and compiler for a
parallel G-machine. It is based on a simplified version of the
parallel G-machine developed as part of ESPRIT project 415; interested
readers are referred to \cite{HDG-machine} for an easily accessible
account.  A general introduction to parallel graph reduction can be
found in \cite{PJ:CJoverview}.

\subsection{Parallel functional programming\index{parallel functional
programming}}

Writing parallel imperative programs is hard. Amongst the reasons for
this are the following:
\begin{itemize}

\item The programmer has to conceive of a {\em parallel
algorithm\/}\index{parallel algorithm} which meets the specification of
the problem.

\item The algorithm must be translated into the programming language
constructs provided by the language. This is likely to entail:
identification of {\em concurrent tasks}\index{task!concurrent},
defining the interfaces between tasks to allow them to {\em
synchronise\/}\index{task!synchronization} and {\em
communicate}\index{task!communication}. {\em Shared data\/}\index{shared memory}
may need to be especially protected, to prevent more than one task
accessing a variable at once.

\item The programmer may be responsible for assigning tasks to
processors, ensuring that tasks that need to communicate with one
another are assigned to processors that are physically connected.

\item Finally, in systems with no programmer control over the {\em
scheduling policy}\index{scheduling!policy}, the programmer must prove
that the collection of concurrent tasks will execute correctly under
all possible {\em interleavings\/}\index{tasks!interleaving} of task
operations.

\end{itemize}

In contrast, when programming in a functional language, only the first
of these points applies. Consider the following (contrived) example
program, $\mbox{\tt psum}\indexTT{psum}~n$, which calculates the sum of the numbers $1\ldots n$.
\begin{verbatim}
        psum n = dsum 1 n;
        dsum lo hi = let mid = (lo+hi)/2 in
                     if (hi==lo) hi ((dsum lo mid)+(dsum (mid+1) hi))
\end{verbatim}
The \mbox{\tt dsum}\indexTT{dsum} function works by dividing the problem into two, roughly
equal, parts. It then combines the two results together to generate
the answer.  This is a classic {\em divide-and-conquer
algorithm}\index{divide-and-conquer algorithm}.

Notice that neither of the functions \mbox{\tt dsum}\indexTT{dsum} or \mbox{\tt psum}\indexTT{psum} includes any
mention of parallel primitives in the language; so why is \mbox{\tt psum}\indexTT{psum} a
parallel algorithm? For comparison, we can write a sequential
algorithm: \mbox{\tt ssum}\indexTT{ssum}.
\begin{verbatim}
        ssum n = if (n==1) 1 (n + ssum (n-1))
\end{verbatim}
This function is a sequential algorithm because its {\em data
dependencies\/}\index{data dependencies} are inherently
sequential\index{inherant sequentiality}. In the \mbox{\tt ssum}\indexTT{ssum} example, what
we mean by this is that the addition in \mbox{\tt ssum\ n} can only take place
once \mbox{\tt ssum\ (n-1)} has been evaluated.  This in turn can only take
place once we have evaluated \mbox{\tt ssum\ (n-2)}, and so on. We may summarise
this distinction as:
\begin{important}
A function implements a parallel algorithm whenever it permits the
concurrent evaluation of two or more sub-expressions of its body.
\end{important}

As with any other programming language, a parallel algorithm is
essential.  Notice, however, the contrasts with parallel imperative
programming:
\begin{itemize}

\item No new language constructs are required to express parallelism,
synchronisation or communication. The concurrency is implicit, and new
tasks are created dynamically to be executed by the machine whenever
it has spare capacity.

\item No special measures are taken to protect data shared between
concurrent tasks. For example, \mbox{\tt mid}\indexTT{mid} is safely shared between the two
concurrent tasks in \mbox{\tt dsum}\indexTT{dsum}.

\item We need no new proof techniques to reason about the parallel
programs, as all of the techniques we use for sequentially executed
functional programs still work. We also note that {\em
deadlock\/}\index{deadlock} can only arise as a result of
a self-dependency, such as
\begin{verbatim}
        letrec a = a+1 in a
\end{verbatim}
Expressions depending on their own value are meaningless, and their
non-termination is the same behaviour as observed in the sequential
implementation.

\item The results of a program are determinate. It is not possible for
the scheduling algorithm to cause answers to differ between two runs
of the same program.

\end{itemize}

In summary, we suggest that these features allow us to
express parallel algorithms conveniently,
without having to solve a large number of
low-level problems. Perhaps we can characterise this as:
\begin{important}
A parallel imperative program specifies in detail {\em
resource allocation decisions\/}\index{resource-allocation decisions}
which a parallel functional program does not even mention.
\end{important}
This means that the machine will have to be able to make resource
allocation decisions automatically.  We will pay, with a loss of
execution efficiency, whenever these decisions are not optimal.

\subsubsection{Annotations}

The high level of abstraction offered by functional languages places
heavy demands on the compile-time and run-time resource allocators.
Rather than leave {\em all\/} resource allocation decisions to the
system, we will introduce an {\em annotation}\index{annotation, in
parallel functional programs}, \mbox{\tt par}\indexTT{par}, which initiates a new parallel
thread. The annotation is a meaning-preserving decoration of the
program text; in the case of \mbox{\tt par}\indexTT{par} it has the following syntax and
meaning:
\[
  \mbox{\tt par}\indexTT{par}~E_1~E_2 = E_1~E_2
\]
That is: \mbox{\tt par}\indexTT{par} is a synonym for application. Where it differs is that
we intend that the expression $E_2$ should be evaluated by a
concurrent task. As an example we will now rewrite the \mbox{\tt dsum}\indexTT{dsum} function
using this annotation:
\begin{verbatim}
        dsum lo hi = let mid = (lo+hi)/2 in
                     let add x y = x+y
                     if (lo==hi) hi (par (add (dsum lo mid))
                                         (dsum (mid+1) hi))
\end{verbatim}

We see that \mbox{\tt par}\indexTT{par} causes the second argument to \mbox{\tt +} to be evaluated in
parallel. The \mbox{\tt par}\indexTT{par} annotation can be inserted by the programmer or,
in principle, by a clever compiler. Such cleverness is, however,
beyond the scope of this book, so we will assume that the \mbox{\tt par}\indexTT{par}s have
already been inserted.

\subsection{Parallel graph reduction\index{parallel graph reduction}}
\label{pgm:ss:paraGR}

We have seen in this book that graph reduction is a useful
implementation technique for sequential machines; it will be no
surprise that it is also suited to the implementation of parallel
machines. In fact it has a number of benefits:
\begin{itemize}

\item There is no sequential concept of program counter; graph reduction
is decentralised and distributed.

\item Reductions may take place concurrently in many places in the
graph, and the relative order in which they are scheduled cannot
affect the result.

\item All communication and synchronisation takes place via the
graph.
\end{itemize}

\subsubsection{A parallel model}

A {\em task\/}\index{task} is a sequential computation whose purpose is
to reduce a particular sub-graph to WHNF. At any moment there may be
many tasks that are able to run; this collection of sparked tasks is
called the {\em spark pool}\index{spark pool}. A processor in search
of work fetches a new task from the spark pool and executes it. A task
can be thought of as a {\em virtual processor}\index{virtual
processor}.

Initially, there is only one task, whose job is to evaluate the whole
program.  During its execution, it will hopefully create new tasks to
evaluate expressions that the main task will later need. These are
placed in the spark pool, so that another processor can pick up the
task if it has run out of work. We call the act of placing a task into
the spark pool {\em sparking\/}\index{sparking, of a child task} a child
task.

It is useful to consider the interaction between parent and child
tasks. In the {evaluate-and-die model}\index{evaluate-and-die model of
parallelism} of task management there is a lock bit on each graph
node. When the bit is on, a task is executing which will evaluate the
node; otherwise the bit is off. When a parent task requires the value
of a sub-graph for which a task has been sparked, there are three
cases to consider:
\begin{itemize}
\item the child task has not started yet,
\item the child task has started, but not stopped yet, or
\item the child task has completed.
\end{itemize}
In the first case, the parent task can evaluate the graph just as if
the child task was not there. Of course this sets the lock bit, so
that when an attempt is made to execute the child task it is
immediately discarded. The interesting case is the second one, in
which both parent and child are executing. When this is the situation,
the parent task must wait for the child task to complete before
proceeding. We say that the child task is {\em
blocking\/}\index{blocking, of tasks} the parent task.  In the
third case, the node is now in WHNF and unlocked, so the parent task
will not take very long to fetch the value of the graph.

The advantage of the evaluate-and-die model is that blocking only
occurs when the parent and child actually collide. In all other cases,
the parent's execution continues unhindered. Notice that the blocking
mechanism is the {\em only\/} form of inter-task communication and
synchronisation. Once a piece of graph has been evaluated to WHNF any
number of tasks can simultaneously inspect it without contention.

\subsubsection{An example\index{parallel G-machine!an example execution}}

We begin with a sample execution of the simplest parallel program:
\begin{verbatim}
        main = par I (I 3)
\end{verbatim}
As we will shortly see, the two reductions of the identity function
can take place in parallel.

A good compiler will produce the following code for the \mbox{\tt main}\indexTT{main}
supercombinator\footnote{Your compiler might not produce this code if
it is not sophisticated enough.}:
\begin{verbatim}
        e1 ++ par
        where e1  = [Pushint 3, Pushglobal "I", Mkap, Push 0]
              par = [Par, Pushglobal "I", Mkap, Update 0, Pop 0, Unwind]
\end{verbatim}
Initially, there is only one task executing; it is constructing the
sub-expression \mbox{\tt (I\ 3)}. After executing the code sequence \mbox{\tt e1}, the
machine will be in the state shown in diagram (a) of
Figure~\ref{pgm:fg:1ex1}.
\begin{figure}
\input{pgm_1x1}
\caption{State after executing \mbox{\tt e1} code sequence and \mbox{\tt Par}\indexTT{Par}}\label{pgm:fg:1ex1}
\end{figure}

After executing \mbox{\tt e1} the machine encounters a \mbox{\tt Par}\indexTT{Par} instruction. This
causes it to create a new task to evaluate the node pointed to from
the top of the stack. We will refer to the new task as \mbox{\tt Task\ 2}; the
original task will be labelled \mbox{\tt Task\ 1}. This situation is illustrated
in diagram (b) of Figure~\ref{pgm:fg:1ex1}.

In diagram (c) (Figure~\ref{pgm:fg:1ex2}), we see \mbox{\tt Task\ 1}
continuing with its evaluation; it is performing a \mbox{\tt Pushglobal\ I}
instruction. The newly created task -- \mbox{\tt Task\ 2} -- starts with the
code sequence: \mbox{\tt [Unwind]}; it therefore starts to unwind the graph it
has been assigned to evaluate. Diagram (d) shows \mbox{\tt Task\ 1}
completing the instantiation of the body of \mbox{\tt main}\indexTT{main}, and \mbox{\tt Task\ 2}
completing its unwinding.
\begin{figure}
\input{pgm_1x2}
\caption{State after \mbox{\tt Task\ 1} executes \mbox{\tt [Pushglobal\ I,\ Mkap]}}
\label{pgm:fg:1ex2}
\end{figure}

Its body instantiated, \mbox{\tt Task\ 1} overwrites the redex node, which is
\mbox{\tt main}\indexTT{main}. \mbox{\tt Task\ 2} performs a \mbox{\tt Push\ 0} instruction, this being the first
instruction in the code for the \mbox{\tt I} supercombinator. This is shown in
diagram (e) (Figure~\ref{pgm:fg:1ex3}). In diagram (f) we see \mbox{\tt Task\ 1}
commence unwinding its spine, whilst \mbox{\tt Task\ 2} performs its updating.
\begin{figure}
\input{pgm_1x3}
\caption{State after \mbox{\tt Task\ 1} executes \mbox{\tt [Update\ 0,\ Unwind]}}
\label{pgm:fg:1ex3}
\end{figure}

In Figure~\ref{pgm:fg:1ex4}, we see \mbox{\tt Task\ 2} run to completion. The
remainder of the execution of \mbox{\tt Task\ 1} is the same as the sequential
G-machine, so we omit it.
\begin{figure}
\input{pgm_1x4}
\caption{State after \mbox{\tt Task\ 1} executes \mbox{\tt [Unwind,\ Unwind]}}
\label{pgm:fg:1ex4}
\end{figure}

This concludes a brief overview of the parallel G-machine's execution
with concurrent tasks. We now provide a minimal parallel G-machine.

\section{Mark 1: A minimal parallel G-machine\index{parallel G-machine!Mark 1}}
\label{minimal-pgm}

The first machine we present can be based on any of the G-machines
from Chapter~\ref{sect:g-machine}, except the Mark~1; we need this
restriction to ensure that updating is done. To this basic machine we
need to add the machinery for parallelism. We make the following basic
assumptions.
\begin{enumerate}
\item There is a shared global graph, which all processors can access.
\item There are an infinite number of processors. This means that
there is always a processor available to execute a task.
\item There is no locking of graph nodes. This means that it is
possible that different tasks will re-evaluate the same expression.
\end{enumerate}

\subsection{Data type definitions}

We will be making considerable use of state access functions in this
chapter. Although they are not particularly interesting, for
completeness they are included in this section.

In a parallel machine the state \mbox{\tt pgmState}\indexTT{pgmState} is split into two
components: a global component, \mbox{\tt pgmGlobalState}\indexTT{pgmGlobalState}; and a local
component, \mbox{\tt pgmLocalState}\indexTT{pgmLocalState}. The \mbox{\tt pgmLocalState}\indexTT{pgmLocalState} contains the
processors that will execute the program. The \mbox{\tt pgmGlobalState}\indexTT{pgmGlobalState}
contains global data structures -- of which the heap is the most
frequently used -- that are manipulated by the processors.

\begin{verbatim}
> type PgmState = (PgmGlobalState,    -- Current global state
>              [PgmLocalState])       -- Current states of processors
\end{verbatim}
\indexDTT{type}%

\subsubsection{The global state component}

To accommodate all of the possible machines you might use as a basis
for your parallel implementation, the global state consists of five
components: \mbox{\tt gmOutput}\indexTT{gmOutput}, which is the output printed as the answer
to the evaluation of the program; \mbox{\tt gmHeap}\indexTT{gmHeap}, which is the heap;
\mbox{\tt gmGlobals}\indexTT{gmGlobals}, which is used to point to the unique node in the heap for
each supercombinator; \mbox{\tt gmSparks}\indexTT{gmSparks}, which is the task pool and will
be used to hold tasks before we begin to execute them; and \mbox{\tt gmStats}\indexTT{gmStats},
which is a global clock.

\begin{verbatim}
> type PgmGlobalState = (GmOutput,          -- output stream
>                    GmHeap,            -- Heap of nodes
>                    GmGlobals,         -- Global addresses in heap
>                    GmSparks,          -- Sparked task pool
>                    GmStats)           -- Statistics
\end{verbatim}
\indexDTT{type}%
We consider each of these components in turn.

\begin{itemize}

\item The \mbox{\tt gmOutput}\indexTT{gmOutput} component was introduced in the Mark~6 G-machine
(see Section~\ref{gm:ss:mark6}). It is used to accumulate the result of
executing programs that can generate structured data, and is simply a string.
\begin{verbatim}
> type GmOutput = [Char]
\end{verbatim}
\indexDTT{type}%
The function to get the \mbox{\tt gmOutput}\indexTT{gmOutput} from a \mbox{\tt pgmState}\indexTT{pgmState} is \mbox{\tt pgmGetOutput}\indexTT{pgmGetOutput} is:
\begin{verbatim}
> pgmGetOutput :: PgmState -> GmOutput
> pgmGetOutput ((o, heap, globals, sparks, stats), locals) = o
\end{verbatim}
\indexDTT{pgmGetOutput}%
\indexDTT{pgmGetOutput}%
\item The heap data structure is the same as we used for the
sequential G-machine.
\begin{verbatim}
> type GmHeap = Heap Node
\end{verbatim}
\indexDTT{type}%
To get the heap from a \mbox{\tt pgmState}\indexTT{pgmState} we use \mbox{\tt pgmGetHeap}\indexTT{pgmGetHeap}.
\begin{verbatim}
> pgmGetHeap :: PgmState -> GmHeap
> pgmGetHeap ((o, heap, globals, sparks, stats), locals) = heap
\end{verbatim}
\indexDTT{pgmGetHeap}%
\indexDTT{pgmGetHeap}%
\item The addresses of global nodes in the heap are stored in
\mbox{\tt gmGlobals}\indexTT{gmGlobals}; this too is the same structure we used in the sequential
G-machine.
\begin{verbatim}
> type GmGlobals = ASSOC Name Addr
\end{verbatim}
\indexDTT{type}%
Obtaining the \mbox{\tt gmGlobals}\indexTT{gmGlobals} from a \mbox{\tt pgmState}\indexTT{pgmState} is performed using the
\mbox{\tt pgmGetGlobals}\indexTT{pgmGetGlobals} function.
\begin{verbatim}
> pgmGetGlobals :: PgmState -> GmGlobals
> pgmGetGlobals ((o, heap, globals, sparks, stats), locals) = globals
\end{verbatim}
\indexDTT{pgmGetGlobals}%
\indexDTT{pgmGetGlobals}%
\item The spark pool is represented by the \mbox{\tt gmSparks}\indexTT{gmSparks} component. It
holds the addresses of nodes in the graph which have been marked by
the \mbox{\tt par}\indexTT{par} annotation as needing to be evaluated concurrently.
\begin{verbatim}
> type GmSparks = [Addr]
\end{verbatim}
\indexDTT{type}%
Access to this component is achieved by using the function
\mbox{\tt pgmGetSparks}\indexTT{pgmGetSparks}.
\begin{verbatim}
> pgmGetSparks :: PgmState -> GmSparks
> pgmGetSparks ((o, heap, globals, sparks, stats), locals) = sparks
\end{verbatim}
\indexDTT{pgmGetSparks}%
\indexDTT{pgmGetSparks}%
\item Finally, in the parallel G-machine we will hold the accumulated
statistics in the global component. It is represented as a list of
numbers; these detail how long each task in the machine ran for,
before it completed.
\begin{verbatim}
> type GmStats = [Int]
\end{verbatim}
\indexDTT{type}%
Access to this component is accomplished using \mbox{\tt pgmGetStats}\indexTT{pgmGetStats}.
\begin{verbatim}
> pgmGetStats :: PgmState -> GmStats
> pgmGetStats ((o, heap, globals, sparks, stats), locals) = stats
\end{verbatim}
\indexDTT{pgmGetStats}%
\indexDTT{pgmGetStats}%
\end{itemize}

\subsubsection{The local state component}

The local component of the parallel G-machine consists of a list of
processors; a processor is represented as a task. Again, to make the
parallel machine capable of executing with any G-machine as its basis,
we make the state of each processor a 5-tuple:

\begin{verbatim}
> type PgmLocalState = (GmCode,    -- Instruction stream
>                   GmStack,       -- Pointer stack
>                   GmDump,        -- Stack of dump items
>                   GmVStack,      -- Value stack
>                   GmClock)       -- Number of ticks the task
>                                  --          has been active
\end{verbatim}
\indexDTT{type}%
We now consider each component in turn.

\begin{itemize}
\item The code sequence is simply a list of instructions.
\begin{verbatim}
> type GmCode = [Instruction]
\end{verbatim}
\indexDTT{type}%
\item As in the sequential G-machine, the stack is a list of addresses
in heap.
\begin{verbatim}
> type GmStack = [Addr]
\end{verbatim}
\indexDTT{type}%
\item If you are using a Mark~4 G-machine (or any later mark) as a
basis for your parallel implementation, then a dump is needed. This is
used as a stack of dump items\index{dump items}, each of type
\mbox{\tt gmDumpItem}\indexTT{gmDumpItem}.
\begin{verbatim}
> type GmDump     = [GmDumpItem]
> type GmDumpItem = (GmCode, GmStack)
\end{verbatim}
\indexDTT{type}%
\indexDTT{type}%
\item If you have used the Mark~7 G-machine as the basis of your
implementation you will need a V-stack for each processor.
\begin{verbatim}
> type GmVStack = [Int]
\end{verbatim}
\indexDTT{type}%

\item We also provide each processor with a clock. This records how
many instructions the task has executed:
\begin{verbatim}
> type GmClock = Int
\end{verbatim}
\indexDTT{type}%
\end{itemize}

\subsubsection{State access functions}

Although we have already defined some state access functions for the
parallel machine, we will find it convenient to define a few more.
Each processor will make one state transition, during
which it behaves as if it were a sequential machine.  If we make the
state \mbox{\tt gmState}\indexTT{gmState} a pair consisting of the global component of the
current machine state and a single processor state, then we have a
superset of the state components of any of the sequential G-machines.
\begin{verbatim}
> type GmState = (PgmGlobalState, PgmLocalState)
\end{verbatim}
\indexDTT{type}%
\par
It follows that we can simply redefine the state access functions we
used in the sequential machine to work with the new type of state.
Here are the type signatures of the global \mbox{\tt put}\indexTT{put} functions:

\begin{verbatim}
> putOutput :: GmOutput -> GmState -> GmState
> putHeap :: GmHeap   -> GmState -> GmState
> putSparks :: GmSparks -> GmState -> GmState
> putStats :: GmStats  -> GmState -> GmState
\end{verbatim}
\indexDTT{putOutput}%
\indexDTT{putHeap}%
\indexDTT{putSparks}%
\indexDTT{putStats}%
\par
The corresponding \mbox{\tt get}\indexTT{get} functions have type-signatures:

\begin{verbatim}
> getOutput :: GmState  -> GmOutput
> getHeap :: GmState  -> GmHeap
> getGlobals :: GmState  -> GmGlobals
> getSparks :: GmState  -> GmSparks
> getStats :: GmState  -> GmStats
\end{verbatim}
\indexDTT{getOutput}%
\indexDTT{getHeap}%
\indexDTT{getGlobals}%
\indexDTT{getSparks}%
\indexDTT{getStats}%
\par
For access to the components local to a processor we need \mbox{\tt put}\indexTT{put} functions
with the following type-signatures:

\begin{verbatim}
> putCode :: GmCode   -> GmState -> GmState
> putStack :: GmStack  -> GmState -> GmState
> putDump :: GmDump   -> GmState -> GmState
> putVStack :: GmVStack -> GmState -> GmState
> putClock :: GmClock  -> GmState -> GmState
\end{verbatim}
\indexDTT{putCode}%
\indexDTT{putStack}%
\indexDTT{putDump}%
\indexDTT{putVStack}%
\indexDTT{putClock}%
\par
The \mbox{\tt get}\indexTT{get} functions have types:

\begin{verbatim}
> getCode :: GmState  -> GmCode
> getStack :: GmState  -> GmStack
> getDump :: GmState  -> GmDump
> getVStack :: GmState  -> GmVStack
> getClock :: GmState  -> GmClock
\end{verbatim}
\indexDTT{getCode}%
\indexDTT{getStack}%
\indexDTT{getDump}%
\indexDTT{getVStack}%
\indexDTT{getClock}%
\begin{exercise}\label{pgm:X:access}
Write the access functions with the types given above.
\end{exercise}

GOT HERE ZZZZ KH

\par\subsection{The evaluator}

The structure of the evaluator \mbox{\tt eval}\indexTT{eval} will be familiar; it is the
similar to the one used in the G-machine.
\begin{verbatim}
> eval :: PgmState -> [PgmState]
> eval state = state: restStates
>              where
>              restStates | gmFinal state  = []
>                         | otherwise = eval (doAdmin (steps state))
\end{verbatim}
\indexDTT{eval}%
\indexDTT{eval}%
The difference is that we call \mbox{\tt steps}\indexTT{steps} instead of \mbox{\tt step}\indexTT{step}.  The \mbox{\tt steps}\indexTT{steps}
function must run down the list of processors doing a single step on
each. The precise sequence of events is:
\begin{enumerate}

\item First we extract the addresses that were sparked in the previous
call to \mbox{\tt steps}\indexTT{steps}, from the state.

\item Next, we turn them into processes. This is labelled \mbox{\tt newtasks}\indexTT{newtasks}.

\item The spark pool component of the state is set to empty.

\item We increment the clock for each processor that is about to
execute.

\item Finally, we use \mbox{\tt mapAccuml}\indexTT{mapAccuml} to perform a sequence of \mbox{\tt step}\indexTT{step}
transitions, one for each processor.

\end{enumerate}
\begin{verbatim}
> steps :: PgmState -> PgmState
> steps state
>  = mapAccuml step global' local'
>    where ((out, heap, globals, sparks, stats), local) = state
>          newtasks = [makeTask a | a <- sparks]
>          global'  = (out, heap, globals, [], stats)
>          local'   = map tick (local ++ newtasks)
\end{verbatim}
\indexDTT{steps}%
\indexDTT{steps}%
\par
To create a task to evaluate a node at address \mbox{\tt addr}\indexTT{addr} you must define
a \mbox{\tt makeTask}\indexTT{makeTask} function. For machines based on G-machines 2~or~3 this
will be:
\begin{verbatim}
> makeTask :: Addr -> PgmLocalState
> makeTask addr = ([Unwind], [addr], [], [], 0)
\end{verbatim}
\indexDTT{makeTask}%
\indexDTT{makeTask}%
For later marks of the G-machine we use:
\begin{verbatim}
> makeTask addr = ([Eval], [addr], [], [], 0)
\end{verbatim}
\indexDTT{makeTask}%
Incrementing the clock component of a processor is accomplished using
\mbox{\tt tick}\indexTT{tick}.
\begin{verbatim}
> tick (i, stack, dump, vstack, clock) = (i, stack, dump, vstack, clock+1)
\end{verbatim}
\indexDTT{tick}%
\par
The machine has terminated when there are no more sparks in the spark pool,
and there are no more processors executing tasks.

\begin{verbatim}
> gmFinal :: PgmState -> Bool
> gmFinal s = second s == [] && pgmGetSparks s == []
\end{verbatim}
\indexDTT{gmFinal}%
\indexDTT{gmFinal}%
We use the \mbox{\tt step}\indexTT{step} function to perform a single step on a processor.
\begin{verbatim}
> step :: PgmGlobalState -> PgmLocalState -> GmState
> step global local = dispatch i (putCode is state)
>                     where (i:is) = getCode state
>                           state = (global, local)
\end{verbatim}
\indexDTT{step}%
\indexDTT{step}%
\par
The \mbox{\tt doAdmin}\indexTT{doAdmin} function eliminates processors that have finished
executing.  A processor has finished when the code component is empty.
When this is the case, we must update the statistics component of the
state, with the number of instructions it took the processor to
complete its task.
\begin{verbatim}
> doAdmin :: PgmState -> PgmState
> doAdmin ((out, heap, globals, sparks, stats), local)
>  = ((out, heap, globals, sparks, stats'), local')
>    where (local', stats') = foldr filter ([], stats) local
>          filter (i, stack, dump, vstack, clock) (local, stats)
>           | i == [] = (local, clock:stats)
>           | otherwise = ((i, stack, dump, vstack, clock): local, stats)
\end{verbatim}
\indexDTT{doAdmin}%
\indexDTT{doAdmin}%
We now consider the new instruction transitions.

\subsubsection{The transition for the \mbox{\tt Par}\indexTT{Par} instruction}

The only new instruction that must be added is \mbox{\tt Par}\indexTT{Par}. Its
effect is to mark the node at the top of the stack so that the machine
may create a task to evaluate the node to WHNF. To do this, the
instruction must modify the global component of the state by adding
the address of the node to the spark pool.

\pgmrule%
{\pgmstate{h}{m}{t}{\mbox{\tt Par}\indexTT{Par}:i}{a:s}}%
{\pgmstate{h}{m}{a:t}{i}{s}}

The first tuple -- consisting of $h$, $m$ and $t$ -- is the global
state component, with $h$, $m$ and $t$ being the heap, global address
map and spark pool respectively. The second tuple -- consisting of an
instruction stream and a stack -- is a particular task's local state;
depending on the version of the G-machine you have used as a basis you
may need to add other components to the local state.

The effect of \mbox{\tt Par}\indexTT{Par} is to add the address $a$ to the spark pool. It is
implemented as follows:
\begin{verbatim}
> par :: GmState -> GmState
> par s = s
\end{verbatim}
\indexDTT{par}%
\indexDTT{par}%
\begin{exercise}\label{pgm:X:par1}
Modify \mbox{\tt showInstruction}\indexTT{showInstruction}, \mbox{\tt dispatch}\indexTT{dispatch} and \mbox{\tt instruction}\indexTT{instruction} so that
\mbox{\tt Par}\indexTT{Par} is correctly handled.
\end{exercise}

\subsection{Compiling a program}

A simple compiler can be constructed for a parallel machine based on
any of the sequential machines (except the Mark~7), by providing a
compiled primitive for the \mbox{\tt par}\indexTT{par} function. More extensive
modifications are required for the Mark~7 based machines.

The other modification required lies in the \mbox{\tt compile}\indexTT{compile} function, where
the various components are now to be found in different locations, and
of course there are now a number of processors. The new definition is:
\begin{verbatim}
> compile :: CoreProgram -> PgmState
> compile program
>  = (([], heap, globals, [], []), [initialTask addr])
>    where (heap, globals) = buildInitialHeap program
>          addr            = aLookup globals "main" (error "main undefined")
\end{verbatim}
\indexDTT{compile}%
\indexDTT{compile}%
This sets the global component to hold the heap and global map, as
used in the sequential G-machine. We also place a task in the local
component, to initiate the execution.
\begin{verbatim}
> initialTask :: Addr -> PgmLocalState
> initialTask addr = (initialCode, [addr], [], [], 0)
\end{verbatim}
\indexDTT{initialTask}%
\indexDTT{initialTask}%
\par
If you use the Mark~2 or Mark~3 sequential G-machine as a basis you need to
define \mbox{\tt initialCode}\indexTT{initialCode} as:
\begin{verbatim}
> initialCode :: GmCode
> initialCode = [Unwind]
\end{verbatim}
\indexDTT{initialCode}%
\indexDTT{initialCode}%
For the Mark~4 or Mar~5 machine this is changed to:
\begin{verbatim}
> initialCode = [Eval]
\end{verbatim}
\indexDTT{initialCode}%
And, to deal with data structures, the Mark~6 and Mark~7 machine has the
following \mbox{\tt initialCode}\indexTT{initialCode}:
\begin{verbatim}
> initialCode = [Eval, Print]
\end{verbatim}
\indexDTT{initialCode}%
\par
We now consider how to add \mbox{\tt par}\indexTT{par} to the primitive functions of the
machine. We begin by considering those machines based on sequential
G-machines Marks 2 through to 6.

\subsubsection{Using the Marks 2--6 G-machine as a basis}

We need to include the following in the definition of \mbox{\tt compiledPrimitives}\indexTT{compiledPrimitives}:
\begin{verbatim}
> ("par", 2, [Push 1, Push 1, Mkap, Push 2, Par, Update 2, Pop 2, Unwind])
\end{verbatim}
This rather cryptic piece of code performs the following task when the
function \mbox{\tt par}\indexTT{par} is applied to the two arguments: $E_1$ and $E_2$.
\begin{enumerate}
\item First we construct the application of $E_1$ to $E_2$; this is
the job of the sequence:
\begin{verbatim}
        [Push 1, Push 1, Mkap]
\end{verbatim}

\item Next, \mbox{\tt Push\ 2} makes a copy of the pointer to $E_2$. The \mbox{\tt Par}\indexTT{Par}
instruction then adds this address to the spark pool.

\item Finally, we perform the usual updating and tidying-up after an
instantiation.

\end{enumerate}

\subsubsection{Using the Mark~7 G-machine as a basis}

As mentioned above, this is a slightly trickier operation. We need to
modify the compiler functions \mbox{\tt compileR}\indexTT{compileR} and \mbox{\tt compileE}\indexTT{compileE} to recognise
the special cases involving \mbox{\tt par}\indexTT{par}. First \mbox{\tt compileR}\indexTT{compileR} needs the
following case added:
\begin{verbatim}
> compileR (EAp (EAp (EVar "par") e1) e2) args
>  = compileC e2 args ++ [Push 0, Par] ++
>    compileC e1 (argOffset 1 args) ++ [Mkap, Update n, Pop n, Unwind]
>    where n = #args
\end{verbatim}
\indexDTT{compileR}%
This uses the \tC{} scheme to compile \mbox{\tt e2}, which is then sparked. The
expression \mbox{\tt e1} is compiled using the \tC{} scheme, before we make the
application node. Finally, we perform the updating and tidying-up of
the stack.

Next, we modify \mbox{\tt compileE}\indexTT{compileE} so that it has a case:
\begin{verbatim}
> compileE (EAp (EAp (EVar "par") e1) e2) args
>  = compileC e2 args ++ [Push 0, Par] ++
>    compileC e1 (argOffset 1 args) ++ [Mkap, Eval]
\end{verbatim}
\indexDTT{compileE}%
This only differs from the case given for \mbox{\tt compileR}\indexTT{compileR} because it uses
\mbox{\tt Eval}\indexTT{Eval} to force the application node that is created to WHNF.

With these two modifications to the compiler, it suffices to add the
following to the \mbox{\tt primitives}\indexTT{primitives}:
\begin{verbatim}
>  ("par", ["x","y"], (EAp (EAp (EVar "par") (EVar "x")) (EVar "y")))
\end{verbatim}
Notice that we could use this approach with the Mark~5 or Mark~6 based
machines.

\begin{exercise}\label{pgm:X:primitives1}
Make the modifications to your compiler, so that there is a \mbox{\tt par}\indexTT{par}
function defined.
\end{exercise}

\begin{exercise}\label{pgm:X:primitives1comp}
Why do we perform the sparking of the graph for the second argument
before constructing the first argument in the special cases for

\mbox{\tt compileR}\indexTT{compileR} and \mbox{\tt compileE}\indexTT{compileE}?
\end{exercise}

\subsection{Printing the results}

Once we have computed the states of the machine we control the display
of them by using \mbox{\tt showResults}\indexTT{showResults}. This prints out: the code for the
supercombinators, the state transitions and the statistics. It has
type:
\begin{verbatim}
> showResults :: [PgmState] -> [Char]
\end{verbatim}
\indexDTT{showResults}%
\par
To print the supercombinator code we use the same \mbox{\tt showSC}\indexTT{showSC} functions as
that for the G-machine; it has type:
\begin{verbatim}
> showSC :: PgmState -> (Name, Addr) -> Iseq
\end{verbatim}
\indexDTT{showSC}%
\par
The function \mbox{\tt showState}\indexTT{showState} is used to display the state of local processes
during the transitions. Because we have a parallel machine there is now
likely to be more than one task executing at once. It has type:
\begin{verbatim}
> showState :: PgmState -> Iseq
\end{verbatim}
\indexDTT{showState}%
\par
Two other functions need to be modified: \mbox{\tt showStats}\indexTT{showStats} and
\mbox{\tt showOutput}\indexTT{showOutput}. They have types:
\begin{verbatim}
> showStats :: PgmState -> Iseq
> showOutput :: GmOutput -> Iseq
\end{verbatim}
\indexDTT{showStats}%
\indexDTT{showOutput}%
\begin{exercise}\label{pgm:X:show1}
Modify the functions: \mbox{\tt showResults}\indexTT{showResults}, \mbox{\tt showSC}\indexTT{showSC}, \mbox{\tt showState}\indexTT{showState}, \mbox{\tt showStats}\indexTT{showStats} and
\mbox{\tt showOutput}\indexTT{showOutput}.

Define a new display function \mbox{\tt showSparks}\indexTT{showSparks} with type:
\begin{verbatim}
> showSparks :: GmSparks -> Iseq
\end{verbatim}
\indexDTT{showSparks}%
\end{exercise}

\begin{exercise}\label{pgm:X:trivial}
Try running the parallel G-machine on the following program.
\begin{verbatim}
main = par (S K K) (S K K 3)
\end{verbatim}
How long does it take in machine cycles? How long does it take for the
equivalent sequential program:
\begin{verbatim}
main = S K K (S K K 3)
\end{verbatim}
\end{exercise}

\begin{exercise}\label{pgm:X:trivial1:1}
What happens when we run the program:
\begin{verbatim}
main = par I (I 3)
\end{verbatim}
Is the use of \mbox{\tt par}\indexTT{par} justified in this program?
\end{exercise}

\section{Mark~2: The evaluate-and-die model\index{parallel G-machine!Mark 2}}
\label{pgm:mark2}

A problem with the Mark~1 machine is that it risks creating many tasks
to reduce the same node in the heap, thereby duplicating the work done
by the machine. The way to prevent this is to {\em
lock\/}\index{locking, of node} the nodes during unwinding. This will
{\em block\/}\index{blocking, of tasks} any other task that encounters
the same node. We must also remember to {\em unlock\/}\index{unlocking,
of node} nodes once they become free again; this allows blocked tasks
to resume.

The only instruction that causes an unlocked node to become locked is
\mbox{\tt Unwind}\indexTT{Unwind}; it does this to each node on the spine that it encounters.
The reason we choose this instruction, rather than \mbox{\tt Eval}\indexTT{Eval}, is that it
is possible to encounter a locked node part way through unlocking a
spine; using \mbox{\tt Eval}\indexTT{Eval} we would not catch this case.  Similarly, the only
instruction that will block a task is \mbox{\tt Unwind}\indexTT{Unwind}.  After all, it is the
only instruction that needs to inspect a node's value.

Conversely, the only instruction that will unlock a locked node is
\mbox{\tt Update}\indexTT{Update}. A previously locked node is unlocked when it is known to be
in WHNF. But we know that {\em all\/} nodes in the spine below the redex
are in WHNF, when we are about to update the redex, and hence we
should unlock all nodes below the root of the redex.

\subsection{The node data structure}
\label{pgm:ss:mark2mc}

The improvements we intend to incorporate require very few changes
to be made to the machine's data structures. First, we must add two sorts of
new nodes to the node data type: these will be \mbox{\tt NLAp}\indexTT{NLAp}, the locked
application nodes; and \mbox{\tt NLGlobal}\indexTT{NLGlobal}, locked supercombinator nodes. We
will see how they are used in the section on the new instruction
transitions.

\begin{verbatim}
> data Node = NNum Int            -- Numbers
>          | NAp  Addr Addr       -- Applications
>          | NGlobal Int GmCode   -- Globals
>          | NInd Addr            -- Indirections
>          | NConstr Int [Addr]   -- Constructors
>          | NLAp Addr Addr       -- Locked applications
>          | NLGlobal Int GmCode  -- Locked globals
\end{verbatim}
\indexDTT{data}%
\indexDTT{NAp}%
\indexDTT{NGlobal}%
\indexDTT{NInd}%
\indexDTT{NConstr}%
\indexDTT{NLAp}%
\indexDTT{NLGlobal}%
\begin{exercise}\label{pgm:X:2nodeprint}
Rewrite the \mbox{\tt showNode}\indexTT{showNode} function to deal with the new locked nodes.
\end{exercise}

\subsection{The instruction set}

The only change that needs to be made is to lock and unlock nodes at
the right places. We must lock application nodes and supercombinators
with zero arguments as we unwind them. When a node is updated all of
the nodes in the spine below it must be unlocked.
We use two functions to perform locking and unlocking of heap nodes.
The \mbox{\tt lock}\indexTT{lock} function turns an unlocked, but possibly updatable, node
into a locked one. The nodes that need to be locked are application
nodes, and global nodes with no arguments.
\begin{verbatim}
> lock :: Addr -> GmState -> GmState
> lock addr state
>  = putHeap (newHeap (hLookup heap addr)) state
>    where
>    heap = getHeap state
>    newHeap (NAp a1 a2)   = hUpdate heap addr (NLAp a1 a2)
>    newHeap (NGlobal n c) | n == 0 = hUpdate heap addr (NLGlobal n c)
>                          | otherwise = heap
\end{verbatim}
\indexDTT{lock}%
\indexDTT{lock}%
\par
When we \mbox{\tt unlock}\indexTT{unlock} a locked application node we need to ensure that the
spine that it points to is also unlocked; \mbox{\tt unlock}\indexTT{unlock} is therefore recursive.
\begin{verbatim}
> unlock :: Addr -> GmState -> GmState
> unlock addr state
>  = newState (hLookup heap addr)
>    where
>    heap = getHeap state
>    newState (NLAp a1 a2)
>      = unlock a1 (putHeap (hUpdate heap addr (NAp a1 a2)) state)
>    newState (NLGlobal n c)
>      = putHeap (hUpdate heap addr (NGlobal n c)) state
>    newState n = state
\end{verbatim}
\indexDTT{unlock}%
\indexDTT{unlock}%
\par
The new step transitions for \mbox{\tt Unwind}\indexTT{Unwind} and \mbox{\tt Update}\indexTT{Update} should be defined
in terms of \mbox{\tt lock}\indexTT{lock} and \mbox{\tt unlock}\indexTT{unlock}; these transitions are now defined. In
the Rule~\ref{pgm:rule:2unwind}, we see that apart from locking the
application node -- which is represented by \mbox{\tt *NAp} -- \mbox{\tt Unwind}\indexTT{Unwind} has the
same transition as it did in the Mark~1 machine.

\pgmrule%
{\pgmstate{h[a: \mbox{\tt NAp}\indexTT{NAp}\  a_1\  a_2]}{m}{t}{[\mbox{\tt Unwind}\indexTT{Unwind}]}{a:s}}%
{\pgmstate{h[a: \mbox{\tt *NAp}\  a_1\  a_2]}{m}{t}{[\mbox{\tt Unwind}\indexTT{Unwind}]}{a_1:a:s}}
\label{pgm:rule:2unwind}
The same is also true of the transition rule for \mbox{\tt Unwind}\indexTT{Unwind} when it has
a supercombinator of arity zero on top of the stack. In this case
\mbox{\tt *NGlobal} is a locked global node.

\pgmrule%
{\pgmstate{h[a: \mbox{\tt NGlobal}\indexTT{NGlobal}\ 0\ c]}{m}{t}{[\mbox{\tt Unwind}\indexTT{Unwind}]}{a:s}}%
{\pgmstate{h[a: \mbox{\tt *NGlobal}\ 0\ c]}{m}{t}{c}{a:s}}

In the new transition rule for \mbox{\tt Update}\indexTT{Update}, when we update the root of
the redex, whose address is $a$, we must unlock all of the nodes in the
spine descending from $a$. The transition rule is therefore:

\pgmrule%
{\pgmstate{h\left[\begin{array}{lll}
a_1' & : & \mbox{\tt NGlobal}\indexTT{NGlobal}\ n\  c\\
a_2' & : & \mbox{\tt *NAp}\ a_1'\ a_1\\
\multicolumn{3}{c}{\cdots}\\
a_{n-1}' & : & \mbox{\tt *NAp}\ a_{n-2}'\ a_{n-2}\\
a_n & : & \mbox{\tt *NAp}\ a_{n-1}'\ a_{n-1}
\end{array}\right]}{m}{t}{\mbox{\tt Update}\indexTT{Update}\ n:i}{a:a_1:\ldots:a_n:s}}%
{\pgmstate{h\left[\begin{array}{lll}
a_1' & : & \mbox{\tt NGlobal}\indexTT{NGlobal}\ n\  c\\
a_2' & : & \mbox{\tt NAp}\indexTT{NAp}\ a_1'\ a_1\\
\multicolumn{3}{c}{\cdots}\\
a_{n-1}' & : & \mbox{\tt NAp}\indexTT{NAp}\ a_{n-2}'\ a_{n-2}\\
a_n & : & \mbox{\tt NInd}\indexTT{NInd}\ a
\end{array}\right]}{m}{t}{i}{a_1:\ldots:a_n:s}}

\begin{exercise}\label{pgm:X:instructions2}
Modify the definitions of the transition functions \mbox{\tt unwind}\indexTT{unwind} and
\mbox{\tt update}\indexTT{update}. You should use the \mbox{\tt lock}\indexTT{lock} and \mbox{\tt unlock}\indexTT{unlock} functions in your
code.

You will also need to `look through' locked nodes on rearranging the
stack, so \mbox{\tt getArg}\indexTT{getArg} becomes:
\begin{verbatim}
> getArg (NLAp a1 a2) = a2
\end{verbatim}
\indexDTT{getArg}%
\end{exercise}

\begin{exercise}\label{pgm:X:run2}
Try running this program on your machine:
\begin{verbatim}
main = twice' (twice' (twice' (S K K))) 3
twice' f x = par f (f x)
\end{verbatim}
\end{exercise}

\begin{exercise}\label{pgm:X:phys-limit}\index{limits to parallelism!physical}
A divide-and-conquer program executes a \mbox{\tt par}\indexTT{par} instruction in each
running process every thirty instructions. The processes do not die.
How many simulated clock ticks pass before we have one task for
each electron in the universe? (Hint: there are approximately $10^{85}$
electrons in the universe.)
\end{exercise}

\begin{exercise}\label{pgm:X:econ-limit}\index{limits to parallelism!economic}
If a processor costs \pounds 0.01, how long can the program of
Exercise~\ref{pgm:X:phys-limit} run before nobody can
afford the machine? (Hint: the US federal budget is approximately
\pounds $5\times10^{12}$.)
\end{exercise}

\section{Mark~3: A realistic parallel G-machine\index{parallel G-machine}}
\label{pgm:sc:mark3}

As Exercises \ref{pgm:X:phys-limit}~and~\ref{pgm:X:econ-limit} will
have shown there are physical and economic limitations to the amount
of parallelism available in the real world\index{limits to
parallelism}. The model of parallelism that we are using is not very
realistic. We are creating a new processor to execute each parallel
task, and in the real world we will very quickly run out of resources.

\subsection{Scheduling policy\index{scheduling!policy}}

A more realistic model must involve restricting the amount of
parallelism to that provided by the hardware. This is easily
accomplished by placing an upper limit on the number of processors
that can run at any one time. As a consequence, whenever there are no
processors available to execute a task, the task will remain unchanged.

When there are more processors than tasks, some processors will be
idle. On the other hand, when the reverse is the case, we will be
faced with the problem of deciding which task we will execute next.
This decision is called a {\em scheduling policy}\index{scheduling
policy}.

\subsection{Conservative and speculative parallelism\index{conservative parallelism}\index{speculative parallelism}}

Some tasks will be more important than others. We can usefully
classify tasks into one of two groups:
\begin{itemize}
\item tasks whose results will {\em definitely\/} be needed; and
\item tasks whose results {\em may\/} be needed.
\end{itemize}
We refer to tasks in the first category as {\em conservative
tasks}\index{conservative parallelism}, whilst those in the second are
termed {\em speculative tasks}\index{speculative parallelism}.

If we choose to allow speculative parallel tasks, then we must address
issues of {\em priority\/}\index{scheduling!priority, of tasks} in scheduling
tasks. That is: we must rank the tasks in order of importance. We
must also allow different tasks with the same priority the same amount
of computing time. To see why this is desirable, consider evaluating
two branches of a case expression -- \mbox{\tt e1} and \mbox{\tt e2} -- in parallel with
the evaluation of the discriminant expression \mbox{\tt e0}.
\begin{verbatim}
        case e0 of
             <1> -> e1
             <2> -> e2
\end{verbatim}
Until \mbox{\tt e0} has completed its evaluation, we do not know which of \mbox{\tt e1}
or \mbox{\tt e2} will be required; it therefore makes sense to try to evaluate
an equal amount of each. This sort of scheduling policy is termed
{\em fair scheduling}\index{fair scheduling, of tasks}.
Notice, in this example, that once \mbox{\tt e0} {\em has\/} completed, one of
the tasks (\mbox{\tt e1} or \mbox{\tt e2}) will become needed, and the other should be
killed. The priorities of a task therefore need to be adjusted during
the execution of a task.

In this book we make no attempt to implement a machine suitable for
speculative parallelism. We make the excuse that this is a `hard
problem', and leave the matter alone. Henceforth, all uses of the
\mbox{\tt par}\indexTT{par} primitive are assumed to occur in situations which give rise only to
conservative parallelism.

In the Mark~3 parallel G-machine, we will only have a limited number
of active tasks within the machine. These are executed by the {\em
processors\/}\index{processors} in the machine. There will be only a
fixed number of processors. This fixed number is: \mbox{\tt machineSize}\indexTT{machineSize}, which
we have currently set to \mbox{\tt 4}.
\begin{verbatim}
> machineSize :: Int
> machineSize = 4
\end{verbatim}
\indexDTT{machineSize}%
\indexDTT{machineSize}%
\par
The major change to the evaluator lies in the \mbox{\tt steps}\indexTT{steps} function, which
does not just add all of the tasks that were created into the machine.
Instead it now uses \mbox{\tt scheduler}\indexTT{scheduler} to pick replacement tasks for any task
that cannot proceed.
\begin{enumerate}
\item First, we extract the sparks in the task pool from the global
state component.
\item New tasks are then created for the sparks, and are added to
the already executing tasks.
\item The \mbox{\tt scheduler}\indexTT{scheduler} function then selects which tasks to execute.
\end{enumerate}

Here is the way we code \mbox{\tt steps}\indexTT{steps}:
\begin{verbatim}
> steps :: PgmState -> PgmState
> steps state
>  = scheduler global' local'
>    where ((out, heap, globals, sparks, stats), local) = state
>          newtasks = [makeTask a | a <- sparks]
>          global'  = (out, heap, globals, [], stats)
>          local'   = local ++ newtasks
\end{verbatim}
\indexDTT{steps}%
\indexDTT{steps}%
\par
The scheduling policy is very simple: we select the first
\mbox{\tt machineSize}\indexTT{machineSize} tasks and run them. These tasks are then placed at the
end of the scheduling queue. This scheduling policy is usually called
a {\em round-robin\/}\index{scheduling policy!round-robin} scheduling
policy.
\begin{verbatim}
> scheduler :: PgmGlobalState -> [PgmLocalState] -> PgmState
> scheduler global tasks
>  = (global', nonRunning ++ tasks')
>    where running    = map tick (take machineSize tasks)
>          nonRunning = drop machineSize tasks
>          (global', tasks') = mapAccuml step global running
\end{verbatim}
\indexDTT{scheduler}%
\indexDTT{scheduler}%
\begin{exercise}\label{pgm:X:sched1}
What happens if the tasks that are executed are not placed at the end
of the scheduling queue for the next step. (Hint: try it!)
\end{exercise}

\begin{exercise}\label{pgm:X:sched2}
One improvement we can make is to create tasks from the spark
pool only when there are idle processors. Modify \mbox{\tt steps}\indexTT{steps} to do this.
\end{exercise}

\begin{exercise}\label{pgm:X:sched3}
Another improvement is only to schedule tasks that can proceed. At
the moment we schedule tasks to evaluate nodes that are already in
WHNF. We also schedule tasks that are blocked because they are
attempting to unwind locked nodes. Modify \mbox{\tt scheduler}\indexTT{scheduler} so that this
no longer happens.
\end{exercise}

\begin{exercise}\label{pgm:X:scheduling}
Investigate the use of other scheduling strategies. For example, try
scheduling the last task in the task pool.

Does the scheduling strategy make any difference to the execution time
of the program?
\end{exercise}

\section{Mark~4: A better way to handle blocking}
\label{pgm:sc:mark4}

So far we have left blocked\index{blocking, of tasks} tasks in the
machine's local state, and required the \mbox{\tt scheduler}\indexTT{scheduler} function to select
runnable tasks. This means that the \mbox{\tt scheduler}\indexTT{scheduler} function may have to
skip over a considerable number of blocked tasks before coming across
one that it can run.

We can do better than this!
It would be a much better idea to attach a blocked task to the node
That caused it to block. Because a locked\index{locking, of node} node can
cause an arbitrary number of tasks to block, we will need to allow a
list of tasks to be placed on the locked node. We call this the {\em
pending list}\index{pending list, of node}.

How do we use the pending list?
\begin{enumerate}
\item When a node is locked, it has its pending list set to \mbox{\tt []}.
\item When a task encounters a locked node, the task places
itself on the pending list of the locked node.
\item When a locked node is unlocked, all of the tasks in its pending
list become runnable, and are transferred to the machine's local state.
\end{enumerate}

To implement the Mark~4 machine we must make the following changes to
the data structures of the machine.

\subsection{Data structures}

First, each locked node must now have a pending list; this means that
the \mbox{\tt node}\indexTT{node} data type is now:

\begin{verbatim}
> data Node = NNum Int                          -- Numbers
>          | NAp  Addr Addr                     -- Applications
>          | NGlobal Int GmCode                 -- Globals
>          | NInd Addr                          -- Indirections
>          | NConstr Int [Addr]                 -- Constructors
>          | NLAp Addr Addr PgmPendingList      -- Locked applications
>          | NLGlobal Int GmCode PgmPendingList -- Locked globals
\end{verbatim}
\indexDTT{data}%
\indexDTT{NAp}%
\indexDTT{NGlobal}%
\indexDTT{NInd}%
\indexDTT{NConstr}%
\indexDTT{NLAp}%
\indexDTT{NLGlobal}%
A pending list is just a list of tasks. The tasks in a locked node's
pending list will be those that have been blocked on the locked node.
\begin{verbatim}
> type PgmPendingList = [PgmLocalState]
\end{verbatim}
\indexDTT{type}%
\begin{exercise}\label{pgm:X:4:showNode}
Modify \mbox{\tt showNode}\indexTT{showNode} to work with the new definition of \mbox{\tt node}\indexTT{node}.
\end{exercise}

The other change is to the type of \mbox{\tt gmSparks}\indexTT{gmSparks}. Instead of being a list of
addresses -- as it was in previous parallel G-machines -- this is now
a list of tasks.
\begin{verbatim}
> type GmSparks = [PgmLocalState]
\end{verbatim}
\indexDTT{type}%
\begin{exercise}\label{pgm:X:showSparks}
Modify the \mbox{\tt par}\indexTT{par} transition so that it places tasks into the spark
pool and not addresses. You should modify \mbox{\tt showSparks}\indexTT{showSparks} function to
print the number of tasks in the spark pool, and the \mbox{\tt steps}\indexTT{steps} function
will need to be modified because it no longer needs to turn items in
the spark pool into tasks.
\end{exercise}

\subsection{Locking and unlocking}

We have been building up to a new way to handle locking and unlocking
of nodes. Let us first consider what happens when a locked node is
about to be updated. The \mbox{\tt Update}\indexTT{Update} instruction will be implemented
using a call to \mbox{\tt unlock}\indexTT{unlock}. Each node in the spine of the expression
about to be overwritten will need to have the tasks in its pending
list transferred to the spark pool. To do this we define the following
function that transfers tasks to the spark pool:
\begin{verbatim}
> emptyPendingList :: [PgmLocalState] -> GmState -> GmState
> emptyPendingList tasks state
>  = putSparks (tasks ++ getSparks state) state
\end{verbatim}
\indexDTT{emptyPendingList}%
\indexDTT{emptyPendingList}%
\begin{exercise}\label{pgm:X:unlock}
Modify the \mbox{\tt unlock}\indexTT{unlock} function so that it empties the tasks in the
pending lists into the spark pool.
\end{exercise}

The locking operation occurs as part of the \mbox{\tt Unwind}\indexTT{Unwind} instruction. As
previously, we use the \mbox{\tt lock}\indexTT{lock} function to perform the locking
operation. Now it must give each newly locked node an empty pending
list.

\begin{exercise}\label{pgm:X:lock}
Modify the \mbox{\tt lock}\indexTT{lock} function so that it gives locked nodes an empty
pending list.
\end{exercise}

Finally, we discuss what happens when a task attempts to unwind a
locked node. Clearly, we place the task onto the node's pending list.
But what do we replace the task with? Remember that the type of \mbox{\tt step}\indexTT{step}
is:
\begin{verbatim}
> step :: gmState -> gmState
\end{verbatim}
\indexDTT{step}%
The solution we have adopted is to replace the task with an
\mbox{\tt emptyTask}\indexTT{emptyTask}:
\begin{verbatim}
> emptyTask :: PgmLocalState
> emptyTask = ([], [], [], [], 0)
\end{verbatim}
\indexDTT{emptyTask}%
\indexDTT{emptyTask}%
\par
So we need two new transitions for \mbox{\tt Unwind}\indexTT{Unwind}. We begin with the one for
locked application nodes, in which we see that the task is placed on
the node's pending list and we see that the task is replaced by the
\mbox{\tt emptyTask}\indexTT{emptyTask}.

\refstepcounter{rulenumber}
    \begin{flushleft}
        \hspace{1em} (\therulenumber) \hspace{1em}
       $\begin{array}{|l@{~ \langle }lll@{\rangle ~ }rr|}
       \hline
&h[a: \mbox{\tt *NAp}\  a_1\  a_2\ pl]& m& t & \langle [\mbox{\tt Unwind}\indexTT{Unwind}] & a:s\rangle\\
\Longrightarrow &
h[a: \mbox{\tt *NAp}\  a_1\  a_2\ \langle [\mbox{\tt Unwind}\indexTT{Unwind}],\,a:s\rangle:pl] &
  {m} & t & \multicolumn{2}{l|}{\mbox{\tt emptyTask}\indexTT{emptyTask}}\\ \hline
\end{array}$
\end{flushleft}

The rule for locked global nodes is similar: we see that the task is
placed onto the node's pending list, and is itself replaced by the
\mbox{\tt emptyTask}\indexTT{emptyTask}.

\refstepcounter{rulenumber}
    \begin{flushleft}
        \hspace{1em} (\therulenumber) \hspace{1em}
       $\begin{array}{|l@{~ \langle }lll@{\rangle ~ }rr|}
       \hline &
h[a: \mbox{\tt *NGlobal}\ 0\ c\ pl]& m&t &\langle [\mbox{\tt Unwind}\indexTT{Unwind}] & a:s\rangle \\
\Longrightarrow &
h[a: \mbox{\tt *NGlobal}\ 0\ c\ \langle [\mbox{\tt Unwind}\indexTT{Unwind}],\,a:s\rangle:pl] &
  {m} & {t} & \multicolumn{2}{l|}{\mbox{\tt emptyTask}\indexTT{emptyTask}}\\ \hline
\end{array}$
\end{flushleft}

\begin{exercise}\label{pgm:X:4:unwind}
Modify the \mbox{\tt unwind}\indexTT{unwind} function to implement the new transitions for the
\mbox{\tt Unwind}\indexTT{Unwind} instruction. You will also need to make the \mbox{\tt getArg}\indexTT{getArg} function:
\begin{verbatim}
> getArg (NLAp a1 a2 pl) = a2
\end{verbatim}
\indexDTT{getArg}%
\end{exercise}

\begin{exercise}\label{pgm:X:4:scheduler}
Modify the \mbox{\tt scheduler}\indexTT{scheduler} function to place non-running tasks into the
spark pool.
\end{exercise}

\begin{exercise}\label{pgm:X:4:doAdmin}
Modify the \mbox{\tt doAdmin}\indexTT{doAdmin} function to filter out \mbox{\tt emptyTask}\indexTT{emptyTask}'s from the
local state.
\end{exercise}

\section{Conclusions}

This chapter has shown that, in principle, a shared memory
implementation of lazy functional languages is straightforward.  Of
course, we have also seen that there are payoffs to be had by carefully
considering optimisations to the simple scheme we used initially in
the Mark~1 machine.
In all of our parallel machines, the graph acts as a communication and
synchronisation medium; and in the Mark~2 and Mark~3 machines,
individual processes will be blocked when trying to access locked
nodes in the heap.

So where are the current challenges in the parallel implementation of
lazy functional languages?
The mechanisms for parallelism included in this book do not handle the
deletion of processes. If speculative parallelism is going to be used
then realistic implementations will need to deal with this problem.
On the other hand, finding the non-speculative parallelism is often
difficult, and in large programs this may even be intractable.
Attempts have been made to use abstract interpretation for this
purpose, and although the results look promising, they should be
regarded tentatively.

One final area that we have not covered is that of distributed memory
parallel machines. Again, in principle they are similar to shared
memory machines, but the practicalities are quite different.
Arranging the message passing so as to avoid deadlock is something of
a black art.

\theendnotes

% end of chap05
r