%\chapter{Related Work}
%
%Lots of work trying to preserve the security of a source languages when compiling exists.
%The idea of using full abstraction to formalize secure compilation is introduced by Abadi~\cite{Abadi}.
%Different techniques to achieve this were developed, for example using \emph{Adress Space Layout Randomisation} or ASLR.% introduced by Abadi~\cite{AbadiASLR} as well.
%The idea of ASLR catched on, and ASLR saw implementations in common operating systems such as Windows Vista, OS X Mountain Lion and some Linux distributions. 
%The idea also raised scientific study, for example by Abadi and Plotkin~\cite{AbadiASLR} or Jagadeesan, et al.\cite{Jagadeesan} and criticism~\cite{Shacham:2004:EAR:1030083.1030124,Strackx:2009:BMS:1519144.1519145}.
%
%Other techniques work by introducing security guarantees to memory access.
%For example, Agten et al.~\cite{Agten:2012:SCM:2354412.2355247} already provide a secure compilation scheme for an object based language, when access to memory is restricted based on the value of the program counter. This technique is called \emph{Program Counter Based Access Control}, or \emph{PCBAC}~\cite{PCBAC}.
%Later work by Patrignani et al.~\cite{Patrignani} introduced additional object oriented to the fully abstract compilation scheme.
%
%The restricted access of memory can be implemented in hardware\cite{Sancus,SGX} or using software\cite{Fides,Salus}.
%This choice effects the size of the trusted computing base or \emph{TCB}.
%Even with fully abstract compilation, security issues in the TCB could lead to low-level attaques.
%A recent innovation in restricting access on a hardway level is Intel\textregistered Software Guard Extensions, or \emph{SGX}~\cite{SGX}.