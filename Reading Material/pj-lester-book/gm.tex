% $Date: 92/01/13 16:22:55 $
% $Revision: 1.21 $
% (c) 1991 Simon Peyton Jones & David Lester.
% Note: some definitions for functions have been indented diffrently
%       from the original file, because otherwise gofer would reject them.
\begin{verbatim}
> module GM where
> import Language
> import Utils
\end{verbatim}
\indexDTT{module}%
\indexDTT{import}%
\indexDTT{import}%
\chapter{The G-machine\index{G-machine}}
\label{sect:g-machine}

In this chapter we introduce our first compiler-based implementation,
the G-machine, which was developed at the Chalmers
Institute of Technology, G\"oteborg, Sweden, by Augustsson and
Johnsson. The material in this chapter is based on their series of
papers \cite{Augustsson:LFP84,Johnsson:CC84} culminating in their Ph.D.
theses \cite{Augustsson:thesis87,Johnsson:thesis87}.

\section{Introduction to the G-machine}

The fundamental operation of the template instantiation machine was to
construct an instance of a supercombinator body, implemented by the
\mbox{\tt instantiate}\indexTT{instantiate} function. This is a rather slow operation, because
\mbox{\tt instantiate}\indexTT{instantiate} must recursively traverse the template {\em each time an
instantiation is performed}. When we think of the machine instructions
that are executed by \mbox{\tt instantiate}\indexTT{instantiate}, we see that they will be of two
kinds: those concerned with traversing the template, and those
concerned with actually constructing the instance.

The `Big Idea' of the G-machine\index{G-machine!the Big Idea}, and other
compiled implementations, is this:
\begin{important}
Before running the program, translate each supercombinator body to a
sequence of instructions which, when executed, will construct an
instance of the supercombinator body.
\end{important}

Executing this code should be faster than calling an instantiation
function, because all the instructions are concerned with constructing
the instance.  There are no instructions required to traverse the
template, because all that has been done during the translation
process. Running a program is thereby split into two stages. In the
first stage a compiler is used to produce some intermediate form of the
program; this is referred to as {\em compile-time}\index{compile-time}.
 In the second stage the intermediate form is executed; this is called
{\em run-time}\index{run-time}.

Since all we ever do to a supercombinator is to instantiate it, we can
discard the original supercombinators once the translation is done,
keeping only the compiled code.

In principle, then, we use a {\em G-machine compiler\/}\index{G-machine
compiler} to turn a program in our source language into a sequence of
{\em machine language instructions}\index{machine language
instructions}. Because we may wish to implement our language on many
different pieces of hardware (68000 based, or VAX, etc.) it is
useful to have an {\em abstract machine}\index{abstract machine}. A
good abstract machine has two properties: firstly, it can be easily
translated into any concrete machine code (for example 68000
assembler); secondly, it is easy to generate the abstract machine code
from the source.

Notice that we are faced with a trade-off here. We can ideally satisfy
the first property (easy concrete code generation) by making the
abstract machine the {\em same\/} as the real machine. But this makes
the second property much harder to fulfil. An abstract machine is
therefore a stepping-stone between the source language and a
particular machine code.

\subsection{An example}

Here is a small example of the G-machine compiler in
action\index{Example execution!of G-machine}. Consider the function
\begin{verbatim}
        f g x = K (g x)
\end{verbatim}
This would be compiled to the sequence of G-code instructions:
\begin{verbatim}
        Push 1
        Push 1
        Mkap
        Pushglobal K
        Mkap
        Slide 3
        Unwind
\end{verbatim}
In Figure~\ref{gm:fg:1example}, we show how this code will execute. On
the left-hand side of each diagram is the stack, which grows
downwards. The remainder of each diagram is the heap. The application
nodes are represented by an @ character, expressions are labelled
with lower-case letters, and supercombinators are labelled with
upper-case letters.
%\begin{figure}[htbp]
\begin{figure} % \raggedright
\setlength{\unitlength}{59204sp}%
%
\begingroup\makeatletter\ifx\SetFigFont\undefined%
\gdef\SetFigFont#1#2#3#4#5{%
  \reset@font\fontsize{#1}{#2pt}%
  \fontfamily{#3}\fontseries{#4}\fontshape{#5}%
  \selectfont}%
\fi\endgroup%
\begin{picture}(464,125)(18,702)
\thinlines
\multiput(455,785)(-0.25000,-0.50000){21}{\makebox(0.1111,0.7778){\SetFigFont{5}{6}{\rmdefault}{\mddefault}{\updefault}.}}
\put(465,805){\line(-1,-3){  5}}
\multiput(340,785)(0.25000,-0.50000){21}{\makebox(0.1111,0.7778){\SetFigFont{5}{6}{\rmdefault}{\mddefault}{\updefault}.}}
\put(345,805){\line(-1,-3){  5}}
\multiput(335,785)(-0.25000,-0.50000){21}{\makebox(0.1111,0.7778){\SetFigFont{5}{6}{\rmdefault}{\mddefault}{\updefault}.}}
\multiput(220,780)(0.25000,-0.50000){21}{\makebox(0.1111,0.7778){\SetFigFont{5}{6}{\rmdefault}{\mddefault}{\updefault}.}}
\multiput(215,785)(-0.25000,-0.50000){21}{\makebox(0.1111,0.7778){\SetFigFont{5}{6}{\rmdefault}{\mddefault}{\updefault}.}}
\multiput(225,805)(-0.25000,-0.50000){21}{\makebox(0.1111,0.7778){\SetFigFont{5}{6}{\rmdefault}{\mddefault}{\updefault}.}}
\put(380,760){\line( 1, 0){ 40}}
\put(380,825){\line( 0,-1){ 85}}
\put(380,740){\line( 1, 0){ 40}}
\put(420,740){\line( 0, 1){ 85}}
\put(400,750){\vector( 1, 0){ 40}}
\put(400,770){\vector( 1, 0){ 35}}
\put(280,730){\line( 1, 0){ 70}}
\put(350,730){\vector( 0, 1){ 30}}
\put(280,750){\line( 1, 0){ 80}}
\put(360,750){\vector( 0, 1){ 30}}
\put(280,770){\vector( 1, 0){ 40}}
\put(260,740){\line( 1, 0){ 40}}
\put(260,825){\line( 0,-1){105}}
\put(260,720){\line( 1, 0){ 40}}
\put(300,720){\line( 0, 1){105}}
\put(140,760){\line( 1, 0){ 40}}
\put(140,825){\line( 0,-1){ 85}}
\put(140,740){\line( 1, 0){ 40}}
\put(180,740){\line( 0, 1){ 85}}
\put(160,750){\line( 1, 0){ 80}}
\put(240,750){\vector( 0, 1){ 30}}
\put(160,770){\vector( 1, 0){ 40}}
\put( 40,770){\vector( 1, 0){ 40}}
\multiput( 95,780)(-0.25000,-0.50000){21}{\makebox(0.1111,0.7778){\SetFigFont{5}{6}{\rmdefault}{\mddefault}{\updefault}.}}
\multiput(100,780)(0.25000,-0.50000){21}{\makebox(0.1111,0.7778){\SetFigFont{5}{6}{\rmdefault}{\mddefault}{\updefault}.}}
\multiput(110,800)(0.25000,-0.50000){21}{\makebox(0.1111,0.7778){\SetFigFont{5}{6}{\rmdefault}{\mddefault}{\updefault}.}}
\multiput(105,800)(-0.25000,-0.50000){21}{\makebox(0.1111,0.7778){\SetFigFont{5}{6}{\rmdefault}{\mddefault}{\updefault}.}}
\put( 20,780){\line( 1, 0){ 40}}
\put( 20,825){\line( 0,-1){ 65}}
\put( 20,760){\line( 1, 0){ 40}}
\put( 60,760){\line( 0, 1){ 65}}
\put(460,785){\line( 1,-2){ 10}}
\put(470,805){\line( 2,-3){ 10}}
\put(350,805){\line( 2,-3){ 10}}
\put(230,805){\line( 2,-3){ 10}}
\multiput(450,745)(0.41667,-0.41667){13}{\makebox(0.1111,0.7778){\SetFigFont{5}{6}{\rmdefault}{\mddefault}{\updefault}.}}
\put(455,740){\line( 1, 0){ 25}}
\put(480,740){\vector( 0, 1){ 40}}
\multiput(440,745)(-0.40000,-0.40000){26}{\makebox(0.1111,0.7778){\SetFigFont{5}{6}{\rmdefault}{\mddefault}{\updefault}.}}
\put(430,735){\line( 0,-1){  5}}
\put(430,730){\line( 1, 0){ 40}}
\put(470,730){\vector( 0, 1){ 25}}
\put(400,790){\vector( 1, 0){ 50}}
\put(400,810){\vector( 1, 0){ 60}}
\put(280,790){\vector( 1, 0){ 50}}
\put(280,810){\vector( 1, 0){ 60}}
\put(160,790){\vector( 1, 0){ 50}}
\put(160,810){\vector( 1, 0){ 60}}
\put(380,820){\line( 1, 0){ 40}}
\put(380,800){\line( 1, 0){ 40}}
\put(380,780){\line( 1, 0){ 40}}
\put(260,760){\line( 1, 0){ 40}}
\put(260,780){\line( 1, 0){ 40}}
\put(260,800){\line( 1, 0){ 40}}
\put(260,820){\line( 1, 0){ 40}}
\put(140,780){\line( 1, 0){ 40}}
\put(140,800){\line( 1, 0){ 40}}
\put(140,820){\line( 1, 0){ 40}}
\put( 40,790){\vector( 1, 0){ 50}}
\put( 40,810){\vector( 1, 0){ 60}}
\put( 20,800){\line( 1, 0){ 40}}
\put( 20,820){\line( 1, 0){ 40}}
\put(385,705){\makebox(0,0)[lb]{\smash{\SetFigFont{11}{13.2}{\familydefault}{\mddefault}{\updefault}(d) Mkap}}}
\put(265,705){\makebox(0,0)[lb]{\smash{\SetFigFont{11}{13.2}{\familydefault}{\mddefault}{\updefault}(c) Push 1}}}
\put(145,705){\makebox(0,0)[lb]{\smash{\SetFigFont{11}{13.2}{\familydefault}{\mddefault}{\updefault}(b) Push 1}}}
\put( 25,705){\makebox(0,0)[lb]{\smash{\SetFigFont{11}{13.2}{\familydefault}{\mddefault}{\updefault}(a)}}}
\put(445,765){\makebox(0,0)[lb]{\smash{\SetFigFont{11}{13.2}{\familydefault}{\mddefault}{\updefault}f}}}
\put(325,765){\makebox(0,0)[lb]{\smash{\SetFigFont{11}{13.2}{\familydefault}{\mddefault}{\updefault}f}}}
\put(205,765){\makebox(0,0)[lb]{\smash{\SetFigFont{11}{13.2}{\familydefault}{\mddefault}{\updefault}f}}}
\put(445,750){\makebox(0,0)[lb]{\smash{\SetFigFont{11}{13.2}{\familydefault}{\mddefault}{\updefault}@}}}
\put( 85,765){\makebox(0,0)[lb]{\smash{\SetFigFont{11}{13.2}{\familydefault}{\mddefault}{\updefault}f}}}
\put(465,765){\makebox(0,0)[lb]{\smash{\SetFigFont{11}{13.2}{\familydefault}{\mddefault}{\updefault}g}}}
\put(345,765){\makebox(0,0)[lb]{\smash{\SetFigFont{11}{13.2}{\familydefault}{\mddefault}{\updefault}g}}}
\put(225,765){\makebox(0,0)[lb]{\smash{\SetFigFont{11}{13.2}{\familydefault}{\mddefault}{\updefault}g}}}
\put(105,765){\makebox(0,0)[lb]{\smash{\SetFigFont{11}{13.2}{\familydefault}{\mddefault}{\updefault}g}}}
\put(475,785){\makebox(0,0)[lb]{\smash{\SetFigFont{11}{13.2}{\familydefault}{\mddefault}{\updefault}x}}}
\put(355,785){\makebox(0,0)[lb]{\smash{\SetFigFont{11}{13.2}{\familydefault}{\mddefault}{\updefault}x}}}
\put(235,785){\makebox(0,0)[lb]{\smash{\SetFigFont{11}{13.2}{\familydefault}{\mddefault}{\updefault}x}}}
\put(115,785){\makebox(0,0)[lb]{\smash{\SetFigFont{11}{13.2}{\familydefault}{\mddefault}{\updefault}x}}}
\put( 95,785){\makebox(0,0)[lb]{\smash{\SetFigFont{11}{13.2}{\familydefault}{\mddefault}{\updefault}@}}}
\put(215,785){\makebox(0,0)[lb]{\smash{\SetFigFont{11}{13.2}{\familydefault}{\mddefault}{\updefault}@}}}
\put(335,785){\makebox(0,0)[lb]{\smash{\SetFigFont{11}{13.2}{\familydefault}{\mddefault}{\updefault}@}}}
\put(455,785){\makebox(0,0)[lb]{\smash{\SetFigFont{11}{13.2}{\familydefault}{\mddefault}{\updefault}@}}}
\put(465,805){\makebox(0,0)[lb]{\smash{\SetFigFont{11}{13.2}{\familydefault}{\mddefault}{\updefault}@}}}
\put(345,805){\makebox(0,0)[lb]{\smash{\SetFigFont{11}{13.2}{\familydefault}{\mddefault}{\updefault}@}}}
\put(225,805){\makebox(0,0)[lb]{\smash{\SetFigFont{11}{13.2}{\familydefault}{\mddefault}{\updefault}@}}}
\put(105,805){\makebox(0,0)[lb]{\smash{\SetFigFont{11}{13.2}{\familydefault}{\mddefault}{\updefault}@}}}
\end{picture}


\vspace{0.25in}

\setlength{\unitlength}{59204sp}%
%
\begingroup\makeatletter\ifx\SetFigFont\undefined%
\gdef\SetFigFont#1#2#3#4#5{%
  \reset@font\fontsize{#1}{#2pt}%
  \fontfamily{#3}\fontseries{#4}\fontshape{#5}%
  \selectfont}%
\fi\endgroup%
\begin{picture}(349,165)(18,662)
\thinlines
\put(260,825){\line( 0,-1){ 25}}
\put(260,800){\line( 1, 0){ 40}}
\put(300,800){\line( 0, 1){ 25}}
\put( 40,730){\vector( 1, 0){ 25}}
\put( 40,750){\vector( 1, 0){ 40}}
\put(195,740){\line(-2,-3){ 10}}
\multiput(200,745)(0.25000,-0.50000){21}{\makebox(0.1111,0.7778){\SetFigFont{5}{6}{\rmdefault}{\mddefault}{\updefault}.}}
\put(225,805){\line(-1,-2){ 15}}
\put(105,805){\line(-1,-2){ 15}}
\put(160,750){\vector( 1, 0){ 30}}
\put(160,770){\vector( 1, 0){ 40}}
\put( 40,770){\vector( 1, 0){ 40}}
\put( 20,825){\line( 0,-1){105}}
\put( 20,720){\line( 1, 0){ 40}}
\put( 60,720){\line( 0, 1){105}}
\put(140,760){\line( 1, 0){ 40}}
\put(140,825){\line( 0,-1){ 85}}
\put(140,740){\line( 1, 0){ 40}}
\put(180,740){\line( 0, 1){ 85}}
\put( 20,740){\line( 1, 0){ 40}}
\put( 20,760){\line( 1, 0){ 40}}
\put( 20,820){\line( 1, 0){ 40}}
\put( 20,800){\line( 1, 0){ 40}}
\put( 20,780){\line( 1, 0){ 40}}
\put(140,800){\line( 1, 0){ 40}}
\put(140,780){\line( 1, 0){ 40}}
\put(140,820){\line( 1, 0){ 40}}
\put(260,820){\line( 1, 0){ 40}}
\put( 40,810){\vector( 1, 0){ 60}}
\put( 40,790){\vector( 1, 0){ 50}}
\put(160,790){\vector( 1, 0){ 50}}
\put(160,810){\vector( 1, 0){ 60}}
\multiput( 85,745)(-0.41667,-0.41667){13}{\makebox(0.1111,0.7778){\SetFigFont{5}{6}{\rmdefault}{\mddefault}{\updefault}.}}
\put( 80,740){\line( 0,-1){ 15}}
\put( 80,725){\line( 1, 0){ 30}}
\put(110,725){\vector( 0, 1){ 30}}
\multiput( 95,745)(0.25000,-0.50000){21}{\makebox(0.1111,0.7778){\SetFigFont{5}{6}{\rmdefault}{\mddefault}{\updefault}.}}
\put(100,735){\line( 1, 0){ 20}}
\put(120,735){\vector( 0, 1){ 45}}
\multiput(215,720)(0.25000,-0.50000){21}{\makebox(0.1111,0.7778){\SetFigFont{5}{6}{\rmdefault}{\mddefault}{\updefault}.}}
\put(220,710){\line( 1, 0){ 20}}
\put(240,710){\vector( 0, 1){ 70}}
\multiput(205,720)(-0.41667,-0.41667){13}{\makebox(0.1111,0.7778){\SetFigFont{5}{6}{\rmdefault}{\mddefault}{\updefault}.}}
\put(200,715){\line( 0,-1){ 10}}
\put(200,705){\line( 1, 0){ 30}}
\put(230,705){\vector( 0, 1){ 55}}
\put(280,810){\vector( 1, 0){ 60}}
\multiput(345,805)(-0.25000,-0.50000){21}{\makebox(0.1111,0.7778){\SetFigFont{5}{6}{\rmdefault}{\mddefault}{\updefault}.}}
\multiput(350,805)(0.25000,-0.50000){21}{\makebox(0.1111,0.7778){\SetFigFont{5}{6}{\rmdefault}{\mddefault}{\updefault}.}}
\multiput(355,785)(-0.25000,-0.50000){21}{\makebox(0.1111,0.7778){\SetFigFont{5}{6}{\rmdefault}{\mddefault}{\updefault}.}}
\multiput(360,785)(0.25000,-0.50000){21}{\makebox(0.1111,0.7778){\SetFigFont{5}{6}{\rmdefault}{\mddefault}{\updefault}.}}
\put(220,785){\line( 1,-3){  5}}
\put(230,805){\line( 1,-3){  5}}
\put(110,805){\line( 1,-2){ 10}}
\put(100,785){\line( 1,-2){ 10}}
\put(265,665){\makebox(0,0)[lb]{\smash{\SetFigFont{11}{13.2}{\familydefault}{\mddefault}{\updefault}(g) Slide 3}}}
\put(145,665){\makebox(0,0)[lb]{\smash{\SetFigFont{11}{13.2}{\familydefault}{\mddefault}{\updefault}(f) Mkap}}}
\put( 25,665){\makebox(0,0)[lb]{\smash{\SetFigFont{11}{13.2}{\familydefault}{\mddefault}{\updefault}(e) Pushglobal K}}}
\put(175,710){\makebox(0,0)[lb]{\smash{\SetFigFont{11}{13.2}{\familydefault}{\mddefault}{\updefault}K}}}
\put( 65,725){\makebox(0,0)[lb]{\smash{\SetFigFont{11}{13.2}{\familydefault}{\mddefault}{\updefault}K}}}
\put(195,745){\makebox(0,0)[lb]{\smash{\SetFigFont{11}{13.2}{\familydefault}{\mddefault}{\updefault}@}}}
\put(205,765){\makebox(0,0)[lb]{\smash{\SetFigFont{11}{13.2}{\familydefault}{\mddefault}{\updefault}f}}}
\put( 85,765){\makebox(0,0)[lb]{\smash{\SetFigFont{11}{13.2}{\familydefault}{\mddefault}{\updefault}f}}}
\put(105,805){\makebox(0,0)[lb]{\smash{\SetFigFont{11}{13.2}{\familydefault}{\mddefault}{\updefault}@}}}
\put(225,805){\makebox(0,0)[lb]{\smash{\SetFigFont{11}{13.2}{\familydefault}{\mddefault}{\updefault}@}}}
\put(345,805){\makebox(0,0)[lb]{\smash{\SetFigFont{11}{13.2}{\familydefault}{\mddefault}{\updefault}@}}}
\put(115,785){\makebox(0,0)[lb]{\smash{\SetFigFont{11}{13.2}{\familydefault}{\mddefault}{\updefault}x}}}
\put(235,785){\makebox(0,0)[lb]{\smash{\SetFigFont{11}{13.2}{\familydefault}{\mddefault}{\updefault}x}}}
\put( 95,785){\makebox(0,0)[lb]{\smash{\SetFigFont{11}{13.2}{\familydefault}{\mddefault}{\updefault}@}}}
\put(215,785){\makebox(0,0)[lb]{\smash{\SetFigFont{11}{13.2}{\familydefault}{\mddefault}{\updefault}@}}}
\put(225,765){\makebox(0,0)[lb]{\smash{\SetFigFont{11}{13.2}{\familydefault}{\mddefault}{\updefault}g}}}
\put(105,765){\makebox(0,0)[lb]{\smash{\SetFigFont{11}{13.2}{\familydefault}{\mddefault}{\updefault}g}}}
\put( 85,750){\makebox(0,0)[lb]{\smash{\SetFigFont{11}{13.2}{\familydefault}{\mddefault}{\updefault}@}}}
\put(205,725){\makebox(0,0)[lb]{\smash{\SetFigFont{11}{13.2}{\familydefault}{\mddefault}{\updefault}@}}}
\put(335,785){\makebox(0,0)[lb]{\smash{\SetFigFont{11}{13.2}{\familydefault}{\mddefault}{\updefault}K}}}
\put(355,785){\makebox(0,0)[lb]{\smash{\SetFigFont{11}{13.2}{\familydefault}{\mddefault}{\updefault}@}}}
\put(345,765){\makebox(0,0)[lb]{\smash{\SetFigFont{11}{13.2}{\familydefault}{\mddefault}{\updefault}g}}}
\put(365,765){\makebox(0,0)[lb]{\smash{\SetFigFont{11}{13.2}{\familydefault}{\mddefault}{\updefault}x}}}
\end{picture}

\caption{Execution of code for the \mbox{\tt f} supercombinator}\label{gm:fg:1example}
\end{figure}

In Figure~\ref{gm:fg:1example}, diagram (a), we see the state
of the machine before executing the sequence of instructions for \mbox{\tt f}.
The spine has been unwound, just as it was in the template machine.
The top two items on the stack are pointers to the application nodes,
whose right-hand parts are the expressions to be bound for \mbox{\tt g} and
\mbox{\tt x}.

The \mbox{\tt Push}\indexTT{Push} instruction uses addressing relative to the top of the
stack. Ignoring the pointer to the supercombinator node \mbox{\tt f}, the first
stack item is numbered \mbox{\tt 0}, the next is numbered \mbox{\tt 1} and so on. The
next diagram (b) shows the changed stack, after executing a \mbox{\tt Push\ 1}
instruction.  This pushes a pointer to the expression \mbox{\tt x} onto the
stack, \mbox{\tt x} being two stack items down the stack. After another \mbox{\tt Push\ 1}
we have a pointer to \mbox{\tt g} on top of the stack; again this is two
stack items down the stack, because the previous instruction pushed a
new pointer onto the stack.  The new diagram is (c).

Diagram (d) shows what happens when a \mbox{\tt Mkap}\indexTT{Mkap} instruction is
executed.  It takes two pointers from the stack and makes an
application node from them; leaving a pointer to the result on the
stack. In diagram (e) we execute a \mbox{\tt Pushglobal\ K} instruction,
with the effect of pushing a pointer to the \mbox{\tt K} supercombinator.
Another \mbox{\tt Mkap}\indexTT{Mkap} instruction completes the instantiation of the body of
\mbox{\tt f}, as shown in diagram (f).

We can now replace the original expression, \mbox{\tt f\ g\ x}, with the newly
instantiated body: \mbox{\tt K\ (g\ x)}. In the first version of the G-machine
-- which is not lazy -- we simply slide the body down three places on the
stack, discarding the three pointers that were there. This is achieved
by using a \mbox{\tt Slide\ 3} instruction, as shown in diagram (g). The final
\mbox{\tt Unwind}\indexTT{Unwind} instruction
will cause the machine to continue to evaluate.

This concludes a brief overview of the execution of the G-machine.

\subsection{Further optimisations}

A modest performance gain can be achieved by eliminating the
interpretive overhead of traversing the template\index{interpretive
overhead!template traversal}, as we have discussed.  However, it turns
out that compilation also opens the door to a whole host of short-cuts
and optimisations which are simply not available to the
template instantiation machine. For example, consider the following
definition:
\begin{verbatim}
        f x = x + x
\end{verbatim}
The template machine would evaluate \mbox{\tt x} twice; on the second occasion
it would of course find that it was already evaluated.  A compiled
implementation can spot at compile-time that \mbox{\tt x} will already be
evaluated, and omit the evaluation step.

\section{Code sequences for building templates}
\label{background-gm}

We recall that the template instantiator operates in the following
way:
\begin{itemize}

\item The machine has {\em terminated\/}\index{termination condition!of
G-machine} when the single item on top of the stack is a pointer to an
integer.

\item If this is not the case then we {\em unwind\/}\index{unwind!in
G-machine} any application nodes we come across until we reach a
supercombinator node. We then {\em instantiate\/}\index{instantiation!in
G-machine} a copy of the supercombinator body, making substitutions
for its arguments.

\end{itemize}

At the heart of the Mark~1 template machine are the two functions
\mbox{\tt scStep}\indexTT{scStep}~and~\mbox{\tt instantiate}\indexTT{instantiate}, which are defined on pages
\pageref{page:sc-step}~and~\pageref{page:instantiate}. If we take a
look at the definitions of \mbox{\tt scStep}\indexTT{scStep} and \mbox{\tt instantiate}\indexTT{instantiate}, we can give the
following description to the operation of instantiating a
supercombinator:

\begin{enumerate}

\item Construct a {\em local environment\/}\index{local environment!of
G-machine} of variable names to addresses in the heap.

\item Using this local environment, make an instance of the
supercombinator body in the heap.  Variables are not copied; instead
the corresponding address is used.

\item Remove the pointers to the application nodes and the
supercombinator node from the stack.

\item Push the address of the newly created instance of the
supercombinator onto the stack.

\end{enumerate}

In the template instantiator, making an instance of a supercombinator
involves traversing the tree structure of the expression which is the
body of the supercombinator.  Because expressions are defined
recursively, the tree-traversal function \mbox{\tt instantiate}\indexTT{instantiate} is defined
recursively. For example, look at the definition of \mbox{\tt instantiate}\indexTT{instantiate} -- on
page~\pageref{page:instantiate} -- for the case of \mbox{\tt EAp}\indexTT{EAp}~\mbox{\tt e1}~\mbox{\tt e2}. First
we call \mbox{\tt instantiate}\indexTT{instantiate} for \mbox{\tt e1} and then for \mbox{\tt e2}, holding on to the
addresses of the graph for each sub-expression.  Finally we combine
the two addresses by building an application node in the graph.

We would like to compile a {\em linear sequence of
instructions\/}\index{linear sequence of instructions} to perform the
operation of instantiating an expression.

\subsection{Postfix evaluation of arithmetic\index{postfix
evalution!of arithmetic expressions}}

The desire to construct a linear sequence of instructions to
instantiate an expression is reminiscent of the postfix evaluation of
arithmetic expressions. We explore this analogy further before
returning to the G-machine.

The language of arithmetic expressions consists of: numbers, addition
and multiplication.  We can represent this language as the type
\mbox{\tt aExpr}\indexTT{aExpr}.
\begin{verbatim}
> data AExpr         = Num Int
>            | Plus AExpr AExpr
>            | Mult AExpr AExpr
\end{verbatim}
\indexDTT{data}%
\indexDTT{Plus}%
\indexDTT{Mult}%
It is intended that the language should have an `obvious' meaning;
we can give this using the function \mbox{\tt aInterpret}\indexTT{aInterpret}.
\begin{verbatim}
> aInterpret :: AExpr -> Int
> aInterpret (Num n)      = n
> aInterpret (Plus e1 e2) = aInterpret e1 + aInterpret e2
> aInterpret (Mult e1 e2) = aInterpret e1 * aInterpret e2
\end{verbatim}
\indexDTT{aInterpret}%
\indexDTT{aInterpret}%
\indexDTT{aInterpret}%
\indexDTT{aInterpret}%
Alternatively, we can compile the expression into a postfix sequence
of operators (or instructions). To evaluate the expression we use the
compiled operators and a stack of values. For example, the arithmetic
expression $2+3\times 4$ would be represented as the sequence
\[
[\mbox{\tt INum}\indexTT{INum}~2,\,\mbox{\tt INum}\indexTT{INum}~3,\,\mbox{\tt INum}\indexTT{INum}~4,\,\mbox{\tt IMult}\indexTT{IMult},\,\mbox{\tt IPlus}\indexTT{IPlus}]
\]
We can give the instructions for our postfix machine as the type
\mbox{\tt aInstruction}\indexTT{aInstruction}.
\begin{verbatim}
> data AInstruction  = INum Int
>                    | IPlus
>                    | IMult
\end{verbatim}
\indexDTT{data}%
\indexDTT{IPlus}%
\indexDTT{IMult}%
\par
The state of the evaluator is a pair, which is a sequence of operators
and a stack of numbers. The meaning of a code sequence is then given
in the following transition rules.

\aerule{\aestate{[]}{[n]}}{\aestate{{ }}{n}}
\aerule{\aestate{\mbox{\tt INum}\indexTT{INum}~n:i}{ns}}{\aestate{i}{n:ns}}
\aerule{\aestate{\mbox{\tt IPlus}\indexTT{IPlus}:i}{n_0:n_1:ns}}{\aestate{i}{(n_1+n_0):ns}}
\aerule{\aestate{\mbox{\tt IMult}\indexTT{IMult}~:i}{n_0:n_1:ns}}{\aestate{i}{(n_1\times n_0):ns}}

Translating these transition rules into Miranda gives:
\begin{verbatim}
> aEval :: ([AInstruction], [Int]) -> Int
> aEval ([],        [n])     = n
> aEval (INum n:is, s)       = aEval (is, n:    s)
> aEval (IPlus: is, n0:n1:s) = aEval (is, n1+n0:s)
> aEval (IMult: is, n0:n1:s) = aEval (is, n1*n0:s)
\end{verbatim}
\indexDTT{aEval}%
\indexDTT{aEval}%
\indexDTT{aEval}%
\indexDTT{aEval}%
\indexDTT{aEval}%
\par
To generate the sequence of postfix code for an expression we must
define a compiler. This takes an expression and delivers a sequence of
instructions, which when executed will compute the value of the
expression.
\begin{verbatim}
> aCompile :: AExpr -> [AInstruction]
> aCompile (Num n)      = [INum n]
> aCompile (Plus e1 e2) = aCompile e1 ++ aCompile e2 ++ [IPlus]
> aCompile (Mult e1 e2) = aCompile e1 ++ aCompile e2 ++ [IMult]
\end{verbatim}
\indexDTT{aCompile}%
\indexDTT{aCompile}%
\indexDTT{aCompile}%
\indexDTT{aCompile}%
The key idea from this is given by the type of the \mbox{\tt aCompile}\indexTT{aCompile} function.
It returns a list of instructions.

\begin{important}
The postfix representation of expressions is a way of {\em
flattening}\index{flattening, of a tree} or {\em
linearising}\index{linearising, of a tree} an expression tree, so that
the expression can be represented by a flat sequence of operators.
\end{important}

\begin{exercise}\label{gm:X:structural-induction}
Using structural induction, or otherwise, prove that the postfix evaluation
of arithmetic expressions results in the same answer as the tree evaluation
of expressions. That is: prove that for all expressions \mbox{\tt e} of type \mbox{\tt aExpr}\indexTT{aExpr},
\[
\mbox{\tt aInterpret}\indexTT{aInterpret}~\mbox{\tt e} = \mbox{\tt aEval}\indexTT{aEval}~(\mbox{\tt aCompile}\indexTT{aCompile}~\mbox{\tt e},~[])
\]
This is an example of a {\em congruence proof}\index{congruence proof, of compiler correctness}.
\end{exercise}

\begin{exercise}\label{gm:X:structural-induction-hard}
Extend the functions \mbox{\tt aInterpret}\indexTT{aInterpret}, \mbox{\tt aCompile}\indexTT{aCompile} and \mbox{\tt aEval}\indexTT{aEval} to handle
\mbox{\tt let}\indexTT{let} expressions. Prove that for all expressions in \mbox{\tt e} of type \mbox{\tt aExpr}\indexTT{aExpr}, these
new functions satisfy the relation:
\[
\mbox{\tt aInterpret}\indexTT{aInterpret}~\mbox{\tt e} = \mbox{\tt aEval}\indexTT{aEval}~(\mbox{\tt aCompile}\indexTT{aCompile}~\mbox{\tt e},~[])
\]
Can you extend the language to even more complicated expressions,
e.g. \mbox{\tt letrec}\indexTT{letrec} expressions? Can you prove that you have correctly
implemented these extensions?
\end{exercise}

\subsection{Using postfix code to construct graphs\index{postfix
code!to construct graphs}}

We can use the same technique to create an instance of a
supercombinator body. In this case the `values' on the stack will be
addresses of parts of the expression being instantiated.

The operations of the template construction instructions will be
different from those we saw in the arithmetic example above, in that
the instructions generally have the side-effect of allocating nodes in
the heap. As an example, consider introducing an \mbox{\tt Mkap}\indexTT{Mkap} instruction.
This instruction makes an application node, in the heap, from the top
two addresses on the stack. It leaves a pointer to this new node on
the stack upon completion.

There is no reason to invent a new evaluation stack of addresses, as
our template instantiation machine already has such a stack. However,
there is an important point to remember if we do make use of this
stack:
\begin{important}
The map of the {\em stack locations\/}\index{stack locations!in G-machine}
corresponding to variable names will change as we pop and push objects
from the stack. We must therefore keep track of this when we are
compiling expressions.
\end{important}
Our access to items in the stack is relative to the top of the stack.
So, if an item is added, the offset to reach that item is increased by
one; similarly, when an item is popped, the offset is decreased by one.

\subsection{What happens after an instantiation has been made?}

Once the instantiation of the supercombinator body has been made we
must tidy up the stack, and arrange the continuation of the evaluation
process. On completing the evaluation of the postfix sequence for a
supercombinator with $n$ arguments, the stack will have the following
form:
\begin{itemize}

\item On top there will be the address in heap of the newly
instantiated body, \mbox{\tt e}.

\item Next there are the $n+1$ pointers. From these we can access the
arguments used in the instantiation process.

\item The last of the $n+1$ pointers points to the root of the
expression we have just instantiated.

\end{itemize}
This is shown in Figure~\ref{gm:fg:stack}.
\begin{figure} %\centering
\setlength{\unitlength}{59204sp}%
%
\begingroup\makeatletter\ifx\SetFigFont\undefined%
\gdef\SetFigFont#1#2#3#4#5{%
  \reset@font\fontsize{#1}{#2pt}%
  \fontfamily{#3}\fontseries{#4}\fontshape{#5}%
  \selectfont}%
\fi\endgroup%
\begin{picture}(229,247)(18,585)
\thinlines
\multiput(230,800)(5.00000,6.66667){4}{\makebox(0.1111,0.7778){\SetFigFont{5}{6}{\rmdefault}{\mddefault}{\updefault}.}}
\put( 40,610){\vector( 2,-1){ 40}}
\put( 40,630){\vector( 3,-1){ 60}}
\put( 20,620){\line( 1, 0){ 40}}
\put( 20,665){\line( 0,-1){ 65}}
\put( 20,600){\line( 1, 0){ 40}}
\put( 60,600){\line( 0, 1){ 65}}
\multiput(220,785)(-0.25000,-0.50000){21}{\makebox(0.1111,0.7778){\SetFigFont{5}{6}{\rmdefault}{\mddefault}{\updefault}.}}
\put(215,775){\vector(-1,-2){0}}
\multiput(230,780)(0.40000,-0.40000){26}{\makebox(0.1111,0.7778){\SetFigFont{5}{6}{\rmdefault}{\mddefault}{\updefault}.}}
\put(240,770){\vector( 1,-1){0}}
\put(215,760){\vector( 2,-3){ 10}}
\put( 40,790){\vector( 1, 0){175}}
\put( 40,770){\vector( 1, 0){160}}
\multiput(150,680)(5.55556,8.33333){10}{\makebox(0.1111,0.7778){\SetFigFont{5}{6}{\rmdefault}{\mddefault}{\updefault}.}}
\put(145,675){\vector(-2,-3){ 10}}
\put(140,640){\vector( 2,-3){ 10}}
\put(120,640){\vector(-2,-3){ 10}}
\put( 40,650){\vector( 1, 0){ 80}}
\put( 60,805){\line( 0,-1){ 55}}
\put( 20,805){\line( 0,-1){ 55}}
\put( 20,640){\line( 1, 0){ 40}}
\put( 20,660){\line( 1, 0){ 40}}
\put( 20,760){\line( 1, 0){ 40}}
\put( 20,780){\line( 1, 0){ 40}}
\put( 20,800){\line( 1, 0){ 40}}
\put( 20,750){\line( 0,-1){ 85}}
\put( 60,750){\line( 0,-1){ 85}}
\put( 20,805){\line( 0, 1){ 25}}
\put( 60,805){\line( 0, 1){ 25}}
\put( 85,585){\makebox(0,0)[lb]{\smash{\SetFigFont{11}{13.2}{\familydefault}{\mddefault}{\updefault}e}}}
\put(245,760){\makebox(0,0)[lb]{\smash{\SetFigFont{11}{13.2}{\familydefault}{\mddefault}{\updefault}en}}}
\put(230,735){\makebox(0,0)[lb]{\smash{\SetFigFont{11}{13.2}{\familydefault}{\mddefault}{\updefault}en-1}}}
\put(220,785){\makebox(0,0)[lb]{\smash{\SetFigFont{11}{13.2}{\familydefault}{\mddefault}{\updefault}@}}}
\put(205,765){\makebox(0,0)[lb]{\smash{\SetFigFont{11}{13.2}{\familydefault}{\mddefault}{\updefault}@}}}
\put(105,605){\makebox(0,0)[lb]{\smash{\SetFigFont{11}{13.2}{\familydefault}{\mddefault}{\updefault}f}}}
\put(155,610){\makebox(0,0)[lb]{\smash{\SetFigFont{11}{13.2}{\familydefault}{\mddefault}{\updefault}e1}}}
\put(125,645){\makebox(0,0)[lb]{\smash{\SetFigFont{11}{13.2}{\familydefault}{\mddefault}{\updefault}@}}}
\end{picture}

\caption{The stack layout for the Mark~1 machine\index{G-machine stack
layout!Mark 1}}\label{gm:fg:stack} \end{figure}

We must replace the redex with the newly instantiated body, and pop
off $n$ items from the stack, using the \mbox{\tt Slide}\indexTT{Slide} instruction. To find
the next supercombinator we must now start unwinding again, using the
\mbox{\tt Unwind}\indexTT{Unwind} instruction. By adding operations to do the tidying and
unwinding to the postfix operator sequence, we have transformed the
template instantiator into our Mark~1 G-machine.

The code for the function \mbox{\tt f\ x1\ ...\ xn\ =\ e} is:
\begin{verbatim}
        <code to construct an instance of e>
        Slide n+1
        Unwind
\end{verbatim}

\section{Mark 1: A minimal G-machine\index{G-machine!Mark 1}}
\label{minimal-gm}

We now present the code for a complete G-machine and its compiler. It
does not perform updates (which are introduced in
Section~\ref{gm:sc:mark2}) or arithmetic (which is introduced in
Section~\ref{gm:sc:primitives}).

\subsection{Overall structure\index{G-machine!toplevel}}

At the top level the G-machine is very similar to the template instantiator;
as usual the whole system is knitted together with a \mbox{\tt run}\indexTT{run} function.
\begin{verbatim}
> -- The function run is already defined in gofers standard.prelude
> runProg :: [Char] -> [Char]
> runProg = showResults . eval . compile . parse
\end{verbatim}
\indexDTT{runProg}%
\indexDTT{runProg}%

The parser data structures and functions are included because we will
need access to them.
\begin{verbatim}
> -- :a language.lhs -- parser data types
\end{verbatim}
\subsection{Data type definitions}

Fundamental to graph reduction implementation techniques is the graph.
We use the \mbox{\tt heap}\indexTT{heap} data type, amongst others, from the utilities
provided in Appendix~\ref{sect:utils}.
\begin{verbatim}
> -- :a util.lhs -- heap data type and other library functions
\end{verbatim}
The Mark~1 G-machine uses the five-tuple, \mbox{\tt gmState}\indexTT{gmState}, as its state. A
\mbox{\tt gmState}\indexTT{gmState} holds all the information that we need during the execution
of the compiled program.
\begin{verbatim}
> type GmState 
>  = (GmCode,      -- Current instruction stream
>  GmStack,        -- Current stack
>  GmHeap,         -- Heap of nodes
>  GmGlobals,      -- Global addresses in heap
>  GmStats)        -- Statistics
\end{verbatim}
\indexDTT{type}%
In describing the G-machine, we will make use of state {\em access
functions\/}\index{access functions} to access the components of a
state. The advantage of this approach is that when we modify the state
to accommodate new components, we may reuse most of the original code
we have written. We will use the prefix \mbox{\tt get}\indexTT{get} to denote an access
function that gets a component from a state, and the prefix \mbox{\tt put}\indexTT{put} to
replace a component in a state.

We consider the type definitions of each of the five components of the
state, and their access functions, in turn.

\begin{itemize}
\item The instruction stream is of type \mbox{\tt gmCode}\indexTT{gmCode} and is simply a list
of \mbox{\tt instruction}\indexTT{instruction}s.
\begin{verbatim}
> type GmCode = [Instruction]
\end{verbatim}
\indexDTT{type}%
To get convenient access to the code, when the state is later
augmented with extra components, we define two functions: \mbox{\tt getCode}\indexTT{getCode} and
\mbox{\tt putCode}\indexTT{putCode}.
\begin{verbatim}
> getCode :: GmState -> GmCode
> getCode (i, stack, heap, globals, stats) = i
\end{verbatim}
\indexDTT{getCode}%
\indexDTT{getCode}%
\begin{verbatim}
> putCode :: GmCode -> GmState -> GmState
> putCode i' (i, stack, heap, globals, stats)
>    = (i', stack, heap, globals, stats)
\end{verbatim}
\indexDTT{putCode}%
\indexDTT{putCode}%
\par
There are only six instructions initially. We will describe these in
more detail in subsection~\ref{gm:ss:eval1}.
\begin{verbatim}
> data Instruction 
>     = Unwind
>     | Pushglobal Name
>     | Pushint Int
>     | Push Int
>     | Mkap
>     | Slide Int
> instance Eq Instruction 
>     where
>     Unwind          == Unwind               = True
>     Pushglobal a    == Pushglobal b         = a == b
>     Pushint a       == Pushint b            = a == b
>     Push a          == Push b               = a == b
>     Mkap            == Mkap                 = True
>     Slide a         == Slide b              = a == b
>     _               == _                    = False
\end{verbatim}
\indexDTT{data}%
\indexDTT{Pushglobal}%
\indexDTT{Pushint}%
\indexDTT{Push}%
\indexDTT{Mkap}%
\indexDTT{Slide}%
\indexDTT{instance}%
\item The G-machine stack \mbox{\tt gmStack}\indexTT{gmStack} is a list of addresses in the heap.
\begin{verbatim}
> type GmStack = [Addr]
\end{verbatim}
\indexDTT{type}%
To get convenient access to the stack, when the state is later
augmented with extra components, we define two functions \mbox{\tt getStack}\indexTT{getStack} and
\mbox{\tt putStack}\indexTT{putStack}
\begin{verbatim}
> getStack :: GmState -> GmStack
> getStack (i, stack, heap, globals, stats) = stack
\end{verbatim}
\indexDTT{getStack}%
\indexDTT{getStack}%
\begin{verbatim}
> putStack :: GmStack -> GmState -> GmState
> putStack stack' (i, stack, heap, globals, stats)
>    = (i, stack', heap, globals, stats)
\end{verbatim}
\indexDTT{putStack}%
\indexDTT{putStack}%
\item Just as we did in the case of the template instantiator, we use
the heap data structure from \mbox{\tt utils}\indexTT{utils} to implement heaps.
\begin{verbatim}
> type GmHeap = Heap Node
\end{verbatim}
\indexDTT{type}%
Again, to access this component of the state we define access
functions.
\begin{verbatim}
> getHeap :: GmState -> GmHeap
> getHeap (i, stack, heap, globals, stats) = heap
\end{verbatim}
\indexDTT{getHeap}%
\indexDTT{getHeap}%
\begin{verbatim}
> putHeap :: GmHeap -> GmState -> GmState
> putHeap heap' (i, stack, heap, globals, stats)
>    = (i, stack, heap', globals, stats)
\end{verbatim}
\indexDTT{putHeap}%
\indexDTT{putHeap}%
In the minimal G-machine there are only three types of nodes: numbers,
\mbox{\tt NNum}\indexTT{NNum}; applications, \mbox{\tt NAp}\indexTT{NAp}; and globals, \mbox{\tt NGlobal}\indexTT{NGlobal}.
\begin{verbatim}
> data Node 
>    = NNum Int              -- Numbers
>    | NAp Addr Addr         -- Applications
>    | NGlobal Int GmCode    -- Globals
\end{verbatim}
\indexDTT{data}%
\indexDTT{NAp}%
\indexDTT{NGlobal}%
Number nodes contain the relevant number; application nodes apply the
function at the first address to the expression at the second address.
The \mbox{\tt NGlobal}\indexTT{NGlobal} node contains the number of arguments that the global
expects and the code sequence to be executed when the global has
enough arguments. This replaces the \mbox{\tt NSupercomb}\indexTT{NSupercomb} nodes of the template
instantiator, which held a template instead of the arity and code.

\item Because we will later be making a lazy implementation it is
important that there is only one node for each global. The address of
a global can be determined by looking up its value in the association
list \mbox{\tt gmGlobals}\indexTT{gmGlobals}. This corresponds to the \mbox{\tt tiGlobals}\indexTT{tiGlobals} component
of the template machine.
\begin{verbatim}
> type GmGlobals = ASSOC Name Addr
\end{verbatim}
\indexDTT{type}%
The access function we use is \mbox{\tt getGlobals}\indexTT{getGlobals}; in the Mark~1 machine, this
component is constant so we do not need a corresponding put function.
\begin{verbatim}
> getGlobals :: GmState -> GmGlobals
> getGlobals (i, stack, heap, globals, stats) = globals
\end{verbatim}
\indexDTT{getGlobals}%
\indexDTT{getGlobals}%
\item The statistics component of the state is implemented as an
abstract data type.
\begin{verbatim}
> statInitial  :: GmStats
> statIncSteps :: GmStats -> GmStats
> statGetSteps :: GmStats -> Int
\end{verbatim}
\indexDTT{statInitial}%
\indexDTT{statIncSteps}%
\indexDTT{statGetSteps}%
The implementation of \mbox{\tt gmStats}\indexTT{gmStats} is now given.
\begin{verbatim}
> type GmStats = Int
> statInitial    = 0
> statIncSteps s = s+1
> statGetSteps s = s
\end{verbatim}
\indexDTT{type}%
\indexDTT{statInitial}%
\indexDTT{statIncSteps}%
\indexDTT{statGetSteps}%
To access this component we define \mbox{\tt getStats}\indexTT{getStats} and \mbox{\tt putStats}\indexTT{putStats}:
\begin{verbatim}
> getStats :: GmState -> GmStats
> getStats (i, stack, heap, globals, stats) = stats
\end{verbatim}
\indexDTT{getStats}%
\indexDTT{getStats}%
\begin{verbatim}
> putStats :: GmStats -> GmState -> GmState
> putStats stats' (i, stack, heap, globals, stats)
>    = (i, stack, heap, globals, stats')
\end{verbatim}
\indexDTT{putStats}%
\indexDTT{putStats}%
\end{itemize}

\subsection{The evaluator\index{G-machine!evaluator}}
\label{gm:ss:eval1}

The G-machine evaluator, \mbox{\tt eval}\indexTT{eval}, is defined to produce a list of
states.  The first one is the one constructed by the compiler. If
there is a last state, then the result of the evaluation will be on
the top of the stack component of the last state.
\begin{verbatim}
> eval :: GmState -> [GmState]
> eval state = state: restStates
>              where
>              restStates | gmFinal state     = []
>                         | otherwise         = eval nextState
>              nextState  = doAdmin (step state)
\end{verbatim}
\indexDTT{eval}%
\indexDTT{eval}%
The function \mbox{\tt doAdmin}\indexTT{doAdmin} uses \mbox{\tt statIncSteps}\indexTT{statIncSteps} to modify the statistics
component of the state.
\begin{verbatim}
> doAdmin :: GmState -> GmState
> doAdmin s = putStats (statIncSteps (getStats s)) s
\end{verbatim}
\indexDTT{doAdmin}%
\indexDTT{doAdmin}%
The important parts of the evaluator are the functions \mbox{\tt gmFinal}\indexTT{gmFinal} and
\mbox{\tt step}\indexTT{step} which we will now look at.

\subsubsection{Testing for a final state\index{termination
condition!in G-machine}}

The G-machine interpreter has finished when the code sequence that it
is executing is empty. We express this condition in the \mbox{\tt gmFinal}\indexTT{gmFinal}
function.
\begin{verbatim}
> gmFinal :: GmState -> Bool
> gmFinal s = case (getCode s) of
>                    []        -> True
>                    otherwise -> False
\end{verbatim}
\indexDTT{gmFinal}%
\indexDTT{gmFinal}%
\subsubsection{Taking a step}

The \mbox{\tt step}\indexTT{step} function is defined so that it makes a state transition
based on the instruction it is executing.
\begin{verbatim}
> step :: GmState -> GmState
> step state = dispatch i (putCode is state)
>              where (i:is) = getCode state
\end{verbatim}
\indexDTT{step}%
\indexDTT{step}%
\par
We \mbox{\tt dispatch}\indexTT{dispatch} on the current instruction \mbox{\tt i} and replace the current
code sequence with the code sequence \mbox{\tt is}; this corresponds to
advancing the program counter in a real machine.
\begin{verbatim}
> dispatch :: Instruction -> GmState -> GmState
> dispatch (Pushglobal f) = pushglobal f
> dispatch (Pushint n)    = pushint n
> dispatch Mkap           = mkap
> dispatch (Push n)       = push n
> dispatch (Slide n)      = slide n
> dispatch Unwind         = unwind
\end{verbatim}
\indexDTT{dispatch}%
\indexDTT{dispatch}%
\indexDTT{dispatch}%
\indexDTT{dispatch}%
\indexDTT{dispatch}%
\indexDTT{dispatch}%
\indexDTT{dispatch}%
As we can see, the \mbox{\tt dispatch}\indexTT{dispatch} function simply selects a state
transition to execute.

Let us begin by looking at the transition rules for the postfix
instructions. There will be one for each syntactic object in
\mbox{\tt instruction}\indexTT{instruction}. We begin with the \mbox{\tt Pushglobal}\indexTT{Pushglobal} instruction, which uses the
\mbox{\tt globals}\indexTT{globals} component of the state to find the unique \mbox{\tt NGlobal}\indexTT{NGlobal} node in the
\mbox{\tt heap}\indexTT{heap} that holds the global $f$. If it cannot find one, it prints a
suitable error message.

\gmrule%
{\gmstate{\mbox{\tt Pushglobal}\indexTT{Pushglobal}\ f:i}{s}{h}{m[f:a]}}%
{\gmstate{i}{a:s}{h}{m}}

We implement this rule using the \mbox{\tt pushglobal}\indexTT{pushglobal} function.
\begin{verbatim}
> pushglobal :: Name -> GmState -> GmState
> pushglobal f state
>       = putStack (a: getStack state) state
>       where a = aLookup (getGlobals state) f (error ("Undeclared global " ++ f))
\end{verbatim}
\indexDTT{pushglobal}%
\indexDTT{pushglobal}%
\par
The remaining transitions are for constructing the body of a
supercombinator. The transition for \mbox{\tt Pushint}\indexTT{Pushint} places an integer node
into the heap.

\gmrule%
{\gmstate{\mbox{\tt Pushint}\indexTT{Pushint}\ n:i}{s}{h}{m}}%
{\gmstate{i}{a:s}{h[a:\mbox{\tt NNum}\indexTT{NNum}\ n]}{m}}

The corresponding function is \mbox{\tt pushint}\indexTT{pushint}.
The number is placed in the new heap \mbox{\tt heap'} with address \mbox{\tt a}. We then
place the heap and stack back into the state.
\begin{verbatim}
> pushint :: Int -> GmState -> GmState
> pushint n state
>       = putHeap heap' (putStack (a: getStack state) state)
>       where (heap', a) = hAlloc (getHeap state) (NNum n)
\end{verbatim}
\indexDTT{pushint}%
\indexDTT{pushint}%
\par
The \mbox{\tt Mkap}\indexTT{Mkap} instruction uses the two addresses on the top of the stack
to construct an application node in the heap. It has the following
transition rule.

\gmrule%
{\gmstate{\mbox{\tt Mkap}\indexTT{Mkap}:i}{a_1:a_2:s}{h}{m}}%
{\gmstate{i}{a:s}{h[a: \mbox{\tt NAp}\indexTT{NAp}\ a_1\  a_2]}{m}}

This transition becomes \mbox{\tt mkap}\indexTT{mkap}. Again \mbox{\tt heap'} and \mbox{\tt a} are respectively
the new heap and the address of the new node.
\begin{verbatim}
> mkap :: GmState -> GmState
> mkap state
>       = putHeap heap' (putStack (a:as') state)
>       where (heap', a)  = hAlloc (getHeap state) (NAp a1 a2)
>             (a1:a2:as') = getStack state
\end{verbatim}
\indexDTT{mkap}%
\indexDTT{mkap}%
\par
The \mbox{\tt Push}\indexTT{Push} instruction is used to take a copy of an argument which was
passed to a function. To do this it has to `look through' the
application node which is pointed to from the stack. We must also
remember to skip over the supercombinator node which is on the stack.

\gmrule%
{\gmstate{\mbox{\tt Push}\indexTT{Push}\ n:i}{a_0:\ldots:a_{n+1}:s}{h[a_{n+1}:\mbox{\tt NAp}\indexTT{NAp}\ a_n \ a'_n]}{m}}%
{\gmstate{i}{a'_n:a_0:\ldots:a_{n+1}:s}{h}{m}}
\begin{verbatim}
> push :: Int -> GmState -> GmState
> push n state
>    = putStack (a:as) state
>    where   as = getStack state
>            a  = getArg (hLookup (getHeap state) (as !! (n+1)))
\end{verbatim}
\indexDTT{push}%
\indexDTT{push}%
This uses the auxiliary function \mbox{\tt getArg}\indexTT{getArg} to select the required
expression from an application node.
\begin{verbatim}
> getArg :: Node -> Addr
> getArg (NAp a1 a2) = a2
\end{verbatim}
\indexDTT{getArg}%
\indexDTT{getArg}%
\begin{important}
Because of the stack structure we have changed the addressing mode of
the \mbox{\tt Push}\indexTT{Push} instruction from that used in \cite{PJBook}.
\end{important}

Next, the tidying up of the stack, which occurs after a supercombinator
has been instantiated and before continuing unwinding, is performed by
the \mbox{\tt Slide}\indexTT{Slide} instruction.

\gmrule%
{\gmstate{\mbox{\tt Slide}\indexTT{Slide}\ n:i}{a_0:\ldots:a_n:s}{h}{m}}%
{\gmstate{i}{a_0:s}{h}{m}}
\begin{verbatim}
> slide :: Int -> GmState -> GmState
> slide n state
>       = putStack (a: drop n as) state
>       where (a:as) = getStack state
\end{verbatim}
\indexDTT{slide}%
\indexDTT{slide}%
\par
\mbox{\tt Unwind}\indexTT{Unwind} is the most complex instruction because it replaces the outer
loop of our template instantiator. The \mbox{\tt Unwind}\indexTT{Unwind} instruction is always
the last instruction of a sequence, as we shall see in the next
section. The \mbox{\tt newState}\indexTT{newState} constructed depends on the item on top of the
stack; this depends on the transition rule that is fired, which also
depends on the item on top of the stack.
\begin{verbatim}
> unwind :: GmState -> GmState
> unwind state
>      = newState (hLookup heap a)
>      where
>              (a:as) = getStack state
>              heap   = getHeap state
\end{verbatim}
\indexDTT{unwind}%
\indexDTT{unwind}%
\par
We first consider the case where there is a number on top of the
stack. In this case, we are finished; the G-machine has terminated,
and we place $[]$ in the code component to signify this fact.

\gmrule%
{\gmstate{[\mbox{\tt Unwind}\indexTT{Unwind}]}{a:s}{h[a: \mbox{\tt NNum}\indexTT{NNum}\  n]}{m}}%
{\gmstate{[]}{a:s}{h}{m}}
\begin{verbatim}
>              newState (NNum n)      = state
\end{verbatim}
\par
If there is an application node on top of the stack then we must
continue to unwind from the next node.

\gmrule%
{\gmstate{[\mbox{\tt Unwind}\indexTT{Unwind}]}{a:s}{h[a: \mbox{\tt NAp}\indexTT{NAp}\  a_1\  a_2]}{m}}%
{\gmstate{[\mbox{\tt Unwind}\indexTT{Unwind}]}{a_1:a:s}{h}{m}}
\begin{verbatim}
>              newState (NAp a1 a2)   = putCode [Unwind] (putStack (a1:a:as) state)
\end{verbatim}
\par
The most complicated rule occurs when there is a global node on top of
the stack. There are two cases to consider, depending on whether there are
enough arguments to reduce the supercombinator application.

Firstly, if there are not enough arguments to reduce the
supercombinator application then the program was ill-typed. We will
ignore this case for the Mark~1 G-machine. Alternatively, when there
are enough arguments, it is possible to reduce the supercombinator, by
`jumping to' the code for the supercombinator. In the transition
rule this is expressed by moving the supercombinator code into the
code component of the machine.

\gmrule%
{\gmstate{[\mbox{\tt Unwind}\indexTT{Unwind}]}{a_0:\ldots:a_n:s}%
{h[a_0  :  \mbox{\tt NGlobal}\indexTT{NGlobal}\ n\  c]}{m}}%
{\gmstate{c}{a_0:\ldots:a_n:s}{h}{m}}
\begin{verbatim}
>              newState (NGlobal n c) 
>                     | length as < n        = error "Unwinding with too few arguments"
>                     | otherwise    = putCode c state
\end{verbatim}
We have now seen how the instructions are defined, but we have not
seen how to generate the postfix sequences of operators, or
instruction sequences as we shall refer to them from now on. This is
the subject of the next subsection.

\subsection{Compiling a program\index{G-machine!compiler}}
\label{gm:ss:compile}

We describe the compiler using a set of {\em compilation
schemes}\index{compilation schemes!in G-machine}.  Each
supercombinator definition is compiled using the compilation scheme
\tSC{}.
The compiled code generated for each supercombinator is defined in
Figure~\ref{gm:fg:schemes1}. Corresponding to the compilation schemes
\tSC{}, \tR{} and \tC{} are {\em compiler functions\/}\index{compiler
functions!in G-machine} \mbox{\tt compileSc}\indexTT{compileSc}, \mbox{\tt compileR}\indexTT{compileR} and \mbox{\tt compileC}\indexTT{compileC}. We
consider each of these in turn.
\begin{figure*} %\centering
$\begin{array}{|rcll|}
\hline
&&&\\
\multicolumn{4}{|l|}{\parbox[t]{29pc}{%
$\SC{d}$ is the G-machine code for the supercombinator definition $d$.}}\\
&&&\\
\multicolumn{4}{|l|}{\SC{f\ x_1\ \ldots\ x_n\ =\ e}
        \: = \: \R{e}\ [x_1\mapsto 0,\,\ldots,\,x_n\mapsto n-1]\ n}\\
&&&\\
\hline
&&&\\
\multicolumn{4}{|l|}{\parbox[t]{29pc}{%
$\R{e}~\rho~d$ generates code which instantiates the expression $e$ in
environment $\rho$, for a supercombinator of arity $d$, and then proceeds
to unwind the resulting stack.}}\\
&&&\\
\indexD{$\protect\cal \pR$}\R{e}\ \rho\ d  & = & \C{e}\ \rho \append [\mbox{\tt Slide}\indexTT{Slide}\ d+1,\, \mbox{\tt Unwind}\indexTT{Unwind}] & \\
&&&\\
\hline
&&&\\
\multicolumn{4}{|l|}{\parbox[t]{29pc}{%
$\C{e}~\rho$ generates code which constructs the graph of $e$ in environment
$\rho$, leaving a pointer to it on top of the stack.}}\\
&&&\\
\indexD{$\protect\cal \pC$}\C{f}\ \rho     & = & [\mbox{\tt Pushglobal}\indexTT{Pushglobal}\ f] &
                \tr{where $f$ is a supercombinator} \\
%
\indexD{$\protect\cal \pC$}\C{x}\ \rho     & = & [\mbox{\tt Push}\indexTT{Push}\ (\rho\ x)] &
                        \tr{where $x$ is a local variable}      \\
%
\indexD{$\protect\cal \pC$}\C{i}\ \rho     & = & [\mbox{\tt Pushint}\indexTT{Pushint}\ i] &\\
\indexD{$\protect\cal \pC$}\C{e_0~ e_1}\ \rho     & = & \C{e_1}\ \rho \append
                            \C{e_0}\ \rho^{+1} \append [\mbox{\tt Mkap}\indexTT{Mkap}] &
                  \tr{where $\rho^{+n}\ x = (\rho\ x) + n$} \\

&&&\\
\hline
\end{array}$
\caption{The \tSC{}, \tR{} and \tC{} compilation schemes}
\label{gm:fg:schemes1}
\end{figure*}

The \mbox{\tt compile}\indexTT{compile} function turns a program into an initial state for the
G-machine.  The initial code sequence finds the global \mbox{\tt main}\indexTT{main} and then
evaluates it. The heap is initialised so that it contains a node for
each global declared. \mbox{\tt globals}\indexTT{globals} contains the map from global names to the
\mbox{\tt NGlobal}\indexTT{NGlobal} nodes provided for them.
\begin{verbatim}
> compile :: CoreProgram -> GmState
> compile program
>    = (initialCode, [], heap, globals, statInitial)
>    where (heap, globals) = buildInitialHeap program
\end{verbatim}
\indexDTT{compile}%
\indexDTT{compile}%
\par
To construct the initial heap and to provide the map of the global
nodes for each global defined we use \mbox{\tt buildInitialHeap}\indexTT{buildInitialHeap}. This is just
as it was in the template machine.
\begin{verbatim}
> buildInitialHeap :: CoreProgram -> (GmHeap, GmGlobals)
> buildInitialHeap program
>    = mapAccuml allocateSc hInitial compiled
>    --where compiled = map compileSc (preludeDefs ++ program) ++
>    --                compiledPrimitives
>    where compiled = map compileSc program
\end{verbatim}
\indexDTT{buildInitialHeap}%
\indexDTT{buildInitialHeap}%
The \mbox{\tt buildInitialHeap}\indexTT{buildInitialHeap} function uses \mbox{\tt mapAccuml}\indexTT{mapAccuml} to allocate nodes for
each compiled global; the compilation occurring (where necessary) in
\mbox{\tt compiled}\indexTT{compiled}, which has type \mbox{\tt [gmCompiledSC]}.
\begin{verbatim}
> type GmCompiledSC = (Name, Int, GmCode)
\end{verbatim}
\indexDTT{type}%
The function \mbox{\tt allocateSc}\indexTT{allocateSc} allocates a new global for its compiled
supercombinator argument, returning the new heap and the address where
the global is stored.
\begin{verbatim}
> allocateSc :: GmHeap -> GmCompiledSC -> (GmHeap, (Name, Addr))
> allocateSc heap (name, nargs, instns)
>       = (heap', (name, addr))
>       where (heap', addr) = hAlloc heap (NGlobal nargs instns)
\end{verbatim}
\indexDTT{allocateSc}%
\indexDTT{allocateSc}%
\par
In the initial state, we want the machine to evaluate the value of the
program. We recall that this is just the value of the global \mbox{\tt main}\indexTT{main}.
\begin{verbatim}
> initialCode :: GmCode
> initialCode = [Pushglobal "main", Unwind]
\end{verbatim}
\indexDTT{initialCode}%
\indexDTT{initialCode}%
\par
Each supercombinator is compiled using \mbox{\tt compileSc}\indexTT{compileSc}, which implements
the \tSC{} scheme of Figure~\ref{gm:fg:schemes1}. It returns a triple
containing the supercombinator name, the number of arguments the
supercombinator needs before it can be reduced, and the code sequence
associated with the supercombinator.
\begin{verbatim}
> compileSc :: (Name, [Name], CoreExpr) -> GmCompiledSC
> compileSc (name, env, body)
>       = (name, length env, compileR body (zip2 env [0..]))
\end{verbatim}
\indexDTT{compileSc}%
\indexDTT{compileSc}%
This in turn uses \mbox{\tt compileR}\indexTT{compileR}, which corresponds to the \tR{} scheme of
Figure~\ref{gm:fg:schemes1}.
\begin{verbatim}
> compileR :: GmCompiler
> compileR e env = compileC e env ++ [Slide (length env + 1), Unwind]
> compileR e env = compileC e env ++ [Slide (length env + 1), Unwind]
\end{verbatim}
\indexDTT{compileR}%
\indexDTT{compileR}%
\indexDTT{compileR}%
Each of the compiler schemes has the same type: \mbox{\tt gmCompiler}\indexTT{gmCompiler}.
\begin{verbatim}
> type GmCompiler = CoreExpr -> GmEnvironment -> GmCode
\end{verbatim}
\indexDTT{type}%
\par
We use the fact that we can represent the map $\rho$ from the
compilation scheme as an association list. Not only can we look up the
offsets for a variable from this list, but we may also calculate how
many arguments there are on the stack. This is used in \mbox{\tt compileR}\indexTT{compileR} to
find out how many stack elements to squeeze out with a \mbox{\tt Slide}\indexTT{Slide}
instruction. The list has type \mbox{\tt gmEnvironment}\indexTT{gmEnvironment}, which is defined as:
\begin{verbatim}
> type GmEnvironment = ASSOC Name Int
\end{verbatim}
\indexDTT{type}%
This constructs the instantiation of the supercombinator body using
\mbox{\tt compileC}\indexTT{compileC}, which corresponds to the \tC{} scheme
of Figure~\ref{gm:fg:schemes1}.
\begin{verbatim}
> compileC :: GmCompiler
> compileC (EVar v)    env 
>  | elem v (aDomain env)          = [Push n]
>  | otherwise                     = [Pushglobal v]
>  where n = aLookup env v (error "Can't happen")
> compileC (ENum n)    env = [Pushint n]
> compileC (EAp e1 e2) env = compileC e2 env ++
>                            compileC e1 (argOffset 1 env) ++
>                            [Mkap]
\end{verbatim}
\indexDTT{compileC}%
\indexDTT{compileC}%
\indexDTT{compileC}%
\indexDTT{compileC}%
We can change the stack offsets using the function \mbox{\tt argOffset}\indexTT{argOffset}. 
If \mbox{\tt env}\indexTT{env} implements
$\rho$, then \mbox{\tt (argOffset\ n\ env)} implements $\rho^{+n}$.
\begin{verbatim}
> argOffset :: Int -> GmEnvironment -> GmEnvironment
> argOffset n env = [(v, n+m) | (v,m) <- env]
\end{verbatim}
\indexDTT{argOffset}%
\indexDTT{argOffset}%
\subsubsection{An example compilation\index{compilation!G-machine example}}

Let us look at the compilation of the \mbox{\tt K} combinator. When compiling
this function, we will begin by evaluating the following expression.
\begin{verbatim}
  compileSc ("K", ["x", "y"], EVar "x")
\end{verbatim}
The first element of the tuple is the name (\mbox{\tt K} in this case); the
second is the argument list (in this case we have two variables: \mbox{\tt x}
and \mbox{\tt y}); and the third component of the tuple is the body of the
supercombinator (which for this example is just the variable \mbox{\tt x}).

When we rewrite this expression, we get:
\begin{verbatim}
  ("K", 2, compileR (EVar "x") [("x", 0), ("y", 1)])
\end{verbatim}
The resulting triple consists of the name (\mbox{\tt K}), the number of
arguments we need to reduce the supercombinator (two in this case),
and the code sequence to perform the instantiation. When we rewrite
this expression we will generate the code sequence for this
supercombinator. Notice that the environment is represented by the
expression \mbox{\tt [("x",\ 0),\ ("y",\ 1)]}; this tells us that when we
instantiate the body, a pointer to \mbox{\tt x} will be on top of the argument
stack and a pointer to \mbox{\tt y} will be immediately below \mbox{\tt x} on the stack.
\begin{verbatim}
  ("K", 2, compileC (EVar "x") [("x", 0), ("y", 1)] ++ [Slide 3, Unwind])
\end{verbatim}
The \mbox{\tt compileR}\indexTT{compileR} function is defined to compile the body using
\mbox{\tt compileC}\indexTT{compileC}, and to add a \mbox{\tt Slide}\indexTT{Slide} and an \mbox{\tt Unwind}\indexTT{Unwind} instruction at the
end.

To compile the body we look up \mbox{\tt x} and find that it is on top of the stack.
We generate code to make a copy of the top of the stack, using \mbox{\tt Push\ 0}.
\begin{verbatim}
  ("K", 2, [Push 0, Slide 3, Unwind])
\end{verbatim}

\begin{exercise}\label{gm:X:symbolic-compilation}
Write out the equivalent sequence of transformations for the \mbox{\tt S}
combinator from the prelude definitions. Recall that \mbox{\tt S} is defined as:
\begin{verbatim}
S f g x = f x (g x)
\end{verbatim}
Check the final result by running the compiler and machine with any of
the simple programs given in Appendix~\ref{sect:core-progs}. (\mbox{\tt S} is
in the standard prelude.)
\end{exercise}

\subsubsection{Primitives\index{primitives!in G-machine}}

In this minimal G-machine there are no primitives, so there is nothing
to implement!
\begin{verbatim}
> compiledPrimitives :: [GmCompiledSC]
> compiledPrimitives = []
\end{verbatim}
\indexDTT{compiledPrimitives}%
\indexDTT{compiledPrimitives}%
\subsection{Printing the results\index{execution traces!in G-machine}}

Because a number of the state components are abstract data types (and
are therefore not directly printable) we must define a pretty-printer
for the states that the machine produces. It is also a fact that the
output is voluminous and not very informative if it is all displayed
at once.
The printing is controlled by \mbox{\tt showResults}\indexTT{showResults}. It produces three pieces
of output: the super-combinator code sequences, the state
transitions and the final statistics.
\begin{verbatim}
> showResults :: [GmState] -> [Char]
> showResults states
>       = iDisplay (iConcat [
>       iStr "Supercombinator definitions", iNewline,
>       iInterleave iNewline (map (showSC s) (getGlobals s)),
>       iNewline, iNewline, iStr "State transitions", iNewline, iNewline,
>       iLayn (map showState states),
>       iNewline, iNewline,
>       showStats (last states)])
>       where (s:ss) = states
\end{verbatim}
\indexDTT{showResults}%
\indexDTT{showResults}%
\par
Taking each of these in turn, we begin with \mbox{\tt showSC}\indexTT{showSC}. This finds the
code for the supercombinator in the unique global heap node
associated with the global, and prints the code sequence using
\mbox{\tt showInstructions}\indexTT{showInstructions}.
\begin{verbatim}
> showSC :: GmState -> (Name, Addr) -> Iseq
> showSC s (name, addr)
>       = iConcat [ iStr "Code for ", iStr name, iNewline,
>             showInstructions code, iNewline, iNewline]
>       where (NGlobal arity code) = (hLookup (getHeap s) addr)
\end{verbatim}
\indexDTT{showSC}%
\indexDTT{showSC}%
Then \mbox{\tt showInstructions}\indexTT{showInstructions} is used to output a code sequence.
\begin{verbatim}
> showInstructions :: GmCode -> Iseq
> showInstructions is
>       = iConcat [iStr "  Code:{",
>            iIndent (iInterleave iNewline (map showInstruction is)),
>            iStr "}", iNewline]
\end{verbatim}
\indexDTT{showInstructions}%
\indexDTT{showInstructions}%
The output for each individual instruction is given by
\mbox{\tt showInstruction}\indexTT{showInstruction}.
\begin{verbatim}
> showInstruction :: Instruction -> Iseq
> showInstruction Unwind         = iStr  "Unwind"
> showInstruction (Pushglobal f) = (iStr "Pushglobal ") `iAppend` (iStr f)
> showInstruction (Push n)       = (iStr "Push ")       `iAppend` (iNum n)
> showInstruction (Pushint n)    = (iStr "Pushint ")    `iAppend` (iNum n)
> showInstruction Mkap           = iStr  "Mkap"
> showInstruction (Slide n)      = (iStr "Slide ")      `iAppend` (iNum n)
\end{verbatim}
\indexDTT{showInstruction}%
\indexDTT{showInstruction}%
\indexDTT{showInstruction}%
\indexDTT{showInstruction}%
\indexDTT{showInstruction}%
\indexDTT{showInstruction}%
\indexDTT{showInstruction}%
\par
The next major piece of output is the state transitions; these are
individually dealt with using \mbox{\tt showState}\indexTT{showState}.
\begin{verbatim}
> showState :: GmState -> Iseq
> showState s
>    = iConcat [showStack s,         iNewline,
>    showInstructions (getCode s),   iNewline]
\end{verbatim}
\indexDTT{showState}%
\indexDTT{showState}%
To correspond with our diagrams, we would like to have the top of
stack at the bottom of the printed stack. To this end we reverse the
stack.
\begin{verbatim}
> showStack :: GmState -> Iseq
> showStack s
>       = iConcat [iStr " Stack:[",
>            iIndent (iInterleave iNewline
>                        (map (showStackItem s) (reverse (getStack s)))),
>            iStr "]"]
\end{verbatim}
\indexDTT{showStack}%
\indexDTT{showStack}%
\par
Each stack item is displayed using \mbox{\tt showStackItem}\indexTT{showStackItem}. It prints the
address stored in the stack and the object in the heap to which it
points.
\begin{verbatim}
> showStackItem :: GmState -> Addr -> Iseq
> showStackItem s a
>       = iConcat [iStr (showaddr a), iStr ": ",
>            showNode s a (hLookup (getHeap s) a)]
\end{verbatim}
\indexDTT{showStackItem}%
\indexDTT{showStackItem}%
\par
The function \mbox{\tt showNode}\indexTT{showNode} needs to invert the association list of global
names and heap addresses to display the global nodes it comes across.
\begin{verbatim}
> showNode :: GmState -> Addr -> Node -> Iseq
> showNode s a (NNum n)      = iNum n
> showNode s a (NGlobal n g) = iConcat [iStr "Global ", iStr v]
>    where v = head [n | (n,b) <- getGlobals s, a==b]
> showNode s a (NAp a1 a2)   = iConcat [iStr "Ap ", iStr (showaddr a1),
>                                       iStr " ",   iStr (showaddr a2)]
\end{verbatim}
\indexDTT{showNode}%
\indexDTT{showNode}%
\indexDTT{showNode}%
\indexDTT{showNode}%
\par
Finally, we print the accumulated statistics, using \mbox{\tt showStats}\indexTT{showStats}.
\begin{verbatim}
> showStats :: GmState -> Iseq
> showStats s
>       = iConcat [ iStr "Steps taken = ", iNum (statGetSteps (getStats s))]
\end{verbatim}
\indexDTT{showStats}%
\indexDTT{showStats}%
\par
This concludes the description of the basic G-machine. We now move on
to consider ways to make it more sophisticated.

\subsection{Improvements to the Mark~1 G-machine}

\begin{exercise}\label{gm:X:run}
Run the program \mbox{\tt main\ =\ S\ K\ K\ 3}.  How many steps does it take? Why
is it different from that obtained for the template machine? Do you
think that comparing steps taken is a fair comparison of the machines?
\end{exercise}

\begin{exercise}\label{gm:X:runagain}
Try running some other programs from Appendix~\ref{sect:core-progs}.
Remember, there is no arithmetic in this simple machine.
\end{exercise}

\begin{exercise}\label{gm:X:pushint}
It is possible to use the same trick we used for \mbox{\tt Pushglobal}\indexTT{Pushglobal} to
implement \mbox{\tt Pushint}\indexTT{Pushint}: for each distinct number we create a unique node
in the heap. For example, when we first execute \mbox{\tt Pushint\ 2}, we update
\mbox{\tt gmGlobals}\indexTT{gmGlobals} so that it associates \mbox{\tt "2"} with the address in heap of
the node \mbox{\tt NNum\ 2}.

In the transition rules, if there is already a global called $n$,
then we can reuse this global node.
\gmrule%
{\gmstate{\mbox{\tt Pushint}\indexTT{Pushint}\ n:i}{s}{h}{m[n: a]}}%
{\gmstate{i}{a:s}{h}{m}}
Alternatively, when this is not the case, we will create a new node
and add it to the global map.
\gmrule%
{\gmstate{\mbox{\tt Pushint}\indexTT{Pushint}\ n:i}{s}{h}{m}}%
{\gmstate{i}{a:s}{h[a: \mbox{\tt NNum}\indexTT{NNum}\ n]}{m[n: a]}}

The advantage of this scheme is that we can reuse the same number node in
the heap each time a \mbox{\tt Pushint}\indexTT{Pushint} is executed.

Implement this new transition \mbox{\tt pushint}\indexTT{pushint} for the \mbox{\tt Pushint}\indexTT{Pushint} instruction.
You should define an access function for the global component, calling
it \mbox{\tt putGlobals}\indexTT{putGlobals}.
\end{exercise}

\section{Mark 2: Making it lazy}
\index{G-machine!Mark 2}\index{G-machine!laziness}\index{updates!in G-machine}
\label{gm:sc:mark2}

We will now make a number of small changes to the Mark~1 G-machine in
order to make it lazy. The Mark~1 machine is not lazy at the moment
because it does not overwrite the root node of the original expression
before unwinding. This updating is described in
Section~\ref{sect:templ:update-review}.
In the Mark~2 machine, the idea is that after instantiating the body
of the supercombinator, we will {\em overwrite\/}\index{overwrite} the
root of the original redex\index{redex} with an {\em indirection
node\/}\index{indirections} pointing to the newly constructed
instance. The effect is that the machine `remembers' the value that
was instantiated last time the redex was reduced, and hence does not
need to recalculate it.

We implement this change as follows. In the Mark~1 machine the code
for each supercombinator concluded with $[\mbox{\tt Slide}\indexTT{Slide}\ (n+1),\ \mbox{\tt Unwind}\indexTT{Unwind}]$.
To capture updating we replace this with $[\mbox{\tt Update}\indexTT{Update}\ n,\ \mbox{\tt Pop}\indexTT{Pop}\ n,\
\mbox{\tt Unwind}\indexTT{Unwind}]$. This is illustrated in the following diagrams, in which we
use \mbox{\tt {\char'43}} to represent indirection nodes.

Figure~\ref{gm:fig:slide} shows how the Mark~1 machine executes a
$\mbox{\tt Slide}\indexTT{Slide}\ n+1$ instruction. In Figure~\ref{gm:fig:upd} we see the
Mark~2 machine executing the sequence $[\mbox{\tt Update}\indexTT{Update}\ n,\, \mbox{\tt Pop}\indexTT{Pop}\ n]$;
this being the sequence we propose to use as a lazy replacement for
$[\mbox{\tt Slide}\indexTT{Slide}\ n+1]$. The \mbox{\tt Update}\indexTT{Update} instruction is responsible for
overwriting the root node with the newly created instance of the body
of the supercombinator.  The \mbox{\tt Pop}\indexTT{Pop} instruction is used to remove the
arguments from the stack, as they are now no longer needed.
\begin{figure} %\centering
\setlength{\unitlength}{59204sp}%
%
\begingroup\makeatletter\ifx\SetFigFont\undefined%
\gdef\SetFigFont#1#2#3#4#5{%
  \reset@font\fontsize{#1}{#2pt}%
  \fontfamily{#3}\fontseries{#4}\fontshape{#5}%
  \selectfont}%
\fi\endgroup%
\begin{picture}(324,225)(18,602)
\thinlines
\put( 20,705){\line( 0,-1){ 65}}
\put( 20,640){\line( 1, 0){ 40}}
\put( 60,640){\line( 0, 1){ 65}}
\put( 40,650){\vector( 1, 0){ 55}}
\multiput(120,685)(-0.25000,-0.50000){21}{\makebox(0.1111,0.7778){\SetFigFont{5}{6}{\rmdefault}{\mddefault}{\updefault}.}}
\put( 20,660){\line( 1, 0){ 40}}
\multiput(160,770)(0.40000,-0.40000){26}{\makebox(0.1111,0.7778){\SetFigFont{5}{6}{\rmdefault}{\mddefault}{\updefault}.}}
\put(170,760){\vector( 1,-1){0}}
\put(160,800){\vector(-1,-3){  5}}
\multiput(150,770)(-3.12500,-9.37500){9}{\makebox(0.1111,0.7778){\SetFigFont{5}{6}{\rmdefault}{\mddefault}{\updefault}.}}
\put(260,820){\line( 1, 0){ 40}}
\put(260,825){\line( 0,-1){ 25}}
\put(260,800){\line( 1, 0){ 40}}
\put(300,800){\line( 0, 1){ 25}}
\put( 20,680){\line( 1, 0){ 40}}
\put( 20,700){\line( 1, 0){ 40}}
\put( 20,800){\line( 1, 0){ 40}}
\put( 60,825){\line( 0,-1){ 30}}
\put( 20,825){\line( 0,-1){ 30}}
\put(340,810){\vector( 0,-1){130}}
\put(280,810){\line( 1, 0){ 60}}
\put( 20,820){\framebox(40,0){}}
\multiput(170,805)(0.40000,-0.40000){26}{\makebox(0.1111,0.7778){\SetFigFont{5}{6}{\rmdefault}{\mddefault}{\updefault}.}}
\put(180,795){\vector( 1,-1){0}}
\multiput(130,685)(0.40000,-0.40000){26}{\makebox(0.1111,0.7778){\SetFigFont{5}{6}{\rmdefault}{\mddefault}{\updefault}.}}
\put(140,675){\vector( 1,-1){0}}
\put(130,685){\makebox(0.1111,0.7778){\SetFigFont{5}{6}{\rmdefault}{\mddefault}{\updefault}.}}
\put( 40,690){\vector( 1, 0){ 80}}
\put( 40,670){\vector( 1, 0){ 60}}
\put( 40,810){\vector( 1, 0){120}}
\put( 40,670){\framebox(60,0){}}
\put( 40,670){\framebox(0,0){}}
\put( 40,810){\framebox(120,0){}}
\put( 20,795){\line( 0,-1){ 90}}
\put( 60,800){\line( 0,-1){ 95}}
\put(265,605){\makebox(0,0)[lb]{\smash{\SetFigFont{11}{13.2}{\familydefault}{\mddefault}{\updefault}After executing Slide (n+1)}}}
\put( 25,605){\makebox(0,0)[lb]{\smash{\SetFigFont{11}{13.2}{\familydefault}{\mddefault}{\updefault}Before executing Slide (n+1)}}}
\put(105,645){\makebox(0,0)[lb]{\smash{\SetFigFont{11}{13.2}{\familydefault}{\mddefault}{\updefault}e}}}
\put(105,665){\makebox(0,0)[lb]{\smash{\SetFigFont{11}{13.2}{\familydefault}{\mddefault}{\updefault}f}}}
\put(175,750){\makebox(0,0)[lb]{\smash{\SetFigFont{11}{13.2}{\familydefault}{\mddefault}{\updefault}en-1}}}
\put(150,775){\makebox(0,0)[lb]{\smash{\SetFigFont{11}{13.2}{\familydefault}{\mddefault}{\updefault}@}}}
\put(335,665){\makebox(0,0)[lb]{\smash{\SetFigFont{11}{13.2}{\familydefault}{\mddefault}{\updefault}e}}}
\put(180,780){\makebox(0,0)[lb]{\smash{\SetFigFont{11}{13.2}{\familydefault}{\mddefault}{\updefault}en}}}
\put(145,660){\makebox(0,0)[lb]{\smash{\SetFigFont{11}{13.2}{\familydefault}{\mddefault}{\updefault}e1}}}
\put(120,685){\makebox(0,0)[lb]{\smash{\SetFigFont{11}{13.2}{\familydefault}{\mddefault}{\updefault}@}}}
\put(160,805){\makebox(0,0)[lb]{\smash{\SetFigFont{11}{13.2}{\familydefault}{\mddefault}{\updefault}@}}}
\end{picture}

\caption{Mark~1 G-machine (executing \mbox{\tt Slide\ n+1})}
\label{gm:fig:slide}
\end{figure}
\begin{figure*} %\centering
\setlength{\unitlength}{59204sp}%
%
\begingroup\makeatletter\ifx\SetFigFont\undefined%
\gdef\SetFigFont#1#2#3#4#5{%
  \reset@font\fontsize{#1}{#2pt}%
  \fontfamily{#3}\fontseries{#4}\fontshape{#5}%
  \selectfont}%
\fi\endgroup%
\begin{picture}(484,245)(18,582)
\thinlines
\put(475,800){\line( 1,-1){ 25}}
\put(500,775){\line( 0,-1){145}}
\put(500,630){\vector(-1, 0){ 60}}
\put(325,800){\line( 3,-4){ 15}}
\put(340,780){\line( 0,-1){150}}
\put(340,630){\vector(-1, 0){ 60}}
\put(220,650){\vector( 1, 0){ 40}}
\put( 40,630){\vector( 1, 0){ 40}}
\put( 20,640){\line( 1, 0){ 40}}
\put(200,685){\line( 0,-1){ 45}}
\put(200,640){\line( 1, 0){ 40}}
\put(240,640){\line( 0, 1){ 45}}
\put( 20,685){\line( 0,-1){ 65}}
\put( 20,620){\line( 1, 0){ 40}}
\put( 60,620){\line( 0, 1){ 65}}
\put(200,660){\line( 1, 0){ 40}}
\put( 40,650){\vector( 1, 0){ 40}}
\put(360,825){\line( 0,-1){ 25}}
\put(360,800){\line( 1, 0){ 40}}
\put(400,800){\line( 0, 1){ 25}}
\put(200,680){\line( 1, 0){ 40}}
\put(360,820){\line( 1, 0){ 40}}
\put(200,800){\line( 1, 0){ 40}}
\put(200,820){\line( 1, 0){ 40}}
\put(240,825){\line( 0,-1){ 30}}
\put(200,825){\line( 0,-1){ 30}}
\put( 20,680){\line( 1, 0){ 40}}
\put( 20,660){\line( 1, 0){ 40}}
\put( 20,800){\line( 1, 0){ 40}}
\put( 60,825){\line( 0,-1){ 30}}
\put( 20,820){\line( 1, 0){ 40}}
\put( 20,825){\line( 0,-1){ 30}}
\multiput(280,680)(2.30769,9.23077){14}{\makebox(0.1111,0.7778){\SetFigFont{5}{6}{\rmdefault}{\mddefault}{\updefault}.}}
\put(220,810){\line( 1, 0){ 90}}
\put(380,810){\vector( 1, 0){ 75}}
\put(220,670){\vector( 1, 0){ 50}}
\put(220,820){\makebox(0.1111,0.7778){\SetFigFont{5}{6}{\rmdefault}{\mddefault}{\updefault}.}}
\put(150,795){\vector( 2,-3){ 10}}
\put( 40,810){\vector( 1, 0){100}}
\put(105,660){\vector( 2,-3){ 10}}
\multiput(100,680)(3.07692,9.23077){14}{\makebox(0.1111,0.7778){\SetFigFont{5}{6}{\rmdefault}{\mddefault}{\updefault}.}}
\put(100,680){\makebox(0.1111,0.7778){\SetFigFont{5}{6}{\rmdefault}{\mddefault}{\updefault}.}}
\put( 40,670){\vector( 1, 0){ 50}}
\put( 20,795){\line( 0,-1){110}}
\put( 60,795){\line( 0,-1){110}}
\put(200,795){\line( 0,-1){110}}
\put(240,795){\line( 0,-1){110}}
\put(215,585){\makebox(0,0)[lb]{\smash{\SetFigFont{11}{13.2}{\familydefault}{\mddefault}{\updefault}After Update n}}}
\put( 25,585){\makebox(0,0)[lb]{\smash{\SetFigFont{11}{13.2}{\familydefault}{\mddefault}{\updefault}Before executing Update n}}}
\put(465,805){\makebox(0,0)[lb]{\smash{\SetFigFont{11}{13.2}{\familydefault}{\mddefault}{\updefault}\#}}}
\put(265,645){\makebox(0,0)[lb]{\smash{\SetFigFont{11}{13.2}{\familydefault}{\mddefault}{\updefault}f}}}
\put(425,625){\makebox(0,0)[lb]{\smash{\SetFigFont{11}{13.2}{\familydefault}{\mddefault}{\updefault}e}}}
\put(365,585){\makebox(0,0)[lb]{\smash{\SetFigFont{11}{13.2}{\familydefault}{\mddefault}{\updefault}After Pop n}}}
\put(265,625){\makebox(0,0)[lb]{\smash{\SetFigFont{11}{13.2}{\familydefault}{\mddefault}{\updefault}e}}}
\put(315,805){\makebox(0,0)[lb]{\smash{\SetFigFont{11}{13.2}{\familydefault}{\mddefault}{\updefault}\#}}}
\put( 85,625){\makebox(0,0)[lb]{\smash{\SetFigFont{11}{13.2}{\familydefault}{\mddefault}{\updefault}e}}}
\put( 85,645){\makebox(0,0)[lb]{\smash{\SetFigFont{11}{13.2}{\familydefault}{\mddefault}{\updefault}f}}}
\put(165,770){\makebox(0,0)[lb]{\smash{\SetFigFont{11}{13.2}{\familydefault}{\mddefault}{\updefault}en}}}
\put(120,640){\makebox(0,0)[lb]{\smash{\SetFigFont{11}{13.2}{\familydefault}{\mddefault}{\updefault}e1}}}
\put(275,665){\makebox(0,0)[lb]{\smash{\SetFigFont{11}{13.2}{\familydefault}{\mddefault}{\updefault}@}}}
\put(140,800){\makebox(0,0)[lb]{\smash{\SetFigFont{11}{13.2}{\familydefault}{\mddefault}{\updefault}@}}}
\put( 95,665){\makebox(0,0)[lb]{\smash{\SetFigFont{11}{13.2}{\familydefault}{\mddefault}{\updefault}@}}}
\end{picture}

\caption{Mark~2 G-machine (executing \mbox{\tt [Update\ n,\ Pop\ n]})}
\label{gm:fig:upd}
\end{figure*}

Let us first consider the necessary modifications to the data structures.

\subsection{Data structures}

In place of the single instruction $\mbox{\tt Slide}\indexTT{Slide}~n+1$ that we generated last
time we now generate the sequence of instructions $[\mbox{\tt Update}\indexTT{Update}\ n, \
\mbox{\tt Pop}\indexTT{Pop}\ n]$. Therefore we are going to have to include these
instructions in the new instruction set.
\begin{verbatim}
> data Instruction = Unwind
>                 | Pushglobal Name
>                 | Pushint Int
>                 | Push Int
>                 | Mkap
>                 | Update Int
>                 | Pop Int
> instance Eq Instruction 
>    where
>    Unwind          == Unwind               = True
>    Pushglobal a    == Pushglobal b         = a == b
>    Pushint a       == Pushint b            = a == b
>    Push a          == Push b               = a == b
>    Mkap            == Mkap                 = True
>    Update a        == Update b             = a == b
>    _               == _                    = False
\end{verbatim}
\indexDTT{data}%
\indexDTT{Pushglobal}%
\indexDTT{Pushint}%
\indexDTT{Push}%
\indexDTT{Mkap}%
\indexDTT{Update}%
\indexDTT{Pop}%
\indexDTT{instance}%
\begin{exercise}\label{gm:X:showinst2}
Modify the function \mbox{\tt showInstruction}\indexTT{showInstruction}, so that it displays the new
instructions.
\end{exercise}

To implement the indirection nodes we must have a new node type in the
heap: \mbox{\tt NInd}\indexTT{NInd} which we use for indirections.
\begin{verbatim}
> data Node 
>  = NNum Int              -- Numbers
>  | NAp Addr Addr         -- Applications
>  | NGlobal Int GmCode    -- Globals
>  | NInd Addr             -- Indirections
> instance Eq Node
>   where
>   NNum a       == NNum b          = a == b    -- needed to check conditions
>   NAp a b      == NAp c d         = False     -- not needed
>   NGlobal a b  == NGlobal c d     = False     -- not needed
>   NInd a       == NInd b          = False     -- not needed
\end{verbatim}
\indexDTT{data}%
\indexDTT{NAp}%
\indexDTT{NGlobal}%
\indexDTT{NInd}%
\indexDTT{instance}%
Again we must redefine the display function \mbox{\tt showNode}\indexTT{showNode}, so that it
reflects the extension of the data type.

\begin{exercise}\label{gm:X:shownode2}
Make the necessary change to \mbox{\tt showNode}\indexTT{showNode}.
\end{exercise}

We have not yet given a semantics to the two new instructions. This is
done below.

\subsection{The evaluator}
\label{gm:sc:eval2}

The effect of an \mbox{\tt Update}\indexTT{Update}~$n$ instruction is to overwrite the
$n+1^{\mbox{\footnotesize th}}$ stack item with an indirection to the
item on top of the stack. Notice that this addressing mode is
different from that used in \cite{PJBook}. For the intended application
of this instruction the $a_1\ldots a_n$ are the $n$ application nodes
forming the spine, and $a_0$ is the function node.

\gmrule%
{\gmstate{\mbox{\tt Update}\indexTT{Update}\ n:i}{a:a_0:\ldots:a_n:s}{h}{m}}%
{\gmstate{i}{a_0:\ldots:a_n:s}{h[a_n: \mbox{\tt NInd}\indexTT{NInd}\ a]}{m}}

The \mbox{\tt Pop}\indexTT{Pop}~$n$ instruction simply removes $n$ stack items.
Again, in the Mark~2 G-machine $a_1\ldots a_n$ are the application
nodes forming the spine of the redex.

\gmrule%
{\gmstate{\mbox{\tt Pop}\indexTT{Pop}\ n:i}{a_1:\ldots:a_n:s}{h}{m}}%
{\gmstate{i}{s}{h}{m}}

We must also define a transition for \mbox{\tt Unwind}\indexTT{Unwind} when the top of stack
item is an indirection. The effect is to replace the current stack
item with the item that the indirection points to.

\gmrule%
{\gmstate{[\mbox{\tt Unwind}\indexTT{Unwind}]}{a_0:s}{h[a_0: \mbox{\tt NInd}\indexTT{NInd}\ a]}{m}}%
{\gmstate{[\mbox{\tt Unwind}\indexTT{Unwind}]}{a:s}{h}{m}}

\begin{exercise}\label{gm:X:step2}
Modify the \mbox{\tt dispatch}\indexTT{dispatch} function of the Mark~1 machine to incorporate the new
instructions; implement the new transition rules.
\end{exercise}

\subsection{The compiler\index{G-machine!compiler Mark 2}}
\label{gm:sc:compiler2}

The only change to the compiler lies in the code generated by the
\tR{} scheme. The new definition is given in
Figure~\ref{gm:fg:schemes2}.
\begin{figure*} %\centering
$\begin{array}{|rcll|}
\hline
&&&\\
\multicolumn{4}{|l|}{\parbox[t]{29pc}{%
$\R{e}~\rho~d$ generates code which instantiates the expression $e$ in
environment $\rho$, for a supercombinator of arity $d$, and then proceeds
to unwind the resulting stack.}}\\
&&&\\
\indexD{$\protect\cal \pR$}\R{e}\ \rho\ d &
     = & \C{e}\ \rho \append [\mbox{\tt Update}\indexTT{Update}\ d,\, \mbox{\tt Pop}\indexTT{Pop}\ d,\,\mbox{\tt Unwind}\indexTT{Unwind}] & \\
&&&\\
\hline
\end{array}$
\caption{The \tR{} compilation scheme for Mark~2 G-machine}
\label{gm:fg:schemes2}
\end{figure*}

\begin{exercise}\label{gm:X:compileR}
Modify \mbox{\tt compileR}\indexTT{compileR} to implement the new \tR{} scheme.
\end{exercise}

\begin{exercise}\label{gm:X:run2}
Run the lazy evaluator on the program:
\begin{verbatim}
        twice f x = f (f x)
        id x = x
        main = twice twice id 3
\end{verbatim}
How many steps does it take? Why is it different from that obtained for
the Mark 1 machine? Is it fair to compare the number of steps taken
for the machines?
\end{exercise}

\section{Mark 3: \mbox{\tt let(rec)}\indexTT{let(rec)} expressions\index{G-machine!Mark 3}}
\label{gm:sc:mark3}
\index{let expressions@\mbox{\tt let}\indexTT{let} expressions!in G-machine}
\index{letrec expressions@\mbox{\tt letrec}\indexTT{letrec} expressions!in G-machine}

We now extend the language so that the compiler will accept
supercombinators whose body includes {\em \mbox{\tt let(rec)}\indexTT{let(rec)}-bound
variables}\index{let(rec)-bound variables@\mbox{\tt let(rec)}\indexTT{let(rec)}-bound variables}.  
These are represented
in the data type \mbox{\tt coreExpr}\indexTT{coreExpr} by the constructor \mbox{\tt ELet}\indexTT{ELet}.  It takes three
arguments: a boolean flag which says whether the definitions are to be
treated recursively, the definitions themselves and an expression in
which the definitions are to be used.

Before we attempt to extend the machine by adding local definitions,
we will have another look at the stack. In particular we will try to
define a more efficient {\em access method\/}\index{access method, for
variables} for variables. Besides the efficiency argument, we also
wish to make access to locally bound variables the same as that used
to bind function parameters.

\subsubsection{Argument access from the stack}

Suppose that the unwinding process has reached a supercombinator node
\mbox{\tt f}, and that the supercombinator takes $n$ arguments. In the Mark~1
machine the stack will be in the state shown in the left-hand diagram of
Figure~\ref{gm:X:arg}.
\begin{figure} %\centering
\setlength{\unitlength}{59204sp}%
%
\begingroup\makeatletter\ifx\SetFigFont\undefined%
\gdef\SetFigFont#1#2#3#4#5{%
  \reset@font\fontsize{#1}{#2pt}%
  \fontfamily{#3}\fontseries{#4}\fontshape{#5}%
  \selectfont}%
\fi\endgroup%
\begin{picture}(399,362)(18,465)
\thinlines
\put(240,565){\line( 0,-1){ 45}}
\put(240,520){\line( 1, 0){ 40}}
\put(280,520){\line( 0, 1){ 45}}
\put(290,540){\vector( 1, 0){ 35}}
\put(290,540){\line(-3, 2){ 15}}
\put(275,550){\line(-1, 0){ 15}}
\put(285,530){\vector( 3,-1){ 30}}
\put(285,530){\line(-1, 0){ 25}}
\put(345,690){\vector( 2,-1){ 30}}
\put(345,690){\line(-1, 0){ 85}}
\put(280,725){\line( 0,-1){ 55}}
\put(240,725){\line( 0,-1){ 55}}
\put(240,680){\line( 1, 0){ 40}}
\put( 20,560){\line( 1, 0){ 40}}
\put( 20,580){\line( 1, 0){ 40}}
\put( 20,800){\line( 1, 0){ 40}}
\put( 20,720){\line( 1, 0){ 40}}
\put( 20,700){\line( 1, 0){ 40}}
\put( 40,710){\vector( 1, 0){ 95}}
\put( 40,810){\vector( 1, 0){130}}
\multiput(100,580)(3.07692,9.23077){14}{\makebox(0.1111,0.7778){\SetFigFont{5}{6}{\rmdefault}{\mddefault}{\updefault}.}}
\multiput(150,720)(3.12500,9.37500){9}{\makebox(0.1111,0.7778){\SetFigFont{5}{6}{\rmdefault}{\mddefault}{\updefault}.}}
\put( 95,560){\makebox(0.1111,0.7778){\SetFigFont{5}{6}{\rmdefault}{\mddefault}{\updefault}.}}
\put( 95,560){\makebox(0.1111,0.7778){\SetFigFont{5}{6}{\rmdefault}{\mddefault}{\updefault}.}}
\multiput( 95,565)(-0.25000,-0.50000){21}{\makebox(0.1111,0.7778){\SetFigFont{5}{6}{\rmdefault}{\mddefault}{\updefault}.}}
\put(105,565){\line( 1,-3){  5}}
\put(150,700){\line( 2,-3){ 10}}
\put(185,805){\line( 2,-3){ 10}}
\put( 95,540){\line( 1,-4){  5}}
\put( 40,570){\makebox(0.1111,0.7778){\SetFigFont{5}{6}{\rmdefault}{\mddefault}{\updefault}.}}
\put( 20,730){\line( 0,-1){ 45}}
\put( 60,730){\line( 0,-1){ 45}}
\put( 40,570){\vector( 1, 0){ 50}}
\put( 40,550){\vector( 1, 0){ 40}}
\put( 20,820){\line( 1, 0){ 40}}
\put( 20,825){\line( 0,-1){ 30}}
\put( 60,825){\line( 0,-1){ 30}}
\put( 20,540){\line( 1, 0){ 40}}
\put( 20,585){\line( 0,-1){ 65}}
\put( 20,520){\line( 1, 0){ 40}}
\put( 60,520){\line( 0, 1){ 65}}
\multiput( 85,540)(-0.25000,-0.50000){21}{\makebox(0.1111,0.7778){\SetFigFont{5}{6}{\rmdefault}{\mddefault}{\updefault}.}}
\put( 40,530){\vector( 1, 0){ 25}}
\multiput(305,540)(-0.25000,-0.50000){21}{\makebox(0.1111,0.7778){\SetFigFont{5}{6}{\rmdefault}{\mddefault}{\updefault}.}}
\put(240,540){\line( 1, 0){ 40}}
\put(280,825){\line( 0,-1){ 30}}
\put(240,825){\line( 0,-1){ 30}}
\put(240,820){\line( 1, 0){ 40}}
\put(315,540){\line( 1,-4){  5}}
\put(405,805){\line( 2,-3){ 10}}
\put(370,700){\line( 2,-3){ 10}}
\put(325,565){\line( 1,-3){  5}}
\multiput(315,565)(-0.25000,-0.50000){21}{\makebox(0.1111,0.7778){\SetFigFont{5}{6}{\rmdefault}{\mddefault}{\updefault}.}}
\put(315,560){\makebox(0.1111,0.7778){\SetFigFont{5}{6}{\rmdefault}{\mddefault}{\updefault}.}}
\put(315,560){\makebox(0.1111,0.7778){\SetFigFont{5}{6}{\rmdefault}{\mddefault}{\updefault}.}}
\multiput(370,720)(3.12500,9.37500){9}{\makebox(0.1111,0.7778){\SetFigFont{5}{6}{\rmdefault}{\mddefault}{\updefault}.}}
\multiput(320,580)(3.07692,9.23077){14}{\makebox(0.1111,0.7778){\SetFigFont{5}{6}{\rmdefault}{\mddefault}{\updefault}.}}
\put(260,810){\vector( 1, 0){130}}
\put(260,710){\vector( 1, 0){ 95}}
\put(240,700){\line( 1, 0){ 40}}
\put(240,720){\line( 1, 0){ 40}}
\put(240,800){\line( 1, 0){ 40}}
\put(240,560){\line( 1, 0){ 40}}
\put( 20,685){\line( 0,-1){100}}
\put( 60,685){\line( 0,-1){100}}
\put(240,670){\line( 0,-1){105}}
\put(280,670){\line( 0,-1){105}}
\put(280,795){\line( 0,-1){ 70}}
\put(240,725){\line( 0, 1){ 70}}
\put( 60,795){\line( 0,-1){ 65}}
\put( 20,795){\line( 0,-1){ 65}}
\put(245,465){\makebox(0,0)[lb]{\smash{\SetFigFont{11}{13.2}{\familydefault}{\mddefault}{\updefault}Mark 3 onwards}}}
\put( 25,465){\makebox(0,0)[lb]{\smash{\SetFigFont{11}{13.2}{\familydefault}{\mddefault}{\updefault}Marks 1 and 2}}}
\put( 95,565){\makebox(0,0)[lb]{\smash{\SetFigFont{11}{13.2}{\familydefault}{\mddefault}{\updefault}@}}}
\put( 85,545){\makebox(0,0)[lb]{\smash{\SetFigFont{11}{13.2}{\familydefault}{\mddefault}{\updefault}@}}}
\put(140,705){\makebox(0,0)[lb]{\smash{\SetFigFont{11}{13.2}{\familydefault}{\mddefault}{\updefault}@}}}
\put(175,805){\makebox(0,0)[lb]{\smash{\SetFigFont{11}{13.2}{\familydefault}{\mddefault}{\updefault}@}}}
\put( 95,510){\makebox(0,0)[lb]{\smash{\SetFigFont{11}{13.2}{\familydefault}{\mddefault}{\updefault}e1}}}
\put(110,535){\makebox(0,0)[lb]{\smash{\SetFigFont{11}{13.2}{\familydefault}{\mddefault}{\updefault}e2}}}
\put(160,670){\makebox(0,0)[lb]{\smash{\SetFigFont{11}{13.2}{\familydefault}{\mddefault}{\updefault}en}}}
\put( 70,520){\makebox(0,0)[lb]{\smash{\SetFigFont{11}{13.2}{\familydefault}{\mddefault}{\updefault}f}}}
\put(290,520){\makebox(0,0)[lb]{\smash{\SetFigFont{11}{13.2}{\familydefault}{\mddefault}{\updefault}f}}}
\put(380,670){\makebox(0,0)[lb]{\smash{\SetFigFont{11}{13.2}{\familydefault}{\mddefault}{\updefault}en}}}
\put(330,535){\makebox(0,0)[lb]{\smash{\SetFigFont{11}{13.2}{\familydefault}{\mddefault}{\updefault}e2}}}
\put(315,510){\makebox(0,0)[lb]{\smash{\SetFigFont{11}{13.2}{\familydefault}{\mddefault}{\updefault}e1}}}
\put(395,805){\makebox(0,0)[lb]{\smash{\SetFigFont{11}{13.2}{\familydefault}{\mddefault}{\updefault}@}}}
\put(360,705){\makebox(0,0)[lb]{\smash{\SetFigFont{11}{13.2}{\familydefault}{\mddefault}{\updefault}@}}}
\put(305,545){\makebox(0,0)[lb]{\smash{\SetFigFont{11}{13.2}{\familydefault}{\mddefault}{\updefault}@}}}
\put(315,565){\makebox(0,0)[lb]{\smash{\SetFigFont{11}{13.2}{\familydefault}{\mddefault}{\updefault}@}}}
\end{picture}

\caption{Stack layout on entry to function \mbox{\tt f}}
\label{gm:X:arg}
\end{figure}

Having reached the supercombinator \mbox{\tt f}, in the Mark~3 G-machine, the
stack is slightly modified. The equivalent Mark~3 stack is shown in
the right-hand diagram of Figure~\ref{gm:X:arg}; the top $n$ elements
now point directly to the expressions {\tt e1} \ldots {\tt en}. The
important point here is that we have faster access to the
variables (provided that the variable is accessed at least
once). This is because we only look at the application node once, to
get its right-hand argument.

This improves the efficiency of access to the expressions that will be 
substituted for the formal parameters in the supercombinator. In terms
of the Mark~1 machine:
\begin{itemize}
\item we no longer need the function \mbox{\tt getArg}\indexTT{getArg} in the \mbox{\tt Push}\indexTT{Push}
instruction,
\item but we do need to rearrange the stack when we \mbox{\tt Unwind}\indexTT{Unwind} a
supercombinator with sufficient arguments.
\end{itemize}

Notice that we have retained a pointer to the root of the redex so
that we can perform an \mbox{\tt Update}\indexTT{Update}.

\subsubsection{The effects on instructions\index{G-machine stack layout!revised}}

When we choose to use the new stack layout, we necessarily have to
modify certain of the machine instructions to cope. The instructions
affected are \mbox{\tt Push}\indexTT{Push} and \mbox{\tt Unwind}\indexTT{Unwind}. The \mbox{\tt Push}\indexTT{Push} instruction will have to
change because we do not need to `look through'\index{look through,
of application nodes} the application node to get at the argument.

\gmrule%
{\gmstate{\mbox{\tt Push}\indexTT{Push}\ n:i}{a_0:\ldots:a_n:s}{h}{m}}%
{\gmstate{i}{a_n:a_0:\ldots:a_n:s}{h}{m}}

The other modification required for the new stack layout is that
\mbox{\tt Unwind}\indexTT{Unwind} must rearrange the stack. This {\em
rearrangement}\index{rearrange} is required whenever a supercombinator
with sufficient arguments is found on the top of the stack. The new
transition rule for \mbox{\tt Unwind}\indexTT{Unwind} is:

\gmrule%
{\gmstate{[\mbox{\tt Unwind}\indexTT{Unwind}]}{a_0:\ldots:a_n:s}%
{h\left[\begin{array}{lll}
a_0 & : & \mbox{\tt NGlobal}\indexTT{NGlobal}\ n\  c\\
a_1 & : & \mbox{\tt NAp}\indexTT{NAp}\ a_0\ a_1'\\
\multicolumn{3}{c}{\cdots}\\
a_n & : & \mbox{\tt NAp}\indexTT{NAp}\ a_{n-1}\ a_n'
\end{array}\right]}{m}}%
{\gmstate{c}{a_1':\ldots:a_n':a_n:s}{h}{m}}
Notice that this definition of \mbox{\tt Unwind}\indexTT{Unwind} will work properly for the
case where $n$ is zero.

\begin{exercise}\label{gm:X:mark3stackinst}
Rewrite the \mbox{\tt dispatch}\indexTT{dispatch} function and the new transitions for the new
instruction set.  You should make use of the function \mbox{\tt rearrange}\indexTT{rearrange} to
rearrange the stack.
\begin{verbatim}
> rearrange :: Int -> GmHeap -> GmStack -> GmStack
> rearrange n heap as
>       = take n as' ++ drop n as
>       where as' = map (getArg . hLookup heap) (tl as)
\end{verbatim}
\indexDTT{rearrange}%
\indexDTT{rearrange}%
\end{exercise}

\begin{exercise}\label{gm:X:mark3R}
Test the compiler and new abstract machine on some sample programs
from Appendix~\ref{sect:core-progs}, to ensure that the implementation
still works.
\end{exercise}

\subsection{Locally bound variables}

Now we return to the implementation of \mbox{\tt let(rec)}\indexTT{let(rec)} expressions,
considering the non-recursive case first. The variables $x_1
\ldots x_n$, in the expression
$\mbox{\tt let}\indexTT{let}~x_1\mbox{\tt =}e_1\mbox{\tt ;}~\ldots\mbox{\tt ;}~x_n=e_n~\mbox{\tt in}~e$, can be treated in the
same way as the arguments to a supercombinator, once the expressions
$e_1\ldots e_n$ have been created. That is, we access the variables
$x_1 \ldots x_n$ via offsets into the stack, using the environment to
record their locations.

Suppose that the code to build the local definitions is called \mbox{\tt Code}\indexTT{Code},
then the sequence of actions shown in Figure~\ref{gm:fig:let} will be
necessary. Initially, the stack will contain pointers to the arguments
to the supercombinator. After the code to build the local definitions
has executed we will have $n$ new pointers on the stack. We can now
proceed to build the body of the \mbox{\tt let}\indexTT{let} expression, in a new
environment that maps $x_i$ to the pointer to $e_i$. Finally, we need
to throw away the pointers to the expressions $e_1\ldots e_n$ from the
stack.
\begin{figure} %\centering
\setlength{\unitlength}{59204sp}%
%
\begingroup\makeatletter\ifx\SetFigFont\undefined%
\gdef\SetFigFont#1#2#3#4#5{%
  \reset@font\fontsize{#1}{#2pt}%
  \fontfamily{#3}\fontseries{#4}\fontshape{#5}%
  \selectfont}%
\fi\endgroup%
\begin{picture}(370,200)(15,622)
\thinlines
\multiput(290,760)(0.00000,-10.00000){8}{\makebox(0.1111,0.7778){\SetFigFont{5}{6}{\rmdefault}{\mddefault}{\updefault}.}}
\multiput(190,760)(0.00000,-9.28571){8}{\makebox(0.1111,0.7778){\SetFigFont{5}{6}{\rmdefault}{\mddefault}{\updefault}.}}
\put(340,770){\vector( 1, 0){ 40}}
\put(340,790){\vector( 1, 0){ 40}}
\put(240,670){\vector( 1, 0){ 40}}
\put(240,690){\vector( 1, 0){ 40}}
\put(240,770){\vector( 1, 0){ 40}}
\put(240,790){\vector( 1, 0){ 40}}
\put(140,690){\vector( 1, 0){ 40}}
\put(140,770){\vector( 1, 0){ 40}}
\put(140,790){\vector( 1, 0){ 40}}
\put( 40,790){\vector( 1, 0){ 40}}
\put(320,780){\line( 1, 0){ 40}}
\put(320,800){\line( 1, 0){ 40}}
\put(320,805){\line( 0,-1){ 45}}
\put(320,760){\line( 1, 0){ 40}}
\put(360,760){\line( 0, 1){ 45}}
\put(220,760){\line( 1, 0){ 40}}
\put(220,780){\line( 1, 0){ 40}}
\put(220,800){\line( 1, 0){ 40}}
\put(220,680){\line( 1, 0){ 40}}
\put(220,700){\line( 1, 0){ 40}}
\put(220,705){\line( 0,-1){ 45}}
\put(220,660){\line( 1, 0){ 40}}
\put(260,660){\line( 0, 1){ 45}}
\put(260,805){\line( 0,-1){ 57}}
\put(220,805){\line( 0,-1){ 55}}
\put(120,700){\line( 1, 0){ 40}}
\put(120,710){\line( 0,-1){ 30}}
\put(120,680){\line( 1, 0){ 40}}
\put(160,680){\line( 0, 1){ 28}}
\put(120,800){\line( 1, 0){ 40}}
\put(120,780){\line( 1, 0){ 40}}
\put(120,760){\line( 1, 0){ 40}}
\put(160,805){\line( 0,-1){ 55}}
\put(120,805){\line( 0,-1){ 55}}
\put( 20,800){\line( 1, 0){ 40}}
\put( 20,805){\line( 0,-1){ 25}}
\put( 20,780){\line( 1, 0){ 40}}
\put( 60,780){\line( 0, 1){ 25}}
\put( 20,820){\makebox(0.1111,0.7778){\SetFigFont{5}{6}{\rmdefault}{\mddefault}{\updefault}.}}
\put(120,750){\line( 0,-1){ 40}}
\put(160,750){\line( 0,-1){ 40}}
\put(220,750){\line( 0,-1){ 45}}
\put(260,750){\line( 0,-1){ 45}}
\put(310,730){\makebox(0,0)[lb]{\smash{\SetFigFont{11}{13.2}{\familydefault}{\mddefault}{\updefault}On completion}}}
\put(205,625){\makebox(0,0)[lb]{\smash{\SetFigFont{11}{13.2}{\familydefault}{\mddefault}{\updefault}After Building e}}}
\put(105,625){\makebox(0,0)[lb]{\smash{\SetFigFont{11}{13.2}{\familydefault}{\mddefault}{\updefault}After Code}}}
\put( 15,625){\makebox(0,0)[lb]{\smash{\SetFigFont{11}{13.2}{\familydefault}{\mddefault}{\updefault}Initially}}}
\put(285,665){\makebox(0,0)[lb]{\smash{\SetFigFont{11}{13.2}{\familydefault}{\mddefault}{\updefault}e}}}
\put(285,685){\makebox(0,0)[lb]{\smash{\SetFigFont{11}{13.2}{\familydefault}{\mddefault}{\updefault}en}}}
\put(185,685){\makebox(0,0)[lb]{\smash{\SetFigFont{11}{13.2}{\familydefault}{\mddefault}{\updefault}en}}}
\put(385,765){\makebox(0,0)[lb]{\smash{\SetFigFont{11}{13.2}{\familydefault}{\mddefault}{\updefault}e}}}
\put(285,765){\makebox(0,0)[lb]{\smash{\SetFigFont{11}{13.2}{\familydefault}{\mddefault}{\updefault}e1}}}
\put(185,765){\makebox(0,0)[lb]{\smash{\SetFigFont{11}{13.2}{\familydefault}{\mddefault}{\updefault}e1}}}
\end{picture}

\caption{Stack usage in non-recursive local definitions}
\label{gm:fig:let}
\end{figure}

Because we have added $n$ new variables to the stack ($x_1 \ldots
x_n$) we must note this fact in the variable map we use to compile
$e$. The code to construct the local bindings -- which we have called
\mbox{\tt Code}\indexTT{Code} -- will simply build the graph of each expression $e_1 \ldots
e_n$ in turn, leaving the address of the piece of graph on the stack.

After building the body expression $e$ -- which may use any of the
variables $x_1 \ldots x_n$ -- we must remove the pointers to $e_1
\ldots e_n$ from the stack. This is accomplished by using a \mbox{\tt Slide}\indexTT{Slide}
instruction. The complete scheme for compiling a non-recursive local
definition is given in Figure~\ref{gm:fg:schemes3}
(p.\pageref{gm:fg:schemes3}).

The situation with recursive local definitions is more complicated:
each of the expressions $e_1 \ldots e_n$ must be compiled so that the
variables $x_1\ldots x_n$ are in scope. To do this we create empty
nodes in the graph, leaving pointers to them on the stack. Each
expression $e_1 \ldots e_n$ is then compiled using the same variable
map that we used for the compilation of the body of the non-recursive
case. At the end of each expression's compiled code we place an
\mbox{\tt Update}\indexTT{Update} instruction that will overwrite the empty node with the
correct piece of graph.
To do this we need one new instruction -- \mbox{\tt Alloc\ n} -- which will
create \mbox{\tt n} empty graph nodes for us. In Figure~\ref{gm:fig:letrec} the
empty graph nodes are represented by a \mbox{\tt ?} symbol.
\begin{figure} %\centering
\setlength{\unitlength}{59204sp}%
%
\begingroup\makeatletter\ifx\SetFigFont\undefined%
\gdef\SetFigFont#1#2#3#4#5{%
  \reset@font\fontsize{#1}{#2pt}%
  \fontfamily{#3}\fontseries{#4}\fontshape{#5}%
  \selectfont}%
\fi\endgroup%
\begin{picture}(407,185)(15,622)
\thinlines
\put(395,770){\line( 1, 0){ 25}}
\put(420,770){\line( 0,-1){100}}
\put(420,670){\vector(-1, 0){ 20}}
\put(340,690){\vector( 1, 0){ 40}}
\put(320,760){\line( 1, 0){ 40}}
\put(360,805){\line( 0,-1){ 50}}
\put(320,805){\line( 0,-1){ 50}}
\put(320,700){\line( 1, 0){ 40}}
\put(320,705){\line( 0,-1){ 25}}
\put(320,680){\line( 1, 0){ 40}}
\put(360,680){\line( 0, 1){ 25}}
\multiput(290,760)(0.00000,-10.00000){7}{\makebox(0.1111,0.7778){\SetFigFont{5}{6}{\rmdefault}{\mddefault}{\updefault}.}}
\multiput(190,755)(0.00000,-9.16667){7}{\makebox(0.1111,0.7778){\SetFigFont{5}{6}{\rmdefault}{\mddefault}{\updefault}.}}
\put(340,770){\vector( 1, 0){ 40}}
\put(340,790){\vector( 1, 0){ 40}}
\put(240,670){\vector( 1, 0){ 40}}
\put(240,690){\vector( 1, 0){ 40}}
\put(240,770){\vector( 1, 0){ 40}}
\put(240,790){\vector( 1, 0){ 40}}
\put(140,690){\vector( 1, 0){ 40}}
\put(140,770){\vector( 1, 0){ 40}}
\put(140,790){\vector( 1, 0){ 40}}
\put( 40,790){\vector( 1, 0){ 40}}
\put(320,780){\line( 1, 0){ 40}}
\put(320,800){\line( 1, 0){ 40}}
\put(220,760){\line( 1, 0){ 40}}
\put(220,780){\line( 1, 0){ 40}}
\put(220,800){\line( 1, 0){ 40}}
\put(220,680){\line( 1, 0){ 40}}
\put(220,700){\line( 1, 0){ 40}}
\put(220,705){\line( 0,-1){ 45}}
\put(220,660){\line( 1, 0){ 40}}
\put(260,660){\line( 0, 1){ 45}}
\put(260,805){\line( 0,-1){ 50}}
\put(220,805){\line( 0,-1){ 50}}
\put(120,700){\line( 1, 0){ 40}}
\put(120,705){\line( 0,-1){ 25}}
\put(120,680){\line( 1, 0){ 40}}
\put(160,680){\line( 0, 1){ 25}}
\put(120,800){\line( 1, 0){ 40}}
\put(120,780){\line( 1, 0){ 40}}
\put(120,760){\line( 1, 0){ 40}}
\put(160,805){\line( 0,-1){ 55}}
\put(120,805){\line( 0,-1){ 55}}
\put( 20,800){\line( 1, 0){ 40}}
\put( 20,805){\line( 0,-1){ 25}}
\put( 20,780){\line( 1, 0){ 40}}
\put( 60,780){\line( 0, 1){ 25}}
\put(120,750){\line( 0,-1){ 45}}
\put(160,750){\line( 0,-1){ 45}}
\put(220,755){\line( 0,-1){ 50}}
\put(260,755){\line( 0,-1){ 50}}
\put(320,755){\line( 0,-1){ 50}}
\put(360,755){\line( 0,-1){ 50}}
\put(325,625){\makebox(0,0)[lb]{\smash{\SetFigFont{11}{13.2}{\familydefault}{\mddefault}{\updefault}After x1 is updated}}}
\put(385,765){\makebox(0,0)[lb]{\smash{\SetFigFont{11}{13.2}{\familydefault}{\mddefault}{\updefault}\#}}}
\put(190,625){\makebox(0,0)[lb]{\smash{\SetFigFont{11}{13.2}{\familydefault}{\mddefault}{\updefault}After building e1}}}
\put(385,685){\makebox(0,0)[lb]{\smash{\SetFigFont{11}{13.2}{\familydefault}{\mddefault}{\updefault}?}}}
\put(385,665){\makebox(0,0)[lb]{\smash{\SetFigFont{11}{13.2}{\familydefault}{\mddefault}{\updefault}e1}}}
\put(110,625){\makebox(0,0)[lb]{\smash{\SetFigFont{11}{13.2}{\familydefault}{\mddefault}{\updefault}Alloc n}}}
\put(285,665){\makebox(0,0)[lb]{\smash{\SetFigFont{11}{13.2}{\familydefault}{\mddefault}{\updefault}e1}}}
\put(285,765){\makebox(0,0)[lb]{\smash{\SetFigFont{11}{13.2}{\familydefault}{\mddefault}{\updefault}?}}}
\put(285,685){\makebox(0,0)[lb]{\smash{\SetFigFont{11}{13.2}{\familydefault}{\mddefault}{\updefault}?}}}
\put(185,685){\makebox(0,0)[lb]{\smash{\SetFigFont{11}{13.2}{\familydefault}{\mddefault}{\updefault}?}}}
\put(185,765){\makebox(0,0)[lb]{\smash{\SetFigFont{11}{13.2}{\familydefault}{\mddefault}{\updefault}?}}}
\put( 15,625){\makebox(0,0)[lb]{\smash{\SetFigFont{11}{13.2}{\familydefault}{\mddefault}{\updefault}Initially}}}
\end{picture}

\caption{Constructing a recursively defined expression: \mbox{\tt e1}}
\label{gm:fig:letrec}
\end{figure}

The process shown in Figure~\ref{gm:fig:letrec} needs to be repeated
until each of the expressions $e_1\ldots e_n$ has been processed.
Compiling code for the body $e$ is then the same as the previous case
for non-recursive local definitions. We now add the new data types for
the Mark~3 machine.

\subsection{Data structures\index{data structures!G-machine Mark 3}}

The instruction data type includes all of the instructions of the
Mark~2 machine, with the new \mbox{\tt Alloc}\indexTT{Alloc} instruction and the \mbox{\tt Slide}\indexTT{Slide}
instruction from the Mark~1 machine.

\begin{exercise}\label{gm:X:instruction3}
Modify the data type \mbox{\tt instruction}\indexTT{instruction} so that it includes \mbox{\tt Alloc}\indexTT{Alloc} and
\mbox{\tt Slide}\indexTT{Slide}.  You will also need to modify the function \mbox{\tt showInstruction}\indexTT{showInstruction},
to accommodate these new instructions.
\end{exercise}

\subsection{The evaluator\index{evaluator!G-machine Mark 3}}
\label{gm:sc:evaluator3}

For the Mark~3 G-machine we will need to add the \mbox{\tt Alloc}\indexTT{Alloc} instruction
which creates \mbox{\tt n} locations in the heap. We use these locations to
mark the places we will store the locally bound expressions. These
nodes are initially created as indirection nodes that point to an
illegal heap address: \mbox{\tt hNull}\indexTT{hNull}. Because these nodes created by \mbox{\tt Alloc}\indexTT{Alloc}
are going to be overwritten, it does not really matter what value we
assign them.

\gmrule%
{\gmstate{\mbox{\tt Alloc}\indexTT{Alloc}\ n:i}{s}{h}{m}}%
{\gmstate{i}{a_1:\ldots:a_n:s}{h\left[\begin{array}{lll}
a_1 & : & \mbox{\tt NInd\ hNull}\\
\multicolumn{3}{c}{\cdots}\\
a_n & : & \mbox{\tt NInd\ hNull}
\end{array}\right]}{m}}

To implement \mbox{\tt alloc}\indexTT{alloc}, the transition function for the \mbox{\tt Alloc}\indexTT{Alloc} instruction, we
use an auxiliary function \mbox{\tt allocNodes}\indexTT{allocNodes}. Given the number of nodes
required and the current heap, it returns a pair consisting of the
modified heap and the list of addresses of the indirection nodes.
\begin{verbatim}
> allocNodes :: Int -> GmHeap -> (GmHeap, [Addr])
> allocNodes 0     heap = (heap,  [])
> allocNodes (n+1) heap = (heap2, a:as)
>                         where (heap1, as) = allocNodes n heap
>                               (heap2, a)  = hAlloc heap1 (NInd hNull)
\end{verbatim}
\indexDTT{allocNodes}%
\indexDTT{allocNodes}%
\indexDTT{allocNodes}%
\begin{exercise}\label{gm:X:step3}
Extend the \mbox{\tt dispatch}\indexTT{dispatch} function, with cases for the new instructions.
You should use \mbox{\tt allocNodes}\indexTT{allocNodes} to implement \mbox{\tt alloc}\indexTT{alloc}, the transition
function for the \mbox{\tt Alloc}\indexTT{Alloc} instruction.
\end{exercise}

\subsection{The compiler\index{G-machine compiler!Mark 3}}

The only change to the compiler is that there are now two more cases for which
the \tC{} scheme can compile code. The modification to \mbox{\tt compileC}\indexTT{compileC} is simple.
It can now cope with a wider range of \mbox{\tt coreExpr}\indexTT{coreExpr}s. We need two new functions:
\mbox{\tt compileLetrec}\indexTT{compileLetrec} and \mbox{\tt compileLet}\indexTT{compileLet}.
\begin{verbatim}
> compileC :: GmCompiler
> compileC (EVar v)      args 
>  | elem v (aDomain args) = [Push n]
>  | otherwise             = [Pushglobal v]
>          where n = aLookup args v (error "")
> compileC (ENum n)    env = [Pushint n]
> compileC (EAp e1 e2) env = compileC e2 env ++
>                            compileC e1 (argOffset 1 env) ++
>                            [Mkap]
> compileC (ELet recursive defs e) args 
>          | recursive     = compileLetrec compileC defs e args
>          | otherwise     = compileLet    compileC defs e args
\end{verbatim}
\indexDTT{compileC}%
\indexDTT{compileC}%
\indexDTT{compileC}%
\indexDTT{compileC}%
\indexDTT{compileC}%
\par
The definition of \mbox{\tt compileLet}\indexTT{compileLet} follows the specification given in
Figure~\ref{gm:fg:schemes3}. It takes as arguments: the compilation
scheme \mbox{\tt comp}\indexTT{comp} for the body \mbox{\tt e}, the definitions \mbox{\tt defs}\indexTT{defs} and the current
environment \mbox{\tt env}\indexTT{env}. We have provided the compiler parameter so that in
later versions of the machine we do not have to rewrite this function.
\begin{figure*}
$\begin{array}{|lcl@{}l|}
\hline
&&&\\
\multicolumn{4}{|l|}{\parbox[t]{29pc}{%
$\C{e}~\rho$ generates code which constructs the graph of $e$ in environment
$\rho$, leaving a pointer to it on top of the stack.}}\\
&&&\\
\indexD{$\protect\cal \pC$}\C{f}\ \rho     & = & [\mbox{\tt Pushglobal}\indexTT{Pushglobal}\ f] &
                \tr{where $f$ is a supercombinator} \\
%
\indexD{$\protect\cal \pC$}\C{x}\ \rho     & = & [\mbox{\tt Push}\indexTT{Push}\ (\rho\ x)] &
                        \tr{where $x$ is a local variable}      \\
%
\indexD{$\protect\cal \pC$}\C{i}\ \rho     & = & [\mbox{\tt Pushint}\indexTT{Pushint}\ i] &\\
\indexD{$\protect\cal \pC$}\C{e_0~e_1}\ \rho     & = & \C{e_1}\ \rho \append
                            \C{e_0}\ \rho^{+1} \append [\mbox{\tt Mkap}\indexTT{Mkap}] &
                  \tr{where $\rho^{+n}\ x$ is $(\rho\ x) + n$} \\
%
\multicolumn{4}{|l|}{%
\indexD{$\protect\cal \pC$}\C{\mbox{\tt let}\indexTT{let}~ x_1\mbox{\tt =}e_1\mbox{\tt ;}~\ldots\mbox{\tt ;}~ x_n\mbox{\tt =}e_n ~\mbox{\tt in}~ e}~ \rho} \\
 & = & \C{e_1}~ \rho^{+0} \append \ldots \append &\\
    &&  \C{e_n}~\rho^{+(n-1)} \append & \\
    &&  \C{e}~\rho' \append [\mbox{\tt Slide}\indexTT{Slide}~ n]
       &\tr{where $\rho' = \rho^{+n}[x_1\mapsto n-1,\ldots,x_n\mapsto 0]$}\\
%
\multicolumn{4}{|l|}{%
\indexD{$\protect\cal \pC$}\C{\mbox{\tt letrec}\indexTT{letrec}~ x_1\mbox{\tt =}e_1\mbox{\tt ;}~\ldots\mbox{\tt ;}~ x_n \mbox{\tt =} e_n ~\mbox{\tt in}~ e}~ \rho} \\
 & = & [\mbox{\tt Alloc}\indexTT{Alloc}~ n] \append & \\
    &&\C{e_1}~ \rho' \append [\mbox{\tt Update}\indexTT{Update}\ n-1]\append \ldots \append &\\
                 &&       \C{e_n}~\rho' \append [\mbox{\tt Update}\indexTT{Update}\ 0] \append &\\
               &&         \C{e}~\rho' \append [\mbox{\tt Slide}\indexTT{Slide}~ n] &
        \tr{where $\rho' = \rho^{+n}[x_1\mapsto n-1,\ldots,x_n\mapsto 0]$}\\
&&&\\
\hline
\end{array}$
\caption{The modified \tC{} compilation scheme for \mbox{\tt let}\indexTT{let} and \mbox{\tt letrec}\indexTT{letrec}}
\label{gm:fg:schemes3}
\end{figure*}
\begin{verbatim}
> compileLet :: GmCompiler -> [(Name, CoreExpr)] -> GmCompiler
> compileLet comp defs expr env
>   = compileLet' defs env ++ comp expr env' ++ [Slide (length defs)]
>     where env' = compileArgs defs env
\end{verbatim}
\indexDTT{compileLet}%
\indexDTT{compileLet}%
\par
The compilation of the new definitions is accomplished by the function
\mbox{\tt compileLet'}.
\begin{verbatim}
> compileLet' :: [(Name, CoreExpr)] -> GmEnvironment -> GmCode
> compileLet' []                  env = []
> compileLet' ((name, expr):defs) env
>     = compileC expr env ++ compileLet' defs (argOffset 1 env)
\end{verbatim}
\mbox{\tt compileLet}\indexTT{compileLet} also uses \mbox{\tt compileArgs}\indexTT{compileArgs} to modify the offsets into the
stack for the compilation of the body, \mbox{\tt e}.
\begin{verbatim}
> compileArgs :: [(Name, CoreExpr)] -> GmEnvironment -> GmEnvironment
> compileArgs defs env
>     = zip (map first defs) [n-1, n-2 .. 0] ++ argOffset n env
>             where n = length defs
\end{verbatim}
\indexDTT{compileArgs}%
\indexDTT{compileArgs}%
\subsubsection{An example}

In this example we will show how the code for the fixpoint combinator
\mbox{\tt Y}\index{Y combinator@\mbox{\tt Y} combinator} is compiled.
The definition we will use is:
\begin{verbatim}
Y f = letrec x = f x in x
\end{verbatim}
This is the so-called `knot-tying' fixpoint
combinator\index{fixpoint combinator!knot-tying}; we will see why it has this
name when we run the resulting code.  When the above definition is
compiled, the \mbox{\tt compileSc}\indexTT{compileSc} function will need to produce code for the
supercombinator.
\begin{verbatim}
compileSc ("Y", ["f"], ELet True [("x", EAp (EVar "f") (EVar "x"))] (EVar "x"))
\end{verbatim}
This in turn calls the \mbox{\tt compileR}\indexTT{compileR} function with an environment for the
variable \mbox{\tt f}; having first created a name for the supercombinator
(\mbox{\tt Y}) and its number of arguments (\mbox{\tt 1}).
\begin{verbatim}
("Y", 1, compileR e [("f", 0)])
where e = ELet True [("x", EAp (EVar "f") (EVar "x"))] (EVar "x")
\end{verbatim}
For convenience we will refer to the body of the expression as \mbox{\tt e}.
The function \mbox{\tt compileR}\indexTT{compileR} calls \mbox{\tt compileC}\indexTT{compileC}, placing the tidying-up code
at the end.
\begin{verbatim}
("Y", 1, compileC e [("f", 0)] ++ [Update 1, Pop 1, Unwind])
\end{verbatim}

Referring to the compilation scheme in Figure~\ref{gm:fg:schemes3}, we
see that to compile a \mbox{\tt letrec}\indexTT{letrec} we first create a new environment. In
the figure this is called $\rho'$; in this example we will call it
\mbox{\tt p}.  It is an extension of the initial environment in which we also
give a stack location to the local variable \mbox{\tt x}.
\begin{verbatim}
("Y", 1, [Alloc 1] ++
         compileC (EAp (EVar "f") (EVar "x")) p ++ [Update 0] ++
         compileC (EVar "x") p ++ [Slide 1] ++
         [Update 1, Pop 1, Unwind])
         where p = [("x", 0), ("f", 1)]
\end{verbatim}

The code generation is laid out in the same fashion as the compilation
scheme. When the expressions involving \mbox{\tt compileC}\indexTT{compileC} are simplified we
get:
\begin{verbatim}
("Y", 1, [Alloc 1] ++
         [Push 0, Push 2, Mkap] ++ [Update 0] ++
         [Push 0] ++ [Slide 1] ++
         [Update 1, Pop 1, Unwind])
\end{verbatim}
Which gives the following code sequence:
\begin{verbatim}
("Y", 1, [Alloc 1, Push 0, Push 2, Mkap, Update 0, Push 0,
          Slide 1, Update 1, Pop 1, Unwind])
\end{verbatim}

We can see the way in which this code executes in
Figure~\ref{gm:fg:y3}.  This definition of the \mbox{\tt Y} supercombinator is
called `knot-tying' because we are tying a knot in the graph when we
do the \mbox{\tt Update\ 0} as the fifth instruction. We have not shown the
remainder of the instructions, as this is left as
Exercise~\ref{gm:X:runY3}.
\begin{figure} %\centering
\setlength{\unitlength}{59204sp}%
%
\begingroup\makeatletter\ifx\SetFigFont\undefined%
\gdef\SetFigFont#1#2#3#4#5{%
  \reset@font\fontsize{#1}{#2pt}%
  \fontfamily{#3}\fontseries{#4}\fontshape{#5}%
  \selectfont}%
\fi\endgroup%
\begin{picture}(464,125)(18,702)
\thinlines
\put(435,800){\line( 2,-3){ 10}}
\put(425,800){\line(-1,-2){ 10}}
\put(275,800){\line( 2,-3){ 10}}
\put(265,800){\line(-1,-2){ 10}}
\put(105,800){\line(-1,-2){ 10}}
\put(115,800){\line( 2,-3){ 10}}
\put(385,750){\vector( 2,-1){ 20}}
\put(385,750){\line(-1, 0){ 25}}
\put(390,770){\vector( 1,-2){ 15}}
\put(390,770){\line(-1, 0){ 30}}
\put(420,790){\vector( 3,-1){ 30}}
\put(420,790){\line(-1, 0){ 60}}
\put(360,810){\vector( 1, 0){ 60}}
\put(230,770){\vector( 1,-2){ 15}}
\put(230,770){\line(-1, 0){ 30}}
\put(450,780){\line(-3,-4){ 30}}
\put(420,740){\line( 1, 0){ 60}}
\put(480,740){\line(-3, 4){ 30}}
\put(340,760){\line( 1, 0){ 40}}
\put(340,780){\line( 1, 0){ 40}}
\put(340,800){\line( 1, 0){ 40}}
\put(340,820){\line( 1, 0){ 40}}
\put(340,825){\line( 0,-1){ 85}}
\put(340,740){\line( 1, 0){ 40}}
\put(380,740){\line( 0, 1){ 85}}
\put(200,810){\vector( 1, 0){ 60}}
\put(260,790){\vector( 3,-1){ 30}}
\put(260,790){\line(-1, 0){ 60}}
\put(290,780){\line(-3,-4){ 30}}
\put(260,740){\line( 1, 0){ 60}}
\put(320,740){\line(-3, 4){ 30}}
\put(180,780){\line( 1, 0){ 40}}
\put(180,800){\line( 1, 0){ 40}}
\put(180,820){\line( 1, 0){ 40}}
\put(180,825){\line( 0,-1){ 65}}
\put(180,760){\line( 1, 0){ 40}}
\put(220,760){\line( 0, 1){ 65}}
\put(100,790){\vector( 3,-1){ 30}}
\put(100,790){\line(-1, 0){ 60}}
\put( 40,810){\vector( 1, 0){ 60}}
\put(130,780){\line(-3,-4){ 30}}
\put(100,740){\line( 1, 0){ 60}}
\put(160,740){\line(-3, 4){ 30}}
\put( 20,800){\line( 1, 0){ 40}}
\put( 20,820){\line( 1, 0){ 40}}
\put( 20,825){\line( 0,-1){ 45}}
\put( 20,780){\line( 1, 0){ 40}}
\put( 60,780){\line( 0, 1){ 45}}
\put(350,705){\makebox(0,0)[lb]{\smash{\SetFigFont{11}{13.2}{\familydefault}{\mddefault}{\updefault}After Push 0}}}
\put(185,705){\makebox(0,0)[lb]{\smash{\SetFigFont{11}{13.2}{\familydefault}{\mddefault}{\updefault}After Alloc 1}}}
\put( 25,705){\makebox(0,0)[lb]{\smash{\SetFigFont{11}{13.2}{\familydefault}{\mddefault}{\updefault}On entering Y}}}
\put(245,725){\makebox(0,0)[lb]{\smash{\SetFigFont{11}{13.2}{\familydefault}{\mddefault}{\updefault}?}}}
\put(405,725){\makebox(0,0)[lb]{\smash{\SetFigFont{11}{13.2}{\familydefault}{\mddefault}{\updefault}?}}}
\put(405,765){\makebox(0,0)[lb]{\smash{\SetFigFont{11}{13.2}{\familydefault}{\mddefault}{\updefault}Y}}}
\put(245,765){\makebox(0,0)[lb]{\smash{\SetFigFont{11}{13.2}{\familydefault}{\mddefault}{\updefault}Y}}}
\put(265,805){\makebox(0,0)[lb]{\smash{\SetFigFont{11}{13.2}{\familydefault}{\mddefault}{\updefault}@}}}
\put(425,805){\makebox(0,0)[lb]{\smash{\SetFigFont{11}{13.2}{\familydefault}{\mddefault}{\updefault}@}}}
\put(445,745){\makebox(0,0)[lb]{\smash{\SetFigFont{11}{13.2}{\familydefault}{\mddefault}{\updefault}f}}}
\put(285,745){\makebox(0,0)[lb]{\smash{\SetFigFont{11}{13.2}{\familydefault}{\mddefault}{\updefault}f}}}
\put(125,745){\makebox(0,0)[lb]{\smash{\SetFigFont{11}{13.2}{\familydefault}{\mddefault}{\updefault}f}}}
\put( 85,765){\makebox(0,0)[lb]{\smash{\SetFigFont{11}{13.2}{\familydefault}{\mddefault}{\updefault}Y}}}
\put(105,805){\makebox(0,0)[lb]{\smash{\SetFigFont{11}{13.2}{\familydefault}{\mddefault}{\updefault}@}}}
\end{picture}


\vspace{0.25in}

\setlength{\unitlength}{59204sp}%
%
\begingroup\makeatletter\ifx\SetFigFont\undefined%
\gdef\SetFigFont#1#2#3#4#5{%
  \reset@font\fontsize{#1}{#2pt}%
  \fontfamily{#3}\fontseries{#4}\fontshape{#5}%
  \selectfont}%
\fi\endgroup%
\begin{picture}(474,145)(18,682)
\thinlines
\multiput(230,740)(-0.25000,-0.50000){21}{\makebox(0.1111,0.7778){\SetFigFont{5}{6}{\rmdefault}{\mddefault}{\updefault}.}}
\put(225,730){\line( 0,-1){ 10}}
\put(225,720){\line( 1, 0){100}}
\put(325,720){\line( 0, 1){ 60}}
\multiput(325,780)(-0.41667,0.41667){13}{\makebox(0.1111,0.7778){\SetFigFont{5}{6}{\rmdefault}{\mddefault}{\updefault}.}}
\put(320,785){\line(-1, 0){ 25}}
\multiput(295,785)(-0.41667,-0.41667){13}{\makebox(0.1111,0.7778){\SetFigFont{5}{6}{\rmdefault}{\mddefault}{\updefault}.}}
\put(290,780){\vector(-1,-1){0}}
\multiput(390,730)(-0.41667,-0.41667){13}{\makebox(0.1111,0.7778){\SetFigFont{5}{6}{\rmdefault}{\mddefault}{\updefault}.}}
\put(385,725){\line( 0,-1){ 10}}
\put(385,715){\line( 1, 0){105}}
\put(490,715){\line( 0, 1){ 75}}
\multiput(490,790)(-0.50000,0.25000){21}{\makebox(0.1111,0.7778){\SetFigFont{5}{6}{\rmdefault}{\mddefault}{\updefault}.}}
\put(480,795){\line(-1, 0){ 15}}
\multiput(465,795)(-0.40000,-0.40000){26}{\makebox(0.1111,0.7778){\SetFigFont{5}{6}{\rmdefault}{\mddefault}{\updefault}.}}
\put(455,785){\vector(-1,-1){0}}
\multiput(400,730)(0.41667,-0.41667){13}{\makebox(0.1111,0.7778){\SetFigFont{5}{6}{\rmdefault}{\mddefault}{\updefault}.}}
\put(405,725){\line( 1, 0){  5}}
\put(410,725){\line( 0, 1){ 25}}
\multiput(410,750)(-0.41667,0.41667){13}{\makebox(0.1111,0.7778){\SetFigFont{5}{6}{\rmdefault}{\mddefault}{\updefault}.}}
\put(405,755){\vector(-1, 1){0}}
\multiput(400,755)(-0.25000,-0.50000){21}{\makebox(0.1111,0.7778){\SetFigFont{5}{6}{\rmdefault}{\mddefault}{\updefault}.}}
\put(395,745){\vector(-1,-2){0}}
\put(435,800){\line( 3,-4){ 15}}
\put(275,800){\line( 2,-3){ 10}}
\put(110,800){\line( 1,-1){ 15}}
\put(105,800){\line(-3,-4){ 15}}
\put(360,770){\vector( 1, 0){ 35}}
\put(340,825){\line( 0,-1){ 65}}
\put(340,760){\line( 1, 0){ 40}}
\put(380,760){\line( 0, 1){ 65}}
\put(200,750){\vector( 1, 0){ 25}}
\put(200,770){\vector( 1, 0){ 35}}
\put(180,760){\line( 1, 0){ 40}}
\put(180,825){\line( 0,-1){ 85}}
\put(180,740){\line( 1, 0){ 40}}
\put(220,740){\line( 0, 1){ 85}}
\put( 70,750){\vector( 2, 1){ 60}}
\put( 70,750){\line(-1,-4){  5}}
\put( 65,730){\line(-1, 0){ 25}}
\put( 70,750){\vector( 3,-2){ 15}}
\put( 70,750){\line(-1, 0){ 30}}
\put( 75,770){\vector( 1,-3){ 10}}
\put( 75,770){\line(-1, 0){ 35}}
\put( 20,740){\line( 1, 0){ 40}}
\put( 20,760){\line( 1, 0){ 40}}
\put( 20,780){\line( 1, 0){ 40}}
\put( 20,825){\line( 0,-1){105}}
\put( 20,720){\line( 1, 0){ 40}}
\put( 60,720){\line( 0, 1){105}}
\put(420,790){\vector( 3,-1){ 30}}
\put(420,790){\line(-1, 0){ 60}}
\put(360,810){\vector( 1, 0){ 60}}
\put(450,780){\line(-3,-4){ 30}}
\put(420,740){\line( 1, 0){ 60}}
\put(480,740){\line(-3, 4){ 30}}
\put(340,780){\line( 1, 0){ 40}}
\put(340,800){\line( 1, 0){ 40}}
\put(340,820){\line( 1, 0){ 40}}
\put(200,810){\vector( 1, 0){ 60}}
\put(260,790){\vector( 3,-1){ 30}}
\put(260,790){\line(-1, 0){ 60}}
\put(290,780){\line(-3,-4){ 30}}
\put(260,740){\line( 1, 0){ 60}}
\put(320,740){\line(-3, 4){ 30}}
\put(180,780){\line( 1, 0){ 40}}
\put(180,800){\line( 1, 0){ 40}}
\put(180,820){\line( 1, 0){ 40}}
\put(100,790){\vector( 3,-1){ 30}}
\put(100,790){\line(-1, 0){ 60}}
\put( 40,810){\vector( 1, 0){ 60}}
\put(130,780){\line(-3,-4){ 30}}
\put(100,740){\line( 1, 0){ 60}}
\put(160,740){\line(-3, 4){ 30}}
\put( 20,800){\line( 1, 0){ 40}}
\put( 20,820){\line( 1, 0){ 40}}
\multiput(240,740)(0.41667,-0.41667){13}{\makebox(0.1111,0.7778){\SetFigFont{5}{6}{\rmdefault}{\mddefault}{\updefault}.}}
\put(245,735){\line( 1, 0){  5}}
\put(250,735){\vector( 0, 1){ 25}}
\put(416,784){\line( 2, 3){ 10}}
\put(254,786){\line( 2, 3){ 10}}
\put(390,735){\makebox(0,0)[lb]{\smash{\SetFigFont{11}{13.2}{\familydefault}{\mddefault}{\updefault}@}}}
\put(400,760){\makebox(0,0)[lb]{\smash{\SetFigFont{11}{13.2}{\familydefault}{\mddefault}{\updefault}\#}}}
\put(345,685){\makebox(0,0)[lb]{\smash{\SetFigFont{11}{13.2}{\familydefault}{\mddefault}{\updefault}After Update 0}}}
\put(185,685){\makebox(0,0)[lb]{\smash{\SetFigFont{11}{13.2}{\familydefault}{\mddefault}{\updefault}After Mkap}}}
\put( 25,685){\makebox(0,0)[lb]{\smash{\SetFigFont{11}{13.2}{\familydefault}{\mddefault}{\updefault}After Push 2}}}
\put(230,745){\makebox(0,0)[lb]{\smash{\SetFigFont{11}{13.2}{\familydefault}{\mddefault}{\updefault}@}}}
\put( 85,725){\makebox(0,0)[lb]{\smash{\SetFigFont{11}{13.2}{\familydefault}{\mddefault}{\updefault}?}}}
\put(265,805){\makebox(0,0)[lb]{\smash{\SetFigFont{11}{13.2}{\familydefault}{\mddefault}{\updefault}@}}}
\put(425,805){\makebox(0,0)[lb]{\smash{\SetFigFont{11}{13.2}{\familydefault}{\mddefault}{\updefault}@}}}
\put(445,745){\makebox(0,0)[lb]{\smash{\SetFigFont{11}{13.2}{\familydefault}{\mddefault}{\updefault}f}}}
\put(285,745){\makebox(0,0)[lb]{\smash{\SetFigFont{11}{13.2}{\familydefault}{\mddefault}{\updefault}f}}}
\put(125,745){\makebox(0,0)[lb]{\smash{\SetFigFont{11}{13.2}{\familydefault}{\mddefault}{\updefault}f}}}
\put( 85,765){\makebox(0,0)[lb]{\smash{\SetFigFont{11}{13.2}{\familydefault}{\mddefault}{\updefault}Y}}}
\put(105,805){\makebox(0,0)[lb]{\smash{\SetFigFont{11}{13.2}{\familydefault}{\mddefault}{\updefault}@}}}
\put( 85,765){\makebox(0,0)[lb]{\smash{\SetFigFont{11}{13.2}{\familydefault}{\mddefault}{\updefault}Y}}}
\put(405,775){\makebox(0,0)[lb]{\smash{\SetFigFont{11}{13.2}{\familydefault}{\mddefault}{\updefault}Y}}}
\put(405,775){\makebox(0,0)[lb]{\smash{\SetFigFont{11}{13.2}{\familydefault}{\mddefault}{\updefault}Y}}}
\put(245,760){\makebox(0,0)[lb]{\smash{\SetFigFont{11}{13.2}{\familydefault}{\mddefault}{\updefault}?}}}
\put(245,775){\makebox(0,0)[lb]{\smash{\SetFigFont{11}{13.2}{\familydefault}{\mddefault}{\updefault}Y}}}
\end{picture}

\caption{Execution of code for \mbox{\tt Y}}\label{gm:fg:y3}
\end{figure}

\begin{exercise}\label{gm:X:letrec3}
The compilation of \mbox{\tt letrec}\indexTT{letrec}s is defined in Figure~\ref{gm:fg:schemes3}.
Implement the function \mbox{\tt compileLetrec}\indexTT{compileLetrec} to perform this operation.
\end{exercise}

\begin{exercise}\label{gm:X:test3}
What test programs would you use to show that the new compiler and
instruction set work properly?
\end{exercise}

\begin{exercise}\label{gm:X:runY3}
By running the code generated for the supercombinator \mbox{\tt Y}, or
otherwise, draw the remainder of the state transitions in the style of
Figure~\ref{gm:fg:y3}.
\end{exercise}

\begin{exercise}\label{gm:X:optY}
Give a shorter, alternative, code sequence for the supercombinator
\mbox{\tt Y}. It should still construct a `knot-tying' version.
\end{exercise}

\begin{exercise}\label{gm:X:Y3}
In the absence of a \mbox{\tt letrec}\indexTT{letrec} construct in the language, how would you
define the fixpoint combinator \mbox{\tt Y}? How is this definition different
from the one we used in the example?
\end{exercise}

\section{Mark 4: Adding primitives\index{G-machine!Mark 4}\index{arithmetic!in G-machine}}
\label{gm:sc:primitives}

In this section we add primitive operations to the G-machine; this
makes it useful. By primitive operations we mean
operations like addition, multiplication and so on. We will use
addition as a running example throughout this section.

The addition instruction will be called \mbox{\tt Add}\indexTT{Add}; which adds two numbers
from the heap, placing the result into a new node in the heap. The
addresses of the two arguments are on top of the stack, and this is
where the address of the result is subsequently placed. It has the
following transition rule.

\gmrule%
{\gmstate{\mbox{\tt Add}\indexTT{Add} :i}{a_0:a_1:s}{h[a_0: \mbox{\tt NNum}\indexTT{NNum}\ n_0,\ a_1: \mbox{\tt NNum}\indexTT{NNum}\ n_1]}{m}}%
{\gmstate{i}{a:s}{h[a: \mbox{\tt NNum}\indexTT{NNum}\ n_0+n_1]}{m}}

We could continue to expand the G-machine with other instructions to
implement the remainder of the operations required, but before we do,
let us pause to consider whether we have missed something here. The
problem is that the rule only applies if the two objects on top of the
stack are numbers. Since we are working on a machine that supports
lazy evaluation there is no good reason to suppose that this will
always be the case. In the template machine \mbox{\tt Add}\indexTT{Add} checked that its
arguments were evaluated. In the G-machine we want to keep the
instructions simple, so we will only use \mbox{\tt Add}\indexTT{Add} in situations where we
guarantee that the arguments are already evaluated.

What we do instead is to augment the instruction set further with an
\mbox{\tt Eval}\indexTT{Eval} instruction. This satisfies the following constraint:

\begin{important}
Suppose that we are in a state:
\[\begin{array}{|lrrll|}\hline
& \gmstate{\mbox{\tt Eval}\indexTT{Eval}: i}{a:s}{h}{m}\\ \hline
\end{array}\]
Whenever execution resumes with instruction sequence $i$, the state
will be:
\[\begin{array}{|lrrll|}\hline
& \gmstate{i}{a:s}{h'}{m}\\ \hline
\end{array}\]
and the item on top of the stack will be in WHNF\index{weak head normal form}.

It is also possible for \mbox{\tt Eval}\indexTT{Eval} to fail to terminate; this will be the
case when the node pointed to from the top of the stack has no WHNF.
\end{important}

If the node whose address is on top of the stack is already in WHNF,
then the \mbox{\tt Eval}\indexTT{Eval} instruction does nothing. If there is reduction to be
performed, then the action of \mbox{\tt Eval}\indexTT{Eval} is to perform an evaluation to
WHNF. If this call terminates then execution resumes with nothing
except the heap component changed.
This is similar to the structure of {\em subroutine call and
return\/}\index{subroutine call and return} traditionally used in
programming language implementation. We recall that the classic way to
implement this feature is to use a stack. The stack will save
sufficient of the machine's current context that it can resume when
the subroutine call completes.

In the Mark~4 G-machine this stack is called the {\em
dump}\index{dump!in G-machine}, and is a stack of pairs, whose first
component is a code sequence, and whose second component is a stack.
This is similar to the dump in the template machine (see
Section~\ref{sect:templ:primitives}), except we now have to restore
the original code sequence as well as the original stack. Hence both
components are kept on the dump.

\subsection{Data structures\index{data structures!G-machine Mark 4}}

We extend the G-machine state by adding a dump component. As previously
discussed, this is used to implement recursive calls to the evaluator.
\begin{verbatim}
> type GmState = ( GmCode,              -- current Instruction
>                 GmStack,      -- current Stack
>                 GmDump,       -- current Dump
>                 GmHeap,       -- Heap of Nodes
>                 GmGlobals,    -- Global adresses in Heap
>                 GmStats)      -- Statistics
\end{verbatim}
\indexDTT{type}%
The dump itself is a stack of \mbox{\tt dumpItem}\indexTT{dumpItem}. Each of these is a pair
consisting of the instruction stream and stack to use when we resume
the original computation.
\begin{verbatim}
> type GmDump = [GmDumpItem]
> type GmDumpItem = (GmCode, GmStack)
\end{verbatim}
\indexDTT{type}%
\indexDTT{type}%
When we add this new component we must change all of the previously
specified access functions. We must also add access functions for the
dump.
\begin{verbatim}
> getDump :: GmState -> GmDump
> getDump (i, stack, dump, heap, globals, stats) = dump
\end{verbatim}
\indexDTT{getDump}%
\indexDTT{getDump}%
\begin{verbatim}
> putDump :: GmDump -> GmState -> GmState
> putDump dump' (i, stack, dump, heap, globals, stats)
>    = (i, stack, dump', heap, globals, stats)
\end{verbatim}
\indexDTT{putDump}%
\indexDTT{putDump}%
\par
Notice that it is only in the access functions that we have done
pattern matching on G-machine states. Changes to other functions as a
result of adding new components to the state are no longer needed.

\begin{exercise}\label{gm:X:access4}
Make the relevant changes to the other access functions.
\end{exercise}

In addition to the new definition of state, we also need some new
instructions.  We reuse all of the instructions from the Mark~3
machine.
\begin{verbatim}
> data Instruction 
>          = Slide Int
>          | Alloc Int
>                  | Update Int
>          | Pop Int
>          | Unwind
>          | Pushglobal Name
>          | Pushint Int
>          | Push Int
>          | Mkap
\end{verbatim}
\indexDTT{data}%
\indexDTT{Alloc}%
\indexDTT{Update}%
\indexDTT{Pop}%
\indexDTT{Unwind}%
\indexDTT{Pushglobal}%
\indexDTT{Pushint}%
\indexDTT{Push}%
\indexDTT{Mkap}%
In addition we include the \mbox{\tt Eval}\indexTT{Eval} instruction,
\begin{verbatim}
>            | Eval
\end{verbatim}
\indexDTT{Eval}%
the following arithmetic instructions:
\begin{verbatim}
>            | Add | Sub | Mul | Div | Neg
\end{verbatim}
\indexDTT{Add}%
\indexDTT{Sub}%
\indexDTT{Mul}%
\indexDTT{Div}%
\indexDTT{Neg}%
and the following comparison instructions:
\begin{verbatim}
>            | Eq | Ne | Lt | Le | Gt | Ge
\end{verbatim}
\indexDTT{Eq}%
\indexDTT{Ne}%
\indexDTT{Lt}%
\indexDTT{Le}%
\indexDTT{Gt}%
\indexDTT{Ge}%
We also include a primitive form of conditional in the \mbox{\tt Cond}\indexTT{Cond}
instruction.
\begin{verbatim}
>          | Cond GmCode GmCode
\end{verbatim}
\indexDTT{Cond}%
\begin{exercise}\label{gm:X:showinstructions4}
Add cases to \mbox{\tt showInstruction}\indexTT{showInstruction} to print all of the new instructions.
\end{exercise}

\subsection{Printing the state}

We take this opportunity to revise the definition of \mbox{\tt showState}\indexTT{showState}, so
that it displays the dump component.
\begin{verbatim}
> showState :: GmState -> Iseq
> showState s
>    = iConcat       [showStack s,                  iNewline,
>                     showDump s,                   iNewline,
>                     showInstructions (getCode s), iNewline]
\end{verbatim}
\indexDTT{showState}%
\indexDTT{showState}%
We therefore need to define \mbox{\tt showDump}\indexTT{showDump}.
\begin{verbatim}
> showDump :: GmState -> Iseq
> showDump s
>     = iConcat       [iStr "  Dump:[",
>                      iIndent (iInterleave iNewline
>                             (map showDumpItem (reverse (getDump s)))),
>                      iStr "]"]
\end{verbatim}
\indexDTT{showDump}%
\indexDTT{showDump}%
This in turn needs the function \mbox{\tt showDumpItem}\indexTT{showDumpItem}.
\begin{verbatim}
> showDumpItem :: GmDumpItem -> Iseq
> showDumpItem (code, stack)
>     = iConcat       [iStr "<",
>                      shortShowInstructions 3 code, iStr ", ",
>                      shortShowStack stack,         iStr ">"]
\end{verbatim}
\indexDTT{showDumpItem}%
\indexDTT{showDumpItem}%
\par
We use the function \mbox{\tt shortShowInstructions}\indexTT{shortShowInstructions} to print only the
first three instructions of the instruction stream in the dump items.
This is usually sufficient to indicate where the computation will
resume.
\begin{verbatim}
> shortShowInstructions :: Int -> GmCode -> Iseq
> shortShowInstructions number code
>     = iConcat [iStr "{", iInterleave (iStr "; ") dotcodes, iStr "}"]
>     where   codes   = map showInstruction (take number code)
>             dotcodes      | length code > number  = codes ++ [iStr "..."]
>                           | otherwise             = codes
\end{verbatim}
\indexDTT{shortShowInstructions}%
\indexDTT{shortShowInstructions}%
Similarly, we do not need the full details of the stack component of
the dump item either, so we use \mbox{\tt shortShowStack}\indexTT{shortShowStack}.
\begin{verbatim}
> shortShowStack :: GmStack -> Iseq
> shortShowStack stack
>     = iConcat [iStr "[",
>            iInterleave (iStr ", ") (map (iStr . showaddr) stack),
>            iStr "]"]
\end{verbatim}
\indexDTT{shortShowStack}%
\indexDTT{shortShowStack}%
\subsection{The new instruction transitions}
\label{gm:sc:inst4}

\subsubsection{Evaluator instructions}

There are actually very few instructions that manipulate the dump.
First, there is \mbox{\tt Eval}\indexTT{Eval} itself, which creates a new dump item whenever
the node on top of the stack is not in WHNF. Secondly, there is a
modification to the \mbox{\tt Unwind}\indexTT{Unwind} instruction that pops a dump item when an
evaluation is completed.

We first describe the new \mbox{\tt Unwind}\indexTT{Unwind} instruction. When the expression
held in the stack is in WHNF, \mbox{\tt Unwind}\indexTT{Unwind} can restore the old context
from the dump, placing the last address in the stack on the restored
old stack. We see this clearly in the transition for the case of
numbers.\footnote{The rule only applies if the dump is non-empty; if
the dump is empty then the machine has terminated.}

\gmruled%
{\gmstated{[\mbox{\tt Unwind}\indexTT{Unwind}]}{a:s}{\langle i',\,s'\rangle:d}{h[a: \mbox{\tt NNum}\indexTT{NNum}\ n]}{m}}%
{\gmstated{i'}{a:s'}{d}{h}{m}}

The expression with address $a$ is in WHNF because it is an integer,
so we restore the old instruction sequence $i'$ and the stack is now
the old stack $s'$ with the address $a$ on top. All other transitions
for \mbox{\tt Unwind}\indexTT{Unwind} remain the same as they were in the Mark~3 machine
(except that they have the dump component in their state).

We are now in a position to specify the rule for \mbox{\tt Eval}\indexTT{Eval}. It saves the
remainder of the stack $s$ and the rest of the instructions $i$ as a
dump item on the dump. The new code sequence is just unwinding and the
new stack contains the singleton $a$.

\gmruled%
{\gmstated{\mbox{\tt Eval}\indexTT{Eval} :i}{a:s}{d}{h}{m}}%
{\gmstated{[\mbox{\tt Unwind}\indexTT{Unwind}]}{[a]}{\langle i,\,s\rangle:d}{h}{m}}

\subsubsection{Arithmetic instructions\index{arithmetic!in G-machine}}

The {\em dyadic arithmetic operators\/}\index{dyadic arithmetic
operators} all have the following generic transition rule.  Let us
suppose that the arithmetic operator we wish to implement is $\odot$;
the transition rule for the instruction $\odot$ is then:

\gmruled%
{\gmstated{\odot :i}{a_0:a_1:s}{d}%
{h[a_0: \mbox{\tt NNum}\indexTT{NNum}\ n_0,\,a_1: \mbox{\tt NNum}\indexTT{NNum}\ n_1]}{m}}%
{\gmstated{i}{a:s}{d}{h[a: \mbox{\tt NNum}\indexTT{NNum}\ (n_0\,\odot\, n_1)]}{m}}

What has happened is that the two numbers on top of the stack have had
the dyadic operator $\odot$ applied to them. The result, which is
entered into the heap, has its address placed on the stack. The \mbox{\tt Neg}\indexTT{Neg}
instruction negates the number on top of the stack, so it has
transition rule:

\gmruled%
{\gmstated{\mbox{\tt Neg}\indexTT{Neg} :i}{a:s}{d}{h[a: \mbox{\tt NNum}\indexTT{NNum}\ n]}{m}}%
{\gmstated{i}{a':s}{d}{h[a': \mbox{\tt NNum}\indexTT{NNum}\ (-n)]}{m}}

Notice how similar all of the dyadic operations are. First we extract
the two numbers from the heap, then we perform the operation, and
finally we place the answer back in the heap. This suggests that we
should write some higher-order functions\index{higher-order function}
that are parameterised over the extraction from heap (which we call
`unboxing'\index{unboxing!of values from the heap} the value), and
insertion back into the heap (which we call `boxing'\index{boxing!of
values into the heap} the value), along with the specific operation we
wish to perform.

Let us write the boxing operations first. \mbox{\tt boxInteger}\indexTT{boxInteger} takes a number
and an initial state, and returns a new state in which the number has
been placed into the heap, and a pointer to this new node left on top
of the stack.
\begin{verbatim}
> boxInteger :: Int -> GmState -> GmState
> boxInteger n state
>       = putStack (a: getStack state) (putHeap h' state)
>       where (h', a) = hAlloc (getHeap state) (NNum n)
\end{verbatim}
\indexDTT{boxInteger}%
\indexDTT{boxInteger}%
\par
Now to extract an integer at address \mbox{\tt a} from a state, we will use
\mbox{\tt unboxInteger}\indexTT{unboxInteger}.
\begin{verbatim}
> unboxInteger :: Addr -> GmState -> Int
> unboxInteger a state
>       = ub (hLookup (getHeap state) a)
>       where   ub (NNum i) = i
>               ub n        = error "Unboxing a non-integer"
\end{verbatim}
\indexDTT{unboxInteger}%
\indexDTT{unboxInteger}%
\par
A generic {\em monadic operator\/}\index{monadic arithmetic operator} can now be
specified in terms of its boxing function, \mbox{\tt box}\indexTT{box}, its unboxing
function \mbox{\tt unbox}\indexTT{unbox}, and the operator \mbox{\tt op} on the unboxed values.
\begin{verbatim}
> primitive1 :: (b -> GmState -> GmState)  -- boxing function
>          -> (Addr -> GmState -> a)       -- unbixing function
>          -> (a -> b)                     -- operator
>          -> (GmState -> GmState)         -- state transition
\end{verbatim}
\indexDTT{primitive1}%
\begin{verbatim}
> primitive1 box unbox op state
>    = box (op (unbox a state)) (putStack as state)
>    where (a:as) = getStack state
\end{verbatim}
\indexDTT{primitive1}%
\par
The generic {\em dyadic operators\/}\index{dyadic arithmetic operator}
can now be implemented in a similar way using \mbox{\tt primitive2}\indexTT{primitive2}.
\begin{verbatim}
> primitive2 :: (b -> GmState -> GmState)  -- boxing function
>          -> (Addr -> GmState -> a)       -- unbixing function
>          -> (a -> a -> b)                -- operator
>          -> (GmState -> GmState)         -- state transition
\end{verbatim}
\indexDTT{primitive2}%
\begin{verbatim}
> primitive2 box unbox op state
>    = box (op (unbox a0 state) (unbox a1 state)) (putStack as state)
>    where (a0:a1:as) = getStack state
\end{verbatim}
\indexDTT{primitive2}%
\par
To be even more explicit, \mbox{\tt arithmetic1}\indexTT{arithmetic1} implements all {\em monadic
arithmetic}\index{monadic arithmetic operator}, and \mbox{\tt arithmetic2}\indexTT{arithmetic2}
implements all {\em dyadic arithmetic}\index{dyadic arithmetic
operator}.
\begin{verbatim}
> arithmetic1 ::   (Int -> Int)            -- arithmetic operator
>                  -> (GmState -> GmState) -- state transition
\end{verbatim}
\indexDTT{arithmetic1}%
\begin{verbatim}
> arithmetic1 = primitive1 boxInteger unboxInteger
\end{verbatim}
\indexDTT{arithmetic1}%
\begin{verbatim}
> arithmetic2 ::   (Int -> Int -> Int)     -- arithmetic operation
>                  -> (GmState -> GmState) -- state transition
\end{verbatim}
\indexDTT{arithmetic2}%
\begin{verbatim}
> arithmetic2 = primitive2 boxInteger unboxInteger
\end{verbatim}
\indexDTT{arithmetic2}%
As the alert reader would expect, we will be taking advantage of the
generality of these functions later in the chapter.

\begin{exercise}\label{gm:X:step4}
Implement all of the new instruction transitions for the machine.
Modify the \mbox{\tt dispatch}\indexTT{dispatch} function to deal with the new instructions. You
should use the higher-order functions \mbox{\tt primitive1}\indexTT{primitive1} and \mbox{\tt primitive2}\indexTT{primitive2}
to implement the operators.
\end{exercise}

\begin{exercise}\label{gm:X:ind4}
Why are indirection nodes never left on top of the stack on completing
an \mbox{\tt Eval}\indexTT{Eval} instruction?
\end{exercise}

\subsubsection{Comparison instructions\index{comparison operators!in G-machine}}

The {\em comparison operators\/}\index{comparison operators} all have
the following generic transition rule.  Let us suppose that the
comparison operator we wish to implement is $\odot$; the transition
rule for the instruction $\odot$ is then:

\gmruled%
{\gmstated{\odot :i}{a_0:a_1:s}{d}%
{h[a_0: \mbox{\tt NNum}\indexTT{NNum}\ n_0,\,a_1: \mbox{\tt NNum}\indexTT{NNum}\ n_1]}{m}}%
{\gmstated{i}{a:s}{d}{h[a: \mbox{\tt NNum}\indexTT{NNum}\ (n_0\,\odot\, n_1)]}{m}}

What has happened is that the two numbers on top of the stack have had
the dyadic operator $\odot$ applied to them. The result, which is
entered into the heap, has its address placed on the stack.
This is almost the same as arithmetic.

The difference is that an operation, \mbox{\tt ==} say, returns a boolean and
not an integer. To fix this we turn booleans into integers using the
following rule:

\begin{itemize}
\item we represent \mbox{\tt True}\indexTT{True} by the integer \mbox{\tt 1};
\item we represent \mbox{\tt False}\indexTT{False} by the integer \mbox{\tt 0}.
\end{itemize}

To make the use of \mbox{\tt primitive2}\indexTT{primitive2} possible, we define \mbox{\tt boxBoolean}\indexTT{boxBoolean}
\begin{verbatim}
> boxBoolean :: Bool -> GmState -> GmState
> boxBoolean b state
>    = putStack (a: getStack state) (putHeap h' state)
>    where   (h',a) = hAlloc (getHeap state) (NNum b')
>            b' | b                = 1
>               | otherwise        = 0
\end{verbatim}
\indexDTT{boxBoolean}%
\indexDTT{boxBoolean}%
Using this definition we can write a generic comparison function,
which we call \mbox{\tt comparison}\indexTT{comparison}. This function takes a {\em
boxing}\index{boxing function} function for the booleans, the unboxing
function for integers (\mbox{\tt unboxInteger}\indexTT{unboxInteger}), and a comparison operator; it
returns a state transition.
\begin{verbatim}
> comparison :: (Int -> Int -> Bool) -> GmState -> GmState
> comparison = primitive2 boxBoolean unboxInteger
\end{verbatim}
\indexDTT{comparison}%
\indexDTT{comparison}%
\par
Finally, we implement the \mbox{\tt Cond}\indexTT{Cond} instruction, which we will use to
compile the \mbox{\tt if} function. It has two transition rules:

\gmruled%
{\gmstated{\mbox{\tt Cond}\indexTT{Cond}\ i_1\ i_2 :i}{a:s}{d}{h[a: \mbox{\tt NNum}\indexTT{NNum}\ 1]}{m}}%
{\gmstated{i_1 \append i}{s}{d}{h}{m}}

In the first case -- where there is the number \mbox{\tt 1} on top of the stack
-- we take the first branch. This means that we execute the
instructions $i_1$ before continuing to execute the instructions $i$.

\gmruled%
{\gmstated{\mbox{\tt Cond}\indexTT{Cond}\ i_1\ i_2 :i}{a:s}{d}{h[a: \mbox{\tt NNum}\indexTT{NNum}\ 0]}{m}}%
{\gmstated{i_2 \append i}{s}{d}{h}{m}}

Alternatively, if the number on top of the stack is \mbox{\tt 0}, we execute
the instruction sequence $i_2$ first, and then the sequence $i$.

\begin{exercise}
Implement the transitions for the comparison instructions and the
\mbox{\tt Cond}\indexTT{Cond} instruction.
\end{exercise}

\subsection{The compiler\index{G-machine compiler!Mark 4}}

The compiler will eventually need to be changed to take advantage of
these new instructions to compile arithmetic expressions. For the
moment we make only the minimum set of changes that will allow us to
use the arithmetic instructions we have so laboriously added. First, the
\mbox{\tt compile}\indexTT{compile} function must create a new initial state in which the
initial dump is empty and in which the initial code sequence differs
from the one we have used so far.
\begin{verbatim}
> compile :: CoreProgram -> GmState
> compile program
>    = (initialCode, [], [], heap, globals, statInitial)
>    where (heap, globals) = buildInitialHeap program
\end{verbatim}
\indexDTT{compile}%
\indexDTT{compile}%
\begin{verbatim}
> initialCode :: GmCode
> initialCode = [Pushglobal "main", Eval]
\end{verbatim}
\indexDTT{initialCode}%
\indexDTT{initialCode}%
\begin{exercise}\label{gm:X:evalForUnwind}
Why has the initial instruction sequence been changed? What happens if
we retain the old one?
\end{exercise}

The simplest way to extend the compiler is simply to add G-machine
code for each of the new built-in functions to the
\mbox{\tt compiledPrimitives}\indexTT{compiledPrimitives}.  The initial four instructions of the sequence
ensure that the arguments have been evaluated to integers.
\begin{verbatim}
> compiledPrimitives :: [GmCompiledSC]
> compiledPrimitives
>    =       [("+", 2, [Push 1, Eval, Push 1, Eval, Add, Update 2, Pop 2, Unwind]),
>             ("-", 2, [Push 1, Eval, Push 1, Eval, Sub, Update 2, Pop 2, Unwind]),
>             ("*", 2, [Push 1, Eval, Push 1, Eval, Mul, Update 2, Pop 2, Unwind]),
>             ("/", 2, [Push 1, Eval, Push 1, Eval, Div, Update 2, Pop 2, Unwind]),
\end{verbatim}
\indexDTT{compiledPrimitives}%
We also need to add the negation function. As this only takes one
argument, we only evaluate one argument.
\begin{verbatim}
>             ("negate", 1, [Push 0, Eval, Neg, Update 1, Pop 1, Unwind]),
\end{verbatim}
\par
The comparison operations are implemented as follows.
\begin{verbatim}
>             ("==", 2, [Push 1, Eval, Push 1, Eval, Eq, Update 2, Pop 2, Unwind]),
>             ("~=", 2, [Push 1, Eval, Push 1, Eval, Ne, Update 2, Pop 2, Unwind]),
>             ("<",  2, [Push 1, Eval, Push 1, Eval, Lt, Update 2, Pop 2, Unwind]),
>             ("<=", 2, [Push 1, Eval, Push 1, Eval, Le, Update 2, Pop 2, Unwind]),
>             (">",  2, [Push 1, Eval, Push 1, Eval, Gt, Update 2, Pop 2, Unwind]),
>             (">=", 2, [Push 1, Eval, Push 1, Eval, Ge, Update 2, Pop 2, Unwind]),
\end{verbatim}
The \mbox{\tt if} function is compiled so that it uses \mbox{\tt Cond}\indexTT{Cond} for the
branching.
\begin{verbatim}
>             ("if", 3, [Push 0, Eval, Cond [Push 1] [Push 2],
>                             Update 3, Pop 3, Unwind])]
\end{verbatim}
\begin{exercise}\label{gm:X:tests4}
What test programs from Appendix~\ref{sect:core-progs} would you use
in order to check that the new instructions and compiler work?
\end{exercise}

\section{Mark 5: Towards better handling of
arithmetic\index{arithmetic!optimised, in G-machine}} \label{gm:ss:Escheme}

The way the G-machine is implemented at the moment, each arithmetic
operator is called via one of the compiled primitives. We can
improve on this arrangement by observing that often we can call the
arithmetic operator directly. For example, consider the following
simple program:
\begin{verbatim}
        main = 3+4*5
\end{verbatim}
This generates the following code when we use the current compiler:
\begin{verbatim}
        [Pushint 5, Pushint 4, Pushglobal "*", Mkap, Mkap,
         Pushint 3, Pushglobal "+", Mkap, Mkap, Eval]
\end{verbatim}
When executed this code will take 33 steps and use 11 heap nodes. Our
first thought must surely be that we can use the instructions \mbox{\tt Add}\indexTT{Add}
and \mbox{\tt Mul}\indexTT{Mul} in place of calls to the functions `\mbox{\tt +}' and `\mbox{\tt *}'. This
leads to the following improved code:
\begin{verbatim}
        [Pushint 5, Pushint 4, Mul, Pushint 3, Add]
\end{verbatim}
This will take only five steps to execute and uses five heap nodes.

\subsection{A problem}

A possible problem arises when we consider our next example program.
\begin{verbatim}
        main = K 1 (1/0)
\end{verbatim}
This generates the following code:
\begin{verbatim}
        [Pushint 0, Pushint 1, Pushglobal "/", Mkap, Mkap,
         Pushint 1, Pushglobal "K", Mkap, Mkap, Eval]
\end{verbatim}

If we follow the pattern of the previous example we might try
generating the code:
\begin{verbatim}
        [Pushint 0, Pushint 1, Div,
         Pushint 1, Pushglobal "K", Mkap, Mkap, Eval]
\end{verbatim}
The problem is that the division operator is applied {\em before\/} we
reduce \mbox{\tt K}, with the result that a {\em division-by-zero
error\/}\index{division-by-zero error} is generated.  A correct compiler
must generate code that will not give such errors.

What has happened is that our code is too strict.  The code is
evaluating expressions that it need not -- which results in errors
arising where they should not. A similar problem will also arise when
{\em non-terminating expressions\/}\index{non-termination of evaluation}
are inadvertently evaluated.

\subsection{The solution}

The solution to the problem is to keep track of the context in which
an expression appears. We will distinguish two
contexts\index{compilation context}\footnote{It is possible to
distinguish more contexts. Projection analysis \cite{Wadler:87} and
evaluation transformers \cite{BURN91:book} are two ways to do this.}:
\begin{description}
\item[Strict\index{compilation context!strict}] The value of the expression will
be required in WHNF.
\item[Lazy\index{compilation context!lazy}] The value of the expression may or
may not be required in WHNF.
\end{description}

Corresponding to each context, we have a compilation scheme which will
compile an expression to a sequence of G-machine instructions. In the
strict context this compilation scheme is the \tE{} scheme; in the lazy
context we will use the \tC{} scheme we have seen already.

We would like to find as many strict contexts as possible, since these
contexts allow us to generate better code. We make the following
observation: whenever a supercombinator is instantiated it is because
we wish to evaluate its value to WHNF. From this we conclude that the
body of a supercombinator can always be evaluated in a strict context.
There are also expressions where we know that some sub-expressions
will be evaluated to WHNF if the expression is evaluated to WHNF.

The class of strict context expressions can be described recursively.
\begin{itemize}
\item The expression in the body of a supercombinator definition is
in a strict context.
\item If $e_0~\odot~e_1$ occurs in a strict context, where $\odot$ is
an arithmetic or comparison operator, then the expressions $e_0$ and
$e_1$ are also in a strict context.
\item If $\mbox{\tt negate}\indexTT{negate}~e$ occurs in a strict context, then the expression
$e$ also does.
\item If the expression $\mbox{\tt if}~e_0~e_1~e_2$ occurs in a strict context, then so do the
expressions $e_0$, $e_1$, and $e_2$.
\item If $\mbox{\tt let(rec)}\indexTT{let(rec)}~\Delta~\mbox{\tt in}~e$ occurs in a strict context then
the expression $e$ is also in a strict context.
\end{itemize}

An example should make this clear; consider the body of the
supercombinator \mbox{\tt f}:
\begin{verbatim}
        f x y = (x+y) + g (x*y)
\end{verbatim}
Both \mbox{\tt (x+y)} and \mbox{\tt g\ (x*y)} will be evaluated in a strict context --
because the body of the supercombinator is. In the first case this
causes \mbox{\tt x} and \mbox{\tt y} to be evaluated in a strict context -- because \mbox{\tt +}
propagates the strict context. In the second expression, the presence
of a user-defined supercombinator means that the sub-expression \mbox{\tt x*y}
will be compiled assuming that the expression may not be evaluated.

This suggests that we can implement the strict-context compiler, \tE{},
in a recursive manner.  Because the body of each supercombinator is
evaluated in a strict context we will need to call the \tE{} scheme
function from the \tR{} scheme. This satisfies the
first of the points above. To propagate the context information into
sub-expressions we will recursively invoke the \tE{} scheme
for arithmetic expressions.

The new compiler schemes are defined in Figure~\ref{gm:fg:schemes5.re}.
\begin{figure*} %\centering
$\begin{array}{|rcll|} \hline
&&&\\
\multicolumn{4}{|l|}{\parbox[t]{29pc}{%
$\R{e}~\rho~d$ generates code
which instantiates the expression $e$ in environment $\rho$, for a
supercombinator of arity $d$, and then proceeds to unwind the
resulting stack.}}\\ &&&\\
\indexD{$\protect\cal \pR$}\R{e}\ \rho~d & = & \E{e}\ \rho
                \append [\mbox{\tt Update}\indexTT{Update}\ d, \mbox{\tt Pop}\indexTT{Pop}\ d, \mbox{\tt Unwind}\indexTT{Unwind}] &\\
&&&\\
\hline
&&&\\
\multicolumn{4}{|l|}{\parbox[t]{29pc}{%
$\E{e}~\rho$ compiles code that evaluates an expression $e$ to WHNF in
environment $\rho$, leaving a pointer to the expression on top of the stack.}}\\
&&&\\
\indexD{$\protect\cal \pE$}\E{i}\ \rho     & = & [\mbox{\tt Pushint}\indexTT{Pushint}\ i] &\\
\multicolumn{4}{|l|}{%
\indexD{$\protect\cal \pE$}\E{\mbox{\tt let}\indexTT{let}~ x_1\mbox{\tt =}e_1\mbox{\tt ;}~\ldots\mbox{\tt ;}~ x_n \mbox{\tt =} e_n ~\mbox{\tt in}~ e}~ \rho} \\
 & = & \C{e_1}~ \rho^{+0} \append \ldots \append &\\
    &&  \C{e_n}~\rho^{+(n-1)} \append & \\
    &&  \E{e}~\rho' \append [\mbox{\tt Slide}\indexTT{Slide}~ n]&\\
    &&\multicolumn{2}{r|}{%
\tr{where $\rho' = \rho^{+n}[x_1\mapsto n-1,\ldots,x_n\mapsto 0]$}}\\
%
\multicolumn{4}{|l|}{%
\indexD{$\protect\cal \pE$}\E{\mbox{\tt letrec}\indexTT{letrec}~ x_1\mbox{\tt =}e_1\mbox{\tt ;}~\ldots\mbox{\tt ;}~ x_n \mbox{\tt =} e_n ~\mbox{\tt in}~ e}~ \rho} \\
 & = & [\mbox{\tt Alloc}\indexTT{Alloc}~ n] \append & \\
    &&\C{e_1}~ \rho' \append [\mbox{\tt Update}\indexTT{Update}\ n-1]\append \ldots \append &\\
                 &&       \C{e_n}~\rho' \append [\mbox{\tt Update}\indexTT{Update}\ 0] \append &\\
               &&         \E{e}~\rho' \append [\mbox{\tt Slide}\indexTT{Slide}~ n] &\\
    &&\multicolumn{2}{r|}{%
        \tr{where $\rho' = \rho^{+n}[x_1\mapsto n-1,\ldots,x_n\mapsto 0]$}}\\
\indexD{$\protect\cal \pE$}\E{e_0~\mbox{\tt +}~e_1}~ \rho& = & \E{e_1}~\rho\append\E{e_0}~ \rho^{+1}
                                                     \append [\mbox{\tt Add}\indexTT{Add}] &\\
\multicolumn{4}{|r|}{\tr{And similarly for other arithmetic and comparison expressions}}\\
\indexD{$\protect\cal \pE$}\E{\mbox{\tt negate}\indexTT{negate} ~ e}~ \rho& = & \E{e}~ \rho \append [\mbox{\tt Neg}\indexTT{Neg}] &\\
\indexD{$\protect\cal \pE$}\E{\mbox{\tt if}~e_0~e_1~e_2}~ \rho& = & \E{e_0}~\rho \append
                [\mbox{\tt Cond}\indexTT{Cond}\ (\E{e_1}~\rho)\ (\E{e_2}~\rho)] &\\
\indexD{$\protect\cal \pE$}\E{e}\ \rho     & = & \C{e}\ \rho \append [\mbox{\tt Eval}\indexTT{Eval}] &
\tr{the default case}\\
&&&\\
\hline
\end{array}$
\caption{The \tR{} and \tE{} compilation schemes for the Mark~5 machine}
\label{gm:fg:schemes5.re}
\end{figure*}

To make it easier to extend the set of built-in operators that can be compiled
using the new scheme, we define \mbox{\tt builtInDyadic}\indexTT{builtInDyadic}.
\begin{verbatim}
> builtInDyadic :: ASSOC Name Instruction
> builtInDyadic
>     =       [("+", Add), ("-", Sub), ("*", Mul), ("div", Div),
>              ("==", Eq), ("~=", Ne), (">=", Ge),
>              (">",  Gt), ("<=", Le), ("<",  Lt)]
\end{verbatim}
\indexDTT{builtInDyadic}%
\begin{exercise}\label{gm:X:compiler5}
Modify the existing compiler functions \mbox{\tt compileR}\indexTT{compileR} and \mbox{\tt compileE}\indexTT{compileE} so
that they implement the \tR{} scheme and \tE{} scheme of
Figure~\ref{gm:fg:schemes5.re}. You should use \mbox{\tt builtInDyadic}\indexTT{builtInDyadic}.
\end{exercise}

Occasionally the machine will fail when the new compiler is used.
What we need is to introduce a new rule for the \mbox{\tt Unwind}\indexTT{Unwind} instruction.

\gmruled%
{\gmstated{[\mbox{\tt Unwind}\indexTT{Unwind}]}{[a_0,\,\ldots,\, a_k]}{\langle i,\, s\rangle:d}%
{h[a_0: \mbox{\tt NGlobal}\indexTT{NGlobal}\ n\ c]}{m}}%
{\gmstated{i}{a_k : s}{d}{h}{m\mbox{ when $k<n$}}}

This allows us to use \mbox{\tt Eval}\indexTT{Eval} to evaluate any object to WHNF, and not just
numbers.

\begin{exercise}\label{gm:X:unwind5}
Implement the new transition for \mbox{\tt Unwind}\indexTT{Unwind}.
Write a program that fails without the new \mbox{\tt Unwind}\indexTT{Unwind} transition.
\end{exercise}

\begin{exercise}\label{gm:X:compiler45}
Compare the execution of the example program used at the start of this
section, with its execution on the Mark~4 machine.
\begin{verbatim}
main = 3+4*5
\end{verbatim}
Try some other programs from Appendix~\ref{sect:core-progs}.
\end{exercise}

The way we implemented the strict context in the compiler is addis
simple example of the {\em inherited
attributes\/}\index{inherited attributes} from compiler theory. If we
regard the strict context as an attribute of an expression, then a
sub-expression {\em inherits\/} its strict context from its parent
expression. The general theory is discussed in \cite{Dragon}.

It is unfortunately not possible -- in general -- to determine at
compile-time whether an expression should be compiled with a strict
context. We therefore have to accept a compromise.  In this book we
have only treated a limited set of expressions -- the arithmetic
expressions -- in a special way. Much active research is concerned
with extending this analysis to cover more general expressions
\cite{BURN91:book}.


\section{Mark 6: Adding data structures\index{data structures!in G-machine Mark 6}}
\label{gm:ss:mark6}

In this section we extend the G-machine to deal with arbitrary data
structures. As discussed in Chapter~\ref{sect:language}, two new
Core-language constructs are required for programs involving data
structures: {\em constructors}\index{constructors!in G-machine},
\mbox{\tt EConstr}\indexTT{EConstr}; and {\em case expressions}\index{case expressions!in
G-machine}, \mbox{\tt ECase}\indexTT{ECase}. Our goal, in producing the Mark~6 machine, is to
compile code for these expressions.

\subsection{Overview}

In Section~\ref{sect:lang:constructors} we saw that a constructor with
tag \mbox{\tt t} and arity \mbox{\tt a} was represented as \mbox{\tt Pack{\char'173}t,a{\char'175}}\indexTT{Pack} in the core
language. For example, the usual list data type has two constructors:
\mbox{\tt Pack{\char'173}1,0{\char'175}}\indexTT{Pack} and \mbox{\tt Pack{\char'173}2,2{\char'175}}\indexTT{Pack}. These correspond to Miranda's $[]$ and
$(:)$ respectively.

A \mbox{\tt case}\indexTT{case} expression, which has no direct counterpart in Miranda, is
used to inspect the values held in a constructor. For example, we
can write the length function for a list as:
\begin{verbatim}
length xs = case xs of
                   <1>      -> 0;
                   <2> y ys -> 1 + length ys
\end{verbatim}
It is instructive to look at the way we execute the \mbox{\tt case}\indexTT{case} expression
\begin{enumerate}

\item To evaluate the \mbox{\tt case}\indexTT{case} expression, we first evaluate \mbox{\tt xs} to
WHNF.  
% The expression \mbox{\tt xs} is called the {\em
% discriminant}\index{discriminant!of \mbox{\tt case}\indexTT{case} expression} of the
% \mbox{\tt case}\indexTT{case} expression.

\item Once this evaluation has occurred, we are able to tell which of
the {\em alternatives\/}\index{alternative!of \mbox{\tt case}\indexTT{case} expression} to
take. The {\em tag\/}\index{tag!of constructor} of the evaluated
expression -- which must be a structured data object -- determines
which alternative we take. In the above example, for \mbox{\tt length}\indexTT{length}:

\begin{itemize}

\item If the tag of the constructor for \mbox{\tt xs} is \mbox{\tt 1} then the list is
empty and we take the first branch. We therefore return \mbox{\tt 0}.

\item If the tag is \mbox{\tt 2}, then the list is non-empty. This time there
are components of the constructor (\mbox{\tt y} and \mbox{\tt ys}). The length of
the list is one more than the length of \mbox{\tt ys}.
\end{itemize}
\end{enumerate}
We will assume that whenever we attempt to dismantle a constructor it
has been applied to the correct number of arguments. A constructor in
this state is said to be {\em saturated}\index{saturated constructor}.
As an example, in Section~\ref{sect:structured-types}, \mbox{\tt Cons}\indexTT{Cons} is
defined to take two arguments, so it is saturated when it is applied
to two expressions.

We also note that a Core-language program can now return a result that
is a structured data object. The Mark~5 G-machine must be able to
print the structured data object in a lazy fashion. Let us first
consider what additions will need to be made to the data structures of
the Mark~5 machine.

\subsection{Data structures\index{data structures!G-machine Mark 6}}

It would be nice to allow the machine to return values which are not
just numbers. We would like to be able to return values that consist
of constructors. This will require us to evaluate the components of
the structure recursively, and then return these values. To do this we
need to add yet another component to the state: \mbox{\tt gmOutput}\indexTT{gmOutput}. This will
hold the result of the program.
\begin{verbatim}
> type GmState =
>    (GmOutput,              -- Current Output
>     GmCode,                -- Current Instruction Stream
>     GmStack,               -- Current Stack
>     GmDump,                -- The Dump
>     GmHeap,                -- Heap of Nodes
>     GmGlobals,             -- Global addresses in Heap
>     GmStats)               -- Statistics
\end{verbatim}
\indexDTT{type}%
This component is defined to be a character string.
\begin{verbatim}
> type GmOutput = [Char]
\end{verbatim}
\indexDTT{type}%
\par
We can write the access functions in the obvious way.
\begin{verbatim}
> getOutput :: GmState -> GmOutput
> getOutput (o, i, stack, dump, heap, globals, stats) = o
\end{verbatim}
\indexDTT{getOutput}%
\indexDTT{getOutput}%
\begin{verbatim}
> putOutput :: GmOutput -> GmState -> GmState
> putOutput o' (o, i, stack, dump, heap, globals, stats)
>      = (o', i, stack, dump, heap, globals, stats)
\end{verbatim}
\indexDTT{putOutput}%
\indexDTT{putOutput}%
\begin{exercise}\label{gm:X:access5}
Make the appropriate changes to the remainder of the access functions.
\end{exercise}

To support constructor nodes in the heap, we augment the type \mbox{\tt node}\indexTT{node}
with \mbox{\tt NConstr}\indexTT{NConstr}; this takes a positive number which will represent a
{\em tag}\index{tag!of constructor}, and a list of {\em
components}\index{components!of constructor} which we represent as the
list of the addresses of the nodes in heap.
\begin{verbatim}
> data Node
>           = NNum Int              -- Numbers
>           | NAp Addr Addr         -- Applications
>           | NGlobal Int GmCode    -- Globals
>           | NInd Addr
>           | NConstr Int [Addr]
> instance Eq Node
>   where
>   NNum a       == NNum b          = a == b    -- needed to check conditions
>   NAp a b      == NAp c d         = False     -- not needed
>   NGlobal a b  == NGlobal c d     = False     -- not needed
>   NInd a       == NInd b          = False     -- not needed
>   NConstr a b  == NConstr c d     = False     -- not needed
\end{verbatim}
\indexDTT{data}%
\indexDTT{NAp}%
\indexDTT{NGlobal}%
\indexDTT{NInd}%
\indexDTT{NConstr}%
\indexDTT{instance}%
\subsection{Printing the result}

Because we have a new state component which we wish to display, we
must redefine the function \mbox{\tt showState}\indexTT{showState}.
\begin{verbatim}
> showState :: GmState -> Iseq
> showState s
>      = iConcat [showOutput s,                 iNewline,
>              showStack s,                  iNewline,
>              showDump s,                   iNewline,
>              showInstructions (getCode s), iNewline]
\end{verbatim}
\indexDTT{showState}%
\indexDTT{showState}%
The \mbox{\tt showOutput}\indexTT{showOutput} function is easy, because the output component is already a
string.
\begin{verbatim}
> showOutput :: GmState -> Iseq
> showOutput s = iConcat [iStr "Output:\"", iStr (getOutput s), iStr "\""]
\end{verbatim}
\indexDTT{showOutput}%
\indexDTT{showOutput}%
\par
The only other change (apart from changing \mbox{\tt showInstruction}\indexTT{showInstruction} for the new
instruction set) occurs in \mbox{\tt showNode}\indexTT{showNode}, because we have extended the data type
to include constructor nodes.
\begin{verbatim}
> showNode :: GmState -> Addr -> Node -> Iseq
> showNode s a (NNum n)      = iNum n
> showNode s a (NGlobal n g) = iConcat [iStr "Global ", iStr v]
>   where v = head [n | (n,b) <- getGlobals s, a==b]
> showNode s a (NAp a1 a2)   = iConcat [iStr "Ap ",  iStr (showaddr a1),
>                                       iStr " ",    iStr (showaddr a2)]
> showNode s a (NInd a1)     = iConcat [iStr "Ind ", iStr (showaddr a1)]
> showNode s a (NConstr t as)
>  = iConcat [iStr "Cons ", iNum t, iStr " [",
>             iInterleave (iStr ", ") (map (iStr.showaddr) as), iStr "]"]
\end{verbatim}
\indexDTT{showNode}%
\indexDTT{showNode}%
\indexDTT{showNode}%
\indexDTT{showNode}%
\indexDTT{showNode}%
\indexDTT{showNode}%
\subsection{The instruction set}

The new instruction set is now defined. It simply adds four new instructions to
the Mark~4 machine.
\begin{verbatim}
> data Instruction 
>    = Slide Int
>    | Alloc Int
>    | Update Int
>    | Pop Int
>    | Unwind
>    | Pushglobal Name
>    | Pushint Int
>    | Push Int
>    | Mkap
>    | Eval
>    | Add | Sub | Mul | Div
>    | Neg
>    | Eq | Ne | Lt | Le | Gt | Ge
>    | Cond GmCode GmCode
\end{verbatim}
\indexDTT{data}%
\indexDTT{Alloc}%
\indexDTT{Update}%
\indexDTT{Pop}%
\indexDTT{Unwind}%
\indexDTT{Pushglobal}%
\indexDTT{Pushint}%
\indexDTT{Push}%
\indexDTT{Mkap}%
\indexDTT{Eval}%
\indexDTT{Add}%
\indexDTT{Sub}%
\indexDTT{Mul}%
\indexDTT{Div}%
\indexDTT{Neg}%
\indexDTT{Eq}%
\indexDTT{Ne}%
\indexDTT{Lt}%
\indexDTT{Le}%
\indexDTT{Gt}%
\indexDTT{Ge}%
\indexDTT{Cond}%
The four new instructions that are added to the machine are as follows:
\begin{verbatim}
>    | Pack Int Int
>    | Casejump [(Int, GmCode)]
>    | Split Int
>    | Print
\end{verbatim}
\indexDTT{Pack}%
\indexDTT{Casejump}%
\indexDTT{Split}%
\indexDTT{Print}%
\begin{exercise}\label{gm:X:showInstruction5}
Extend \mbox{\tt showInstruction}\indexTT{showInstruction} to match the new instruction set.
\end{exercise}

The \mbox{\tt Pack}\indexTT{Pack} instruction is simple; it assumes that there are
sufficient arguments on the stack to construct a {\em saturated
constructor}\index{saturated constructor}.  When there are, it
proceeds to make a saturated constructor; if there are not enough
arguments, then the instruction is undefined.

\gmruleod%
{\gmstateod{o}{\mbox{\tt Pack}\indexTT{Pack}\ t\ n:i}{a_1:\ldots:a_n:s}{d}{h}{m}}%
{\gmstateod{o}{i}{a:s}{d}{h[a:\mbox{\tt NConstr}\indexTT{NConstr}\ t [a_1,\ldots,a_n]]}{m}}

The transition rule for \mbox{\tt Casejump}\indexTT{Casejump} expects (a) that the node
on top of the stack is in WHNF, and (b) that
the node is a structured data object. Using the tag from this
object we select one of the alternative instruction sequences, and
the current instruction stream is then prefixed by the code for the
particular alternative selected.

\gmruleod%
{\gmstateod{o}{\mbox{\tt Casejump}\indexTT{Casejump}\ [\ldots, t\mbox{\tt ->}i', \ldots ]:i}{a:s}{d}{h[a:\mbox{\tt NConstr}\indexTT{NConstr}\ t\ ss]}{m}}%
{\gmstateod{o}{i'\append i}{a:s}{d}{h}{m}}

This is a simple way to specify a multiway jump and join. That is, by
prefixing the current code $i$ by the code for the alternative $i'$,
we achieve the effect of first running the code for the alternative
and then resuming with whatever the remainder of the code for the main
expression requires\footnote{It should be noted that code sequences
using \mbox{\tt Casejump}\indexTT{Casejump} are not flat. We can, however, construct the flat
code sequences we desire by labelling each alternative and jumping to
the labelled code addresses. We have not done this as it unnecessarily
complicates the code generation.}.

The code for each alternative begins with a $\mbox{\tt Split}\indexTT{Split}\ n$ instruction
and terminates with a $\mbox{\tt Slide}\indexTT{Slide}\ n$ instruction. The value of $n$ is
determined by the number of components in the constructor. The \mbox{\tt Split}\indexTT{Split}
instruction is used to gain access to the components of a constructor.

Consider the code sequence generated for the \mbox{\tt length}\indexTT{length} function:
\begin{verbatim}
       [Push 0, Eval,
        Casejump [1 -> [Pushint 0]
                  2 -> [Split 2, Push 1, Pushglobal "length", Mkap,
                        Eval, Pushint 1, Add, Slide 2]],
        Update 1,
        Pop 1,
        Unwind]
\end{verbatim}
The execution of this pattern is shown in Figure~\ref{gm:fg:6alter},
where we see that the \mbox{\tt Slide}\indexTT{Slide} and \mbox{\tt Split}\indexTT{Split} instructions are being used
temporarily to extend the current set of local bindings. Assuming that
the \mbox{\tt length}\indexTT{length} function was applied to a non-nil node, when we execute
the \mbox{\tt Casejump}\indexTT{Casejump} instruction, we take the alternative labelled \mbox{\tt 2}. This
is the initial diagram (a). The
\mbox{\tt Split\ 2} instruction `unpacks' the constructor node onto the stack.
This is shown in diagram (b). After completing the body of
the alternative, i.e.\ the code sequence
\begin{verbatim}
        [Push 1, Pushglobal "length", Mkap, Eval, Pushint 1, Add]
\end{verbatim}
the length of the list argument to this call of \mbox{\tt length}\indexTT{length} will be on
top of the stack labelled \mbox{\tt l} in diagram (c).  To
complete the execution we remove the pointers to \mbox{\tt head}\indexTT{head} and \mbox{\tt tail}\indexTT{tail};
this is shown in diagram (d).
\begin{figure} %\centering
\setlength{\unitlength}{59204sp}%
%
\begingroup\makeatletter\ifx\SetFigFont\undefined%
\gdef\SetFigFont#1#2#3#4#5{%
  \reset@font\fontsize{#1}{#2pt}%
  \fontfamily{#3}\fontseries{#4}\fontshape{#5}%
  \selectfont}%
\fi\endgroup%
\begin{picture}(367,265)(18,542)
\thinlines
\put( 40,610){\vector( 1, 0){ 70}}
\put( 40,630){\line( 1, 0){ 80}}
\put(120,630){\line( 1,-1){ 20}}
\put(140,610){\vector( 1, 0){ 20}}
\put( 40,650){\line( 1, 0){ 80}}
\put(120,650){\line( 1,-1){ 20}}
\put(140,630){\vector( 1, 0){ 20}}
\put(110,610){\line(-1,-1){ 30}}
\put( 80,580){\line( 1, 0){ 60}}
\put(140,580){\line(-1, 1){ 30}}
\put( 20,620){\line( 1, 0){ 40}}
\put( 20,640){\line( 1, 0){ 40}}
\put( 20,660){\line( 1, 0){ 40}}
\put( 20,665){\line( 0,-1){ 65}}
\put( 20,600){\line( 1, 0){ 40}}
\put( 60,600){\line( 0, 1){ 65}}
\put(150,790){\line( 0,-1){ 20}}
\put(150,770){\vector( 1, 0){ 10}}
\put(130,790){\line( 0,-1){ 40}}
\put(130,750){\vector( 1, 0){ 30}}
\put( 40,790){\vector( 1, 0){ 60}}
\put(100,780){\framebox(60,20){}}
\put( 20,800){\line( 1, 0){ 40}}
\put( 20,805){\line( 0,-1){ 25}}
\put( 20,780){\line( 1, 0){ 40}}
\put( 60,780){\line( 0, 1){ 25}}
\put(320,780){\framebox(60,20){}}
\put(240,800){\line( 1, 0){ 40}}
\put(240,780){\line( 1, 0){ 40}}
\put(370,790){\line( 0,-1){ 20}}
\put(370,770){\vector( 1, 0){ 10}}
\put(240,805){\line( 0,-1){ 45}}
\put(240,760){\line( 1, 0){ 40}}
\put(280,760){\line( 0, 1){ 45}}
\put(260,790){\line( 1, 0){ 40}}
\put(300,790){\line( 1,-1){ 20}}
\put(320,770){\line( 1, 0){ 50}}
\put(260,770){\line( 1, 0){ 40}}
\put(300,770){\line( 1,-1){ 20}}
\put(320,750){\line( 1, 0){ 30}}
\put(350,765){\line( 0,-1){ 15}}
\put(350,750){\vector( 1, 0){ 30}}
\put(350,790){\line( 0,-1){ 15}}
\put(240,665){\line( 0,-1){ 25}}
\put(240,640){\line( 1, 0){ 40}}
\put(280,640){\line( 0, 1){ 25}}
\put(240,660){\line( 1, 0){ 40}}
\put(330,610){\line(-1,-1){ 30}}
\put(300,580){\line( 1, 0){ 60}}
\put(360,580){\line(-1, 1){ 30}}
\put(260,650){\line( 1, 0){ 40}}
\put(300,650){\line( 0,-1){ 40}}
\put(300,610){\vector( 1, 0){ 25}}
\put(245,545){\makebox(0,0)[lb]{\smash{\SetFigFont{11}{13.2}{\familydefault}{\mddefault}{\updefault}(d) After Slide 2}}}
\put( 25,545){\makebox(0,0)[lb]{\smash{\SetFigFont{11}{13.2}{\familydefault}{\mddefault}{\updefault}(c) After code for body}}}
\put(245,705){\makebox(0,0)[lb]{\smash{\SetFigFont{11}{13.2}{\familydefault}{\mddefault}{\updefault}(b) After Split 2}}}
\put( 25,705){\makebox(0,0)[lb]{\smash{\SetFigFont{11}{13.2}{\familydefault}{\mddefault}{\updefault}(a) On entering alternative}}}
\put(105,585){\makebox(0,0)[lb]{\smash{\SetFigFont{11}{13.2}{\familydefault}{\mddefault}{\updefault}l}}}
\put(165,605){\makebox(0,0)[lb]{\smash{\SetFigFont{11}{13.2}{\familydefault}{\mddefault}{\updefault}head}}}
\put(165,625){\makebox(0,0)[lb]{\smash{\SetFigFont{11}{13.2}{\familydefault}{\mddefault}{\updefault}tail}}}
\put(165,745){\makebox(0,0)[lb]{\smash{\SetFigFont{11}{13.2}{\familydefault}{\mddefault}{\updefault}head}}}
\put(165,765){\makebox(0,0)[lb]{\smash{\SetFigFont{11}{13.2}{\familydefault}{\mddefault}{\updefault}tail}}}
\put(105,785){\makebox(0,0)[lb]{\smash{\SetFigFont{11}{13.2}{\familydefault}{\mddefault}{\updefault}2}}}
\put(325,785){\makebox(0,0)[lb]{\smash{\SetFigFont{11}{13.2}{\familydefault}{\mddefault}{\updefault}2}}}
\put(385,765){\makebox(0,0)[lb]{\smash{\SetFigFont{11}{13.2}{\familydefault}{\mddefault}{\updefault}tail}}}
\put(385,745){\makebox(0,0)[lb]{\smash{\SetFigFont{11}{13.2}{\familydefault}{\mddefault}{\updefault}head}}}
\put(325,585){\makebox(0,0)[lb]{\smash{\SetFigFont{11}{13.2}{\familydefault}{\mddefault}{\updefault}l}}}
\end{picture}

\caption{Running compiled code for alternatives}\label{gm:fg:6alter}
\end{figure}
The transition for \mbox{\tt Split}\indexTT{Split} is straightforward.

\gmruleod%
{\gmstateod{o}{\mbox{\tt Split}\indexTT{Split}\ n:i}{a:s}{d}{h[a:\mbox{\tt NConstr}\indexTT{NConstr}\ t\ [a_1,\ldots, a_n]]}{m}}%
{\gmstateod{o}{i}{a_1:\ldots :a_n:s}{d}{h}{m}}

Next, we describe the transitions for \mbox{\tt Print}\indexTT{Print}. There are two
transitions for \mbox{\tt Print}\indexTT{Print}; one each for constructors and numbers.

\gmruleod%
{\gmstateod{o}{\mbox{\tt Print}\indexTT{Print}:i}{a:s}{d}{h[a:\mbox{\tt NNum}\indexTT{NNum}\ n]}{m}}%
{\gmstateod{o\append [n]}{i}{s}{d}{h}{m}}

The rule for constructors is more complex, as it must arrange to print
each component of the constructor. For simplicity we will only print
out the components.

\gmruleod%
{\gmstateod{o}{\mbox{\tt Print}\indexTT{Print}:i}{a:s}{d}{h[a:\mbox{\tt NConstr}\indexTT{NConstr}\ t\ [a_1,\ldots, a_n]]}{m}}%
{\gmstateod{o}{i' \append i}{a_1:\ldots :a_n:s}{d}{h}{m}}

The code $i'$ is simply:
\[ [\underbrace{\mbox{\tt Eval}\indexTT{Eval},\ \mbox{\tt Print}\indexTT{Print}, \ldots, \mbox{\tt Eval}\indexTT{Eval},\ \mbox{\tt Print}\indexTT{Print}}_n ].\]

Lastly, we must add a new rule for \mbox{\tt Unwind}\indexTT{Unwind}, that tells it to return
when unwinding an \mbox{\tt NConstr}\indexTT{NConstr}, just like the rule for \mbox{\tt NNum}\indexTT{NNum}.
\gmruled%
{\gmstated{[\mbox{\tt Unwind}\indexTT{Unwind}]}{a:s}{\langle i',\,s'\rangle:d}{h[a: \mbox{\tt NConstr}\indexTT{NConstr}\ n\ as]}{m}}%
{\gmstated{i'}{a:s'}{d}{h}{m}}

\begin{exercise}\label{gm:X:step6}
Implement the new transitions and modify the \mbox{\tt dispatch}\indexTT{dispatch} function.
\end{exercise}

\subsection{The compiler\index{G-machine compiler!Mark 6}}

In Figure~\ref{gm:fg:CE6} the new cases for the \tE{}~and~\tC{}
compilation schemes are presented.  They require auxiliary compilation
schemes, \tD{} and \tA{}, to deal with the alternatives that may be
selected in a \mbox{\tt case}\indexTT{case} expression. The function \mbox{\tt compileAlts}\indexTT{compileAlts}
(corresponding to the \tD{} scheme) compiles a list of alternatives,
using the current environment, and produces a list of tagged code
sequences. It also uses the \mbox{\tt comp}\indexTT{comp} argument (which corresponds to
\tA{}) to compile the body of each of the alternatives. For the moment
this argument will always be \mbox{\tt compileE'}.
\begin{figure*}
$\begin{array}{|rcll|}
\hline
&&&\\
\multicolumn{4}{|l|}{\parbox[t]{29pc}{%
$\E{e}~\rho$ compiles code that evaluates an expression $e$ to WHNF in
environment $\rho$, leaving a pointer to the expression on top of the
stack.}}\\
&&&\\
\indexD{$\protect\cal \pE$}\E{\mbox{\tt case}\indexTT{case}~ e ~\mbox{\tt of}~ alts}~ \rho & = & \E{e}~\rho \append
                                         [\mbox{\tt Casejump}\indexTT{Casejump}\ \D{alts}\ \rho] & \\
\indexD{$\protect\cal \pE$}\E{\mbox{\tt Pack{\char'173}}t\mbox{\tt ,}a\mbox{\tt {\char'175}}\indexTT{Pack}~e_1~\ldots~e_a}~\rho & = & \C{e_a}~\rho^{+0} \append\ldots
                          \C{e_1}~\rho^{+(a-1)}\append [\mbox{\tt Pack}\indexTT{Pack}\ t\ a] &\\
&&&\\
\hline
&&&\\
\multicolumn{4}{|l|}{\parbox[t]{29pc}{%
$\C{e}~\rho$ generates code which constructs the graph of $e$ in environment
$\rho$, leaving a pointer to it on top of the stack.}}\\
&&&\\
\indexD{$\protect\cal \pC$}\C{\mbox{\tt Pack{\char'173}}t\mbox{\tt ,}a\mbox{\tt {\char'175}}\indexTT{Pack}~e_1~\ldots~e_a}~\rho & = & \C{e_a}~\rho^{+0}\append\ldots
                          \C{e_1}~\rho^{+(a-1)}\append [\mbox{\tt Pack}\indexTT{Pack}\ t\ a] &\\
&&&\\
\hline
%
% Two figures combined here by Simon 5/11/91
%
%\end{array}$
%\caption{New cases for the Mark~6 \tC{} and \tE{} schemes}\label{gm:fg:CE6}
%\end{figure*}
%\begin{figure*}[htbp]
%$\begin{array}{|rcll|}
%\hline
&&&\\
\multicolumn{4}{|l|}{\parbox[t]{29pc}{%
$\D{alts}~\rho$ compiles the code for the alternatives in a \mbox{\tt case}\indexTT{case} expression.}}
\\
&&&\\
\indexD{$\protect\cal \pD$}\D{alt_1~ \ldots ~alt_n}~\rho
& = & [\A{alt_1}~\rho~, \ldots ,~ \A{alt_n}~\rho] & \\
&&&\\
\hline
&&&\\
\multicolumn{4}{|l|}{\parbox[t]{29pc}{%
$\A{alt}~\rho$ compiles the code for an alternative in a \mbox{\tt case}\indexTT{case} expression.}}\\
&&&\\
\indexD{$\protect\cal \pA$}\A{ \mbox{\tt <} t \mbox{\tt >}~x_1 \ldots x_n~\mbox{\tt ->}~body}~\rho
& = & t~ \mbox{\tt ->} ~[\mbox{\tt Split}\indexTT{Split}\ n] \append \E{body}~\rho' \append [\mbox{\tt Slide}\indexTT{Slide}\ n] &\\
\multicolumn{4}{|r|}{%
\tr{where $\rho' = \rho^{+n}[x_1\mapsto 0\ldots x_n\mapsto n-1]$}}\\
&&&\\
\hline
\end{array}$
\caption{Compilation schemes for \mbox{\tt case}\indexTT{case} expressions}\label{gm:fg:CE6}
%\caption{The \tD{} and \tA{} compilation schemes}\label{gm:fg:DA6}
\end{figure*}
\begin{verbatim}
> compileAlts ::    (Int -> GmCompiler)     -- compiler for alternative bodies
>                   -> [CoreAlt]            -- the list of alternatives
>                   -> GmEnvironment        -- the current environment
>                   -> [(Int, GmCode)]      -- list of alternative code sequences
> compileAlts comp alts env
>  = [(tag, comp (length names) body (zip names [0..] ++ argOffset (length names) env))
>          | (tag, names, body) <- alts]
\end{verbatim}
\indexDTT{compileAlts}%
\indexDTT{compileAlts}%
\par
The \mbox{\tt compileE'} scheme is a small modification to the \mbox{\tt compileE}\indexTT{compileE}
scheme. It simply places a \mbox{\tt Split}\indexTT{Split} and \mbox{\tt Slide}\indexTT{Slide} around the code
generated by the ordinary \mbox{\tt compileE}\indexTT{compileE} scheme.
\begin{verbatim}
> compileE' :: Int -> GmCompiler
> compileE' offset expr env
>     = [Split offset] ++ compileE expr env ++ [Slide offset]
\end{verbatim}
\begin{exercise}
Make the relevant changes to \mbox{\tt compile}\indexTT{compile}, and modify \mbox{\tt initialCode}\indexTT{initialCode} 
to have a final \mbox{\tt Print}\indexTT{Print} instruction.
\end{exercise}

\begin{exercise}
Add the new cases to the compiler functions \mbox{\tt compileE}\indexTT{compileE} and
\mbox{\tt compileC}\indexTT{compileC}.
\end{exercise}

\begin{exercise}
What changes are required to print out the output in `structured
form'. By this we mean placing the constructors and parentheses into
the output component \mbox{\tt gmOutput}\indexTT{gmOutput}, as well as integers.
\end{exercise}

\subsection{Using the new boolean representation in
comparisons\index{booleans!using constructors}}

In this section we show how the Mark~6 machine we have constructed,
can be modified to use our new representation of booleans\index{booleans}. We first
observe that we can implement the booleans as structured data objects;
with \mbox{\tt True}\indexTT{True} and \mbox{\tt False}\indexTT{False} being represented as constructors of zero
arity and tags 2 and 1 respectively.

How do we implement conditionals? This can be done by adding a new
definition to the program for \mbox{\tt if}. It returns either its second or
third argument, depending on the first argument.
\begin{verbatim}
        if c t f = case c of
                        <1> -> f;
                        <2> -> t
\end{verbatim}
The first change we require lies in the comparison operations. These
have the following generic transition rule.

\gmruleod%
{\gmstateod{o}{\odot :i}{a_0:a_1:s}{d}%
{h[a_0: \mbox{\tt NNum}\indexTT{NNum}\ n_0,\,a_1: \mbox{\tt NNum}\indexTT{NNum}\ n_1]}{m}}%
{\gmstateod{o}{i}{a:s}{d}{h[a: \mbox{\tt Constr}\indexTT{Constr}\ (n_0\,\odot\, n_1)\ []]}{m}}
%
For example, in the \mbox{\tt Eq} instruction, we replace $\odot$ with a
function that returns \mbox{\tt 2} (the tag for \mbox{\tt True}\indexTT{True}) if the two numbers
$n_0$ and $n_1$ are the same, and \mbox{\tt 1} (the tag for \mbox{\tt False}\indexTT{False}) otherwise.

We can implement the transitions quickly, by reusing some of the code
we developed for the Mark~4 machine. In Section~\ref{gm:sc:inst4} we
saw how to represent some generic arithmetic and comparison operators.
In fact, because of the way in which we structured the definition of
the \mbox{\tt comparison}\indexTT{comparison} function, we can almost immediately use our new
representation of booleans.

The boxing function \mbox{\tt boxBoolean}\indexTT{boxBoolean} takes a comparison operation and a
state in which there are two integers on top of the stack. It returns
the new state in which there is the boolean result of comparing the
two integers on top of the stack.
\begin{verbatim}
> boxBoolean :: Bool -> GmState -> GmState
> boxBoolean b state
>  = putStack (a: getStack state) (putHeap h' state)
>    where (h',a) = hAlloc (getHeap state) (NConstr b' [])
>          b' | b = 2              -- 2 is tag of True
>             | otherwise = 1      -- 1 is tag of False
\end{verbatim}
\indexDTT{boxBoolean}%
\indexDTT{boxBoolean}%
\begin{exercise}\label{gm:X:fac6}
Run some example programs from Appendix~\ref{sect:core-progs}; for
example try the factorial program:
\begin{verbatim}
        fac n = if (n==0) 1 (n * fac (n-1))
\end{verbatim}
\end{exercise}

\subsection{Extending the language accepted}

As astute readers might have noticed, there are some legal expressions
involving \mbox{\tt ECase}\indexTT{ECase} and \mbox{\tt EConstr}\indexTT{EConstr} for which our compiler will fail. The
legal expressions that we cannot compile fall into two classes:
\begin{enumerate}
\item Occurrences of \mbox{\tt ECase}\indexTT{ECase} in non-strict contexts; i.e.\ in
expressions compiled by the \tC{} scheme.

\item Occurrences of \mbox{\tt EConstr}\indexTT{EConstr} in expressions where it is applied to
too few arguments.
\end{enumerate}
Both problems can be solved by using program transformation
techniques. The solution for \mbox{\tt ECase}\indexTT{ECase} is to make the offending
expressions into supercombinators which are then applied to their free
variables. For example, the program:
\begin{verbatim}
        f x = Pack{2,2} (case x of <1> -> 1; <2> -> 2) Pack{1,0}
\end{verbatim}
can be transformed into the equivalent program:
\begin{verbatim}
        f x = Pack{2,2} (g x) Pack{1,0}
        g x = case x of <1> -> 1; <2> -> 2
\end{verbatim}
The trick for \mbox{\tt EConstr}\indexTT{EConstr} is to create a supercombinator for each
constructor; this will be generated with enough free variables to
saturate the constructor. Here is an example.
\begin{verbatim}
        prefix p xs = map (Pack{2,2} p) xs
\end{verbatim}
This is transformed to:
\begin{verbatim}
        prefix p xs = map (f p) xs
        f p x = Pack{2,2} p x
\end{verbatim}

Another way to solve this problem is to modify the \mbox{\tt Pushglobal}\indexTT{Pushglobal}
instruction, so that it will work for functions with names of the
form: `\mbox{\tt Pack{\char'173}t,a{\char'175}}\indexTT{Pack}'. We can then simply look for constructor functions,
such as \mbox{\tt f} in the example above, in the \mbox{\tt globals}\indexTT{globals} component of the
state.  If the function is not already present, we can create a new
global node to associate with the function because the node has a
particularly simple structure.
\begin{verbatim}
        NGlobal a [Pack t a, Update 0, Unwind]
\end{verbatim}
The new transitions are, firstly, if the function exists already:

\gmruleod%
{\gmstateod{o}{\mbox{\tt Pushglobal\ Pack{\char'173}}t\mbox{\tt ,}n\mbox{\tt {\char'175}} :i}{s}{d}{h}{m[\mbox{\tt Pack{\char'173}}t\mbox{\tt ,}n \mbox{\tt {\char'175}}\indexTT{Pack} :a]}}%
{\gmstateod{o}{i}{a:s}{d}{h}{m}}
%
and secondly when it does not already exist:

\gmruleod%
{\gmstateod{o}{\mbox{\tt Pushglobal\ Pack{\char'173}}t\mbox{\tt ,}n\mbox{\tt {\char'175}} :i}{s}{d}{h}{m}}%
{\gmstateod{o}{i}{a:s}{d}{h[a:\ \mbox{\tt gNode}\indexTT{gNode}_{t,n}]}{m[\mbox{\tt Pack{\char'173}}t\mbox{\tt ,}n\mbox{\tt {\char'175}}\indexTT{Pack} :a]}}
where $\mbox{\tt gNode}\indexTT{gNode}_{t,n}$ is
\[
\mbox{\tt NGlobal}\indexTT{NGlobal}\ n\ [\mbox{\tt Pack}\indexTT{Pack}\ t\ n,\ \mbox{\tt Update}\indexTT{Update}\ 0,\ \mbox{\tt Unwind}\indexTT{Unwind}]
\]
Our compiler can then generate code for expressions with unsaturated
constructor nodes directly. It does this by generating the following
code for unsaturated constructors.
\[
\indexD{$\protect\cal \pC$}\C{\mbox{\tt Pack{\char'173}} t \mbox{\tt ,} a \mbox{\tt {\char'175}}\indexTT{Pack} }~\rho = [\mbox{\tt Pushglobal}\indexTT{Pushglobal}\ \mbox{\tt "Pack{\char'173}} t \mbox{\tt ,} a \mbox{\tt {\char'175}"}]
\]

\begin{exercise}\label{gm:X:6consext}
Implement the extensions to the \mbox{\tt pushglobal}\indexTT{pushglobal} function for the
\mbox{\tt Pushglobal}\indexTT{Pushglobal} instruction and modify the compiler.
\end{exercise}

\section{Mark 7: Further
improvements\index{G-machine!Mark 7}\index{value stack!in G-machine}}
\label{sect:v-stack}

Let us consider again the example program we saw when we developed the
Mark~5 compiler.
\begin{verbatim}
        main = 3+4*5
\end{verbatim}
This generates the following code when we use the Mark~6 compiler:
\begin{verbatim}
        [Pushint 5, Pushint 4, Mul, Pushint 3, Add]
\end{verbatim}
When executed this code will use five heap nodes. Is it possible to
reduce this still further?

The answer is yes. We can reduce the number of heap accesses for
arithmetic further by using a stack of numbers to represent
intermediate values in the computation. In the Mark~7 machine these
values are held in a new state component called the {\em
V-stack}\index{V-stack}. The problem is that placing numbers into the
heap or extracting them is an expensive operation on a real machine.
It is much more efficient to use the machine's register set or
stack. In the Mark~7 G-machine we will use a stack; this means that we
do not have to worry about running out of registers.

The new code for the program
\begin{verbatim}
        main = 3+4*5
\end{verbatim}
is very similar to that which we previously generated:
\begin{verbatim}
        [Pushbasic 5, Pushbasic 4, Mul, Pushbasic 3, Add, Mkint]
\end{verbatim}
The first instruction \mbox{\tt Pushbasic\ 5} places \mbox{\tt 5} on top of the V-stack.
Next we push \mbox{\tt 4} onto the V-stack, following this by a multiplication.
This instruction now expects its arguments to be in the V-stack. It
will place the answer into the V-stack as well. The next two
instructions add \mbox{\tt 3} to the value on top of the V-stack. The final
instruction, \mbox{\tt Mkint}\indexTT{Mkint}, takes the value on top of the V-stack and places
it into a number node in the heap, leaving a pointer to this new node
on top of the S-stack.

\subsection{Executing the factorial function using the V-stack}

We begin an investigation into the Mark~7 machine, by way of an
example.  We will be looking at the execution of the factorial
function, defined as follows:
\begin{verbatim}
        fac n = if (n==0) 1 (n * fac (n-1))
\end{verbatim}
Using the Mark~7 compiler, we will generate the following code
sequence for the body of the supercombinator.
\begin{verbatim}
[Pushbasic 0, Push 0, Eval, Get, Eq,
 Cond [Pushint 1, Update 1, Pop 1, Unwind]
      [Pushint 1, Push 1, Pushglobal "-", Mkap, Mkap, Pushglobal "fac", Mkap,
       Eval, Get, Push 0, Eval, Get, Mul, Mkint, Update 1, Pop 1, Unwind]
\end{verbatim}

When this code commences execution, the V-stack is empty, and there is
one item on the ordinary stack. We will call the latter the S-stack
from now on to distinguish the two sorts of stack. In
Figure~\ref{gm:fg:7fac1} -- starting with diagram (a) --
we see the initial state in which a pointer to the argument to \mbox{\tt fac}\indexTT{fac}
is on top of the S-stack. In diagram (b) we see what happens when a
\mbox{\tt Pushbasic}\indexTT{Pushbasic} instruction is executed: an integer is pushed onto the
V-stack. Diagram (c)
shows that the argument to \mbox{\tt fac}\indexTT{fac} has now been evaluated, whilst in
diagram (d) we see the effect of a \mbox{\tt Get}\indexTT{Get}
instruction. It has extracted the value from the node in the heap, and
placed it into the V-stack.
\begin{figure}
\setlength{\unitlength}{59204sp}%
%
\begingroup\makeatletter\ifx\SetFigFont\undefined%
\gdef\SetFigFont#1#2#3#4#5{%
  \reset@font\fontsize{#1}{#2pt}%
  \fontfamily{#3}\fontseries{#4}\fontshape{#5}%
  \selectfont}%
\fi\endgroup%
\begin{picture}(404,212)(18,625)
\thinlines
\put(320,690){\vector( 1, 0){ 70}}
\put(300,700){\line( 1, 0){ 40}}
\put(300,705){\line( 0,-1){ 25}}
\put(300,680){\line( 1, 0){ 40}}
\put(340,680){\line( 0, 1){ 25}}
\put(240,680){\line( 1, 0){ 40}}
\put(240,660){\framebox(40,40){}}
\put( 80,705){\line( 0,-1){ 45}}
\put( 80,660){\line( 1, 0){ 40}}
\put(120,660){\line( 0, 1){ 45}}
\put( 80,680){\line( 1, 0){ 40}}
\put(100,670){\vector( 4, 1){ 60}}
\put(100,690){\vector( 1, 0){ 70}}
\put( 80,700){\line( 1, 0){ 40}}
\put( 20,680){\framebox(40,20){}}
\put(400,810){\line(-2,-3){ 20}}
\put(380,780){\line( 1, 0){ 40}}
\put(420,780){\line(-2, 3){ 20}}
\put(300,820){\line( 1, 0){ 40}}
\put(300,825){\line( 0,-1){ 25}}
\put(300,800){\line( 1, 0){ 40}}
\put(340,800){\line( 0, 1){ 25}}
\put(320,810){\vector( 1, 0){ 70}}
\put(240,800){\framebox(40,20){}}
\put(100,810){\vector( 1, 0){ 70}}
\put(180,810){\line(-2,-3){ 20}}
\put(160,780){\line( 1, 0){ 40}}
\put(200,780){\line(-2, 3){ 20}}
\put( 80,820){\line( 1, 0){ 40}}
\put( 80,825){\line( 0,-1){ 25}}
\put( 80,800){\line( 1, 0){ 40}}
\put(120,800){\line( 0, 1){ 25}}
\put( 20,820){\line( 1, 0){ 40}}
\put(245,625){\makebox(0,0)[lb]{\smash{\SetFigFont{11}{13.2}{\familydefault}{\mddefault}{\updefault}(d) After Get}}}
\put( 25,625){\makebox(0,0)[lb]{\smash{\SetFigFont{11}{13.2}{\familydefault}{\mddefault}{\updefault}(c) After Push 0, Eval}}}
\put(245,745){\makebox(0,0)[lb]{\smash{\SetFigFont{11}{13.2}{\familydefault}{\mddefault}{\updefault}(b) After Pushbasic 0}}}
\put( 25,745){\makebox(0,0)[lb]{\smash{\SetFigFont{11}{13.2}{\familydefault}{\mddefault}{\updefault}(a) State on entry to fac}}}
\put(255,665){\makebox(0,0)[lb]{\smash{\SetFigFont{11}{13.2}{\familydefault}{\mddefault}{\updefault}1}}}
\put(255,685){\makebox(0,0)[lb]{\smash{\SetFigFont{11}{13.2}{\familydefault}{\mddefault}{\updefault}0}}}
\put(395,685){\makebox(0,0)[lb]{\smash{\SetFigFont{11}{13.2}{\familydefault}{\mddefault}{\updefault}1}}}
\put(315,705){\makebox(0,0)[lb]{\smash{\SetFigFont{11}{13.2}{\familydefault}{\mddefault}{\updefault}S}}}
\put(255,705){\makebox(0,0)[lb]{\smash{\SetFigFont{11}{13.2}{\familydefault}{\mddefault}{\updefault}V}}}
\put(175,685){\makebox(0,0)[lb]{\smash{\SetFigFont{11}{13.2}{\familydefault}{\mddefault}{\updefault}1}}}
\put( 35,685){\makebox(0,0)[lb]{\smash{\SetFigFont{11}{13.2}{\familydefault}{\mddefault}{\updefault}0}}}
\put( 95,705){\makebox(0,0)[lb]{\smash{\SetFigFont{11}{13.2}{\familydefault}{\mddefault}{\updefault}S}}}
\put( 35,705){\makebox(0,0)[lb]{\smash{\SetFigFont{11}{13.2}{\familydefault}{\mddefault}{\updefault}V}}}
\put(255,805){\makebox(0,0)[lb]{\smash{\SetFigFont{11}{13.2}{\familydefault}{\mddefault}{\updefault}0}}}
\put(395,785){\makebox(0,0)[lb]{\smash{\SetFigFont{11}{13.2}{\familydefault}{\mddefault}{\updefault}e}}}
\put(315,825){\makebox(0,0)[lb]{\smash{\SetFigFont{11}{13.2}{\familydefault}{\mddefault}{\updefault}S}}}
\put(255,825){\makebox(0,0)[lb]{\smash{\SetFigFont{11}{13.2}{\familydefault}{\mddefault}{\updefault}V}}}
\put(175,785){\makebox(0,0)[lb]{\smash{\SetFigFont{11}{13.2}{\familydefault}{\mddefault}{\updefault}e}}}
\put( 95,825){\makebox(0,0)[lb]{\smash{\SetFigFont{11}{13.2}{\familydefault}{\mddefault}{\updefault}S}}}
\put( 35,825){\makebox(0,0)[lb]{\smash{\SetFigFont{11}{13.2}{\familydefault}{\mddefault}{\updefault}V}}}
\end{picture}

\caption{Mark~7 machine running \mbox{\tt fac}\indexTT{fac}}\label{gm:fg:7fac1}
\end{figure}

In diagram (e) (Figure~\ref{gm:fg:7fac2}) we see the state
after an \mbox{\tt Eq} instruction has executed. It has compared the two items
in the V stack, and discovered that they are not equal. The Mark~7
G-machine represents the boolean value \mbox{\tt False}\indexTT{False} by \mbox{\tt 1} in the V-stack.
In diagram (f), the \mbox{\tt Cond}\indexTT{Cond} instruction has inspected this
value, and used it to select which branch to execute.
%\begin{figure}[htbp]
\begin{figure}
\setlength{\unitlength}{59204sp}%
%
\begingroup\makeatletter\ifx\SetFigFont\undefined%
\gdef\SetFigFont#1#2#3#4#5{%
  \reset@font\fontsize{#1}{#2pt}%
  \fontfamily{#3}\fontseries{#4}\fontshape{#5}%
  \selectfont}%
\fi\endgroup%
\begin{picture}(362,75)(18,742)
\thinlines
\put(300,790){\vector( 1, 0){ 70}}
\put(280,800){\line( 1, 0){ 40}}
\put(280,805){\line( 0,-1){ 25}}
\put(280,780){\line( 1, 0){ 40}}
\put(320,780){\line( 0, 1){ 25}}
\put(220,800){\line( 1, 0){ 40}}
\put( 20,780){\framebox(40,20){}}
\put(100,790){\vector( 1, 0){ 70}}
\put( 80,800){\line( 1, 0){ 40}}
\put( 80,805){\line( 0,-1){ 25}}
\put( 80,780){\line( 1, 0){ 40}}
\put(120,780){\line( 0, 1){ 25}}
\put(225,745){\makebox(0,0)[lb]{\smash{\SetFigFont{11}{13.2}{\familydefault}{\mddefault}{\updefault}(f) After Cond}}}
\put( 25,745){\makebox(0,0)[lb]{\smash{\SetFigFont{11}{13.2}{\familydefault}{\mddefault}{\updefault}(e) After Eq}}}
\put(380,785){\makebox(0,0)[lb]{\smash{\SetFigFont{11}{13.2}{\familydefault}{\mddefault}{\updefault}1}}}
\put(295,805){\makebox(0,0)[lb]{\smash{\SetFigFont{11}{13.2}{\familydefault}{\mddefault}{\updefault}S}}}
\put(235,805){\makebox(0,0)[lb]{\smash{\SetFigFont{11}{13.2}{\familydefault}{\mddefault}{\updefault}V}}}
\put( 35,785){\makebox(0,0)[lb]{\smash{\SetFigFont{11}{13.2}{\familydefault}{\mddefault}{\updefault}1}}}
\put(175,785){\makebox(0,0)[lb]{\smash{\SetFigFont{11}{13.2}{\familydefault}{\mddefault}{\updefault}1}}}
\put( 95,805){\makebox(0,0)[lb]{\smash{\SetFigFont{11}{13.2}{\familydefault}{\mddefault}{\updefault}S}}}
\put( 35,805){\makebox(0,0)[lb]{\smash{\SetFigFont{11}{13.2}{\familydefault}{\mddefault}{\updefault}V}}}
\end{picture}

\caption{Mark~7 machine running \mbox{\tt fac}\indexTT{fac}}\label{gm:fg:7fac2}
\end{figure}

In diagram (g) (Figure~\ref{gm:fg:7fac3}) the state after
the construction and evaluation of \mbox{\tt fac\ (1-1)} is shown. The next
instruction is \mbox{\tt Get}\indexTT{Get} which fetches the newly evaluated value into the
V stack. Diagram (i) shows that we evaluate and fetch the value
from the node \mbox{\tt 1} into the V-stack. In diagram (j) a \mbox{\tt Mul}\indexTT{Mul}
instruction has multiplied the two values in the V-stack, placing the
result back there. In diagram (k) a \mbox{\tt Mkint}\indexTT{Mkint} instruction has
moved this result from the V-stack to the heap, recording the address
of the newly created node on the S-stack.
\begin{figure}
\setlength{\unitlength}{59204sp}%
%
\begingroup\makeatletter\ifx\SetFigFont\undefined%
\gdef\SetFigFont#1#2#3#4#5{%
  \reset@font\fontsize{#1}{#2pt}%
  \fontfamily{#3}\fontseries{#4}\fontshape{#5}%
  \selectfont}%
\fi\endgroup%
\begin{picture}(437,332)(18,485)
\thinlines
\put(100,530){\vector( 1, 0){ 70}}
\put(100,550){\vector( 1, 0){ 70}}
\put( 80,540){\line( 1, 0){ 40}}
\put( 80,565){\line( 0,-1){ 45}}
\put( 80,520){\line( 1, 0){ 40}}
\put(120,520){\line( 0, 1){ 45}}
\put( 80,560){\line( 1, 0){ 40}}
\put( 20,560){\line( 1, 0){ 40}}
\put( 20,560){\framebox(0,0){}}
\put( 20,660){\line( 1, 0){ 40}}
\put( 80,680){\line( 1, 0){ 40}}
\put( 80,685){\line( 0,-1){ 25}}
\put( 80,660){\line( 1, 0){ 40}}
\put(120,660){\line( 0, 1){ 25}}
\put(100,670){\vector( 1, 0){ 70}}
\put( 80,680){\framebox(0,0){}}
\put( 20,640){\framebox(40,40){}}
\put(100,770){\vector( 1, 0){ 55}}
\put(100,790){\vector( 1, 0){ 70}}
\put( 80,780){\line( 1, 0){ 40}}
\put( 80,800){\line( 1, 0){ 40}}
\put( 80,805){\line( 0,-1){ 45}}
\put( 80,760){\line( 1, 0){ 40}}
\put(120,760){\line( 0, 1){ 45}}
\put( 20,800){\line( 1, 0){ 40}}
\put(300,780){\framebox(40,20){}}
\put(360,805){\line( 0,-1){ 25}}
\put(360,780){\line( 1, 0){ 40}}
\put(400,780){\line( 0, 1){ 25}}
\put(360,800){\line( 1, 0){ 40}}
\put(380,790){\vector( 1, 0){ 70}}
\put(300,660){\framebox(40,20){}}
\put(360,685){\line( 0,-1){ 25}}
\put(360,660){\line( 1, 0){ 40}}
\put(400,660){\line( 0, 1){ 25}}
\put(360,680){\line( 1, 0){ 40}}
\put(380,670){\vector( 1, 0){ 70}}
\put( 25,485){\makebox(0,0)[lb]{\smash{\SetFigFont{11}{13.2}{\familydefault}{\mddefault}{\updefault}(k) After Mkint}}}
\put(305,605){\makebox(0,0)[lb]{\smash{\SetFigFont{11}{13.2}{\familydefault}{\mddefault}{\updefault}(j) After Mul}}}
\put( 25,605){\makebox(0,0)[lb]{\smash{\SetFigFont{11}{13.2}{\familydefault}{\mddefault}{\updefault}(i) After Push 0, Eval, Get}}}
\put(305,725){\makebox(0,0)[lb]{\smash{\SetFigFont{11}{13.2}{\familydefault}{\mddefault}{\updefault}(h) After Get}}}
\put( 25,725){\makebox(0,0)[lb]{\smash{\SetFigFont{11}{13.2}{\familydefault}{\mddefault}{\updefault}(g) After Pushint 1, ... Mkap, Eval}}}
\put(175,525){\makebox(0,0)[lb]{\smash{\SetFigFont{11}{13.2}{\familydefault}{\mddefault}{\updefault}1}}}
\put(175,545){\makebox(0,0)[lb]{\smash{\SetFigFont{11}{13.2}{\familydefault}{\mddefault}{\updefault}1}}}
\put( 95,565){\makebox(0,0)[lb]{\smash{\SetFigFont{11}{13.2}{\familydefault}{\mddefault}{\updefault}S}}}
\put( 35,565){\makebox(0,0)[lb]{\smash{\SetFigFont{11}{13.2}{\familydefault}{\mddefault}{\updefault}V}}}
\put(175,665){\makebox(0,0)[lb]{\smash{\SetFigFont{11}{13.2}{\familydefault}{\mddefault}{\updefault}1}}}
\put( 35,645){\makebox(0,0)[lb]{\smash{\SetFigFont{11}{13.2}{\familydefault}{\mddefault}{\updefault}1}}}
\put( 35,665){\makebox(0,0)[lb]{\smash{\SetFigFont{11}{13.2}{\familydefault}{\mddefault}{\updefault}1}}}
\put( 95,685){\makebox(0,0)[lb]{\smash{\SetFigFont{11}{13.2}{\familydefault}{\mddefault}{\updefault}S}}}
\put( 35,685){\makebox(0,0)[lb]{\smash{\SetFigFont{11}{13.2}{\familydefault}{\mddefault}{\updefault}V}}}
\put(165,765){\makebox(0,0)[lb]{\smash{\SetFigFont{11}{13.2}{\familydefault}{\mddefault}{\updefault}1}}}
\put(175,785){\makebox(0,0)[lb]{\smash{\SetFigFont{11}{13.2}{\familydefault}{\mddefault}{\updefault}1}}}
\put( 95,805){\makebox(0,0)[lb]{\smash{\SetFigFont{11}{13.2}{\familydefault}{\mddefault}{\updefault}S}}}
\put( 35,805){\makebox(0,0)[lb]{\smash{\SetFigFont{11}{13.2}{\familydefault}{\mddefault}{\updefault}V}}}
\put(315,805){\makebox(0,0)[lb]{\smash{\SetFigFont{11}{13.2}{\familydefault}{\mddefault}{\updefault}V}}}
\put(375,805){\makebox(0,0)[lb]{\smash{\SetFigFont{11}{13.2}{\familydefault}{\mddefault}{\updefault}S}}}
\put(455,785){\makebox(0,0)[lb]{\smash{\SetFigFont{11}{13.2}{\familydefault}{\mddefault}{\updefault}1}}}
\put(315,785){\makebox(0,0)[lb]{\smash{\SetFigFont{11}{13.2}{\familydefault}{\mddefault}{\updefault}1}}}
\put(315,685){\makebox(0,0)[lb]{\smash{\SetFigFont{11}{13.2}{\familydefault}{\mddefault}{\updefault}V}}}
\put(375,685){\makebox(0,0)[lb]{\smash{\SetFigFont{11}{13.2}{\familydefault}{\mddefault}{\updefault}S}}}
\put(455,665){\makebox(0,0)[lb]{\smash{\SetFigFont{11}{13.2}{\familydefault}{\mddefault}{\updefault}1}}}
\put(315,665){\makebox(0,0)[lb]{\smash{\SetFigFont{11}{13.2}{\familydefault}{\mddefault}{\updefault}1}}}
\end{picture}

\caption{Mark~7 machine running \mbox{\tt fac}\indexTT{fac}}\label{gm:fg:7fac3}
\end{figure}

In a machine where there is a performance penalty for creating and
accessing objects in the heap -- compared with keeping the objects in
a stack -- we expect the use of the V-stack to be an improvement.
Having seen how the V-stack works, we now make small modifications to
the G-machine to implement the Mark~7 machine.

\subsection{Data structures\index{data structures!in G-machine Mark 7}}

The use of the V-stack requires that each G-machine state has a new
state component \mbox{\tt gmVStack}\indexTT{gmVStack} added to its state.
\begin{verbatim}
> type GmState = (GmOutput,      -- Current output
>             GmCode,            -- Current instruction stream
>             GmStack,           -- Current stack
>             GmDump,            -- Current dump
>             GmVStack,          -- Current V-stack
>             GmHeap,            -- Heap of nodes
>             GmGlobals,         -- Global addresses in heap
>             GmStats)           -- Statistics
\end{verbatim}
\indexDTT{type}%
As we have already stated this new component behaves as a stack of
numbers.
\begin{verbatim}
> type GmVStack = [Int]
\end{verbatim}
\indexDTT{type}%
We add access functions for this component.
\begin{verbatim}
> getVStack :: GmState -> GmVStack
> getVStack (o, i, stack, dump, vstack, heap, globals, stats) = vstack
\end{verbatim}
\indexDTT{getVStack}%
\indexDTT{getVStack}%
\begin{verbatim}
> putVStack :: GmVStack -> GmState -> GmState
> putVStack vstack' (o, i, stack, dump, vstack, heap, globals, stats)
>      = (o, i, stack, dump, vstack', heap, globals, stats)
\end{verbatim}
\indexDTT{putVStack}%
\indexDTT{putVStack}%
\begin{exercise}\label{gm:X:access7}
Make the relevant changes to the other access functions.
\end{exercise}

\subsubsection{Displaying the states}

The function \mbox{\tt showState}\indexTT{showState} is changed so that it prints out the V-stack
component.
\begin{verbatim}
> showState :: GmState -> Iseq
> showState s
>      = iConcat [showOutput s,                 iNewline,
>            showStack s,                  iNewline,
>            showDump s,                   iNewline,
>            showVStack s,                 iNewline,
>            showInstructions (getCode s), iNewline]
\end{verbatim}
\indexDTT{showState}%
\indexDTT{showState}%
To do this we define the function \mbox{\tt showVStack}\indexTT{showVStack}.
\begin{verbatim}
> showVStack :: GmState -> Iseq
> showVStack s
>      = iConcat [iStr "Vstack:[",
>            iInterleave (iStr ", ") (map iNum (getVStack s)),
>            iStr "]"]
\end{verbatim}
\indexDTT{showVStack}%
\indexDTT{showVStack}%
\subsection{Instruction set}

An obvious first requirement is that each of the arithmetic
transitions will now have to be modified to get their values from, and
return their results to, the V-stack instead of the ordinary stack.
Let us first deal with the case of dyadic primitives. The generic
transition for the operation $\odot$ is given below. It takes two
arguments from the V-stack and places the result of the operation
$\odot$ back onto the V-stack.

\gmruleodv%
{\gmstateodv{o}{\odot :i}{s}{d}{n_0:n_1:v}{h}{m}}%
{\gmstateodv{o}{i}{s}{d}{n_0\,\odot\, n_1:v}{h}{m}}

The \mbox{\tt Neg}\indexTT{Neg} instruction now has the following transition: it simply
replaces the number on top of the V-stack with its negation.

\gmruleodv%
{\gmstateodv{o}{\mbox{\tt Neg}\indexTT{Neg} :i}{s}{d}{n:v}{h}{m}}%
{\gmstateodv{o}{i}{s}{d}{(-n):v}{h}{m}}
%
We also need instructions to move values between the heap and the
V-stack. We begin with \mbox{\tt Pushbasic}\indexTT{Pushbasic}, which pushes an integer $n$ onto
the V-stack.

\gmruleodv%
{\gmstateodv{o}{\mbox{\tt Pushbasic}\indexTT{Pushbasic}\ n :i}{s}{d}{v}{h}{m}}%
{\gmstateodv{o}{i}{s}{d}{n:v}{h}{m}}

To move a value from the V-stack to the heap, we use two instructions:
\mbox{\tt Mkbool}\indexTT{Mkbool} and \mbox{\tt Mkint}\indexTT{Mkint}. These treat the integer on top of the V-stack as
booleans and integers respectively. We begin with \mbox{\tt Mkbool}\indexTT{Mkbool}.

\gmruleodv%
{\gmstateodv{o}{\mbox{\tt Mkbool}\indexTT{Mkbool}:i}{s}{d}{t:v}{h}{m}}%
{\gmstateodv{o}{i}{a:s}{d}{v}{h[a: \mbox{\tt NConstr}\indexTT{NConstr}\  t\ []]}{m}}
%
The transition for \mbox{\tt Mkint}\indexTT{Mkint} is similar, except that it creates a new
integer node in the heap.

\gmruleodv%
{\gmstateodv{o}{\mbox{\tt Mkint}\indexTT{Mkint}:i}{s}{d}{n:v}{h}{m}}%
{\gmstateodv{o}{i}{a:s}{d}{v}{h[a: \mbox{\tt NNum}\indexTT{NNum}\  n]}{m}}

To perform the inverse operation we use \mbox{\tt Get}\indexTT{Get}. This is specified by
two transitions. In the first we see how \mbox{\tt Get}\indexTT{Get} treats a boolean on top
of the stack.

\gmruleodv%
{\gmstateodv{o}{\mbox{\tt Get}\indexTT{Get}:i}{a:s}{d}{v}{h[a: \mbox{\tt NConstr}\indexTT{NConstr}\  t\ []]}{m}}%
{\gmstateodv{o}{i}{s}{d}{t:v}{h}{m}}
%
In the second, we see how \mbox{\tt Get}\indexTT{Get} treats a number.

\gmruleodv%
{\gmstateodv{o}{\mbox{\tt Get}\indexTT{Get}:i}{a:s}{d}{v}{h[a: \mbox{\tt NNum}\indexTT{NNum}\  n]}{m}}%
{\gmstateodv{o}{i}{s}{d}{n:v}{h}{m}}

Finally, to make use of booleans on the V-stack we use a simplified
\mbox{\tt Casejump}\indexTT{Casejump} instruction that inspects the V-stack to determine which
instruction stream to use. This new instruction is called \mbox{\tt Cond}\indexTT{Cond}, and
is specified by the following two transitions. In the first -- with
the value on top of the V-stack being true -- we select the first code
sequence: $t$.

\gmruleodv%
{\gmstateodv{o}{\mbox{\tt Cond}\indexTT{Cond}\ t\ f:i}{s}{d}{2:v}{h}{m}}%
{\gmstateodv{o}{t\append i}{s}{d}{v}{h}{m}}
%
In the second transition we see that with a false value on top of the
V-stack \mbox{\tt Cond}\indexTT{Cond} selects its second code sequence: $f$.

\gmruleodv%
{\gmstateodv{o}{\mbox{\tt Cond}\indexTT{Cond}\ t\ f:i}{s}{d}{1:v}{h}{m}}%
{\gmstateodv{o}{f\append i}{s}{d}{v}{h}{m}}

\begin{exercise}\label{gm:X:instruction}
Extend the instruction data type, redefine the \mbox{\tt showInstruction}\indexTT{showInstruction}
function, implement the new instruction transitions and modify the
\mbox{\tt dispatch}\indexTT{dispatch} function.
\end{exercise}

We now consider the -- rather extensive -- modifications to the
compiler.

\subsection{The compiler\index{G-machine compiler!Mark 7}}

Because of the extra state component, the \mbox{\tt compile}\indexTT{compile} function must
initialise the V-stack component to be empty.
\begin{verbatim}
> compile :: CoreProgram -> GmState
> compile program
>      = ([], initialCode, [], [], [], heap, globals, statInitial)
>      where (heap, globals) = buildInitialHeap program
\end{verbatim}
\indexDTT{compile}%
\indexDTT{compile}%
Strictly speaking, this is all that is necessary to make the machine
work, but we have introduced the V-stack so that we can compile
arithmetic functions `in-line', so this is what we intend our code
to do.
\begin{verbatim}
> buildInitialHeap :: CoreProgram -> (GmHeap, GmGlobals)
> buildInitialHeap program
>      = mapAccuml allocateSc hInitial compiled
>      where compiled = map compileSc (preludeDefs ++ program ++ primitives)
\end{verbatim}
\indexDTT{buildInitialHeap}%
\indexDTT{buildInitialHeap}%
\par
Because of the changes to the transitions for the primitive
instruction, we must change the code for each compiled primitive.
Instead of hand compiling this code -- as we did for the Mark~6
machine -- we can instead give this job to the compiler. This of
course relies on the fact that the compiler is clever enough to
optimise the code it produces, otherwise we never generate any \mbox{\tt Add}\indexTT{Add}
instructions!
\begin{verbatim}
> primitives :: [(Name,[Name],CoreExpr)]
> primitives
>      =       [("+", ["x","y"], (EAp (EAp (EVar "+") (EVar "x")) (EVar "y"))),
>               ("-", ["x","y"], (EAp (EAp (EVar "-") (EVar "x")) (EVar "y"))),
>               ("*", ["x","y"], (EAp (EAp (EVar "*") (EVar "x")) (EVar "y"))),
>               ("/", ["x","y"], (EAp (EAp (EVar "/") (EVar "x")) (EVar "y"))),
\end{verbatim}
\indexDTT{primitives}%
We also need to add the negation function.
\begin{verbatim}
>               ("negate", ["x"], (EAp (EVar "negate") (EVar "x"))),
\end{verbatim}
Comparison functions are almost identical to the dyadic arithmetic
functions.
\begin{verbatim}
>               ("==", ["x","y"], (EAp (EAp (EVar "==") (EVar "x")) (EVar "y"))),
>               ("~=", ["x","y"], (EAp (EAp (EVar "~=") (EVar "x")) (EVar "y"))),
>               (">=", ["x","y"], (EAp (EAp (EVar ">=") (EVar "x")) (EVar "y"))),
>               (">",  ["x","y"], (EAp (EAp (EVar ">")  (EVar "x")) (EVar "y"))),
>               ("<=", ["x","y"], (EAp (EAp (EVar "<=") (EVar "x")) (EVar "y"))),
>               ("<",  ["x","y"], (EAp (EAp (EVar "<")  (EVar "x")) (EVar "y"))),
\end{verbatim}
Finally, we ought to include the conditional function, and some
supercombinators to represent boolean values.
\begin{verbatim}
>               ("if",  ["c","t","f"],
>                      (EAp (EAp (EAp (EVar "if") (EVar "c")) (EVar "t")) (EVar "f"))),
>               ("True",  [], (EConstr 2 0)),
>               ("False", [], (EConstr 1 0))]
\end{verbatim}
\subsubsection{The \tB{} compilation scheme}

The \tB{} scheme, shown in Figure~\ref{gm:fg:schemes7.b}, constitutes
another type of context. To be compiled by the \tB{} scheme, an
expression must not only be known to need evaluation to WHNF, it must
also be an expression of type integer or boolean. The following
expressions will propagate through the \tB{} scheme:
\begin{figure*}
$\begin{array}{|rcll|}
\hline
&&&\\
\multicolumn{4}{|l|}{\parbox[t]{29pc}{%
$\B{e}~\rho$ compiles code that evaluates an expression $e$ to WHNF,
in an environment $\rho$, leaving the result on the V stack.}}\\
&&&\\
\indexD{$\protect\cal \pB$}\B{i}\ \rho     & = & [\mbox{\tt Pushbasic}\indexTT{Pushbasic}\ i] &\\
\multicolumn{4}{|l|}{%
\indexD{$\protect\cal \pB$}\B{ \mbox{\tt let}\indexTT{let} ~ x_1 \mbox{\tt =} e_1 \mbox{\tt ;} ~\ldots \mbox{\tt ;} ~ x_n \mbox{\tt =} e_n ~ \mbox{\tt in} ~ e}~ \rho} \\
 & = & \C{e_1}~ \rho^{+0} \append \ldots \append &\\
    &&  \C{e_n}~\rho^{+(n-1)} \append & \\
    &&  \B{e}~\rho' \append [\mbox{\tt Pop}\indexTT{Pop}~ n]&\\
    &&\multicolumn{2}{r|}{%
\tr{where $\rho' = \rho^{+n}[x_1\mapsto n-1,\ldots,x_n\mapsto 0]$}}\\
%
\multicolumn{4}{|l|}{%
\indexD{$\protect\cal \pB$}\B{ \mbox{\tt letrec}\indexTT{letrec} ~ x_1 \mbox{\tt =} e_1 \mbox{\tt ;} ~\ldots \mbox{\tt ;} ~ x_n \mbox{\tt =} e_n ~ \mbox{\tt in} ~ e}~ \rho} \\
 & = & [\mbox{\tt Alloc}\indexTT{Alloc} ~ n] \append & \\
    &&\C{e_1}~ \rho' \append [\mbox{\tt Update}\indexTT{Update}\ n-1]\append \ldots \append &\\
                 &&       \C{e_n}~\rho' \append [\mbox{\tt Update}\indexTT{Update}\ 0] \append &\\
               &&         \B{e}~\rho' \append [\mbox{\tt Pop}\indexTT{Pop}~ n] &\\
    &&\multicolumn{2}{r|}{%
        \tr{where $\rho' = \rho^{+n}[x_1\mapsto n-1,\ldots,x_n\mapsto 0]$}}\\
\indexD{$\protect\cal \pB$}\B{e_0~ \mbox{\tt +} ~e_1}~ \rho& = & \B{e_1}~ \rho \append\B{e_0}~ \rho \append [\mbox{\tt Add}\indexTT{Add}] &\\
\multicolumn{4}{|r|}{\tr{And similarly for other arithmetic expressions}}\\
\indexD{$\protect\cal \pB$}\B{e_0~ \mbox{\tt ==} ~e_1}~ \rho& = & \B{e_1}~ \rho \append\B{e_0}~ \rho \append [\mbox{\tt Eq}] &\\
\multicolumn{4}{|r|}{\tr{And similarly for other comparison expressions}}\\
\indexD{$\protect\cal \pB$}\B{ \mbox{\tt negate}\indexTT{negate} ~ e}~ \rho& = & \B{e}~ \rho \append [\mbox{\tt Neg}\indexTT{Neg}] &\\
\indexD{$\protect\cal \pB$}\B{ \mbox{\tt if} ~e_0 ~ e_1 ~ e_2} ~\rho & = & \B{e_0}~\rho \append [\mbox{\tt Cond}\indexTT{Cond}\
(\B{e_1}~\rho)~ (\B{e_2}~\rho)] &\\
\indexD{$\protect\cal \pB$}\B{e}\ \rho     & = & \E{e}\ \rho \append [\mbox{\tt Get}\indexTT{Get}] &
\tr{the default case}\\
&&&\\
\hline
\end{array}$
\caption{The \tB{} compilation scheme}
\label{gm:fg:schemes7.b}
\end{figure*}
\begin{itemize}
\item If $\mbox{\tt let(rec)}\indexTT{let(rec)}~\Delta~\mbox{\tt in}~e$ occurs in a \tB{}-strict context
then the expression $e$ is also in a \tB{}-strict context.

\item If the expression $\mbox{\tt if} ~e_0~e_1~e_2$ occurs in a \tB{}-strict
context then the expressions $e_0$, $e_1$ and $e_2$ also occur in
\tB{}-strict contexts.

\item If $e_0~\odot~e_1$ occurs in a \tB{}-strict context, with $\odot$
being a comparison or arithmetic operator, then the expressions $e_0$
and $e_1$ also occur in \tB{}-strict contexts.

\item If $\mbox{\tt negate}\indexTT{negate} ~e~$ occurs in a \tB{}-strict context then so does
the expression $e$.
\end{itemize}

If we cannot recognise any of the special cases of expression, then
the compiled code will evaluate the expression using the \tE{} scheme,
and then perform a \mbox{\tt Get}\indexTT{Get} instruction. The \mbox{\tt Get}\indexTT{Get} instruction unboxes
the value left on top of the stack by the \tE{} scheme and moves it to
the V-stack.

This has left unspecified how we know that an expression is initially
in a \tB{}-strict context. The usual situation is that we generate a
\tB{}-strict context from the usual strict context: \tE{}, with the
additional knowledge that the value is of type integer or boolean.

\begin{exercise}\label{gm:X:7b}
Implement the \tB{} compiler scheme.
\end{exercise}

\subsubsection{The \tE{} compilation scheme}

The \tE{} scheme -- defined in Figure~\ref{gm:fg:schemes7.e} --
specifies that we should give special treatment to arithmetic and
comparison function calls. It differs from the version we used for the
Mark~6 machine because we make use of the \tB{} scheme to perform the
arithmetic and comparison operations. Again, if there are no special
cases, then we must use a default method to compile the expression.
This is simply to build the graph, using the \tC{} scheme, and then
place an \mbox{\tt Eval}\indexTT{Eval} instruction in the code stream. This will ensure that
the graph is evaluated to WHNF.
\begin{figure*}
$\begin{array}{|rcll|}
\hline
&&&\\
\multicolumn{4}{|l|}{\parbox[t]{29pc}{%
$\E{e}~\rho$ compiles code that evaluates an expression $e$ to WHNF in
environment $\rho$, leaving a pointer to the expression on top of the
stack.}}\\
&&&\\
\indexD{$\protect\cal \pE$}\E{i}\ \rho     & = & [\mbox{\tt Pushint}\indexTT{Pushint}\ i] &\\
\multicolumn{4}{|l|}{%
\indexD{$\protect\cal \pE$}\E{ \mbox{\tt let}\indexTT{let} ~x_1 \mbox{\tt =} e_1 \mbox{\tt ;} ~\ldots \mbox{\tt ;} ~x_n \mbox{\tt =} e_n~ \mbox{\tt in} ~ e}~ \rho} \\
 & = & \C{e_1}~ \rho^{+0} \append \ldots \append &\\
    &&  \C{e_n}~\rho^{+(n-1)} \append & \\
    &&  \E{e}~\rho' \append [\mbox{\tt Slide}\indexTT{Slide}~ n]&\\
    &&\multicolumn{2}{r|}{%
\tr{where $\rho' = \rho^{+n}[x_1\mapsto n-1,\ldots,x_n\mapsto 0]$}}\\
%
\multicolumn{4}{|l|}{%
\indexD{$\protect\cal \pE$}\E{ \mbox{\tt letrec}\indexTT{letrec} ~x_1 \mbox{\tt =} e_1 \mbox{\tt ;} ~\ldots \mbox{\tt ;} ~ x_n \mbox{\tt =} e_n~ \mbox{\tt in} ~e}~ \rho} \\
 & = & [\mbox{\tt Alloc}\indexTT{Alloc}~ n] \append & \\
    &&\C{e_1}~ \rho' \append [\mbox{\tt Update}\indexTT{Update}\ n-1]\append \ldots \append &\\
                 &&       \C{e_n}~\rho' \append [\mbox{\tt Update}\indexTT{Update}\ 0] \append &\\
               &&         \E{e}~\rho' \append [\mbox{\tt Slide}\indexTT{Slide}~ n] &\\
    &&\multicolumn{2}{r|}{%
        \tr{where $\rho' = \rho^{+n}[x_1\mapsto n-1,\ldots,x_n\mapsto 0]$}}\\

\indexD{$\protect\cal \pE$}\E{ \mbox{\tt case}\indexTT{case} ~ e ~ \mbox{\tt of} ~ alts}~ \rho & = &
\multicolumn{2}{l|}{%
\indexD{$\protect\cal \pE$}\E{e}~\rho \append  [\mbox{\tt Casejump}\indexTT{Casejump}\ \DE{alts}\ \rho]}\\
\indexD{$\protect\cal \pE$}\E{ \mbox{\tt Pack{\char'173}} t \mbox{\tt ,} a \mbox{\tt {\char'175}}\indexTT{Pack}~e_1~\ldots~e_a }~\rho &
           = & \C{e_a}~\rho\append\ldots
                          \C{e_1}~\rho\append [\mbox{\tt Pack}\indexTT{Pack}\ t\ a] & \\
\indexD{$\protect\cal \pE$}\E{e_0 ~\mbox{\tt +} ~e_1}~ \rho& = & \B{e_0 ~\mbox{\tt +}~ e_1}~ \rho \append [\mbox{\tt Mkint}\indexTT{Mkint}] &\\
\multicolumn{4}{|r|}{\tr{And similarly for other arithmetic expressions}}\\
\indexD{$\protect\cal \pE$}\E{e_0 ~\mbox{\tt ==} ~e_1}~ \rho& = & \B{e_0~ \mbox{\tt ==} ~e_1}~ \rho \append [\mbox{\tt Mkbool}\indexTT{Mkbool}] &\\
\multicolumn{4}{|r|}{\tr{And similarly for other comparison expressions}}\\
\indexD{$\protect\cal \pE$}\E{ \mbox{\tt negate}\indexTT{negate} ~e}~ \rho& = & \B{ \mbox{\tt negate}\indexTT{negate} ~e}~ \rho \append [\mbox{\tt Mkint}\indexTT{Mkint}] &\\
\indexD{$\protect\cal \pE$}\E{ \mbox{\tt if} ~e_0 ~ e_1 ~ e_2} ~\rho & = & \B{e_0}~\rho \append [\mbox{\tt Cond}\indexTT{Cond}\
(\E{e_1}~\rho)~ (\E{e_2}~\rho)] &\\
\indexD{$\protect\cal \pE$}\E{e}\ \rho     & = & \C{e}\ \rho \append [\mbox{\tt Eval}\indexTT{Eval}] &
\tr{the default case}\\
&&&\\
\hline
&&&\\
\multicolumn{4}{|l|}{\parbox[t]{29pc}{%
$\DE{alts}~\rho$ compiles the code for the alternatives in a \mbox{\tt case}\indexTT{case} expression.}}
\\
&&&\\
\indexD{$\protect\cal \pDE$}\DE{alt_1~ \ldots ~alt_n}~\rho
& = & [\AltE{alt_1}~\rho,~ \ldots , ~ \AltE{alt_n}~\rho] & \\
&&&\\
\hline
&&&\\
\multicolumn{4}{|l|}{\parbox[t]{29pc}{%
$\AltE{alt}~\rho$ compiles the code for an alternative in a \mbox{\tt case}\indexTT{case} expression.}}\\
&&&\\
\multicolumn{4}{|l|}{%
\AltE{\mbox{\tt <}t\mbox{\tt >}~x_1\ldots x_n~\mbox{\tt ->}~body}~\rho}\\
& = & \multicolumn{2}{l|}{%
t~ \mbox{\tt ->} ~ [\mbox{\tt Split}\indexTT{Split}\ n] \append \E{body}~\rho' \append [\mbox{\tt Slide}\indexTT{Slide}\ n]}\\
&&\multicolumn{2}{r|}{%
\tr{where $\rho' = \rho^{+n}[x_1\mapsto 0\ldots x_n\mapsto n-1]$}}\\
&&&\\
\hline
\end{array}$
\caption{The Mark~7 \tE{}, \tDE{} and \tAltE{} compilation schemes}
\label{gm:fg:schemes7.e}
\end{figure*}

\begin{exercise}\label{gm:X:e7}
Implement the new \tE{} compiler scheme.
\end{exercise}

\subsubsection{The \tR{} compilation scheme}

We also take this opportunity to improve the \tR{} scheme. Firstly, we
wish to create opportunities for the \tB{} scheme to be used. We are
also attempting to reduce the number of instructions that are executed
at the end of a function's code sequence. The new compilation schemes
for the \tR{} scheme are given in Figure~\ref{gm:fg:schemes7.r}. It has
been expanded from the version we used in the Mark~6 machine; in that
it now works like a {\em context}\index{compilation context}. We refer to this
context as \tR{}-strict. An expression being compiled in a \tR{}-strict
context will be evaluated to WHNF, and it will then be used to
overwrite the current redex. It obeys the following rules of
propagation.
\begin{figure*}
$\begin{array}{|rcll|}
\hline
&&&\\
\multicolumn{4}{|l|}{\parbox[t]{29pc}{%
$\R{e}~\rho~d$ generates code which instantiates the expression $e$ in
environment $\rho$, for a supercombinator of arity $d$, and then proceeds
to unwind the resulting stack.}}\\
&&&\\
\multicolumn{4}{|l|}{%
\indexD{$\protect\cal \pR$}\R{ \mbox{\tt let}\indexTT{let} ~ x_1 \mbox{\tt =} e_1 \mbox{\tt ;} ~\ldots \mbox{\tt ;} ~x_n \mbox{\tt =} e_n~ \mbox{\tt in} ~e}~ \rho~ d} \\
 & = & \C{e_1}~ \rho^{+0} \append \ldots \append &\\
    &&  \C{e_n}~\rho^{+(n-1)} \append & \\
    &&  \R{e}~\rho' ~ (n+d)&\\
    &&\multicolumn{2}{r|}{%
\tr{where $\rho' = \rho^{+n}[x_1\mapsto n-1,\ldots,x_n\mapsto 0]$}}\\
%
\multicolumn{4}{|l|}{%
\indexD{$\protect\cal \pR$}\R{ \mbox{\tt letrec}\indexTT{letrec} ~x_1 \mbox{\tt =} e_1 \mbox{\tt ;} ~\ldots \mbox{\tt ;} ~x_n \mbox{\tt =} e_n ~ \mbox{\tt in} ~ e}~ \rho~ d} \\
 & = & [\mbox{\tt Alloc}\indexTT{Alloc}~ n] \append & \\
    &&\C{e_1}~ \rho' \append [\mbox{\tt Update}\indexTT{Update}\ n-1]\append \ldots \append &\\
                 &&       \C{e_n}~\rho' \append [\mbox{\tt Update}\indexTT{Update}\ 0] \append &\\
               &&         \R{e}~\rho' (n+d)&\\
    &&\multicolumn{2}{r|}{%
        \tr{where $\rho' = \rho^{+n}[x_1\mapsto n-1,\ldots,x_n\mapsto 0]$}}\\
\indexD{$\protect\cal \pR$}\R{ \mbox{\tt if} ~e_0 ~ e_1 ~ e_2} ~\rho~d & = & \B{e_0}~\rho \append [\mbox{\tt Cond}\indexTT{Cond}\
[\R{e_1}~\rho~d,\, \R{e_2}~\rho~d]] &\\
\indexD{$\protect\cal \pR$}\R{\mbox{\tt case}\indexTT{case}~ e ~\mbox{\tt of}~ alts}~ \rho ~d & = &
\indexD{$\protect\cal \pE$}\E{e}~\rho \append  [\mbox{\tt Casejump}\indexTT{Casejump}\ \DR{alts}\ \rho~ d] & \\
\indexD{$\protect\cal \pR$}\R{e}\ \rho~d     & = & \E{e}\ \rho \append [\mbox{\tt Update}\indexTT{Update}\ d, \mbox{\tt Pop}\indexTT{Pop}\ d, \mbox{\tt Unwind}\indexTT{Unwind}] &
\tr{the default case}\\
&&&\\
\hline
&&&\\
\multicolumn{4}{|l|}{\parbox[t]{29pc}{%
$\DR{alts}~\rho~d$ compiles the code for the alternatives in a \mbox{\tt case}\indexTT{case}
expression.}}\\
&&&\\
\indexD{$\protect\cal \pDR$}\DR{alt_1~ \ldots ~alt_n}~\rho~d
& = & [\AR{alt_1}~\rho~d,~ \ldots ,~ \AR{alt_n}~\rho~d] & \\
&&&\\
\hline
&&&\\
\multicolumn{4}{|l|}{\parbox[t]{29pc}{%
$\AR{alt}~\rho~d$ compiles the code for an alternative in a \mbox{\tt case}\indexTT{case}
expression.}}\\
&&&\\
\multicolumn{4}{|l|}{%
\indexD{$\protect\cal \pAR$}\AR{ \mbox{\tt <} t \mbox{\tt >} ~x_1\ldots x_n~ \mbox{\tt ->} ~body}~\rho~d}\\
& = & \multicolumn{2}{l|}{%
t ~ \mbox{\tt ->} ~ [\mbox{\tt Split}\indexTT{Split}\ n] \append \R{body}~\rho'~(n+d)}\\
&&\multicolumn{2}{r|}{%
\tr{where $\rho' = \rho^{+n}[x_1\mapsto 0\ldots x_n\mapsto n-1]$}}\\
&&&\\
\hline
\end{array}$
\caption{The Mark~7 \tR{}, \tDR{}, and \tAR{} compilation schemes}
\label{gm:fg:schemes7.r}
\end{figure*}

\begin{itemize}
\item The expression in the body of a supercombinator definition is in
an \tR{}-strict context.

\item If $\mbox{\tt let(rec)}\indexTT{let(rec)}~\Delta~\mbox{\tt in}~e$ occurs in an \tR{}-strict context
then the expression $e$ is also in an \tR{}-strict context.

\item If the expression $\mbox{\tt if} ~e_0~e_1~e_2$
occurs in an \tR{}-strict context then the
expressions $e_1$ and $e_2$ are also in an \tR{}-strict context.
(The expression $e_0$ will now appear in a \tB{}-strict context.)

\item If $\mbox{\tt case}\indexTT{case} ~e~ \mbox{\tt of} ~alts$ occurs in an \tR{}-strict context then
the expression $e$ is in a strict context. Furthermore, the expression
part of each alternative will occur in an \tR{}-strict context.

\end{itemize}

\begin{exercise}\label{gm:X:comp7r}
Implement the \tR{} scheme in Figure~\ref{gm:fg:schemes7.r}.  Note that
you can use the generality of the \mbox{\tt compileAlts}\indexTT{compileAlts} function to implement
both the \tAR{} and \tAltE{} schemes.
\end{exercise}

One point worth noting about the Mark~7 machine is that we did not
define the \mbox{\tt Eval}\indexTT{Eval} instruction to save the current V-stack on the dump.
This is in contrast to the G-machine described in \cite{PJBook}.
Whether you do this is really a matter of taste in the abstract
machines we have produced in this book. In compiling code for a real
machine we would be likely to try to minimise the number of stacks.
When this is the case, we will wish to use the following alternative
transition for \mbox{\tt Eval}\indexTT{Eval}.

\gmruleodv%
{\gmstateodv{o}{\mbox{\tt Eval}\indexTT{Eval}: i}{a:s}{d}{v}{h}{m}}%
{\gmstateodv{o}{[\mbox{\tt Unwind}\indexTT{Unwind}]}{[a]}{\langle i,\, s,\, v\rangle:d}{[]}{h}{m}}

\begin{exercise}
Implement this alternative transition for \mbox{\tt Eval}\indexTT{Eval}. What are the other
instructions that will need to be changed to allow for this new
transition?
\end{exercise}

\begin{exercise}
What compilation rules would you change to produce optimised code
for the boolean operators \mbox{\tt {\char'46}}, \mbox{\tt |} and \mbox{\tt not}\indexTT{not}?
\end{exercise}

\begin{exercise}
Add a \mbox{\tt Return}\indexTT{Return} instruction with transition:

\gmruleodv%
{\gmstateodv{o}{[\mbox{\tt Return}\indexTT{Return}]}{[a_0,\ldots ,\, a_k]}{\langle i,\, s\rangle:d}{v}{h}{m}}%
{\gmstateodv{o}{i}{a_k:s}{d}{v}{h}{m}}

This is used by the \tR{} scheme in place of \mbox{\tt Unwind}\indexTT{Unwind} when the item on
top of the stack is known to be in WHNF. Modify \mbox{\tt compileR}\indexTT{compileR} to generate
this new instruction.
\end{exercise}

\begin{exercise}
Write out the transition for \mbox{\tt UpdateInt\ n}. This instruction performs
the same actions as the sequence \mbox{\tt [Mkint,\ Update\ n]}. Implement this
transition and a similar one for \mbox{\tt UpdateBool\ n}. Modify \mbox{\tt compileR}\indexTT{compileR} to
generate these instructions instead of the original sequences.

Why are the new instructions preferred to the original sequences?
(Hint: use the statistics from some sample programs.)
\end{exercise}

\section{Conclusions}

The approach that we have taken in this chapter is a very useful one
when designing large pieces of software. First we start with something
very simple, and then by a number of gradual changes we produce a very
large and complicated piece of software.  It would be misleading to
claim that this is always possible by a process of small, incremental,
changes. In fact the material presented as part of the Mark~1 machine
was specifically designed with the Mark~7 machine in mind.

For the moment, however, we have produced a reasonably efficient
machine.  In the next chapter we will look at the TIM.

\theendnotes

% end of chap03
