\batchmode
\documentstyle[11pt,fpstyle]{report}

\begin{document}

\title{Installation guide\\
for\\
Implementing functional languages: a tutorial}
\author{Simon L Peyton Jones \\ 
        Department of Computing Science, University of Glasgow
\and
        and David R Lester \\
         Department of Computer Science, University of Manchester\\
         \copyright{} 1991}

\maketitle

\chapter{Introduction}

This installation guide to ``Implementing functional languages: a tutorial''
provides you with:
\begin{itemize}
\item
Information about how to get the machine-readable version of the book.
\item
A detailed description of how all the files inter-relate, and how to 
use them.
\end{itemize}

To use the material you will need the following programs: 
\mbox{\tt expand}\indexTT{expand}, \mbox{\tt latex}\indexTT{latex}, \mbox{\tt makeindex}\indexTT{makeindex}\footnote{
\mbox{\tt makeindex}\indexTT{makeindex} is part of the \LaTeX{} world; any good installation will have it.
Its one true home is \mbox{\tt science.utah.edu:{\char'176}ftp/pub/tex/pub/makeindex}; I
got ours from \mbox{\tt ucbarpa.berkeley.edu:{\char'176}ftp/pub}.  It's on the Unix \TeX{}
tape in \mbox{\tt tex82/LateX/LateXmakeindex/src}.  The ``supplementary info''
posting in \mbox{\tt comp.text.tex} says \mbox{\tt metsat.met.fsu.edu} is also an FTP site,
and a Mac version is available from \mbox{\tt midway.uchicago.edu}.  The ``archie''
server nearest you will probably tell you about a zillion other sites.

The current version of \mbox{\tt makeindex}\indexTT{makeindex} is 2.11.
}, \mbox{\tt perl}\indexTT{perl}\footnote{
\mbox{\tt perl}\indexTT{perl} can be obtained from any \mbox{\tt comp.sources.unix} 
archive.  These machines, at the very least,
definitely have it available for anonymous FTP:
\begin{tabular}{ll}
\mbox{\tt ftp.uu.net}                    &       137.39.1.2 \\
\mbox{\tt archive.cis.ohio-state.edu}    &       128.146.8.52 \\
\mbox{\tt jpl-devvax.jpl.nasa.gov}       &       128.149.1.143 \\
\mbox{\tt archive.cs.ruu.nl}             &       131.211.80.5
\end{tabular}
}, \mbox{\tt lex}\indexTT{lex}, \mbox{\tt cc} 
and Miranda\footnote{Miranda is a trademark of 
Research Software Ltd.}.

The \mbox{\tt Makefile}\indexTT{Makefile} uses a few non-standard features of \mbox{\tt make}\indexTT{make}, so use Sun \mbox{\tt make}\indexTT{make} or
Gnu \mbox{\tt gmake}\indexTT{gmake}.

\chapter{Getting up-to-date source code}

In case you are reading a paper copy of this document, we begin by
describing how to get hold of the machine-readable sources for the book,
and how to keep them up to date.

Once you have got the sources, read the \mbox{\tt README}\indexTT{README} file.  The version number
of the sources you have is given in the \mbox{\tt Makefile}\indexTT{Makefile}.

\section{Getting the sources}
% Information about getting the source files.
% This is in a separate file because it is included in both the
% main book and the installation guide

You can get all the source code for this book,
by network file transfer (FTP) from the
several sites.  In each case, you need only get the file
$$
\mbox{\tt pjlester-}n\mbox{\tt .}m\mbox{\tt .tar.Z}
$$
where $n\mbox{\tt .}m$ is the current version number of the book. There is always
only one such file, but the $n\mbox{\tt .}m$ may vary as we correct errors and
otherwise improve the material.  Once you have got the file, run the command
$$
\mbox{\tt zcat\ pjlester-}n\mbox{\tt .}m\mbox{\tt .tar.Z\ |\ tar\ xf\ -}
$$
and then read or print the \mbox{\tt README}\indexTT{README} file, and the DVI file \mbox{\tt installation.dvi}.
If you don't have \mbox{\tt zcat}\indexTT{zcat}, do the following instead:
$$
\begin{array}{l}
\mbox{\tt uncompress\ pjlester-}n\mbox{\tt .}m\mbox{\tt .tar.Z} \\
\mbox{\tt tar\ xf\ pjlester-}n\mbox{\tt .}m\mbox{\tt .tar}
\end{array}
$$

The sites where the sources are held are as follows.
\begin{center}
\begin{tabular}{|llll|}
\hline
Site    & Host name               & Host address & Directory \\
\hline
Glasgow & \mbox{\tt ftp.dcs.glasgow.ac.uk} & 130.209.240.50 & \mbox{\tt pub/pj-lester-book}\\
Yale    & \mbox{\tt nebula.cs.yale.edu}    & 128.36.13.1 & \mbox{\tt pub/pj-lester-book}  \\
Chalmers& \mbox{\tt animal.cs.chalmers.se} & 129.16.2.27 & \mbox{\tt pub/pj-lester-book} \\
\hline
\end{tabular}
\end{center}
Log in as \mbox{\tt anonymous}\indexTT{anonymous}, and use your electronic mail address
as password.
Here is a sample Internet FTP session:
\begin{verbatim}
    % ftp nebula.cs.yale.edu
    Connected to NEBULA.SYSTEMSZ.CS.YALE.EDU.
    220 nebula FTP server (SunOS 4.0) ready.
    Name (nebula.cs.yale.edu:guestftp): anonymous
    331 Guest login ok, send ident as password.
    Password: simonpj@dcs.glasgow.ac.uk
    230 Guest login ok, access restrictions apply.
    ftp> type binary
    200 Type set to I.
    ftp> cd pub/pj-lester-book
    250 CWD command successful.
    ftp> get pjlester-1.2.tar.Z
    <messages about a successful transfer would come here>
    ftp> bye
\end{verbatim}
Within the UK, you may get the above file from
\mbox{\tt uk.ac.glasgow.dcs} by anonymous UK NIFTP 
(binary mode; user: \mbox{\tt guest}\indexTT{guest}; password: your e-mail address);
request:
$$
\mbox{\tt <FP>/pj-lester-book/}filename
$$
A typical command you might use on at least some Unix machines is:
\begin{verbatim}
  cpf -b '<FP>/pj-lester-book/pjlester-1.2.tar.Z@uk.ac.glasgow.dcs' 
         pjlester-1.2.tar.Z
\end{verbatim}


\section{Getting the Tutor's Guide}

The files comprising the Tutor's Guide 
contain answers to many of the exercises, so they are not included in
the distribution.

To get them, send a letter on your departmental
letterhead to Simon Peyton Jones, Department of Computing Science,
University of Glasgow, G12 8QQ, UK.  (Or, if you are already in email contact,
send him electronic mail.)

\section{Updating the sources}

From time to time we will doubtless fix errors and improve the distribution.
As well as the main distribution file \mbox{\tt pjlester-}n\mbox{\tt .}m\mbox{\tt .tar.Z}, the
distribution directory contains a collection of files named
$$
\mbox{\tt pjlester-}n\mbox{\tt .}m\mbox{\tt -}p\mbox{\tt .}q\mbox{\tt .diffs.Z}
$$
Each of these is a compressed file of differences between version $n\mbox{\tt .}m$ and
version $p\mbox{\tt .}q$ of the book.  If you already have version $n\mbox{\tt .}m$, you can
change it to version $p\mbox{\tt .}q$ as follows:
\begin{itemize}
\item
\mbox{\tt cd} into the directory containing all the book sources.
\item
Run the following command:
$$
\mbox{\tt zcat\ pjlester-}n\mbox{\tt .}m\mbox{\tt -}p\mbox{\tt .}q\mbox{\tt .diffs.Z\ |\ patch\ -p1}
$$
\end{itemize}
{\em Warning: this does not updated the (binary) file \mbox{\tt installation.dvi}.
If you want the up-to-date version of this you must either get it
from the current main distribution, or re-make it from installation.src.}

You can find out what version number you have by looking in 
the file \mbox{\tt Makefile}\indexTT{Makefile}.
The \mbox{\tt patch}\indexTT{patch} program does this too, and will complain if you try to
apply a set of diffs to the wrong version.

The release history is given in the file \mbox{\tt Release-history}.

\chapter{Source files and their format}

All of the text for the two books is contained in the following files:
\begin{center}
\begin{tabular}{|lll|}
\hline 
Contents        & Student text          & Tutor's Guide \\
\hline
Root file       & \mbox{\tt student.src}         & \mbox{\tt tutor.src} \\
Title           & \mbox{\tt title.src}           & \mbox{\tt title-tutor.src} \\
Preface         & \mbox{\tt preface.src}         & \mbox{\tt preface-tutor.src} \\
Chapter 1       & \mbox{\tt language.src}        & \mbox{\tt language-tutor.src} \\
Chapter 2       & \mbox{\tt template.src}        & \mbox{\tt template-tutor.src} \\
Chapter 3       & \mbox{\tt gm.src}              & \mbox{\tt gm-tutor.src} \\
Chapter 4       & \mbox{\tt tim.src}             & \mbox{\tt tim-tutor.src} \\
Chapter 5       & \mbox{\tt pargm.src}           & \mbox{\tt pargm-tutor.src} \\
Chapter 6       & \mbox{\tt lambda.src}          & \mbox{\tt lambda-tutor.src} \\
Appendix        & \mbox{\tt utils.src}           & \\
Bibliography    & \mbox{\tt bibliography.src}    & \\
Style file      & \multicolumn{2}{c}{\mbox{\tt fpstyle.sty}} \\
\hline
\end{tabular}
\end{center}
The root file in each case is what is given to \LaTeX{}, which then includes
the others.
All of these \mbox{\tt .src} files are what we call {\em literate source files}.
From them we can mechanically construct:
\begin{itemize}
\item
Any of several versions of the Miranda code 
written in the text.
\item
Each of the two books, complete with a mechanically-generated index of
the functions defined in the text.
\end{itemize}
We discuss these two in order.

\section{Generating executable Miranda files}

Lines of Miranda source code are distinguished in the \mbox{\tt .src} files
by the ``inverse comment convention''; that is, lines of Miranda source
code begin with a \mbox{\tt >} sign.

Because Miranda understands the inverse comment convention,
the \mbox{\tt .src} files are very nearly valid Miranda source files.  But not quite.
In the text we develop a succession of versions of each implementation:
Mark 1, Mark 2, and so on.
There are two issues here.  Firstly, all of the Mark 1 code is in the
student text, but much of the code for subsequent versions is in the form
of exercises, and so does not appear in the student text.  The Tutor's Guide
contains all the missing code for the subsequent versions.

Secondly,
in the text we often refine the definition of a particular data type
or function.  As a result, the text may contain several definitions
of a particular data type or function, which Miranda will reject.
So we want to extract from the source file
the code for Mark 1, say, or the code for Mark 3.
This requires some way of indicating which version each line of code
belongs to, for which we use the following convention:
\begin{itemize}
\item
A line like this:
\begin{verbatim}
 > f x = x+1
\end{verbatim}
belongs to every version.
\item
A line like this
\begin{verbatim}
 2> f x = x+1
\end{verbatim}
belongs to version 2 only.  Any single-digit version number can be used.
\item
A line like this
\begin{verbatim}
 2-4> f x = x+1
\end{verbatim}
belongs to versions 2, 3 and 4.  Any pair of single-digit version numbers can
be used.
\item
Finally, a line like this
\begin{verbatim}
 2-> f x = x+1
\end{verbatim}
belongs to version 2 and every subsequent version.
\end{itemize}
Of course, this convention makes the scripts must be pre-processed before
being given to Miranda.  

\subsection{How to generate the Miranda sources}

The \mbox{\tt Makefile}\indexTT{Makefile} knows how to generate each version of each machine, so
all you have to do is say, for example, 
\mbox{\tt make\ template3.m} to generate the Miranda
source (\mbox{\tt template3.m}) for Mark 3 of the template-instantation machine.

The targets \mbox{\tt language.m} and \mbox{\tt utils.m} (with no version numbers) generate
the most advanced versions of the Core-language and Utilities 
modules respecively.
These two modules are used widely through the book and it is inconvenient
to have to say \mbox{\tt language3.m}, for example.

\subsection{How the \mbox{\tt Makefile}\indexTT{Makefile} does its work}

The \mbox{\tt Makefile}\indexTT{Makefile} uses an \mbox{\tt awk}\indexTT{awk} script called \mbox{\tt make-version}, 
which extracts a given version from the source file.  Its usage is as follows:
$$
\mbox{\tt make-version}~n
$$
takes a source file on its standard input, filters out any program lines
which do not belong to version $n$, and puts the result on its standard output.
For example
\begin{verbatim}
        cat template.src template-tutor.src | make-version 3 > template3.m
\end{verbatim}
makes version 3 of the template-instantiation machine, putting the result
in \mbox{\tt template3.m}.  The student and tutor source files are concatenated
so that the required code is included in the output.

\subsection{Different versions of Miranda}

All the Miranda code in the book should run on both Version 1 and Version 2
Miranda.  We have taken the following actions to avoid compatibility problems:
\begin{itemize}
\item
Guards are written without a keyword \mbox{\tt if}.  This is illegal in Version 1
and optional in Version 2.  We always use a ``\mbox{\tt ,\ otherwise}'' guard on the
final clause of a guarded set of equations; this is optional in Version 1
but compulsory in Version 2.
\item
The definition of the standard function \mbox{\tt foldl}\indexTT{foldl} differs between Version 1 and
Version 2.  In Appendix~\ref{sect:utils} we 
define our own function \mbox{\tt foldll}\indexTT{foldll} (which is the same
as Version 2's \mbox{\tt foldl}\indexTT{foldl}), and always use it instead of \mbox{\tt foldl}\indexTT{foldl}.
\item
Version 1 lacks the standard functions \mbox{\tt fst}\indexTT{fst} and \mbox{\tt snd}\indexTT{snd}.  We use 
\mbox{\tt first}\indexTT{first} and \mbox{\tt second}\indexTT{second} instead, which are defined in 
Appendix~\ref{sect:utils}.
\item
We do not use any of the new features introduced in Version 2, nor do
we use strictness annotations and laws which were eliminated in Version 2.
\end{itemize}

\section{Typesetting the book}

The \mbox{\tt .src} files are also very nearly \LaTeX{} source files, but again not quite.
There are two differences: typewriter font text is written more compactly than
in \LaTeX{}, and indexing is added automatically for Miranda functions.

From this material you can typeset two books: the main textbook (\mbox{\tt student.dvi}),
and the Tutor's Guide (\mbox{\tt tutor.dvi}).  Please do not make bulk copies of
the former; we would prefer student to buy the book!

\subsection{Running \LaTeX{}}

\begin{enumerate}
\item
The distribution includes a typeset version of the installation guide (this
document), namely \mbox{\tt installation.dvi}.  Still, if you want to re-make it, 
say \mbox{\tt make\ installation.dvi}.
\item
To typeset the book, just say \mbox{\tt make}\indexTT{make}; the result should be in \mbox{\tt student.dvi}.  

To get started, the file \mbox{\tt student.idx} should exist or the \mbox{\tt Makefile}\indexTT{Makefile} will
complain.  So start by saying \mbox{\tt touch\ student.idx}.  (The distribution
contains a null \mbox{\tt student.idx} for this very reason.)

You'll get a lot of unresolved references and no index the first time; just 
say \mbox{\tt make\ student.dvi} twice more, and everything should get resolved.  (\mbox{\tt make}\indexTT{make} will
re-\LaTeX{} the book, because it spots that the index has altered.)

{\bf Warning.}  You will need a big \LaTeX{} to typeset the book.
\item
To typeset the tutor's guide, say \mbox{\tt make\ tutor.dvi}.
\end{enumerate}

\subsection{Figures}

The figures were drawn using the Fig program, whose output is in files
with extension \mbox{\tt .fig}.  These are translated into \LaTeX{} by the \mbox{\tt fig2dev}\indexTT{fig2dev} 
program.

The release includes the \mbox{\tt .tex} files as well as the \mbox{\tt .fig} ones, 
in case you havn't got \mbox{\tt fig2dev}\indexTT{fig2dev}.

\subsection{Font selection}

Newer \LaTeX{} installations use the New Font Selection Scheme (NFSS).
Old versions don't.  There are a couple of places in the style file \mbox{\tt fpstyle.sty}
where we've put in some conditional code which is meant to do the right thing
regardless of which version you have.  This certainly works if you have NFSS, 
but we can't guarantee it works if you don't.  This section tells you what
we are trying to do, so you can get local help if you need it.

There are two places font selection shows up:
\begin{itemize}
\item
Like many others, we tell \LaTeX{} (or is it \TeX{}?) to make math italic letters appear
as text italic.  
This is a fine point.  It only makes a significant difference when
the letter $f$ is used as part of a word in math italic,
and even then everything is legible either way.

\item
The figures, whose \LaTeX{} code is generated by \mbox{\tt fig}\indexTT{fig}, include
commands, such as \mbox{\tt {\char'134}ninrm} and \mbox{\tt {\char'134}sevtt}, which are meant to select a
font (nine-point Roman and seven-point typewriter font respectively).
This is done directly by NFSS, but \mbox{\tt fpstyle.sty} defines some macros for them
if NFSS isn't there.
\end{itemize}

\subsection{Typewriter font material}

The text contains a great deal of material in typewriter font, both inline
in a paragraph, \mbox{\tt like}\indexTT{like} \mbox{\tt this}\indexTT{this}, and displayed as Miranda source

 > like this

Taking the Miranda source code first, it is already distinguished by 
the inverse comment convention,
so it would be very tiresome to have to surround it with
\mbox{\tt {\char'134}begin{\char'173}verbatim{\char'175}} and \mbox{\tt {\char'134}end{\char'173}verbatim{\char'175}} \LaTeX{} commands as well.
Furthermore, as described above, we have extended the inverse comment
convention to describe versions.  But we do not want these version numbers
to appear in the typeset text!  We have provided, therefore, a program
called \mbox{\tt verbatim}\indexTT{verbatim} which removes the version-numbering, and adds \mbox{\tt begin}\indexTT{begin}/\mbox{\tt end}\indexTT{end}
brackets around Miranda code.

Returning to inline material, it makes source text quite illegible to keep
writing material in typewriter font using the \mbox{\tt {\char'134}tt} command.  For example
\begin{verbatim}
        The variables {\tt a} and {\tt b} in function {\tt fun} are gone.
\end{verbatim}
Instead we write
\begin{verbatim}
        The variables @a@ and @b@ in function @fun@\indexTT{fun} are gone.
\end{verbatim}
and rely on \mbox{\tt verbatim}\indexTT{verbatim} to convert the {\tt @@} pairs
into the appropriate use of \mbox{\tt {\char'134}tt}.  The improvement is even more striking
in the case of mathematics.  Compare
\begin{verbatim}
        The function call $@mapAccuml@\indexTT{mapAccuml}~f~acc~xs$ works well
\end{verbatim}
with
\begin{verbatim}
        The function call $\mbox{\tt mapAccuml}~f~acc~xs$ works well
\end{verbatim}
The \mbox{\tt verbatim}\indexTT{verbatim} program, which is actually a \mbox{\tt lex}\indexTT{lex} script, takes input on
its standard input, adds the extra \LaTeX{} commands described, and puts the
output on its standard output.

\subsection{Indexing}

It is highly desirable to have a complete and correct index of all
the Miranda functions defined in the book.  It would be time-consuming and
error-prone to do so by hand, so we have automated the process.  The program
\mbox{\tt add-index-entries} takes a source file on its standard input ({\em before}
it has been passed through \mbox{\tt verbatim}\indexTT{verbatim}), and adds calls to the
indexing macros \mbox{\tt {\char'134}indexDef} (for a defining use, underlined in the
index) and \mbox{\tt {\char'134}indexUse} (for a non-defining use).
These macros are defined in \mbox{\tt fpstyle.sty}.

\mbox{\tt add-index-entries} chooses what to index as follows:
\begin{description}
\item[Function and type definitions.]
It indexes as a definition
any identifier which follows the \mbox{\tt >} of a program line
(including one decorated with a version number), separated from the \mbox{\tt >}
by a single space.
\item[Constructor definitions.]
It indexes as a definition any constructor name (starting with an upper-case
letter) which follows \mbox{\tt ::=} or \mbox{\tt |}.
\item
It indexes as a use any identifier which appears between @@ signs in
running text.
\item
Only identifiers which are at least three characters long are indexed.
\end{description}

\end{document}


